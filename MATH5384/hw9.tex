% Thank you Josh Davis for this template!
% https://github.com/jdavis/latex-homework-template/blob/master/homework.tex

\documentclass{article}

\newcommand{\hmwkTitle}{HW\ \#9}

% % ----------

% Packages

\usepackage{fancyhdr}
\usepackage{extramarks}
\usepackage{amsmath}
\usepackage{amssymb}
\usepackage{amsthm}
\usepackage{amsfonts}
\usepackage{tikz}
\usepackage[plain]{algorithm}
\usepackage{algpseudocode}
\usepackage{enumitem}
\usepackage{chngcntr}

% Libraries

\usetikzlibrary{automata, positioning, arrows}

%
% Basic Document Settings
%

\topmargin=-0.45in
\evensidemargin=0in
\oddsidemargin=0in
\textwidth=6.5in
\textheight=9.0in
\headsep=0.25in

\linespread{1.1}

\pagestyle{fancy}
\lhead{\hmwkAuthorName}
\chead{}
\rhead{\hmwkClass\ (\hmwkClassInstructor): \hmwkTitle}
\lfoot{\lastxmark}
\cfoot{\thepage}

\renewcommand\headrulewidth{0.4pt}
\renewcommand\footrulewidth{0.4pt}

\setlength\parindent{0pt}
\setcounter{secnumdepth}{0}

\newcommand{\hmwkClass}{MATH 3380 / Analysis 1}        % Class
\newcommand{\hmwkClassInstructor}{Dr. Welsh}           % Instructor
\newcommand{\hmwkAuthorName}{\textbf{Joshua Mitchell}} % Author

%
% Title Page
%

\title{
    \vspace{2in}
    \textmd{\textbf{\hmwkClass:\ \hmwkTitle}}\\
    \normalsize\vspace{0.1in}\small\vspace{0.1in}\large{\textit{\hmwkClassInstructor}}
    \vspace{3in}
}

\author{\hmwkAuthorName}
\date{}

\renewcommand{\part}[1]{\textbf{\large Part \Alph{partCounter}}\stepcounter{partCounter}\\}

% Integral dx
\newcommand{\dx}{\mathrm{d}x}

%
% Various Helper Commands
%

% For derivatives
\newcommand{\deriv}[1]{\frac{\mathrm{d}}{\mathrm{d}x} (#1)}

% For partial derivatives
\newcommand{\pderiv}[2]{\frac{\partial}{\partial #1} (#2)}


% Alias for the Solution section header
\newcommand{\solution}{\textbf{\large Solution}}

% Probability commands: Expectation, Variance, Covariance, Bias
\newcommand{\E}{\mathrm{E}}
\newcommand{\Var}{\mathrm{Var}}
\newcommand{\Cov}{\mathrm{Cov}}
\newcommand{\Bias}{\mathrm{Bias}}

% Formatting commands:

\newcommand{\mt}[1]{\ensuremath{#1}}
\newcommand{\nm}[1]{\textrm{#1}}

\newcommand\bsc[2][\DefaultOpt]{%
  \def\DefaultOpt{#2}%
  \section[#1]{#2}%
}
\newcommand\ssc[2][\DefaultOpt]{%
  \def\DefaultOpt{#2}%
  \subsection[#1]{#2}%
}
\newcommand{\bgpf}{\begin{proof} $ $\newline}

\newcommand{\bgeq}{\begin{equation*}}
\newcommand{\eeq}{\end{equation*}}	

\newcommand{\balist}{\begin{enumerate}[label=\alph*.]}
\newcommand{\elist}{\end{enumerate}}

\newcommand{\bilist}{\begin{enumerate}[label=\roman*)]}	

\newcommand{\bgsp}{\begin{split}}
% \newcommand{\esp}{\end{split}} % doesn't work for some reason.

\newcommand\prs[1]{~~~\textbf{(#1)}}

\newcommand{\lt}[1]{\textbf{Let: } #1}
     							   %  if you're setting it to be true
\newcommand{\supp}[1]{\textbf{Suppose: } #1}
     							   %  Suppose (if it'll end up false)
\newcommand{\wts}[1]{\textbf{Want to show: } #1}
     							   %  Want to show
\newcommand{\as}[1]{\textbf{Assume: } #1}
     							   %  if you think it follows from truth
\newcommand{\bpth}[1]{\textbf{(#1)}}

\newcommand{\step}[2]{\begin{equation}\tag{#2}#1\end{equation}}
\newcommand{\epf}{\end{proof}}

\newcommand{\dbs}[3]{\mt{#1_{#2_#3}}}

\newcommand{\sidenote}[1]{-----------------------------------------------------------------Side Note----------------------------------------------------------------
#1 \

---------------------------------------------------------------------------------------------------------------------------------------------}

% Analysis / Logical commands:

\newcommand{\br}{\mt{\mathbb{R}} }       % |R
\newcommand{\bq}{\mt{\mathbb{Q}} }       % |Q
\newcommand{\bn}{\mt{\mathbb{N}} }       % |N
\newcommand{\bc}{\mt{\mathbb{C}} }       % |C
\newcommand{\bz}{\mt{\mathbb{Z}} }       % |Z

\newcommand{\ep}{\mt{\epsilon} }         % epsilon
\newcommand{\fa}{\mt{\forall} }          % for all
\newcommand{\afa}{\mt{\alpha} }
\newcommand{\bta}{\mt{\beta} }
\newcommand{\mem}{\mt{\in} }
\newcommand{\exs}{\mt{\exists} }

\newcommand{\es}{\mt{\emptyset}}        % empty set
\newcommand{\sbs}{\mt{\subset} }         % subset of
\newcommand{\fs}[2]{\{\uw{#1}{1}, \uw{#1}{2}, ... \uw{#1}{#2}\}}

\newcommand{\lra}{ \mt{\longrightarrow} } % implies ----->
\newcommand{\rar}{ \mt{\Rightarrow} }     % implies -->

\newcommand{\lla}{ \mt{\longleftarrow} }  % implies <-----
\newcommand{\lar}{ \mt{\Leftarrow} }      % implies <--

\newcommand{\eql}{\mt{=} }
\newcommand{\pr}{\mt{^\prime} } 		   % prime (i.e. R')
\newcommand{\uw}[2]{#1\mt{_{#2}}}
\newcommand{\frc}[2]{\mt{\frac{#1}{#2}}}

\newcommand{\bnm}[2]{\mt{#1\setminus{#2}}}
\newcommand{\bnt}[2]{\mt{\textrm{#1}\setminus{\textrm{#2}}}}
\newcommand{\bi}{\bnm{\mathbb{R}}{\mathbb{Q}}}

\newcommand{\urng}[2]{\mt{\bigcup_{#1}^{#2}}}
\newcommand{\nrng}[2]{\mt{\bigcap_{#1}^{#2}}}

\newcommand{\nbho}[3]{\textrm{N(}#1, #2\textrm{) }\cap \textrm{ #3} \neq \emptyset}
     							   %  N(x, eps) intersect S \= emptyset
\newcommand{\nbhe}[3]{\textrm{N(}#1, #2\textrm{) }\cap \textrm{ #3} = \emptyset}
     							   %  N(x, eps) intersect S  = emptyset
\newcommand{\dnbho}[3]{\textrm{N*(}#1, #2\textrm{) }\cap \textrm{ #3} \neq \emptyset}
     							   %  N*(x, eps) intersect S \= emptyset
\newcommand{\dnbhe}[3]{\textrm{N*(}#1, #2\textrm{) }\cap \textrm{ #3} = \emptyset}
     							   %  N*(x, eps) intersect S = emptyset
     							 


% ----------

% ----------

% Packages

\usepackage{fancyhdr}
\usepackage{extramarks}
\usepackage{amsmath}
\usepackage{amssymb}
\usepackage{amsthm}
\usepackage{amsfonts}
\usepackage{tikz}
\usepackage[plain]{algorithm}
\usepackage{algpseudocode}
\usepackage{enumitem}
\usepackage{chngcntr}

% Libraries

\graphicspath{{/Users/jm/iclouddrive/3380pics/}}

\usetikzlibrary{automata, positioning, arrows}

%
% Basic Document Settings
%

\topmargin=-0.45in
\evensidemargin=0in
\oddsidemargin=0in
\textwidth=6.5in
\textheight=9.0in
\headsep=0.25in

\linespread{1.1}

\pagestyle{fancy}
\lhead{\hmwkAuthorName}
\chead{}
\rhead{\hmwkClass\ (\hmwkClassInstructor): \hmwkTitle}
\lfoot{\lastxmark}
\cfoot{\thepage}

\renewcommand\headrulewidth{0.4pt}
\renewcommand\footrulewidth{0.4pt}

\setlength\parindent{0pt}
\setcounter{secnumdepth}{0}

\newcommand{\hmwkClass}{MATH 5384 / Abstract Algebra}        % Class
\newcommand{\hmwkClassInstructor}{Dr. Acosta}           % Instructor
\newcommand{\hmwkAuthorName}{\textbf{Joshua Mitchell}} % Author

%
% Title Page
%

\title{
    \vspace{2in}
    \textmd{\textbf{\hmwkClass:\ \hmwkTitle}}\\
    \normalsize\vspace{0.1in}\small\vspace{0.1in}\large{\textit{\hmwkClassInstructor}}
    \vspace{3in}
}

\author{\hmwkAuthorName}
\date{}

\renewcommand{\part}[1]{\textbf{\large Part \Alph{partCounter}}\stepcounter{partCounter}\\}

% Integral dx
\newcommand{\dx}{\mathrm{d}x}

%
% Various Helper Commands
%

% For derivatives
\newcommand{\deriv}[1]{\frac{\mathrm{d}}{\mathrm{d}x} (#1)}

% For partial derivatives
\newcommand{\pderiv}[2]{\frac{\partial}{\partial #1} (#2)}


% Alias for the Solution section header
\newcommand{\solution}{\textbf{\large Solution}}

% Probability commands: Expectation, Variance, Covariance, Bias
\newcommand{\E}{\mathrm{E}}
\newcommand{\Var}{\mathrm{Var}}
\newcommand{\Cov}{\mathrm{Cov}}
\newcommand{\Bias}{\mathrm{Bias}}

% Formatting commands:

\newcommand{\mt}[1]{\ensuremath{#1}}
\newcommand{\nm}[1]{\textrm{#1}}

\newcommand\bsc[2][\DefaultOpt]{%
  \def\DefaultOpt{#2}%
  \section[#1]{#2}%
}
\newcommand\ssc[2][\DefaultOpt]{%
  \def\DefaultOpt{#2}%
  \subsection[#1]{#2}%
}
\newcommand{\bgpf}{\begin{proof} $ $\newline}

\newcommand{\bgeq}{\begin{equation*}}
\newcommand{\eeq}{\end{equation*}}	

\newcommand{\balist}{\begin{enumerate}[label=\alph*.]}
\newcommand{\elist}{\end{enumerate}}

\newcommand{\bilist}{\begin{enumerate}[label=\roman*)]}	

\newcommand{\bgsp}{\begin{split}}
% \newcommand{\esp}{\end{split}} % doesn't work for some reason.

\newcommand\prs[1]{~~~\textbf{(#1)}}

\newcommand{\lt}[1]{\textbf{Let: } #1}
     							   %  if you're setting it to be true
\newcommand{\supp}[1]{\textbf{Suppose: } #1}
     							   %  Suppose (if it'll end up false)
\newcommand{\wts}[1]{\textbf{Want to show: } #1}
     							   %  Want to show
\newcommand{\as}[1]{\textbf{Assume: } #1}
     							   %  if you think it follows from truth
\newcommand{\bpth}[1]{\textbf{(#1)}}

\newcommand{\step}[2]{\begin{equation}\tag{#2}#1\end{equation}}
\newcommand{\epf}{\end{proof}}

\newcommand{\dbs}[3]{\mt{#1_{#2_#3}}}

\newcommand{\sidenote}[1]{-----------------------------------------------------------------Side Note----------------------------------------------------------------
#1 \

---------------------------------------------------------------------------------------------------------------------------------------------}

% Analysis / Logical commands:

\newcommand{\br}{\mt{\mathbb{R}} }       % |R
\newcommand{\bq}{\mt{\mathbb{Q}} }       % |Q
\newcommand{\bn}{\mt{\mathbb{N}} }       % |N
\newcommand{\bc}{\mt{\mathbb{C}} }       % |C
\newcommand{\bz}{\mt{\mathbb{Z}} }       % |Z

\newcommand{\ep}{\mt{\epsilon} }         % epsilon
\newcommand{\fa}{\mt{\forall} }          % for all
\newcommand{\afa}{\mt{\alpha} }
\newcommand{\bta}{\mt{\beta} }
\newcommand{\dta}{\mt{\delta} }
\newcommand{\mem}{\mt{\in} }
\newcommand{\exs}{\mt{\exists} }

\newcommand{\es}{\mt{\emptyset} }        % empty set
\newcommand{\sbs}{\mt{\subset} }         % subset of
\newcommand{\fs}[2]{\{\uw{#1}{1}, \uw{#1}{2}, ... \uw{#1}{#2}\}}

\newcommand{\lra}{ \mt{\longrightarrow} } % implies ----->
\newcommand{\rar}{ \mt{\Rightarrow} }     % implies -->

\newcommand{\lla}{ \mt{\longleftarrow} }  % implies <-----
\newcommand{\lar}{ \mt{\Leftarrow} }      % implies <--

\newcommand{\av}[1]{\mt{|}#1\mt{|}}  % absolute value

\newcommand{\prn}[1]{(#1)}
\newcommand{\bk}[1]{\{#1\}}
\newcommand{\abk}[1]{\mt{\langle}#1\mt{\rangle}}

\newcommand{\ps}{\mt{+} }
\newcommand{\ms}{\mt{-} }

\newcommand{\ls}{\mt{<} }
\newcommand{\gr}{\mt{>} }

\newcommand{\lse}{\mt{\leq} }
\newcommand{\gre}{\mt{\geq} }

\newcommand{\eql}{\mt{=} }

\newcommand{\pr}{\mt{^\prime} } 		   % prime (i.e. R')
\newcommand{\uw}[2]{#1\mt{_{#2}}}
\newcommand{\uf}[2]{#1\mt{^{#2}}}
\newcommand{\frc}[2]{\mt{\frac{#1}{#2}}}
\newcommand{\lmti}[1]{\mt{\displaystyle{\lim_{#1 \to \infty}}}}
\newcommand{\limt}[2]{\mt{\displaystyle{\lim_{#1 \to #2}}}}

\newcommand{\bnm}[2]{\mt{#1\setminus{#2}}}
\newcommand{\bnt}[2]{\mt{\textrm{#1}\setminus{\textrm{#2}}}}
\newcommand{\bi}{\bnm{\mathbb{R}}{\mathbb{Q}}}

\newcommand{\urng}[2]{\mt{\bigcup_{#1}^{#2}}}
\newcommand{\nrng}[2]{\mt{\bigcap_{#1}^{#2}}}
\newcommand{\nck}[2]{\mt{{#1 \choose #2}}}

\newcommand{\nbho}[3]{\textrm{N(}#1, #2\textrm{) }\cap \textrm{ #3} \neq \emptyset}
     							   %  N(x, eps) intersect S \= emptyset
\newcommand{\nbhe}[3]{\textrm{N(}#1, #2\textrm{) }\cap \textrm{ #3} = \emptyset}
     							   %  N(x, eps) intersect S  = emptyset
\newcommand{\dnbho}[3]{\textrm{N*(}#1, #2\textrm{) }\cap \textrm{ #3} \neq \emptyset}
     							   %  N*(x, eps) intersect S \= emptyset
\newcommand{\dnbhe}[3]{\textrm{N*(}#1, #2\textrm{) }\cap \textrm{ #3} = \emptyset}
     							   %  N*(x, eps) intersect S = emptyset
     							   
\newcommand{\eqn}[1]{\[#1\]}
\newcommand{\splt}[1]{\begin{split}#1\end{split}}

\newcommand{\infy}{\mt{\infty} }
\newcommand{\unn}{\mt{\cup} }
\newcommand{\inn}{\mt{\cap} }
\newcommand\tab[1][1cm]{\hspace*{#1}}
\newcommand{\rln}{ \mt{\sim} }
\newcommand{\dvd}{ \mt{\vert} }
\newcommand{\ndvd}{ \mt{\not\vert} }
\newcommand{\eqw}{ \mt{ \equiv } }
\newcommand{\lcg}{ \mt{\gamma} }

\newcommand{\edp}{\mt{\bigoplus} }

\newcommand{\wit}[1]{\mt{\widetilde{#1}}}
     							 
% ----------

\begin{document}

Due 4/9:

G1 (present): page 150: 1, 7, 8

G2 (present): page 150: 3, 6, 9, 12, 14 (me: 3, 14)

All (turn in): page 150: 17, 19, 29, 36 (me)

Due 4/11:

Present: page 167: 20

All (turn in): page 167: 1, 22

\bsc{Page 150}{
\ssc{Exercise 3}{

\textbf{Let H \eql \bk{0, \mt{\pm3}, \mt{\pm6}, \mt{\pm9}...}. Rewrite the condition \uf{a}{-1}b \mem H given in property 6 of the lemma on page 139 in additive notation. Assume that the group is Abelian. Use this to decide whether or not the following cosets of H are the same.}

\

Property 6: aH \eql bH iff \uf{a}{-1}b \mem H

Rewritten: a \ps H \eql b \ps H iff \uf{a}{-1} \ps b \mem H

\balist
\item \textbf{11 \ps H and 17 \ps H}: \mt{-}11 \ps 17 \eql 6 \mem H, so yes.
\item \textbf{\mt{-}1 \ps H and 5 \ps H}: 1 \ps 5 \eql 6 mem H, so yes.
\item \textbf{7 \ps H and 23 \ps H}: \mt{-}7 \ps 23 \eql 16 $\not\in$ H, so no.
\elist

}
\ssc{Exercise 14}{
\textbf{Let \uf{C}{*} be the group of nonzero complex numbers under multiplication and let H \eql \bk{a \ps bi \mem \uf{C}{*} : \uf{a}{2} \ps \uf{b}{2} \eql 1}. Give a geometric description of the cosets (3 \ps 4i)H and (c \ps di)H.}

Well,

(3 \ps 4i)H \eql \bk{(3 \ps 4i)h : h \mem H}


(3 \ps 4i)H \eql \bk{(3 \ps 4i)(a \ps bi) : a \ps bi \mem \uf{C}{*}, \uf{a}{2} \ps \uf{b}{2} \eql 1}

(3 \ps 4i)H \eql \bk{3a \ps 4ai \ps 3bi - 4b: a \ps bi \mem \uf{C}{*}, \uf{a}{2} \ps \uf{b}{2} \eql 1}

(3 \ps 4i)H \eql \bk{3a \ps (4a \ps 3b)i - 4b: a \ps bi \mem \uf{C}{*}, \uf{a}{2} \ps \uf{b}{2} \eql 1}

thus,

(c \ps di)H \eql \bk{ca \ps (da \ps cb)i - db: a \ps bi \mem \uf{C}{*}, \uf{a}{2} \ps \uf{b}{2} \eql 1}

(c \ps di)H \eql \bk{(ca  - db) \ps (da \ps cb)i: a \ps bi \mem \uf{C}{*}, \uf{a}{2} \ps \uf{b}{2} \eql 1}

It looks like the subset H just indicates the elements that create a unit circle.

When we multiply by some real constant \gr 1, we just get a coset that represents a bigger circle.

When we multiply by some complex constant (e.g. 2i), we just get a coset that represents a flipped circle (where x, y becomes y, x), and if the complex constant has a scaling factor (e.g. 2), then the circle grows by that factor.

As far as the description of a coset with a positive real and positive complex part, I think it transforms it into an ellipse.


}
\ssc{Exercise 17}{

\textbf{Let G be a group with \av{G} \eql pq: p, q are prime. Prove that every proper subgroup of G is cyclic.}

Let H be a proper subgroup of G.

Since G is finite, \av{H} divides \av{G}.

Case:
\bilist
\item \av{H} \eql 1: Then H is cyclic by default.
\item \av{H} $\neq$ 1: Then by the fundamental theorem of arithmetic, \av{H} \eql t: t \mem \bk{p, q}

Notice: \av{H} \gr 1.

Let h \mem H: h $\neq$ e.

Then 1 \ls \av{$<h>$} \lse \av{H}.

Since H is finite, \av{$<h>$} divides \av{H}.

Since \av{H} is prime, its factors are only 1 and \av{H}. 

Since \av{$<h>$} $\neq$ 1, this implies that \av{$<h>$} \eql \av{H}.

Hence, H must be cyclic.

\elist

}
\ssc{Exercise 19}{

\textbf{Compute \uf{5}{15} mod 7 and \uf{7}{13} mod 11.}

\

Fermat's Little Theorem: For every integer a and prime p, \uf{a}{p} mod p \eql a mod p 

\eqn{
	\splt{
		5^{15}\mod 7 & = 5^{3} * 5^5 \mod 7 \\
	}
}
}
\ssc{Exercise 29}{

\textbf{Let \av{G} \eql 33. What are the possible orders for the elements of G? Show that G must have an element of order 3.}

\

Well, if \av{G} \eql 33, then for g \mem G, \av{g} must be some factor of 33: 1, 3, 11, or 33.

If \av{g} \eql 1, then g is the identity, which exists in every group.

\av{g} cannot be 33, since that's the size of the group. The maximum order for an element is n \ms 1 where n is the size of the group.

So the possible orders are 1, 3, and 11.

Let's suppose this group contains elements only of orders 1 and 11. In order for there to be 33 elements, there has to be more than one element of order 11.

However, the moment there are two elements of order 11, we can look at their cross product and see that we get more than 33 elements - a contradiction.

So, we must have an element of order 3.

}
\ssc{Exercise 36}{
\textbf{Let G be a group and \av{G} \eql 21. If g \mem G and \uf{g}{14} \eql e, what are the possibilities for \av{g}?}

\

Well, since g is a generator for H, a cyclic subgroup of G, that means that \av{H} must be a factor of \av{G}.

Since \av{G} \eql 21 and 14 doesn't divide 21, \av{H} must be some factor of both 21 and 14, but lower than 14.

Those possibilities are: 1, 7 
}
}
\bsc{Page 167}{
\ssc{Exercise 1}{
\textbf{Prove that the external direct product of any finite number of groups is a group.}

Let \uw{G}{1}, \uw{G}{2}, ... \uw{G}{n} be a finite collection of groups.

Then \uw{G}{1} \edp \uw{G}{2} \edp ... \edp \uw{G}{n} \eql \bk{(\uw{g}{1}, \uw{g}{2}, ... \uw{g}{n}) : \uw{g}{i} \mem \uw{G}{i}}

and (\uw{g}{1}, \uw{g}{2}, ... , \uw{g}{n})(\uw{g}{1}\pr, \uw{g}{n}\pr, ... , \uw{g}{n}\pr) \eql (\uw{g}{1}\uw{g}{1}\pr, \uw{g}{2}\uw{g}{2}\pr, ... , \uw{g}{n}\uw{g}{n}\pr)

Denote D \eql \bk{(\uw{g}{1}, \uw{g}{2}, ... \uw{g}{n}) : \uw{g}{i} \mem \uw{G}{i}}

Want to show that D is a group on the group product operation.

\textbf{Closure:}

Let a, b \mem D.

So:

a \eql (\uw{g}{1}, \uw{g}{2}, ... , \uw{g}{n})

b \eql (\uw{g}{1}\pr, \uw{g}{2}\pr, ... , \uw{g}{n}\pr)

and 

ab \eql (\uw{g}{1}\uw{g}{1}\pr, \uw{g}{2}\uw{g}{2}\pr, ... , \uw{g}{n}\uw{g}{n}\pr)

Since \uw{g}{i}\uw{g}{i}\pr \mem \uw{G}{i} for i \eql 1, 2, .. n by definition of a group,

(\uw{g}{1}\uw{g}{1}\pr, \uw{g}{2}\uw{g}{2}\pr, ... , \uw{g}{n}\uw{g}{n}\pr) \mem D

\textbf{Associativity:}

Let a, b, c \mem D.

So:

a \eql (\uw{g}{1}, \uw{g}{2}, ... , \uw{g}{n})

b \eql (\uw{g}{1}\pr, \uw{g}{2}\pr, ... , \uw{g}{n}\pr)

c \eql (\uw{g}{1}\pr\pr, \uw{g}{2}\pr\pr, ... , \uw{g}{n}\pr\pr)

(ab)c \eql (\uw{g}{1}\uw{g}{1}\pr, \uw{g}{2}\uw{g}{2}\pr, ... , \uw{g}{n}\uw{g}{n}\pr)(\uw{g}{1}\pr\pr, \uw{g}{2}\pr\pr, ... , \uw{g}{n}\pr\pr)

(ab)c \eql (\uw{g}{1}\uw{g}{1}\pr\uw{g}{1}\pr\pr, \uw{g}{2}\uw{g}{2}\pr\uw{g}{2}\pr\pr, ... , \uw{g}{n}\uw{g}{n}\pr\uw{g}{n}\pr\pr)

(ab)c \eql (\uw{g}{1}, \uw{g}{2}, ... , \uw{g}{n})(\uw{g}{1}\pr\uw{g}{1}\pr\pr, \uw{g}{2}\pr\uw{g}{2}\pr\pr, ... , \uw{g}{n}\pr\uw{g}{n}\pr\pr) \eql a(bc)

\textbf{Identity:}

Let e \eql (\uw{e}{1}, \uw{e}{2}, ... , \uw{e}{n}) and let a \mem D: a \eql (\uw{g}{1}, \uw{g}{2}, ... , \uw{g}{n}) 

Notice: 

ae \eql (\uw{g}{1}, \uw{g}{2}, ... , \uw{g}{n})(\uw{e}{1}, \uw{e}{2}, ... , \uw{e}{n}) \eql (\uw{g}{1}, \uw{g}{2}, ... , \uw{g}{n})

ea \eql (\uw{e}{1}, \uw{e}{2}, ... , \uw{e}{n})\eql (\uw{g}{1}, \uw{g}{2}, ... , \uw{g}{n}) \eql (\uw{g}{1}, \uw{g}{2}, ... , \uw{g}{n})

Hence, D contains an identity element: e 

\textbf{Inverse:}

Let a \mem D: a \eql (\uw{g}{1}, \uw{g}{2}, ... , \uw{g}{n}) 

Define \uf{a}{-1} \eql (\uf{\uw{g}{1}}{-1}, \uf{\uw{g}{2}}{-1}, ... , \uf{\uw{g}{n}}{-1}) 

Notice:

a\uf{a}{-1} \eql (\uw{g}{1}, \uw{g}{2}, ... , \uw{g}{n})(\uf{\uw{g}{1}}{-1}, \uf{\uw{g}{2}}{-1}, ... , \uf{\uw{g}{n}}{-1}) \eql (\uw{g}{1}\uf{\uw{g}{1}}{-1}, \uw{g}{2}\uf{\uw{g}{2}}{-1}, ... , \uw{g}{n}\uf{\uw{g}{n}}{-1}) \eql (\uw{e}{1}, \uw{e}{2}, ... , \uw{e}{n}) \eql e

\uf{a}{-1}a \eql (\uf{\uw{g}{1}}{-1}, \uf{\uw{g}{2}}{-1}, ... , \uf{\uw{g}{n}}{-1})(\uw{g}{1}, \uw{g}{2}, ... , \uw{g}{n}) \eql (\uf{\uw{g}{1}}{-1}\uw{g}{1}, \uf{\uw{g}{2}}{-1}\uw{g}{2}, ... , \uf{\uw{g}{n}}{-1}\uw{g}{n}) \eql (\uw{e}{1}, \uw{e}{2}, ... , \uw{e}{n}) \eql e

Hence, all elements of D have an inverse.

Hence, D is a group on the group product operation.
}

\newpage

\ssc{Exercise 20}{
\textbf{Find a subgroup of \uw{Z}{12} \edp \uw{Z}{18} that is isomorphic to \uw{Z}{9} \edp \uw{Z}{4}.}

Well, we know that \uw{Z}{9} \edp \uw{Z}{4} is isomorphic to \uw{Z}{36} since 9 and 4 don't share any common factors (by Theorem 8.2).

So, let's just pick two elements with orders 4 and 9.

I think 3 from \uw{Z}{12} will work for an order of 4, and 2 from \uw{Z}{18} will work for an order 9.

So our generator becomes (3, 2) \mem \uw{Z}{12} \edp \uw{Z}{18}, and the group isomorphic to \uw{Z}{9} \edp \uw{Z}{4} is simply \mt{<(3, 2)>}

}
\ssc{Exercise 22}{
\textbf{Determine the number of elements of order 15 and the number of cyclic subgroups of order 15 in \uw{Z}{30} \edp \uw{Z}{20}.}

By Theorem 8.1, the number of elements of order 15 is the number of elements (a, b) \mem \uw{Z}{30} \edp \uw{Z}{20} such that 15 \eql \av{(a, b)} \eql lcm(\av{a}, \av{b})

Case:
\bilist
\item \av{a} \eql 1, \av{b} \eql 15 - in this case there are 1 * 3 \eql 3
\item \av{a} \eql 15, \av{b} \eql 1 - in this case there are 15 * 1 \eql 15
\item \av{a} \eql 15, \av{b} \eql 15 - in this case there are 15 * 3 \eql 45
\item \av{a} \eql 3, \av{b} \eql 5 - in this case there are 3 * 5 \eql 15
\item \av{a} \eql 5, \av{b} \eql 3 - in this case there are 5 * 1 \eql 5
\elist

So the sum of all of those is 83.

The number of cyclic subgroups of order 15 in \uw{Z}{30} \edp \uw{Z}{20} is going to be \frc{83}{\phi(15)}
}
}

\end{document}