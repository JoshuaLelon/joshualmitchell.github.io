% Thank you Josh Davis for this template!
% https://github.com/jdavis/latex-homework-template/blob/master/homework.tex

\documentclass{article}

\newcommand{\hmwkTitle}{Lec\ \#12}

% % ----------

% Packages

\usepackage{fancyhdr}
\usepackage{extramarks}
\usepackage{amsmath}
\usepackage{amssymb}
\usepackage{amsthm}
\usepackage{amsfonts}
\usepackage{tikz}
\usepackage[plain]{algorithm}
\usepackage{algpseudocode}
\usepackage{enumitem}
\usepackage{chngcntr}

% Libraries

\usetikzlibrary{automata, positioning, arrows}

%
% Basic Document Settings
%

\topmargin=-0.45in
\evensidemargin=0in
\oddsidemargin=0in
\textwidth=6.5in
\textheight=9.0in
\headsep=0.25in

\linespread{1.1}

\pagestyle{fancy}
\lhead{\hmwkAuthorName}
\chead{}
\rhead{\hmwkClass\ (\hmwkClassInstructor): \hmwkTitle}
\lfoot{\lastxmark}
\cfoot{\thepage}

\renewcommand\headrulewidth{0.4pt}
\renewcommand\footrulewidth{0.4pt}

\setlength\parindent{0pt}
\setcounter{secnumdepth}{0}

\newcommand{\hmwkClass}{MATH 3380 / Analysis 1}        % Class
\newcommand{\hmwkClassInstructor}{Dr. Welsh}           % Instructor
\newcommand{\hmwkAuthorName}{\textbf{Joshua Mitchell}} % Author

%
% Title Page
%

\title{
    \vspace{2in}
    \textmd{\textbf{\hmwkClass:\ \hmwkTitle}}\\
    \normalsize\vspace{0.1in}\small\vspace{0.1in}\large{\textit{\hmwkClassInstructor}}
    \vspace{3in}
}

\author{\hmwkAuthorName}
\date{}

\renewcommand{\part}[1]{\textbf{\large Part \Alph{partCounter}}\stepcounter{partCounter}\\}

% Integral dx
\newcommand{\dx}{\mathrm{d}x}

%
% Various Helper Commands
%

% For derivatives
\newcommand{\deriv}[1]{\frac{\mathrm{d}}{\mathrm{d}x} (#1)}

% For partial derivatives
\newcommand{\pderiv}[2]{\frac{\partial}{\partial #1} (#2)}


% Alias for the Solution section header
\newcommand{\solution}{\textbf{\large Solution}}

% Probability commands: Expectation, Variance, Covariance, Bias
\newcommand{\E}{\mathrm{E}}
\newcommand{\Var}{\mathrm{Var}}
\newcommand{\Cov}{\mathrm{Cov}}
\newcommand{\Bias}{\mathrm{Bias}}

% Formatting commands:

\newcommand{\mt}[1]{\ensuremath{#1}}
\newcommand{\nm}[1]{\textrm{#1}}

\newcommand\bsc[2][\DefaultOpt]{%
  \def\DefaultOpt{#2}%
  \section[#1]{#2}%
}
\newcommand\ssc[2][\DefaultOpt]{%
  \def\DefaultOpt{#2}%
  \subsection[#1]{#2}%
}
\newcommand{\bgpf}{\begin{proof} $ $\newline}

\newcommand{\bgeq}{\begin{equation*}}
\newcommand{\eeq}{\end{equation*}}	

\newcommand{\balist}{\begin{enumerate}[label=\alph*.]}
\newcommand{\elist}{\end{enumerate}}

\newcommand{\bilist}{\begin{enumerate}[label=\roman*)]}	

\newcommand{\bgsp}{\begin{split}}
% \newcommand{\esp}{\end{split}} % doesn't work for some reason.

\newcommand\prs[1]{~~~\textbf{(#1)}}

\newcommand{\lt}[1]{\textbf{Let: } #1}
     							   %  if you're setting it to be true
\newcommand{\supp}[1]{\textbf{Suppose: } #1}
     							   %  Suppose (if it'll end up false)
\newcommand{\wts}[1]{\textbf{Want to show: } #1}
     							   %  Want to show
\newcommand{\as}[1]{\textbf{Assume: } #1}
     							   %  if you think it follows from truth
\newcommand{\bpth}[1]{\textbf{(#1)}}

\newcommand{\step}[2]{\begin{equation}\tag{#2}#1\end{equation}}
\newcommand{\epf}{\end{proof}}

\newcommand{\dbs}[3]{\mt{#1_{#2_#3}}}

\newcommand{\sidenote}[1]{-----------------------------------------------------------------Side Note----------------------------------------------------------------
#1 \

---------------------------------------------------------------------------------------------------------------------------------------------}

% Analysis / Logical commands:

\newcommand{\br}{\mt{\mathbb{R}} }       % |R
\newcommand{\bq}{\mt{\mathbb{Q}} }       % |Q
\newcommand{\bn}{\mt{\mathbb{N}} }       % |N
\newcommand{\bc}{\mt{\mathbb{C}} }       % |C
\newcommand{\bz}{\mt{\mathbb{Z}} }       % |Z

\newcommand{\ep}{\mt{\epsilon} }         % epsilon
\newcommand{\fa}{\mt{\forall} }          % for all
\newcommand{\afa}{\mt{\alpha} }
\newcommand{\bta}{\mt{\beta} }
\newcommand{\mem}{\mt{\in} }
\newcommand{\exs}{\mt{\exists} }

\newcommand{\es}{\mt{\emptyset}}        % empty set
\newcommand{\sbs}{\mt{\subset} }         % subset of
\newcommand{\fs}[2]{\{\uw{#1}{1}, \uw{#1}{2}, ... \uw{#1}{#2}\}}

\newcommand{\lra}{ \mt{\longrightarrow} } % implies ----->
\newcommand{\rar}{ \mt{\Rightarrow} }     % implies -->

\newcommand{\lla}{ \mt{\longleftarrow} }  % implies <-----
\newcommand{\lar}{ \mt{\Leftarrow} }      % implies <--

\newcommand{\eql}{\mt{=} }
\newcommand{\pr}{\mt{^\prime} } 		   % prime (i.e. R')
\newcommand{\uw}[2]{#1\mt{_{#2}}}
\newcommand{\frc}[2]{\mt{\frac{#1}{#2}}}

\newcommand{\bnm}[2]{\mt{#1\setminus{#2}}}
\newcommand{\bnt}[2]{\mt{\textrm{#1}\setminus{\textrm{#2}}}}
\newcommand{\bi}{\bnm{\mathbb{R}}{\mathbb{Q}}}

\newcommand{\urng}[2]{\mt{\bigcup_{#1}^{#2}}}
\newcommand{\nrng}[2]{\mt{\bigcap_{#1}^{#2}}}

\newcommand{\nbho}[3]{\textrm{N(}#1, #2\textrm{) }\cap \textrm{ #3} \neq \emptyset}
     							   %  N(x, eps) intersect S \= emptyset
\newcommand{\nbhe}[3]{\textrm{N(}#1, #2\textrm{) }\cap \textrm{ #3} = \emptyset}
     							   %  N(x, eps) intersect S  = emptyset
\newcommand{\dnbho}[3]{\textrm{N*(}#1, #2\textrm{) }\cap \textrm{ #3} \neq \emptyset}
     							   %  N*(x, eps) intersect S \= emptyset
\newcommand{\dnbhe}[3]{\textrm{N*(}#1, #2\textrm{) }\cap \textrm{ #3} = \emptyset}
     							   %  N*(x, eps) intersect S = emptyset
     							 


% ----------

% ----------

% Packages

\usepackage{fancyhdr}
\usepackage{extramarks}
\usepackage{amsmath}
\usepackage{amssymb}
\usepackage{amsthm}
\usepackage{amsfonts}
\usepackage{tikz}
\usepackage[plain]{algorithm}
\usepackage{algpseudocode}
\usepackage{enumitem}
\usepackage{chngcntr}

% Libraries

\usetikzlibrary{automata, positioning, arrows}

%
% Basic Document Settings
%

\topmargin=-0.45in
\evensidemargin=0in
\oddsidemargin=0in
\textwidth=6.5in
\textheight=9.0in
\headsep=0.25in

\linespread{1.1}

\pagestyle{fancy}
\lhead{\hmwkAuthorName}
\chead{}
\rhead{\hmwkClass\ (\hmwkClassInstructor): \hmwkTitle}
\lfoot{\lastxmark}
\cfoot{\thepage}

\renewcommand\headrulewidth{0.4pt}
\renewcommand\footrulewidth{0.4pt}

\setlength\parindent{0pt}
\setcounter{secnumdepth}{0}

\newcommand{\hmwkClass}{MATH 3380 / Analysis 1}        % Class
\newcommand{\hmwkClassInstructor}{Dr. Welsh}           % Instructor
\newcommand{\hmwkAuthorName}{\textbf{Joshua Mitchell}} % Author

%
% Title Page
%

\title{
    \vspace{2in}
    \textmd{\textbf{\hmwkClass:\ \hmwkTitle}}\\
    \normalsize\vspace{0.1in}\small\vspace{0.1in}\large{\textit{\hmwkClassInstructor}}
    \vspace{3in}
}

\author{\hmwkAuthorName}
\date{}

\renewcommand{\part}[1]{\textbf{\large Part \Alph{partCounter}}\stepcounter{partCounter}\\}

% Integral dx
\newcommand{\dx}{\mathrm{d}x}

%
% Various Helper Commands
%

% For derivatives
\newcommand{\deriv}[1]{\frac{\mathrm{d}}{\mathrm{d}x} (#1)}

% For partial derivatives
\newcommand{\pderiv}[2]{\frac{\partial}{\partial #1} (#2)}


% Alias for the Solution section header
\newcommand{\solution}{\textbf{\large Solution}}

% Probability commands: Expectation, Variance, Covariance, Bias
\newcommand{\E}{\mathrm{E}}
\newcommand{\Var}{\mathrm{Var}}
\newcommand{\Cov}{\mathrm{Cov}}
\newcommand{\Bias}{\mathrm{Bias}}

% Formatting commands:

\newcommand{\mt}[1]{\ensuremath{#1}}
\newcommand{\nm}[1]{\textrm{#1}}

\newcommand\bsc[2][\DefaultOpt]{%
  \def\DefaultOpt{#2}%
  \section[#1]{#2}%
}
\newcommand\ssc[2][\DefaultOpt]{%
  \def\DefaultOpt{#2}%
  \subsection[#1]{#2}%
}
\newcommand{\bgpf}{\begin{proof} $ $\newline}

\newcommand{\bgeq}{\begin{equation*}}
\newcommand{\eeq}{\end{equation*}}	

\newcommand{\balist}{\begin{enumerate}[label=\alph*.]}
\newcommand{\elist}{\end{enumerate}}

\newcommand{\bilist}{\begin{enumerate}[label=\roman*)]}	

\newcommand{\bgsp}{\begin{split}}
% \newcommand{\esp}{\end{split}} % doesn't work for some reason.

\newcommand\prs[1]{~~~\textbf{(#1)}}

\newcommand{\lt}[1]{\textbf{Let: } #1}
     							   %  if you're setting it to be true
\newcommand{\supp}[1]{\textbf{Suppose: } #1}
     							   %  Suppose (if it'll end up false)
\newcommand{\wts}[1]{\textbf{Want to show: } #1}
     							   %  Want to show
\newcommand{\as}[1]{\textbf{Assume: } #1}
     							   %  if you think it follows from truth
\newcommand{\bpth}[1]{\textbf{(#1)}}

\newcommand{\step}[2]{\begin{equation}\tag{#2}#1\end{equation}}
\newcommand{\epf}{\end{proof}}

\newcommand{\dbs}[3]{\mt{#1_{#2_#3}}}

\newcommand{\sidenote}[1]{-----------------------------------------------------------------Side Note----------------------------------------------------------------
#1 \

---------------------------------------------------------------------------------------------------------------------------------------------}

% Analysis / Logical commands:

\newcommand{\br}{\mt{\mathbb{R}} }       % |R
\newcommand{\bq}{\mt{\mathbb{Q}} }       % |Q
\newcommand{\bn}{\mt{\mathbb{N}} }       % |N
\newcommand{\bc}{\mt{\mathbb{C}} }       % |C
\newcommand{\bz}{\mt{\mathbb{Z}} }       % |Z

\newcommand{\ep}{\mt{\epsilon} }         % epsilon
\newcommand{\fa}{\mt{\forall} }          % for all
\newcommand{\afa}{\mt{\alpha} }
\newcommand{\bta}{\mt{\beta} }
\newcommand{\mem}{\mt{\in} }
\newcommand{\exs}{\mt{\exists} }

\newcommand{\es}{\mt{\emptyset} }        % empty set
\newcommand{\sbs}{\mt{\subset} }         % subset of
\newcommand{\fs}[2]{\{\uw{#1}{1}, \uw{#1}{2}, ... \uw{#1}{#2}\}}

\newcommand{\lra}{ \mt{\longrightarrow} } % implies ----->
\newcommand{\rar}{ \mt{\Rightarrow} }     % implies -->

\newcommand{\lla}{ \mt{\longleftarrow} }  % implies <-----
\newcommand{\lar}{ \mt{\Leftarrow} }      % implies <--

\newcommand{\av}[1]{\mt{|}#1\mt{|}}  % absolute value

\newcommand{\prn}[1]{(#1)}
\newcommand{\bk}[1]{\{#1\}}

\newcommand{\ps}{\mt{+} }
\newcommand{\ms}{\mt{-} }

\newcommand{\ls}{\mt{<} }
\newcommand{\gr}{\mt{>} }

\newcommand{\lse}{\mt{\leq} }
\newcommand{\gre}{\mt{\geq} }

\newcommand{\eql}{\mt{=} }

\newcommand{\pr}{\mt{^\prime} } 		   % prime (i.e. R')
\newcommand{\uw}[2]{#1\mt{_{#2}}}
\newcommand{\uf}[2]{#1\mt{^{#2}}}
\newcommand{\frc}[2]{\mt{\frac{#1}{#2}}}
\newcommand{\lmti}[1]{\mt{\displaystyle{\lim_{#1 \to \infty}}}}
\newcommand{\limt}[2]{\mt{\displaystyle{\lim_{#1 \to #2}}}}

\newcommand{\bnm}[2]{\mt{#1\setminus{#2}}}
\newcommand{\bnt}[2]{\mt{\textrm{#1}\setminus{\textrm{#2}}}}
\newcommand{\bi}{\bnm{\mathbb{R}}{\mathbb{Q}}}

\newcommand{\urng}[2]{\mt{\bigcup_{#1}^{#2}}}
\newcommand{\nrng}[2]{\mt{\bigcap_{#1}^{#2}}}
\newcommand{\nck}[2]{\mt{{#1 \choose #2}}}

\newcommand{\nbho}[3]{\textrm{N(}#1, #2\textrm{) }\cap \textrm{ #3} \neq \emptyset}
     							   %  N(x, eps) intersect S \= emptyset
\newcommand{\nbhe}[3]{\textrm{N(}#1, #2\textrm{) }\cap \textrm{ #3} = \emptyset}
     							   %  N(x, eps) intersect S  = emptyset
\newcommand{\dnbho}[3]{\textrm{N*(}#1, #2\textrm{) }\cap \textrm{ #3} \neq \emptyset}
     							   %  N*(x, eps) intersect S \= emptyset
\newcommand{\dnbhe}[3]{\textrm{N*(}#1, #2\textrm{) }\cap \textrm{ #3} = \emptyset}
     							   %  N*(x, eps) intersect S = emptyset
     							   
\newcommand{\eqn}[1]{\[#1\]}
\newcommand{\splt}[1]{\begin{split}#1\end{split}}
     							 
% ----------

\begin{document}

Homework Due 10/12/17: (13 problems) Section 4.2 pages 177 - 178; 1, 2, 4, 5(a)(c)(e)(g)(i)(k), 9, 10, 17, 18 (for 5(i) define tn to be 1 over sm, and then show that 1 over sm goes to 0)

\bsc{Corollary 4.2.5}{
If \bk{\uw{t}{n}} converges to t and \uw{t}{n} \gre 0 \fa n \mem \bn, then t \gre 0
}
\bsc{Example 4.2.6}{
If \bk{\uw{t}{n}} converges to t and \uw{t}{n} \gre 0 \fa n \mem \bn, then
\eqn{\lmti{n}\sqrt{\uw{t}{n}} = \sqrt{t}}
\bgpf

\sidenote{
For \ep \gr 0, \exs N \mem \bn st

\av{$\sqrt{\uw{t}{n}} - \sqrt{t}$} \ls \ep \fa m \gre N
}

Notice that \lmti{n} \uw{t}{n} \eql t, t \gre 0

Case \bpth{i}:

\eqn{
	\splt{
		\av{\sqrt{t_n} - \sqrt{t}} & = \frac{\av{(\sqrt{\uw{t}{n}} - \sqrt{t})( \sqrt{\uw{t}{n}} + \sqrt{t})}}{\av{\sqrt{\uw{t}{n}} + \sqrt{t}}} \\
		& = \frac{\av{\uw{t}{n} - t}}{\sqrt{\uw{t}{n}} + \sqrt{t}} \\
		& \lse \frac{\av{\uw{t}{n} - t}}{\sqrt{t}} \\
		& = (\frac{1}{\sqrt{t}})\av{\uw{t}{n} - t}
		}
} 

For \ep \gr 0, \exs N \mem \bn st

\av{\uw{t}{n} - t} \ls $\sqrt{t}\times\epsilon$, \fa n \gre N \bpth{2}

\sidenote{
\eqn{\sqrt{t} + \sqrt{t_n} \gre \sqrt{t}}
\eqn{\frac{1}{\sqrt{t} + \sqrt{t_n}} \lse \frac{1}{\sqrt{t}}}
\eqn{\sqrt{\uw{t}{n}} \gre 0}
\eqn{\sqrt{t} \gr 0}
So, $\sqrt{\uw{t}{n}}$ \ps $\sqrt{t}$ \gr 0 \fa n \mem \bn
}

From \bpth{1} and \bpth{2}, \

\

\av{$\sqrt{\uw{t}{n}} - \sqrt{t}$} \lse $\frac{\av{\uw{t}{n} \ms t}}{\sqrt{t}}$ \ls $\frac{\sqrt{t}\times\epsilon}{\sqrt{t}}$ \eql \ep, \fa n \gre N

Hence, result in this case.

\sidenote{
If

\av{\uw{s}{n} - s} \lse k\av{\uw{a}{n}} \fa n \gre N

and if

\lmti{n} \uw{a}{n} \eql 0,

then \lmti{n} \uw{s}{n} \eql s
}

Case \bpth{ii}: t \eql 0

Then, for \ep \gr 0, \exs N \mem \bn st

\uw{t}{n} \eql \av{\uw{t}{n} - 0} \ls \ep$^2$, \fa n \gre N

Thus, $\sqrt{\uw{t}{n}}$ \ls \ep, \fa n \gre N

In other words,

\av{$\sqrt{\uw{t}{n}}$ \ms 0} \ls \ep, \fa n \gre N

So, \lmti{n} $\sqrt{\uw{t}{n}}$ \eql 0 \eql $\sqrt{t}$

Hence, result.

\epf
}

\bsc{Theorem 4.2.7 - "The Ratio Test"}{
Suppose that \bk{\uw{s}{n}} is a sequence of \textbf{positive} terms (i.e. \uw{s}{n} \gr 0, \fa n \mem \bn) and \lmti{n} \frc{\uw{s}{n \ps 1}}{\uw{s}{n}} \eql L.

If L \ls 1, then \lmti{n}\uw{s}{n} \eql 0

\bgpf

For \ep \eql \frc{1 \ms L}{2} \gr 0,

\exs N \mem \bn st

\av{\frc{\uw{s}{n \ps 1}}{\uw{s}{n}} \ms L} \ls \frc{1 \ms L}{2}, \fa n \gre N

So,

\frc{\uw{s}{n \ps 1}}{\uw{s}{n}} \eql \av{\frc{\uw{s}{n \ps 1}}{\uw{s}{n}}} \eql \av{(\frc{\uw{s}{n \ps 1}}{\uw{s}{n}} \ms L) \ps L} \lse \av{\frc{\uw{s}{n \ps 1}}{\uw{s}{n}} \ms L} \ps \av{L} \ls \prn{\frc{1 \ms L}{2}} \ps L \eql \frc{1 \ps L}{2} \eql \frc{1}{2} \ps \frc{L}{2} \ls \frc{1}{2} \ps \frc{1}{2} (which is 1)

Define c \eql \frc{1 \ps L}{2}

Then,

\uw{s}{n} $\times$ \frc{\uw{s}{n \ps 1}}{\uw{s}{n}} \ls \sbs \uw{s}{n}, \fa n \gre N where c \ls 1

So, \uw{s}{n \ps 1} \ls \sbs \uw{s}{n}, \fa n \gre N

Now

\eqn{\uw{s}{N + 1} < c^1\uw{s}{N}}
\eqn{\uw{s}{N + 2} < c\uw{s}{N + 1} < c^2\uw{s}{N}}
\eqn{\uw{s}{N + 3} < c\uw{s}{N + 2} < c^3\uw{s}{N}\textrm{, etc.}}

So,

\uw{s}{N \ps K} \lse c$^k$\uw{s}{N}, \fa k \mem \bn $\cup$ \bk{0}

Thus,

\uw{s}{m} \lse c$^{m - N}$\uw{s}{N} \fa m \gre N

\uw{s}{m} \lse c$^{m}$\frc{\uw{s}{N}}{\uf{c}{N}} \fa m $\gre$ N

N \ps k \eql m

k \eql m \ms N

\av{\uw{s}{m} \ms 0} \eql \prn{\frc{\uw{s}{N}}{\uf{c}{N}}} \bpth{1}

\sidenote{
Theorem 4.1.8

If \av{\uw{s}{m} \ms s} \lse k\av{\uw{a}{m}} and

\lmti{m} \uw{a}{m} \eql 0

then lim \uw{s}{m} \eql s

Also, recall HW 5 7(f): If \av{x} \ls 1, then \lmti{n} \prn{\uf{x}{n}} \eql 0
}

From \bpth{1}, it follows by Example 7(f) pg 170 and Theorem 4.1.8, that \lmti{n} \uw{s}{n} \eql 0

\epf

\textbf{Example 5(g):} \uw{s}{n} \eql \frc{1 \ms n}{\uf{2}{n}} \eql \frc{1}{\uf{2}{n}} \ms \frc{n}{\uf{2}{n}} (or, \uw{v}{n} \ms \uw{u}{n})

Suppose \uw{u}{n} \eql \frc{n}{\uf{2}{n}} \gr 0 \fa n \mem \bn,

\frc{\uw{t}{n \ps 1}}{\uw{t}{n}} \eql \frc{n \ps 1}{\uf{2}{n \ps 1}}$\times$\frc{\uf{2}{n}}{n} \eql \frc{1}{2}\frc{n(1 \ps \frc{1}{n}}{n} \eql \frc{1}{2}\frc{1 \ps \frc{1}{n}}{1} \eql \frc{1 \ps \frc{1}{n}}{2} \eql \frc{1}{2} \ps \frc{1}{2n}

Which approaches \frc{1}{2} as n \lra $\infty$ \

\ssc{Definition 4.2.9}{

\textbf{Infinite Limits:}

A sequence \bk{\uw{s}{n}} is said to \textbf{diverge} to $\infty$, written as \lmti{n} \uw{s}{n} \eql $\infty$, provided that

\fa M \mem \br, \exs N \mem \bn st

\uw{s}{n} \gr M, \fa n \gre N

(i.e. \uw{s}{n} \eql (\ms1)$^n$

Similarly, \bk{\uw{s}{n}} diverges to \ms$\infty$, written as \lmti{n} \uw{s}{n} \eql \ms$\infty$, if, provided that

for every M \mem \br, \exs N(M) \mem \bn st

\uw{s}{n} \ls M, \fa n \gre N
}

\ssc{Theorem 4.2.12}{
Suppose that \bk{\uw{s}{n}}, \bk{\uw{t}{n}} are sequences st \uw{s}{n} \lse \uw{t}{n} \fa n \mem \bn 

\balist
\item If \lmti{n} \uw{s}{n} \eql $\infty$, then \lmti{n} \uw{t}{n} \eql $\infty$
\item If \lmti{n} \uw{s}{n} \eql \ms$\infty$, then \lmti{n} \uw{t}{n} \eql \ms$\infty$
\elist

On the homework, the proof, using the definition, about "one comment away" from being done.
}

\ssc{Theorem 4.2.13}{
\lt{\bk{\uw{s}{n}} be a sequence of \textbf{positive} numbers}

Then \lmti{n} \uw{s}{n} \eql $\infty$

iff \lmti{n} \frc{1}{\uw{s}{n}} \eql 0

\bgpf

\lra

Suppose that \lmti{n} \uw{s}{n} \eql $\infty$

\wts{\lmti{n} \frc{1}{\uw{s}{n}} \eql 0}

\sidenote{
\fa \ep \gr 0, \exs N \mem \bn st

\av{\frc{1}{\uw{s}{n}} \ms 0} \eql \frc{1}{\uw{s}{n}} \ls \ep

(which implies that \uw{s}{n} \gr \frc{1}{\ep})

\fa n \gre N}

\fa \ep \gr 0, \exs N \mem \bn st

\uw{s}{n} \gr \frc{1}{\ep}, \fa n \gre N

Hence,

\av{\frc{1}{\uw{s}{n}} \ms 0} \eql \frc{1}{\uw{s}{n}} \ls \ep, \fa n \gre N

Which shows that \lmti{n} \frc{1}{\uw{s}{n}} \eql 0

\lla

Conversely, assume that \lmti{n} \frc{1}{\uw{s}{n}} \eql 0

\wts{\lmti{n} \uw{s}{n} \eql $\infty$}

\sidenote{
For M \mem \br, \exs N \mem \bn st

\frc{1}{\uw{s}{n}} \ls \frc{1}{M}

\uw{s}{n} \gr M

\fa n \gre N
}

\lt{M \mem \br, M \gr 0}

Then \exs N(M) \mem \bn st

\frc{1}{\uw{s}{n}} \eql \av{\frc{1}{\uw{s}{n}} \ms 0} \ls \frc{1}{M} \fa n \gre N

Hence, \uw{s}{n} \gr M, \fa n \gre N.

Hence, result.

\epf
}


\end{document}