% Thank you Josh Davis for this template!
% https://github.com/jdavis/latex-homework-template/blob/master/homework.tex

\documentclass{article}

\newcommand{\hmwkTitle}{HW\ \#6}

% % ----------

% Packages

\usepackage{fancyhdr}
\usepackage{extramarks}
\usepackage{amsmath}
\usepackage{amssymb}
\usepackage{amsthm}
\usepackage{amsfonts}
\usepackage{tikz}
\usepackage[plain]{algorithm}
\usepackage{algpseudocode}
\usepackage{enumitem}
\usepackage{chngcntr}

% Libraries

\usetikzlibrary{automata, positioning, arrows}

%
% Basic Document Settings
%

\topmargin=-0.45in
\evensidemargin=0in
\oddsidemargin=0in
\textwidth=6.5in
\textheight=9.0in
\headsep=0.25in

\linespread{1.1}

\pagestyle{fancy}
\lhead{\hmwkAuthorName}
\chead{}
\rhead{\hmwkClass\ (\hmwkClassInstructor): \hmwkTitle}
\lfoot{\lastxmark}
\cfoot{\thepage}

\renewcommand\headrulewidth{0.4pt}
\renewcommand\footrulewidth{0.4pt}

\setlength\parindent{0pt}
\setcounter{secnumdepth}{0}

\newcommand{\hmwkClass}{MATH 3380 / Analysis 1}        % Class
\newcommand{\hmwkClassInstructor}{Dr. Welsh}           % Instructor
\newcommand{\hmwkAuthorName}{\textbf{Joshua Mitchell}} % Author

%
% Title Page
%

\title{
    \vspace{2in}
    \textmd{\textbf{\hmwkClass:\ \hmwkTitle}}\\
    \normalsize\vspace{0.1in}\small\vspace{0.1in}\large{\textit{\hmwkClassInstructor}}
    \vspace{3in}
}

\author{\hmwkAuthorName}
\date{}

\renewcommand{\part}[1]{\textbf{\large Part \Alph{partCounter}}\stepcounter{partCounter}\\}

% Integral dx
\newcommand{\dx}{\mathrm{d}x}

%
% Various Helper Commands
%

% For derivatives
\newcommand{\deriv}[1]{\frac{\mathrm{d}}{\mathrm{d}x} (#1)}

% For partial derivatives
\newcommand{\pderiv}[2]{\frac{\partial}{\partial #1} (#2)}


% Alias for the Solution section header
\newcommand{\solution}{\textbf{\large Solution}}

% Probability commands: Expectation, Variance, Covariance, Bias
\newcommand{\E}{\mathrm{E}}
\newcommand{\Var}{\mathrm{Var}}
\newcommand{\Cov}{\mathrm{Cov}}
\newcommand{\Bias}{\mathrm{Bias}}

% Formatting commands:

\newcommand{\mt}[1]{\ensuremath{#1}}
\newcommand{\nm}[1]{\textrm{#1}}

\newcommand\bsc[2][\DefaultOpt]{%
  \def\DefaultOpt{#2}%
  \section[#1]{#2}%
}
\newcommand\ssc[2][\DefaultOpt]{%
  \def\DefaultOpt{#2}%
  \subsection[#1]{#2}%
}
\newcommand{\bgpf}{\begin{proof} $ $\newline}

\newcommand{\bgeq}{\begin{equation*}}
\newcommand{\eeq}{\end{equation*}}	

\newcommand{\balist}{\begin{enumerate}[label=\alph*.]}
\newcommand{\elist}{\end{enumerate}}

\newcommand{\bilist}{\begin{enumerate}[label=\roman*)]}	

\newcommand{\bgsp}{\begin{split}}
% \newcommand{\esp}{\end{split}} % doesn't work for some reason.

\newcommand\prs[1]{~~~\textbf{(#1)}}

\newcommand{\lt}[1]{\textbf{Let: } #1}
     							   %  if you're setting it to be true
\newcommand{\supp}[1]{\textbf{Suppose: } #1}
     							   %  Suppose (if it'll end up false)
\newcommand{\wts}[1]{\textbf{Want to show: } #1}
     							   %  Want to show
\newcommand{\as}[1]{\textbf{Assume: } #1}
     							   %  if you think it follows from truth
\newcommand{\bpth}[1]{\textbf{(#1)}}

\newcommand{\step}[2]{\begin{equation}\tag{#2}#1\end{equation}}
\newcommand{\epf}{\end{proof}}

\newcommand{\dbs}[3]{\mt{#1_{#2_#3}}}

\newcommand{\sidenote}[1]{-----------------------------------------------------------------Side Note----------------------------------------------------------------
#1 \

---------------------------------------------------------------------------------------------------------------------------------------------}

% Analysis / Logical commands:

\newcommand{\br}{\mt{\mathbb{R}} }       % |R
\newcommand{\bq}{\mt{\mathbb{Q}} }       % |Q
\newcommand{\bn}{\mt{\mathbb{N}} }       % |N
\newcommand{\bc}{\mt{\mathbb{C}} }       % |C
\newcommand{\bz}{\mt{\mathbb{Z}} }       % |Z

\newcommand{\ep}{\mt{\epsilon} }         % epsilon
\newcommand{\fa}{\mt{\forall} }          % for all
\newcommand{\afa}{\mt{\alpha} }
\newcommand{\bta}{\mt{\beta} }
\newcommand{\mem}{\mt{\in} }
\newcommand{\exs}{\mt{\exists} }

\newcommand{\es}{\mt{\emptyset}}        % empty set
\newcommand{\sbs}{\mt{\subset} }         % subset of
\newcommand{\fs}[2]{\{\uw{#1}{1}, \uw{#1}{2}, ... \uw{#1}{#2}\}}

\newcommand{\lra}{ \mt{\longrightarrow} } % implies ----->
\newcommand{\rar}{ \mt{\Rightarrow} }     % implies -->

\newcommand{\lla}{ \mt{\longleftarrow} }  % implies <-----
\newcommand{\lar}{ \mt{\Leftarrow} }      % implies <--

\newcommand{\eql}{\mt{=} }
\newcommand{\pr}{\mt{^\prime} } 		   % prime (i.e. R')
\newcommand{\uw}[2]{#1\mt{_{#2}}}
\newcommand{\frc}[2]{\mt{\frac{#1}{#2}}}

\newcommand{\bnm}[2]{\mt{#1\setminus{#2}}}
\newcommand{\bnt}[2]{\mt{\textrm{#1}\setminus{\textrm{#2}}}}
\newcommand{\bi}{\bnm{\mathbb{R}}{\mathbb{Q}}}

\newcommand{\urng}[2]{\mt{\bigcup_{#1}^{#2}}}
\newcommand{\nrng}[2]{\mt{\bigcap_{#1}^{#2}}}

\newcommand{\nbho}[3]{\textrm{N(}#1, #2\textrm{) }\cap \textrm{ #3} \neq \emptyset}
     							   %  N(x, eps) intersect S \= emptyset
\newcommand{\nbhe}[3]{\textrm{N(}#1, #2\textrm{) }\cap \textrm{ #3} = \emptyset}
     							   %  N(x, eps) intersect S  = emptyset
\newcommand{\dnbho}[3]{\textrm{N*(}#1, #2\textrm{) }\cap \textrm{ #3} \neq \emptyset}
     							   %  N*(x, eps) intersect S \= emptyset
\newcommand{\dnbhe}[3]{\textrm{N*(}#1, #2\textrm{) }\cap \textrm{ #3} = \emptyset}
     							   %  N*(x, eps) intersect S = emptyset
     							 


% ----------

% ----------

% Packages

\usepackage{fancyhdr}
\usepackage{extramarks}
\usepackage{amsmath}
\usepackage{amssymb}
\usepackage{amsthm}
\usepackage{amsfonts}
\usepackage{tikz}
\usepackage[plain]{algorithm}
\usepackage{algpseudocode}
\usepackage{enumitem}
\usepackage{chngcntr}

% Libraries

\usetikzlibrary{automata, positioning, arrows}

%
% Basic Document Settings
%

\topmargin=-0.45in
\evensidemargin=0in
\oddsidemargin=0in
\textwidth=6.5in
\textheight=9.0in
\headsep=0.25in

\linespread{1.1}

\pagestyle{fancy}
\lhead{\hmwkAuthorName}
\chead{}
\rhead{\hmwkClass\ (\hmwkClassInstructor): \hmwkTitle}
\lfoot{\lastxmark}
\cfoot{\thepage}

\renewcommand\headrulewidth{0.4pt}
\renewcommand\footrulewidth{0.4pt}

\setlength\parindent{0pt}
\setcounter{secnumdepth}{0}

\newcommand{\hmwkClass}{MATH 3380 / Analysis 1}        % Class
\newcommand{\hmwkClassInstructor}{Dr. Welsh}           % Instructor
\newcommand{\hmwkAuthorName}{\textbf{Joshua Mitchell}} % Author

%
% Title Page
%

\title{
    \vspace{2in}
    \textmd{\textbf{\hmwkClass:\ \hmwkTitle}}\\
    \normalsize\vspace{0.1in}\small\vspace{0.1in}\large{\textit{\hmwkClassInstructor}}
    \vspace{3in}
}

\author{\hmwkAuthorName}
\date{}

\renewcommand{\part}[1]{\textbf{\large Part \Alph{partCounter}}\stepcounter{partCounter}\\}

% Integral dx
\newcommand{\dx}{\mathrm{d}x}

%
% Various Helper Commands
%

% For derivatives
\newcommand{\deriv}[1]{\frac{\mathrm{d}}{\mathrm{d}x} (#1)}

% For partial derivatives
\newcommand{\pderiv}[2]{\frac{\partial}{\partial #1} (#2)}


% Alias for the Solution section header
\newcommand{\solution}{\textbf{\large Solution}}

% Probability commands: Expectation, Variance, Covariance, Bias
\newcommand{\E}{\mathrm{E}}
\newcommand{\Var}{\mathrm{Var}}
\newcommand{\Cov}{\mathrm{Cov}}
\newcommand{\Bias}{\mathrm{Bias}}

% Formatting commands:

\newcommand{\mt}[1]{\ensuremath{#1}}
\newcommand{\nm}[1]{\textrm{#1}}

\newcommand\bsc[2][\DefaultOpt]{%
  \def\DefaultOpt{#2}%
  \section[#1]{#2}%
}
\newcommand\ssc[2][\DefaultOpt]{%
  \def\DefaultOpt{#2}%
  \subsection[#1]{#2}%
}
\newcommand{\bgpf}{\begin{proof} $ $\newline}

\newcommand{\bgeq}{\begin{equation*}}
\newcommand{\eeq}{\end{equation*}}	

\newcommand{\balist}{\begin{enumerate}[label=\alph*.]}
\newcommand{\elist}{\end{enumerate}}

\newcommand{\bilist}{\begin{enumerate}[label=\roman*)]}	

\newcommand{\bgsp}{\begin{split}}
% \newcommand{\esp}{\end{split}} % doesn't work for some reason.

\newcommand\prs[1]{~~~\textbf{(#1)}}

\newcommand{\lt}[1]{\textbf{Let: } #1}
     							   %  if you're setting it to be true
\newcommand{\supp}[1]{\textbf{Suppose: } #1}
     							   %  Suppose (if it'll end up false)
\newcommand{\wts}[1]{\textbf{Want to show: } #1}
     							   %  Want to show
\newcommand{\as}[1]{\textbf{Assume: } #1}
     							   %  if you think it follows from truth
\newcommand{\bpth}[1]{\textbf{(#1)}}

\newcommand{\step}[2]{\begin{equation}\tag{#2}#1\end{equation}}
\newcommand{\epf}{\end{proof}}

\newcommand{\dbs}[3]{\mt{#1_{#2_#3}}}

\newcommand{\sidenote}[1]{-----------------------------------------------------------------Side Note----------------------------------------------------------------
#1 \

---------------------------------------------------------------------------------------------------------------------------------------------}

% Analysis / Logical commands:

\newcommand{\br}{\mt{\mathbb{R}} }       % |R
\newcommand{\bq}{\mt{\mathbb{Q}} }       % |Q
\newcommand{\bn}{\mt{\mathbb{N}} }       % |N
\newcommand{\bc}{\mt{\mathbb{C}} }       % |C
\newcommand{\bz}{\mt{\mathbb{Z}} }       % |Z

\newcommand{\ep}{\mt{\epsilon} }         % epsilon
\newcommand{\fa}{\mt{\forall} }          % for all
\newcommand{\afa}{\mt{\alpha} }
\newcommand{\bta}{\mt{\beta} }
\newcommand{\mem}{\mt{\in} }
\newcommand{\exs}{\mt{\exists} }

\newcommand{\es}{\mt{\emptyset} }        % empty set
\newcommand{\sbs}{\mt{\subset} }         % subset of
\newcommand{\fs}[2]{\{\uw{#1}{1}, \uw{#1}{2}, ... \uw{#1}{#2}\}}

\newcommand{\lra}{ \mt{\longrightarrow} } % implies ----->
\newcommand{\rar}{ \mt{\Rightarrow} }     % implies -->

\newcommand{\lla}{ \mt{\longleftarrow} }  % implies <-----
\newcommand{\lar}{ \mt{\Leftarrow} }      % implies <--

\newcommand{\av}[1]{\mt{|}#1\mt{|}}  % absolute value

\newcommand{\prn}[1]{(#1)}
\newcommand{\bk}[1]{\{#1\}}

\newcommand{\ps}{\mt{+} }
\newcommand{\ms}{\mt{-} }

\newcommand{\ls}{\mt{<} }
\newcommand{\gr}{\mt{>} }

\newcommand{\lse}{\mt{\leq} }
\newcommand{\gre}{\mt{\geq} }

\newcommand{\eql}{\mt{=} }

\newcommand{\pr}{\mt{^\prime} } 		   % prime (i.e. R')
\newcommand{\uw}[2]{#1\mt{_{#2}}}
\newcommand{\uf}[2]{#1\mt{^{#2}}}
\newcommand{\frc}[2]{\mt{\frac{#1}{#2}}}
\newcommand{\lmti}[1]{\mt{\displaystyle{\lim_{#1 \to \infty}}}}
\newcommand{\limt}[2]{\mt{\displaystyle{\lim_{#1 \to #2}}}}

\newcommand{\bnm}[2]{\mt{#1\setminus{#2}}}
\newcommand{\bnt}[2]{\mt{\textrm{#1}\setminus{\textrm{#2}}}}
\newcommand{\bi}{\bnm{\mathbb{R}}{\mathbb{Q}}}

\newcommand{\urng}[2]{\mt{\bigcup_{#1}^{#2}}}
\newcommand{\nrng}[2]{\mt{\bigcap_{#1}^{#2}}}
\newcommand{\nck}[2]{\mt{{#1 \choose #2}}}

\newcommand{\nbho}[3]{\textrm{N(}#1, #2\textrm{) }\cap \textrm{ #3} \neq \emptyset}
     							   %  N(x, eps) intersect S \= emptyset
\newcommand{\nbhe}[3]{\textrm{N(}#1, #2\textrm{) }\cap \textrm{ #3} = \emptyset}
     							   %  N(x, eps) intersect S  = emptyset
\newcommand{\dnbho}[3]{\textrm{N*(}#1, #2\textrm{) }\cap \textrm{ #3} \neq \emptyset}
     							   %  N*(x, eps) intersect S \= emptyset
\newcommand{\dnbhe}[3]{\textrm{N*(}#1, #2\textrm{) }\cap \textrm{ #3} = \emptyset}
     							   %  N*(x, eps) intersect S = emptyset
     							   
\newcommand{\eqn}[1]{\[#1\]}
\newcommand{\splt}[1]{\begin{split}#1\end{split}}
     							 
% ----------

\begin{document}

Homework Due 10/12/17: (13 problems) Section 4.2 pages 177 - 178; 1, 2, 4, 5(a)(c)(e)(g)(i)(k), 9, 10, 17, 18 (for 5(i) define tn to be 1 over sm, and then show that 1 over sm goes to 0)

\bsc{Problem 1}{
Mark each statement True or False. Justify each answer.

\balist
\item If \prn{\uw{s}{n}} and (\uw{t}{n}) are convergent sequences with \uw{s}{n} \lra s and \uw{t}{n} \lra t, then lim (\uw{s}{n} \ps \uw{t}{n}) \eql s \ps t and lim (\uw{s}{n}\uw{t}{n}) \eql st.
	
	\textbf{True.} By Theorem 4.2.1 (a) and (c).
\item If \prn{\uw{s}{n}} converges to s and \uw{s}{n} \gr 0 \fa n \mem \bn, then s \gr 0.
	
	\textbf{False.} Counter example: \prn{\uw{s}{n}} \eql \frc{1}{n} (s \eql 0, but the moment you define n, \uw{s}{n} \gr 0)
\item The sequence \prn{\uw{s}{n}} converges to s iff lim \uw{s}{n} \eql s.
	
	\textbf{False.} The sequence converges to s iff s exists \textbf{as a real number}. If s \eql $+\infty$ then it can't converge.
\item lim \uw{s}{n} \eql $+\infty$ iff lim (\frc{1}{\uw{s}{n}}) \eql 0.
	
	\textbf{False.} If lim (\frc{1}{\uw{s}{n}}) \eql 0 but \prn{\uw{s}{n}} \eql $-1, -2, -3, ...$ then \uw{s}{n} does not diverge to $+\infty$
\elist
}

\bsc{Problem 2}{
Mark each statement True or False. Justify each answer.

\balist
\item If \uw{s}{n} \eql s and lim \uw{t}{n} \eql t, then lim (\uw{s}{n}\uw{t}{n}) \eql st.
	
	\textbf{False.} We don't know \uw{s}{n}'s limit (which could be, for example, (\uw{s}{n}) \eql n, which diverges)
\item If lim \uw{s}{n} \eql $+\infty$, then \prn{\uw{s}{n}} is said to converge to $+\infty$.
	
	\textbf{False.} You can only converge to a finite number.
\item Given sequences \prn{\uw{s}{n}} and \prn{\uw{t}{n}} with \uw{s}{n} \lse \uw{t}{n} \fa n \mem \bn, if lim \uw{s}{n} \eql $+\infty$, then lim \uw{t}{n} \eql $+\infty$.
	
	\textbf{True.} 
	
	Suppose \exs sequences \prn{\uw{s}{n}} and \prn{\uw{t}{n}} st \uw{s}{n} \lse \uw{t}{n} \fa n \mem \bn where lim \uw{s}{n} \eql $+\infty$ and  lim \uw{t}{n} is NOT $+\infty$.
	
	\uw{t}{n} diverges to $+\infty$ if \fa M \mem \br, \exs N \mem \bn st n \gre N implies \uw{t}{n} \gr M
	
	\lt{M \mem \br}
	
	We know that
	
	\exs N \mem \bn st n \gre N implies \uw{t}{n} \gr M
	
	Since \uw{s}{n} \lse \uw{t}{n} \fa n \mem \bn
	
	\exs N \mem \bn st n \gre N implies \uw{s}{n} \gre \uw{t}{n} \gr M
	
	\exs N \mem \bn st n \gre N implies \uw{s}{n} \gr M
	
	This is the definition of diverging to $+\infty$, a contradiction.
	
	Hence, result.
\item Suppose \prn{\uw{s}{n}} is a sequence st the sequence of ratios (\frc{\uw{s}{n \ps 1}}{\uw{s}{n}}) converges to L. If L \ls 1, then lim \uw{s}{n} \eql 0.
	
	\textbf{False.}
	
	\lt{\uw{s}{n} \eql n(1)$^{-n}$ \lra (\frc{\uw{s}{n \ps 1}}{\uw{s}{n}}) \eql $\frac{(n + 1)(1)^{-(n + 1)}}{n(1)^{-n}}$}
	
	which converges to $-$1 which is less than 1 but does not have a limit of 0. 
\elist
}

\bsc{Problem 4}{
\balist
\item Prove Theorem 4.2.1(b):
	
	Suppose that \prn{\uw{s}{n}} and (\uw{t}{n}) are convergent sequences with lim \uw{s}{n} \eql s and lim \uw{t}{n} \eql t. Then
	
	\bpth{b} lim (k\uw{s}{n}) \eql ks and lim (k \ps \uw{s}{n}) \eql k \ps s, for any k \mem \br
	
	We know that since \uw{s}{n} and \uw{t}{n} are convergent sequences with limits s and t, respectively.
	
	So,
	
	\fa \ep \gr 0, \exs N \mem \bn st n \gre N implies \av{\uw{s}{n} \ms s} \ls \ep
	
	\fa \ep \gr 0, \exs N \mem \bn st n \gre N implies \av{\uw{t}{n} \ms t} \ls \ep
	
	\wts{\fa \ep \gr 0, \exs N \mem \bn st n \gre N implies \av{k\uw{s}{n} \ms ks} \ls \ep}
	
	\av{k\uw{s}{n} \ms ks} \eql \av{k(\uw{s}{n} \ms s)} \eql \av{k}\av{\uw{s}{n} \ms s}
	
	So,
	
	\av{k\uw{s}{n} \ms ks} \eql \av{k}\av{\uw{s}{n} \ms s} \ls \ep
	
	\av{\uw{s}{n} \ms s} \ls \av{k}\ep \eql \uw{\ep}{1}(\ep)
	
	Since
	
	\fa \ep \gr 0, \exs N \mem \bn st n \gre N implies \av{\uw{s}{n} \ms s} \ls \ep
	
	thus,
	
	\fa \uw{\ep}{1} \gr 0, \exs N \mem \bn st n \gre N implies \av{\uw{s}{n} \ms s} \ls \uw{\ep}{1}
	
	and
	
	\fa \ep \gr 0, \exs N \mem \bn st n \gre N implies \av{k\uw{s}{n} \ms ks} \ls \ep
	
	Hence, lim (k\uw{s}{n}) \eql ks
	
	\wts{\fa \ep \gr 0, \exs N \mem \bn st n \gre N implies \av{k \ps \uw{s}{n} \ms (k \ps s)} \ls \ep}
	
	We know:
		
	\fa \ep \gr 0, \exs N \mem \bn st n \gre N implies \av{\uw{s}{n} \ms s} \ls \ep
	
	So,
	
	\fa \ep \gr 0, \exs N \mem \bn st n \gre N implies \av{\uw{s}{n} \ps k \ms s \ms k} \ls \ep
	
	\fa \ep \gr 0, \exs N \mem \bn st n \gre N implies \av{\uw{s}{n} \ps k \ms (s \ps k)} \ls \ep
	
	Since this is true,
	
	lim (\uw{s}{n} \ps k) \eql k \ps s
\item Prove Corollary 4.2.5:
	
	If \prn{\uw{t}{n}} converges to t and \uw{t}{n} \gre 0 \fa n \mem \bn, then t \gre 0.
	
	We know that 
	
	\fa \ep \gr 0, \exs N \mem \bn st n \gre N implies \av{\uw{t}{n} \ms t} \ls \ep
	
	\supp{t \ls 0}
	
	\lt{\ep \eql \av{t}}
	
	\fa n \mem \bn, \exs N \mem \bn st n \gre N implies \av{\uw{t}{n} \ms t} \ls \av{t}
	
	Since t is negative,
	
	\fa n \mem \bn, \exs N \mem \bn st n \gre N implies \av{\uw{t}{n} \ps \av{t}} \ls \av{t}
	
	
	Since \uw{t}{n} \gre 0,
	
	\fa n \mem \bn, \exs N \mem \bn st n \gre N implies \uw{t}{n} \ps \av{t} \ls \av{t}
	
	So,
	
	\fa n \mem \bn, \exs N \mem \bn st n \gre N implies \uw{t}{n} \ls 0
	
	but \uw{t}{n} \gre 0, a contradiction.
	
	Hence, result.
	
\elist
}

\bsc{Problem 5}{
For \uw{s}{n} given by the following formulas, determine the convergence or divergence of the sequence \prn{\uw{s}{n}}. Find any limits that exist.

\balist
\item \uw{s}{n} \eql \frc{3 - 2n}{1 + n} \lra \frc{1}{2}
\item \uw{s}{n} \eql \frc{(-1)^n}{n + 3} \lra 0
\item \uw{s}{n} \eql \frc{(-1)^n}{2n - 1} \lra 0
\item \uw{s}{n} \eql \frc{2^{3n}}{3^{2n}} \eql \frc{8^n}{9^n} \lra 0
\item \uw{s}{n} \eql \frc{n^2 - 2}{n + 1} \lra $\infty$
\item \uw{s}{n} \eql \frc{3 + n - n^2}{1 + 2n} \lra $-\infty$
\item \uw{s}{n} \eql \frc{1 - n}{2^n	} \lra 0
\item \uw{s}{n} \eql \frc{3^n}{n^3 + 5} \lra $\infty$
\item \uw{s}{n} \eql \frc{n!}{2^n} \lra $\infty$
\item \uw{s}{n} \eql \frc{n!}{n^n} \eql \frc{1 * 2 * 3 * 4 * 5}{5 * 5 * 5 * 5 * 5} where n \eql 5 \lra 0
\item \uw{s}{n} \eql \frc{n^2}{2^n} \lra 0
\item \uw{s}{n} \eql \frc{n^2}{n!} \lra 0
\elist
}

\bsc{Problem 9}{
Prove Theorem 4.2.12:

Suppose that \prn{\uw{s}{n}} and \prn{\uw{t}{n}} are sequences st \uw{s}{n} \lse \uw{t}{n} \fa n \mem \bn

\balist
\item If lim \uw{s}{n} \eql $+\infty$ then lim \uw{t}{n} \eql $+\infty$
	
	Suppose \exs sequences \prn{\uw{s}{n}} and \prn{\uw{t}{n}} st \uw{s}{n} \lse \uw{t}{n} \fa n \mem \bn where lim \uw{s}{n} \eql $+\infty$ and  lim \uw{t}{n} is NOT $+\infty$.
	
	\uw{t}{n} diverges to $+\infty$ if \fa M \mem \br, \exs N \mem \bn st n \gre N implies \uw{t}{n} \gr M
	
	\lt{M \mem \br}
	
	We know that
	
	\exs N \mem \bn st n \gre N implies \uw{t}{n} \gr M
	
	Since \uw{s}{n} \lse \uw{t}{n} \fa n \mem \bn,
	
	\exs N \mem \bn st n \gre N implies \uw{s}{n} \gre \uw{t}{n} \gr M
	
	\exs N \mem \bn st n \gre N implies \uw{s}{n} \gr M
	
	This is the definition of diverging to $+\infty$, a contradiction.
	
	Hence, \uw{s}{n} diverges to $+\infty$.
	
\item If lim \uw{t}{n} \eql $-\infty$ then lim \uw{s}{n} \eql $-\infty$
	
	Suppose \exs sequences \prn{\uw{s}{n}} and \prn{\uw{t}{n}} st \uw{s}{n} \lse \uw{t}{n} \fa n \mem \bn where lim \uw{s}{n} \eql $-\infty$ and  lim \uw{t}{n} is NOT $-\infty$.
	
	\uw{t}{n} diverges to $-\infty$ if \fa M \mem \br, \exs N \mem \bn st n \gre N implies \uw{t}{n} \ls M
	
	\lt{M \mem \br}
	
	We know that
	
	\exs N \mem \bn st n \gre N implies \uw{t}{n} \ls M
	
	Since \uw{s}{n} \lse \uw{t}{n} \fa n \mem \bn,
	
	\exs N \mem \bn st n \gre N implies \uw{s}{n} \lse \uw{t}{n} \ls M
	
	\exs N \mem \bn st n \gre N implies \uw{s}{n} \ls M
	
	This is the definition of diverging to $-\infty$, a contradiction.
	
	Hence, \uw{s}{n} diverges to $-\infty$.
\elist
}

\bsc{Problem 10}{
Prove the converse part of Theorem 4.2.13:

Let \prn{\uw{s}{n}} be a sequence of positive numbers. Then, lim \uw{s}{n} \eql $+\infty$ iff lim (\frc{1}{\uw{s}{n}}) \eql 0.

\lra

\as{lim \uw{s}{n} \eql $+\infty$}

Given any \ep \gr 0, let M \eql \frc{1}{\ep}. Then there exists a natural number N st n \gre N implies that \uw{s}{n} \gr M \eql \frc{1}{\ep}.

Since each \uw{s}{n} is positive, we have:

\av{\frc{1}{\uw{s}{n}} \ms 0} \ls \ep, whenever n \gre N

Thus, lim (\frc{1}{\uw{s}{n}}) \eql 0.

\lla 

}

\bsc{Problem 17}{
\balist
\item Show that \lmti{n} \frc{k^n}{n!} \eql 0 \fa k \mem \br
\item What can be said about \lmti{n} \frc{n!}{k^n}?
\elist
}

\bsc{Problem 18}{
Assume that \prn{\uw{s}{n}} is a convergent sequence with a \gre \uw{s}{n} \gre b \fa n \mem \bn.

Prove that a \lse lim \uw{s}{n} \lse b.
}

\end{document}