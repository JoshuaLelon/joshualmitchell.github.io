% Thank you Josh Davis for this template!
% https://github.com/jdavis/latex-homework-template/blob/master/homework.tex

\documentclass{article}

\newcommand{\hmwkTitle}{Lec\ \#24}

% % ----------

% Packages

\usepackage{fancyhdr}
\usepackage{extramarks}
\usepackage{amsmath}
\usepackage{amssymb}
\usepackage{amsthm}
\usepackage{amsfonts}
\usepackage{tikz}
\usepackage[plain]{algorithm}
\usepackage{algpseudocode}
\usepackage{enumitem}
\usepackage{chngcntr}

% Libraries

\usetikzlibrary{automata, positioning, arrows}

%
% Basic Document Settings
%

\topmargin=-0.45in
\evensidemargin=0in
\oddsidemargin=0in
\textwidth=6.5in
\textheight=9.0in
\headsep=0.25in

\linespread{1.1}

\pagestyle{fancy}
\lhead{\hmwkAuthorName}
\chead{}
\rhead{\hmwkClass\ (\hmwkClassInstructor): \hmwkTitle}
\lfoot{\lastxmark}
\cfoot{\thepage}

\renewcommand\headrulewidth{0.4pt}
\renewcommand\footrulewidth{0.4pt}

\setlength\parindent{0pt}
\setcounter{secnumdepth}{0}

\newcommand{\hmwkClass}{MATH 3380 / Analysis 1}        % Class
\newcommand{\hmwkClassInstructor}{Dr. Welsh}           % Instructor
\newcommand{\hmwkAuthorName}{\textbf{Joshua Mitchell}} % Author

%
% Title Page
%

\title{
    \vspace{2in}
    \textmd{\textbf{\hmwkClass:\ \hmwkTitle}}\\
    \normalsize\vspace{0.1in}\small\vspace{0.1in}\large{\textit{\hmwkClassInstructor}}
    \vspace{3in}
}

\author{\hmwkAuthorName}
\date{}

\renewcommand{\part}[1]{\textbf{\large Part \Alph{partCounter}}\stepcounter{partCounter}\\}

% Integral dx
\newcommand{\dx}{\mathrm{d}x}

%
% Various Helper Commands
%

% For derivatives
\newcommand{\deriv}[1]{\frac{\mathrm{d}}{\mathrm{d}x} (#1)}

% For partial derivatives
\newcommand{\pderiv}[2]{\frac{\partial}{\partial #1} (#2)}


% Alias for the Solution section header
\newcommand{\solution}{\textbf{\large Solution}}

% Probability commands: Expectation, Variance, Covariance, Bias
\newcommand{\E}{\mathrm{E}}
\newcommand{\Var}{\mathrm{Var}}
\newcommand{\Cov}{\mathrm{Cov}}
\newcommand{\Bias}{\mathrm{Bias}}

% Formatting commands:

\newcommand{\mt}[1]{\ensuremath{#1}}
\newcommand{\nm}[1]{\textrm{#1}}

\newcommand\bsc[2][\DefaultOpt]{%
  \def\DefaultOpt{#2}%
  \section[#1]{#2}%
}
\newcommand\ssc[2][\DefaultOpt]{%
  \def\DefaultOpt{#2}%
  \subsection[#1]{#2}%
}
\newcommand{\bgpf}{\begin{proof} $ $\newline}

\newcommand{\bgeq}{\begin{equation*}}
\newcommand{\eeq}{\end{equation*}}	

\newcommand{\balist}{\begin{enumerate}[label=\alph*.]}
\newcommand{\elist}{\end{enumerate}}

\newcommand{\bilist}{\begin{enumerate}[label=\roman*)]}	

\newcommand{\bgsp}{\begin{split}}
% \newcommand{\esp}{\end{split}} % doesn't work for some reason.

\newcommand\prs[1]{~~~\textbf{(#1)}}

\newcommand{\lt}[1]{\textbf{Let: } #1}
     							   %  if you're setting it to be true
\newcommand{\supp}[1]{\textbf{Suppose: } #1}
     							   %  Suppose (if it'll end up false)
\newcommand{\wts}[1]{\textbf{Want to show: } #1}
     							   %  Want to show
\newcommand{\as}[1]{\textbf{Assume: } #1}
     							   %  if you think it follows from truth
\newcommand{\bpth}[1]{\textbf{(#1)}}

\newcommand{\step}[2]{\begin{equation}\tag{#2}#1\end{equation}}
\newcommand{\epf}{\end{proof}}

\newcommand{\dbs}[3]{\mt{#1_{#2_#3}}}

\newcommand{\sidenote}[1]{-----------------------------------------------------------------Side Note----------------------------------------------------------------
#1 \

---------------------------------------------------------------------------------------------------------------------------------------------}

% Analysis / Logical commands:

\newcommand{\br}{\mt{\mathbb{R}} }       % |R
\newcommand{\bq}{\mt{\mathbb{Q}} }       % |Q
\newcommand{\bn}{\mt{\mathbb{N}} }       % |N
\newcommand{\bc}{\mt{\mathbb{C}} }       % |C
\newcommand{\bz}{\mt{\mathbb{Z}} }       % |Z

\newcommand{\ep}{\mt{\epsilon} }         % epsilon
\newcommand{\fa}{\mt{\forall} }          % for all
\newcommand{\afa}{\mt{\alpha} }
\newcommand{\bta}{\mt{\beta} }
\newcommand{\mem}{\mt{\in} }
\newcommand{\exs}{\mt{\exists} }

\newcommand{\es}{\mt{\emptyset}}        % empty set
\newcommand{\sbs}{\mt{\subset} }         % subset of
\newcommand{\fs}[2]{\{\uw{#1}{1}, \uw{#1}{2}, ... \uw{#1}{#2}\}}

\newcommand{\lra}{ \mt{\longrightarrow} } % implies ----->
\newcommand{\rar}{ \mt{\Rightarrow} }     % implies -->

\newcommand{\lla}{ \mt{\longleftarrow} }  % implies <-----
\newcommand{\lar}{ \mt{\Leftarrow} }      % implies <--

\newcommand{\eql}{\mt{=} }
\newcommand{\pr}{\mt{^\prime} } 		   % prime (i.e. R')
\newcommand{\uw}[2]{#1\mt{_{#2}}}
\newcommand{\frc}[2]{\mt{\frac{#1}{#2}}}

\newcommand{\bnm}[2]{\mt{#1\setminus{#2}}}
\newcommand{\bnt}[2]{\mt{\textrm{#1}\setminus{\textrm{#2}}}}
\newcommand{\bi}{\bnm{\mathbb{R}}{\mathbb{Q}}}

\newcommand{\urng}[2]{\mt{\bigcup_{#1}^{#2}}}
\newcommand{\nrng}[2]{\mt{\bigcap_{#1}^{#2}}}

\newcommand{\nbho}[3]{\textrm{N(}#1, #2\textrm{) }\cap \textrm{ #3} \neq \emptyset}
     							   %  N(x, eps) intersect S \= emptyset
\newcommand{\nbhe}[3]{\textrm{N(}#1, #2\textrm{) }\cap \textrm{ #3} = \emptyset}
     							   %  N(x, eps) intersect S  = emptyset
\newcommand{\dnbho}[3]{\textrm{N*(}#1, #2\textrm{) }\cap \textrm{ #3} \neq \emptyset}
     							   %  N*(x, eps) intersect S \= emptyset
\newcommand{\dnbhe}[3]{\textrm{N*(}#1, #2\textrm{) }\cap \textrm{ #3} = \emptyset}
     							   %  N*(x, eps) intersect S = emptyset
     							 


% ----------

% ----------

% Packages

\usepackage{fancyhdr}
\usepackage{extramarks}
\usepackage{amsmath}
\usepackage{amssymb}
\usepackage{amsthm}
\usepackage{amsfonts}
\usepackage{tikz}
\usepackage[plain]{algorithm}
\usepackage{algpseudocode}
\usepackage{enumitem}
\usepackage{chngcntr}

% Libraries

\graphicspath{{/Users/jm/iclouddrive/3380pics/}}

\usetikzlibrary{automata, positioning, arrows}

%
% Basic Document Settings
%

\topmargin=-0.45in
\evensidemargin=0in
\oddsidemargin=0in
\textwidth=6.5in
\textheight=9.0in
\headsep=0.25in

\linespread{1.1}

\pagestyle{fancy}
\lhead{\hmwkAuthorName}
\chead{}
\rhead{\hmwkClass\ (\hmwkClassInstructor): \hmwkTitle}
\lfoot{\lastxmark}
\cfoot{\thepage}

\renewcommand\headrulewidth{0.4pt}
\renewcommand\footrulewidth{0.4pt}

\setlength\parindent{0pt}
\setcounter{secnumdepth}{0}

\newcommand{\hmwkClass}{MATH 3380 / Analysis 1}        % Class
\newcommand{\hmwkClassInstructor}{Dr. Welsh}           % Instructor
\newcommand{\hmwkAuthorName}{\textbf{Joshua Mitchell}} % Author

%
% Title Page
%

\title{
    \vspace{2in}
    \textmd{\textbf{\hmwkClass:\ \hmwkTitle}}\\
    \normalsize\vspace{0.1in}\small\vspace{0.1in}\large{\textit{\hmwkClassInstructor}}
    \vspace{3in}
}

\author{\hmwkAuthorName}
\date{}

\renewcommand{\part}[1]{\textbf{\large Part \Alph{partCounter}}\stepcounter{partCounter}\\}

% Integral dx
\newcommand{\dx}{\mathrm{d}x}

%
% Various Helper Commands
%

% For derivatives
\newcommand{\deriv}[1]{\frac{\mathrm{d}}{\mathrm{d}x} (#1)}

% For partial derivatives
\newcommand{\pderiv}[2]{\frac{\partial}{\partial #1} (#2)}


% Alias for the Solution section header
\newcommand{\solution}{\textbf{\large Solution}}

% Probability commands: Expectation, Variance, Covariance, Bias
\newcommand{\E}{\mathrm{E}}
\newcommand{\Var}{\mathrm{Var}}
\newcommand{\Cov}{\mathrm{Cov}}
\newcommand{\Bias}{\mathrm{Bias}}

% Formatting commands:

\newcommand{\mt}[1]{\ensuremath{#1}}
\newcommand{\nm}[1]{\textrm{#1}}

\newcommand\bsc[2][\DefaultOpt]{%
  \def\DefaultOpt{#2}%
  \section[#1]{#2}%
}
\newcommand\ssc[2][\DefaultOpt]{%
  \def\DefaultOpt{#2}%
  \subsection[#1]{#2}%
}
\newcommand{\bgpf}{\begin{proof} $ $\newline}

\newcommand{\bgeq}{\begin{equation*}}
\newcommand{\eeq}{\end{equation*}}	

\newcommand{\balist}{\begin{enumerate}[label=\alph*.]}
\newcommand{\elist}{\end{enumerate}}

\newcommand{\bilist}{\begin{enumerate}[label=\roman*)]}	

\newcommand{\bgsp}{\begin{split}}
% \newcommand{\esp}{\end{split}} % doesn't work for some reason.

\newcommand\prs[1]{~~~\textbf{(#1)}}

\newcommand{\lt}[1]{\textbf{Let: } #1}
     							   %  if you're setting it to be true
\newcommand{\supp}[1]{\textbf{Suppose: } #1}
     							   %  Suppose (if it'll end up false)
\newcommand{\wts}[1]{\textbf{Want to show: } #1}
     							   %  Want to show
\newcommand{\as}[1]{\textbf{Assume: } #1}
     							   %  if you think it follows from truth
\newcommand{\bpth}[1]{\textbf{(#1)}}

\newcommand{\step}[2]{\begin{equation}\tag{#2}#1\end{equation}}
\newcommand{\epf}{\end{proof}}

\newcommand{\dbs}[3]{\mt{#1_{#2_#3}}}

\newcommand{\sidenote}[1]{-----------------------------------------------------------------Side Note----------------------------------------------------------------
#1 \

---------------------------------------------------------------------------------------------------------------------------------------------}

% Analysis / Logical commands:

\newcommand{\br}{\mt{\mathbb{R}} }       % |R
\newcommand{\bq}{\mt{\mathbb{Q}} }       % |Q
\newcommand{\bn}{\mt{\mathbb{N}} }       % |N
\newcommand{\bc}{\mt{\mathbb{C}} }       % |C
\newcommand{\bz}{\mt{\mathbb{Z}} }       % |Z

\newcommand{\ep}{\mt{\epsilon} }         % epsilon
\newcommand{\fa}{\mt{\forall} }          % for all
\newcommand{\afa}{\mt{\alpha} }
\newcommand{\bta}{\mt{\beta} }
\newcommand{\dta}{\mt{\delta} }
\newcommand{\mem}{\mt{\in} }
\newcommand{\exs}{\mt{\exists} }

\newcommand{\es}{\mt{\emptyset} }        % empty set
\newcommand{\sbs}{\mt{\subset} }         % subset of
\newcommand{\fs}[2]{\{\uw{#1}{1}, \uw{#1}{2}, ... \uw{#1}{#2}\}}

\newcommand{\lra}{ \mt{\longrightarrow} } % implies ----->
\newcommand{\rar}{ \mt{\Rightarrow} }     % implies -->

\newcommand{\lla}{ \mt{\longleftarrow} }  % implies <-----
\newcommand{\lar}{ \mt{\Leftarrow} }      % implies <--

\newcommand{\av}[1]{\mt{|}#1\mt{|}}  % absolute value

\newcommand{\prn}[1]{(#1)}
\newcommand{\bk}[1]{\{#1\}}

\newcommand{\ps}{\mt{+} }
\newcommand{\ms}{\mt{-} }

\newcommand{\ls}{\mt{<} }
\newcommand{\gr}{\mt{>} }

\newcommand{\lse}{\mt{\leq} }
\newcommand{\gre}{\mt{\geq} }

\newcommand{\eql}{\mt{=} }

\newcommand{\pr}{\mt{^\prime} } 		   % prime (i.e. R')
\newcommand{\uw}[2]{#1\mt{_{#2}}}
\newcommand{\uf}[2]{#1\mt{^{#2}}}
\newcommand{\frc}[2]{\mt{\frac{#1}{#2}}}
\newcommand{\lmti}[1]{\mt{\displaystyle{\lim_{#1 \to \infty}}}}
\newcommand{\limt}[2]{\mt{\displaystyle{\lim_{#1 \to #2}}}}

\newcommand{\bnm}[2]{\mt{#1\setminus{#2}}}
\newcommand{\bnt}[2]{\mt{\textrm{#1}\setminus{\textrm{#2}}}}
\newcommand{\bi}{\bnm{\mathbb{R}}{\mathbb{Q}}}

\newcommand{\urng}[2]{\mt{\bigcup_{#1}^{#2}}}
\newcommand{\nrng}[2]{\mt{\bigcap_{#1}^{#2}}}
\newcommand{\nck}[2]{\mt{{#1 \choose #2}}}

\newcommand{\nbho}[3]{\textrm{N(}#1, #2\textrm{) }\cap \textrm{ #3} \neq \emptyset}
     							   %  N(x, eps) intersect S \= emptyset
\newcommand{\nbhe}[3]{\textrm{N(}#1, #2\textrm{) }\cap \textrm{ #3} = \emptyset}
     							   %  N(x, eps) intersect S  = emptyset
\newcommand{\dnbho}[3]{\textrm{N*(}#1, #2\textrm{) }\cap \textrm{ #3} \neq \emptyset}
     							   %  N*(x, eps) intersect S \= emptyset
\newcommand{\dnbhe}[3]{\textrm{N*(}#1, #2\textrm{) }\cap \textrm{ #3} = \emptyset}
     							   %  N*(x, eps) intersect S = emptyset
     							   
\newcommand{\eqn}[1]{\[#1\]}
\newcommand{\splt}[1]{\begin{split}#1\end{split}}

\newcommand{\infy}{\mt{\infty} }
\newcommand{\unn}{\mt{\cup} }
\newcommand{\inn}{\mt{\cap} }
\newcommand\tab[1][1cm]{\hspace*{#1}}

     							 
% ----------

\begin{document}

Part of HW 11: 1, 2, and 5 from section 5.3

Hint: \uf{5}{x} and \uf{x}{4} are polynomials, so you can assume g(x) is continuous.

Look at 5.3.8 if you have a chance.

\bsc{Theorem 5.3.2}{

D \sbs \br is compact and f : D \lra \br is continuous \rar f(D) is compact (i.e. D is compact \rar f(D) is compact)

\bgpf

A set is compact iff every open cover of a set has a finite subcover.

\as{f(D) \sbs \urng{\afa \mem I}{} \uw{G}{\afa} and \uw{G}{\afa} is open \fa \afa \mem I}

Thus, D \sbs \urng{\afa \mem I}{} \uf{f}{-1}(\uw{G}{\afa})

\

(i.e. \urng{\afa \mem I}{} \uf{f}{-1}(\uw{G}{\afa}) \eql \urng{\afa \mem I}{} (\uw{H}{\afa} \inn D) \sbs \urng{\afa \mem I}{} \uw{H}{\afa})

\sidenote{
\lt{x \mem D}

Then f(x) \mem f(D)

So, \exs \uw{\afa}{0} \mem I st f(x) \mem \dbs{G}{\afa}{0}

and x \mem \uf{f}{-1}(\dbs{G}{\afa}{0}) \sbs \urng{\afa \mem I}{} \uf{f}{-1}(\uw{G}{\afa})
}

From 
\eqn{f^{-1}(G_\alpha) = H_\alpha \inn D \textrm{, where \uw{H}{\afa} is open (Theorem 5.2.14)}}

Since D is compact,

D \sbs \urng{i = 1}{n} \dbs{H}{\afa}{i}

Thus,

D \sbs \urng{i = 1}{n} (\dbs{H}{\afa}{i} \inn D) \eql \urng{i = 1}{n} \uf{f}{-1}(\dbs{G}{\afa}{i})

f(D) \sbs \urng{i = 1}{n} \dbs{G}{\afa}{i}

Hence, f(D) is compact.

\sidenote{
\lt{x \mem D}

Then x \mem \uf{f}{-1}(\dbs{G}{\afa}{k}) where k \mem \bk{1, 2.. n}

So f(x) \mem \dbs{G}{\afa}{k} \sbs \urng{i = 1}{n} \dbs{G}{\afa}{i}

and

f(D) \sbs \urng{i = 1}{n} \dbs{G}{\afa}{i}
}
\epf
}

\newpage 

\bsc{Corollary 5.3.3}{

f : D \lra \br is continuous and D is compact.

Then \exs \uw{x}{1}, \uw{x}{2} \mem D st
\eqn{f(x_1) \lse f(x) \lse f(x_2) \tab \fa x \mem D}
where f(\uw{x}{2}) is the max of f(D) and f(\uw{x}{1}) is the min.

\bgpf

f(D) is compact, so by Heine-Borel, f(D) is closed and bounded.

Thus, sup f(D) exists.

Now, sup f(D) is an accumulation point of f(D).

Since f(D) is closed,

sup f(D) \mem f(D) and sup f(D) \eql f(\uw{x}{2}) for some \uw{x}{2} \mem D

i.e. y \lse sup f(D) \fa y \mem f(D)

and

\exs \uw{y}{n} \mem f(D) st sup f(D) \ms \frc{1}{n} \ls \uw{y}{n} \lse sup f(D) \ls sup f(D) \ps \frc{1}{n}

\epf

}

\bsc{Section 5.4: Uniform Continuity	}{

\ssc{Definition 5.4.1}{

\lt{f : D \lra \br}

We say that f is uniformly continuous on D if for \ep \gr 0, \exs \dta \gr 0 st \av{f(x) \ms f(y)} \ls \ep whenever \av{x \ms y} \ls \dta and x, y \mem D

\

Recall the old definition of continuity:

f is continuous at c \mem D iff \fa \ep \gr 0, \exs \dta \gr 0 st \av{x \ms c} and x \mem D \rar \av{f(x) \ms f(c)} \ls \ep

\

The old one depends on c and \ep, where as the new definition only depends on \ep
}

Notice that if f is uniformly continuous on D, then f is certainly continuous on D.

\ssc{Example 5.4.2}{

Prove that f : \br \lra \br where f(x) \eql 2x is uniformly continuous on \br

\bgpf

\lt{x, y \mem \br and \ep \gr 0}

\av{f(x) \ms f(y)} \eql \av{2x \ms 2y} \eql 2\av{x \ms y} \ls \ep

whenever \av{x \ms y} \ls \dta \eql \frc{\epsilon}{2}

\epf

}

\ssc{Practice 5.4.3}{

Negating the definition:

\exs \ep \gr 0 st \fa \dta \gr 0, \exs x, y \mem D st \av{x \ms y} \ls \dta but \av{f(x) \ms f(y)} \gre \ep

in logic, but is another word for and (where but tells you something different is coming)
}

\ssc{Example 5.4.4}{

\lt{f : \br \lra \br where f(x) \eql \uf{x}{2}}

Prove that f is not uniformly continuous on \br

\bgpf

\lt{\ep \eql 1, x \mem \uf{\br}{+} and choose \dta \gr 0}

\lt{y \eql x \ps \frc{\dta}{2}}

Then
\eqn{|x - y| = \frac{\delta}{2} < \dta}
and
\step{|f(x) - f(y)| = |x^2 - y^2| = |x - y||x + y| = |x + y|\frac{\delta}{2}}{1}
\lt{x \eql \frc{1}{\dta}}

Then
\step{|x + y| = |\frac{1}{\delta} + \frac{1}{\delta} + \frac{\delta}{2} = \frac{2}{\delta} + \frac{\delta}{2} > \frac{2}{\delta}}{2}
From \bpth{1} and \bpth{2},
\eqn{|f(x) - f(y)| > \frac{2}{\delta} \frac{\delta}{2} = 1}

Hence,

f is NOT uniformly continuous on \br

\epf
}

\ssc{Example 5.4.5}{
\lt{f : [$-$5, 5] \lra \br where f(x) \eql \uf{x}{2}}

Show that f is uniformly continuous on [$-$5, 5]

\bgpf

\lt{\ep \gr 0 and x, y \mem D}

Then
\eqn{|f(x) - f(y)| = |x^2 - y^2| = |x - y||x + y| \lse |x - y|(|x| + |y|) \lse |x - y|10 \ls \ep}
whenever \av{x \ms y} \ls \dta \eql \frc{\ep}{10}

\epf
}

\newpage

\ssc{Theorem 5.4.6}{

\as{f : D \lra \br is continuous on a compact set D.}

Then, f is uniformly continuous on D

\bgpf

\lt{\ep \gr 0 and c \mem D}

Then \exs \dta(c) \gr 0 st 
\eqn{f(x) - f(c) < \frac{\epsilon}{2}}
whenever \av{x \ms c} \ls \dta(c) and x \mem D

(this is just the definition of continuity)

Thus,

D \sbs \urng{c \mem D}{} N(c, \frc{\delta(c)}{2})

Since D is compact and N(c, \dta(c)) is an open set \fa c \mem D (which we proved a long time ago), 

\exs n \mem \bn st
\eqn{D \sbs \urng{i = 1}{n} N(c_i, \frac{\delta(c)_i}{2})}
\lt{\dta \eql min \bk{\frc{\dta(c)_1}{2}, \frc{\dta(c)_2}{2}, ... \frc{\dta(c)_i}{2}}}

Now let x, y \mem D with \av{x \ms y} \ls \dta

Then x \mem N(\uw{c}{k}, \frc{\dta(c)_k}{2} for some k \mem \bk{1, 2, .. n}

Then \av{x \ms \uw{c}{k}} \ls \frc{\dta(c)_k}{2} \ls \uw{\dta(c)}{k}

From \bpth{1},

\av{f(x) \ms f(\uw{c}{k})} \ls \frc{\ep}{2} \bpth{2}

Also, \av{y \ms \uw{c}{k}} \eql \av{(y \ms x) \ps (x \ms \uw{c}{k})} \lse \av{y \ms x} \ps \av{x \ms \uw{c}{k}} \ls \dta \ps \frc{\dta(c)_k}{2} \lse \frc{\dta(c)_k}{2} \ps \frc{\dta(c)_k}{2} \eql \uw{\dta(c)}{k}

From \bpth{1},

\av{f(y) \ms f(\uw{c}{k})} \ls \frc{\epsilon}{2} \bpth{3}

Hence, from \bpth{2} and \bpth{3},

\eqn{|f(x) - f(y)| < |f(x) - f(c_k)| + |f(x) - f(y)| < \frc{\ep}{2} + \frc{\ep}{2} = \ep}

Hence, f is uniformly continuous.

\epf

}

}

\end{document}