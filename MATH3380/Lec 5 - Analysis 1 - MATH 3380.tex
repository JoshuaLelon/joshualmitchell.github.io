% Thank you Josh Davis for this template!
% https://github.com/jdavis/latex-homework-template/blob/master/homework.tex

\documentclass{article}

\newcommand{\hmwkTitle}{HW\ \#3}

% % ----------

% Packages

\usepackage{fancyhdr}
\usepackage{extramarks}
\usepackage{amsmath}
\usepackage{amssymb}
\usepackage{amsthm}
\usepackage{amsfonts}
\usepackage{tikz}
\usepackage[plain]{algorithm}
\usepackage{algpseudocode}
\usepackage{enumitem}
\usepackage{chngcntr}

% Libraries

\usetikzlibrary{automata, positioning, arrows}

%
% Basic Document Settings
%

\topmargin=-0.45in
\evensidemargin=0in
\oddsidemargin=0in
\textwidth=6.5in
\textheight=9.0in
\headsep=0.25in

\linespread{1.1}

\pagestyle{fancy}
\lhead{\hmwkAuthorName}
\chead{}
\rhead{\hmwkClass\ (\hmwkClassInstructor): \hmwkTitle}
\lfoot{\lastxmark}
\cfoot{\thepage}

\renewcommand\headrulewidth{0.4pt}
\renewcommand\footrulewidth{0.4pt}

\setlength\parindent{0pt}
\setcounter{secnumdepth}{0}

\newcommand{\hmwkClass}{MATH 3380 / Analysis 1}        % Class
\newcommand{\hmwkClassInstructor}{Dr. Welsh}           % Instructor
\newcommand{\hmwkAuthorName}{\textbf{Joshua Mitchell}} % Author

%
% Title Page
%

\title{
    \vspace{2in}
    \textmd{\textbf{\hmwkClass:\ \hmwkTitle}}\\
    \normalsize\vspace{0.1in}\small\vspace{0.1in}\large{\textit{\hmwkClassInstructor}}
    \vspace{3in}
}

\author{\hmwkAuthorName}
\date{}

\renewcommand{\part}[1]{\textbf{\large Part \Alph{partCounter}}\stepcounter{partCounter}\\}

% Integral dx
\newcommand{\dx}{\mathrm{d}x}

%
% Various Helper Commands
%

% For derivatives
\newcommand{\deriv}[1]{\frac{\mathrm{d}}{\mathrm{d}x} (#1)}

% For partial derivatives
\newcommand{\pderiv}[2]{\frac{\partial}{\partial #1} (#2)}


% Alias for the Solution section header
\newcommand{\solution}{\textbf{\large Solution}}

% Probability commands: Expectation, Variance, Covariance, Bias
\newcommand{\E}{\mathrm{E}}
\newcommand{\Var}{\mathrm{Var}}
\newcommand{\Cov}{\mathrm{Cov}}
\newcommand{\Bias}{\mathrm{Bias}}

% Formatting commands:

\newcommand{\mt}[1]{\ensuremath{#1}}
\newcommand{\nm}[1]{\textrm{#1}}

\newcommand\bsc[2][\DefaultOpt]{%
  \def\DefaultOpt{#2}%
  \section[#1]{#2}%
}
\newcommand\ssc[2][\DefaultOpt]{%
  \def\DefaultOpt{#2}%
  \subsection[#1]{#2}%
}
\newcommand{\bgpf}{\begin{proof} $ $\newline}

\newcommand{\bgeq}{\begin{equation*}}
\newcommand{\eeq}{\end{equation*}}	

\newcommand{\balist}{\begin{enumerate}[label=\alph*.]}
\newcommand{\elist}{\end{enumerate}}

\newcommand{\bilist}{\begin{enumerate}[label=\roman*)]}	

\newcommand{\bgsp}{\begin{split}}
% \newcommand{\esp}{\end{split}} % doesn't work for some reason.

\newcommand\prs[1]{~~~\textbf{(#1)}}

\newcommand{\lt}[1]{\textbf{Let: } #1}
     							   %  if you're setting it to be true
\newcommand{\supp}[1]{\textbf{Suppose: } #1}
     							   %  Suppose (if it'll end up false)
\newcommand{\wts}[1]{\textbf{Want to show: } #1}
     							   %  Want to show
\newcommand{\as}[1]{\textbf{Assume: } #1}
     							   %  if you think it follows from truth
\newcommand{\bpth}[1]{\textbf{(#1)}}

\newcommand{\step}[2]{\begin{equation}\tag{#2}#1\end{equation}}
\newcommand{\epf}{\end{proof}}

\newcommand{\dbs}[3]{\mt{#1_{#2_#3}}}

\newcommand{\sidenote}[1]{-----------------------------------------------------------------Side Note----------------------------------------------------------------
#1 \

---------------------------------------------------------------------------------------------------------------------------------------------}

% Analysis / Logical commands:

\newcommand{\br}{\mt{\mathbb{R}} }       % |R
\newcommand{\bq}{\mt{\mathbb{Q}} }       % |Q
\newcommand{\bn}{\mt{\mathbb{N}} }       % |N
\newcommand{\bc}{\mt{\mathbb{C}} }       % |C
\newcommand{\bz}{\mt{\mathbb{Z}} }       % |Z

\newcommand{\ep}{\mt{\epsilon} }         % epsilon
\newcommand{\fa}{\mt{\forall} }          % for all
\newcommand{\afa}{\mt{\alpha} }
\newcommand{\bta}{\mt{\beta} }
\newcommand{\mem}{\mt{\in} }
\newcommand{\exs}{\mt{\exists} }

\newcommand{\es}{\mt{\emptyset}}        % empty set
\newcommand{\sbs}{\mt{\subset} }         % subset of
\newcommand{\fs}[2]{\{\uw{#1}{1}, \uw{#1}{2}, ... \uw{#1}{#2}\}}

\newcommand{\lra}{ \mt{\longrightarrow} } % implies ----->
\newcommand{\rar}{ \mt{\Rightarrow} }     % implies -->

\newcommand{\lla}{ \mt{\longleftarrow} }  % implies <-----
\newcommand{\lar}{ \mt{\Leftarrow} }      % implies <--

\newcommand{\eql}{\mt{=} }
\newcommand{\pr}{\mt{^\prime} } 		   % prime (i.e. R')
\newcommand{\uw}[2]{#1\mt{_{#2}}}
\newcommand{\frc}[2]{\mt{\frac{#1}{#2}}}

\newcommand{\bnm}[2]{\mt{#1\setminus{#2}}}
\newcommand{\bnt}[2]{\mt{\textrm{#1}\setminus{\textrm{#2}}}}
\newcommand{\bi}{\bnm{\mathbb{R}}{\mathbb{Q}}}

\newcommand{\urng}[2]{\mt{\bigcup_{#1}^{#2}}}
\newcommand{\nrng}[2]{\mt{\bigcap_{#1}^{#2}}}

\newcommand{\nbho}[3]{\textrm{N(}#1, #2\textrm{) }\cap \textrm{ #3} \neq \emptyset}
     							   %  N(x, eps) intersect S \= emptyset
\newcommand{\nbhe}[3]{\textrm{N(}#1, #2\textrm{) }\cap \textrm{ #3} = \emptyset}
     							   %  N(x, eps) intersect S  = emptyset
\newcommand{\dnbho}[3]{\textrm{N*(}#1, #2\textrm{) }\cap \textrm{ #3} \neq \emptyset}
     							   %  N*(x, eps) intersect S \= emptyset
\newcommand{\dnbhe}[3]{\textrm{N*(}#1, #2\textrm{) }\cap \textrm{ #3} = \emptyset}
     							   %  N*(x, eps) intersect S = emptyset
     							 


% ----------

% ----------

% Packages

\usepackage{fancyhdr}
\usepackage{extramarks}
\usepackage{amsmath}
\usepackage{amssymb}
\usepackage{amsthm}
\usepackage{amsfonts}
\usepackage{tikz}
\usepackage[plain]{algorithm}
\usepackage{algpseudocode}
\usepackage{enumitem}
\usepackage{chngcntr}

% Libraries

\usetikzlibrary{automata, positioning, arrows}

%
% Basic Document Settings
%

\topmargin=-0.45in
\evensidemargin=0in
\oddsidemargin=0in
\textwidth=6.5in
\textheight=9.0in
\headsep=0.25in

\linespread{1.1}

\pagestyle{fancy}
\lhead{\hmwkAuthorName}
\chead{}
\rhead{\hmwkClass\ (\hmwkClassInstructor): \hmwkTitle}
\lfoot{\lastxmark}
\cfoot{\thepage}

\renewcommand\headrulewidth{0.4pt}
\renewcommand\footrulewidth{0.4pt}

\setlength\parindent{0pt}
\setcounter{secnumdepth}{0}

\newcommand{\hmwkClass}{MATH 3380 / Analysis 1}        % Class
\newcommand{\hmwkClassInstructor}{Dr. Welsh}           % Instructor
\newcommand{\hmwkAuthorName}{\textbf{Joshua Mitchell}} % Author

%
% Title Page
%

\title{
    \vspace{2in}
    \textmd{\textbf{\hmwkClass:\ \hmwkTitle}}\\
    \normalsize\vspace{0.1in}\small\vspace{0.1in}\large{\textit{\hmwkClassInstructor}}
    \vspace{3in}
}

\author{\hmwkAuthorName}
\date{}

\renewcommand{\part}[1]{\textbf{\large Part \Alph{partCounter}}\stepcounter{partCounter}\\}

% Integral dx
\newcommand{\dx}{\mathrm{d}x}

%
% Various Helper Commands
%

% For derivatives
\newcommand{\deriv}[1]{\frac{\mathrm{d}}{\mathrm{d}x} (#1)}

% For partial derivatives
\newcommand{\pderiv}[2]{\frac{\partial}{\partial #1} (#2)}


% Alias for the Solution section header
\newcommand{\solution}{\textbf{\large Solution}}

% Probability commands: Expectation, Variance, Covariance, Bias
\newcommand{\E}{\mathrm{E}}
\newcommand{\Var}{\mathrm{Var}}
\newcommand{\Cov}{\mathrm{Cov}}
\newcommand{\Bias}{\mathrm{Bias}}

% Formatting commands:

\newcommand{\mt}[1]{\ensuremath{#1}}
\newcommand{\nm}[1]{\textrm{#1}}

\newcommand\bsc[2][\DefaultOpt]{%
  \def\DefaultOpt{#2}%
  \section[#1]{#2}%
}
\newcommand\ssc[2][\DefaultOpt]{%
  \def\DefaultOpt{#2}%
  \subsection[#1]{#2}%
}
\newcommand{\bgpf}{\begin{proof} $ $\newline}

\newcommand{\bgeq}{\begin{equation*}}
\newcommand{\eeq}{\end{equation*}}	

\newcommand{\balist}{\begin{enumerate}[label=\alph*.]}
\newcommand{\elist}{\end{enumerate}}

\newcommand{\bilist}{\begin{enumerate}[label=\roman*)]}	

\newcommand{\bgsp}{\begin{split}}
% \newcommand{\esp}{\end{split}} % doesn't work for some reason.

\newcommand\prs[1]{~~~\textbf{(#1)}}

\newcommand{\lt}[1]{\textbf{Let: } #1}
     							   %  if you're setting it to be true
\newcommand{\supp}[1]{\textbf{Suppose: } #1}
     							   %  Suppose (if it'll end up false)
\newcommand{\wts}[1]{\textbf{Want to show: } #1}
     							   %  Want to show
\newcommand{\as}[1]{\textbf{Assume: } #1}
     							   %  if you think it follows from truth
\newcommand{\bpth}[1]{\textbf{(#1)}}

\newcommand{\step}[2]{\begin{equation}\tag{#2}#1\end{equation}}
\newcommand{\epf}{\end{proof}}

\newcommand{\dbs}[3]{\mt{#1_{#2_#3}}}

\newcommand{\sidenote}[1]{-----------------------------------------------------------------Side Note----------------------------------------------------------------
#1 \

---------------------------------------------------------------------------------------------------------------------------------------------}

% Analysis / Logical commands:

\newcommand{\br}{\mt{\mathbb{R}} }       % |R
\newcommand{\bq}{\mt{\mathbb{Q}} }       % |Q
\newcommand{\bn}{\mt{\mathbb{N}} }       % |N
\newcommand{\bc}{\mt{\mathbb{C}} }       % |C
\newcommand{\bz}{\mt{\mathbb{Z}} }       % |Z

\newcommand{\ep}{\mt{\epsilon} }         % epsilon
\newcommand{\fa}{\mt{\forall} }          % for all
\newcommand{\afa}{\mt{\alpha} }
\newcommand{\bta}{\mt{\beta} }
\newcommand{\mem}{\mt{\in} }
\newcommand{\exs}{\mt{\exists} }

\newcommand{\es}{\mt{\emptyset} }        % empty set
\newcommand{\sbs}{\mt{\subset} }         % subset of
\newcommand{\fs}[2]{\{\uw{#1}{1}, \uw{#1}{2}, ... \uw{#1}{#2}\}}

\newcommand{\lra}{ \mt{\longrightarrow} } % implies ----->
\newcommand{\rar}{ \mt{\Rightarrow} }     % implies -->

\newcommand{\lla}{ \mt{\longleftarrow} }  % implies <-----
\newcommand{\lar}{ \mt{\Leftarrow} }      % implies <--

\newcommand{\eql}{\mt{=} }
\newcommand{\pr}{\mt{^\prime} } 		   % prime (i.e. R')
\newcommand{\uw}[2]{#1\mt{_{#2}}}
\newcommand{\frc}[2]{\mt{\frac{#1}{#2}}}

\newcommand{\bnm}[2]{\mt{#1\setminus{#2}}}
\newcommand{\bnt}[2]{\mt{\textrm{#1}\setminus{\textrm{#2}}}}
\newcommand{\bi}{\bnm{\mathbb{R}}{\mathbb{Q}}}

\newcommand{\urng}[2]{\mt{\bigcup_{#1}^{#2}}}
\newcommand{\nrng}[2]{\mt{\bigcap_{#1}^{#2}}}

\newcommand{\nbho}[3]{\textrm{N(}#1, #2\textrm{) }\cap \textrm{ #3} \neq \emptyset}
     							   %  N(x, eps) intersect S \= emptyset
\newcommand{\nbhe}[3]{\textrm{N(}#1, #2\textrm{) }\cap \textrm{ #3} = \emptyset}
     							   %  N(x, eps) intersect S  = emptyset
\newcommand{\dnbho}[3]{\textrm{N*(}#1, #2\textrm{) }\cap \textrm{ #3} \neq \emptyset}
     							   %  N*(x, eps) intersect S \= emptyset
\newcommand{\dnbhe}[3]{\textrm{N*(}#1, #2\textrm{) }\cap \textrm{ #3} = \emptyset}
     							   %  N*(x, eps) intersect S = emptyset
     							
% ----------

\begin{document}

\bsc{Definition 3.4.6 - Def of Open/Closed Set}{

\lt{S \sbs \br} \

\

	if bd S \sbs S, then S is closed.
	
	if bd S \sbs (\bnt{\br}{S}), then S is open.

}

\bsc{Theorem 3.4.7}{

\balist
\item A set S is open iff S \eql int S; i.e. iff \fa s \mem S, s is an \textbf{interior point}.
\item A set S is closed iff its compliment, \bnt{\br}{S} is open.
	
	Equivalently, a set s is open iff \bnt{\br}{S} is closed.
\elist

\bgpf

\bpth{a}:

\lra

\as{S is open}

\wts{S \eql int S}

By definition, int S \sbs S.

\wts{S \sbs int S}

\lt{x \mem S \bpth{1}}

\wts{x \mem int S}

Since S is open, bd S \sbs \bnt{\br}{S}

So, x $\not\in$ bd S.

Thus, \exs \ep $>$ 0 st $\nbho{x}{\ep}{\bnt{\br}{S}}$

\fa \ep $>$ 0, $\nbho{x}{\ep}{S}$

x \mem bd S if \fa \ep $>$ 0,

$\nbho{x}{\ep}{S}$ and $\nbho{x}{\ep}{(\bnt{\br}{S})}$

Thus, N(x, \ep) \sbs S.

So, x \mem int S.

This proves that S \sbs int S

\lla

\as{S \eql int S}

\wts{S is open}

\lt{x \mem bd S}

\wts{x \mem \bnt{\br}{S}}

Since x \mem bd S, we conclude that x $\not\in$ int S.

\sidenote{
x \mem bd S if, \fa \ep $>$ 0, 

$\nbho{x}{\ep}{S}$ and $\nbho{x}{\ep}{(\bnt{\br}{S})}$
}

Thus, x \mem \bnt{\br}{S}.

So, bd S \sbs \bnt{\br}{S}.

So, by definition, 

S is open.

\newpage

\bpth{b}: S is closed iff \bnt{\br}{S} is open.

So, x $\not\in$ bd S.

Thus, \exs \ep $>$ 0 st  $\nbho{x}{\ep}{S}$

Hence, N(x, \ep) \sbs \bnt{\br}{S}

So, \bnt{\br}{S} is open from \bpth{a}.

\lla

\as{\bnt{\br}{S} is open}

\wts{S is closed}

\lt{x \mem bd S}

\wts{x \mem S}

Since x \mem bd S, \fa \ep $>$ 0,

\step{\nbho{x}{\ep}{S}}{1}
and
\step{\nbho{x}{\ep}{(\bnt{\br}{S})}}{2}
Since \bnt{\br}{S} is open, \fa s \mem \bnt{\br}{S}, s is an \textbf{interior point} of \bnt{\br}{S}.

Thus, x \mem S. 

We have shown that bd S \sbs S.

By definition, S is closed.

\epf

}


\bsc{Example 3.4.8}{

\balist
\item `[0, 5] is a closed set. (\bnt{\br}{[0, 5]} \eql ($-\infty$, 0) $\cup$ (5, $\infty$) )
\item (0, 5) is an open set.
\item `[0, 5) is neither open nor closed.
\item `[2, $\infty$) is a closed set.
\item \br is both open and closed.

	bd \br \eql \es \sbs \br
	
	Also, int \br \eql \br
	
	Also, \es is both open and closed.
\elist

}

\bsc{Theorem 2 (not in book)}{

\lt{x \mem \br, \ep $>$ 0}

Then:

\balist
\item N(x, \ep) is an open set
\item N*(x, \ep) is an open set
\elist

\bpth{a}

\bgpf

N(x, \ep) \eql \{y : $|y - x|$ $<$ \ep\} i.e. $-$\ep $<$ $y - x$ $<$ \ep

So, y \mem N(x, \ep) iff x $-$ \ep $<$ y $<$ x $+$ \ep

\lt{y \mem N(x, \ep)}

We shall find $\hat\epsilon$ $>$ 0 st

N(y, $\hat\epsilon$) \sbs N(x, \ep), which will show that

N(x, \ep) is open.

\sidenote{
----(---|---|----|--|----)---

x-ep, y-ephat, y, yplusEphat, x,  xplusEp
}

\lt{$\hat\epsilon$ \eql min \{$y - (x - \epsilon)$,  $x + \epsilon - y$\}} \bpth{1}

\wts{N(y, $\hat\epsilon$) \sbs N(x, \ep)}

\lt{z \mem N(y, $\hat\epsilon$)}

Then, y $-$ $\hat\epsilon$ $<$ z $<$ y $+$ $\hat\epsilon$ \bpth{2}

From \bpth{1}, $\hat\epsilon$ $\leq$ $y - (x - \epsilon)$ \bpth{3}

and

$\hat\epsilon$ $\leq$ x $+$ \ep $-$ y \bpth{4}

So from \bpth{4},

y $+$ $\hat\epsilon$ $\leq$ y $+$ x $+$ \ep $-$ \uw{y}{i}

y $+$ $\hat\epsilon$ $\leq$ x $+$ \ep

From \bpth{3}, (x $-$ \ep) $-$ y $\leq$ $-\hat\epsilon$ \bpth{5}

Then, 

y $+$ (x $-$ \ep) $-$ y $\leq$ y $-$ $\hat\epsilon$ 

x $-$ \ep $\leq$ y $-$ $\hat\epsilon$ \bpth{6}

From \bpth{2}, \bpth{5}, \bpth{6},

x $-$ \ep $\leq$ y $-$ $\hat\epsilon$ $<$ z $<$ y $+$ $\hat\epsilon$ $\leq$ x $+$ \ep

Therefore,

x $-$ \ep $<$ z $<$ x $+$ \ep

Thus, z \mem N(x, \ep).

Hence, 

N(y, $\hat\epsilon$) \sbs N(x, \ep)

Which proves that

N(x, \ep) is open. \

\

\bpth{b}: N*(x, \ep) is an open set. Similar to \bpth{a}.

\epf

}
\bsc{Theorem 3.4.10}{

\lt{I be an index set. I \sbs \bn \sbs \br}

\supp{\uw{G}{\afa} \sbs \br is an open set \fa \afa \mem I}

Then, 

\balist
\item \urng{\afa \mem I}{} \uw{G}{\afa} is an open set.
\item If \uw{G}{i} \sbs \br is open \fa i \eql 1, 2, ... n where n \mem \bn
	
	Then \nrng{i = 1}{n} \uw{G}{i} is open.
\elist

\bgpf

\bpth{a}: \

\

\lt{x \mem \urng{\afa \mem I}{} \uw{G}{i}}

Thus, \exs \uw{\afa}{0} \mem I st x \mem \dbs{G}{\afa}{0}.

Since \dbs{G}{\afa}{0} is open, \exs \uw{$\epsilon$}{0} $>$ 0 st N(x, \uw{\ep}{0}) \sbs \dbs{G}{\afa}{0}

Thus, N(x, \uw{\ep}{0}) \sbs \urng{\afa \mem I}{} \uw{G}{\afa}

This proves that x \mem int (\urng{\afa \mem I}{} \uw{G}{\afa})

By Theorem 3.4.7 a),

\urng{\afa \mem I}{} \uw{G}{\afa} is open. \

\

\bpth{b}: \

\

\lt{x \mem \nrng{i = 1}{n} \uw{G}{i}}

Thus, x \mem \uw{G}{i} \fa i \eql 1, 2, ... n

Since \uw{G}{i} is open \fa i \eql 1, 2, ... n

\exs \uw{\ep}{i} $>$ 0 st N(x, \uw{\ep}{i}) \sbs \uw{G}{i}  \fa i from 1 to n.

Choose \ep \eql min \{\uw{\ep}{1}, \uw{\ep}{2}, ... \uw{\ep}{n}\} $>$ 0

Then N(x, \ep) \sbs N(x, \uw{\ep}{i}) \fa i from 1 to n.

Hence, N(x, \ep) \sbs \nrng{i = 1}{n} \uw{G}{i}

Hence, \nrng{i = 1}{n} \uw{G}{i} is open.
\epf

\sidenote{
----(---(---|---)---)----

x-epi, x-ep, x, xplusEp, xplusEpi
}

}

\bsc{Corollary 3.4.11}{

\balist
\item Let \uw{F}{\afa} be closed \fa \afa \mem I, I is an index set.

	Then \nrng{\afa \mem I}{} \uw{F}{\afa} is closed.
\item Let \uw{F}{i} be closed \fa i from 1 to n.
	
	Then ( \urng{i = 1}{n} \uw{F}{i} ) is closed.

\elist

}

\end{document}