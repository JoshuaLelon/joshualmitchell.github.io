% Thank you Josh Davis for this template!
% https://github.com/jdavis/latex-homework-template/blob/master/homework.tex

\documentclass{article}

\newcommand{\hmwkTitle}{Lec\ \#10}

% % ----------

% Packages

\usepackage{fancyhdr}
\usepackage{extramarks}
\usepackage{amsmath}
\usepackage{amssymb}
\usepackage{amsthm}
\usepackage{amsfonts}
\usepackage{tikz}
\usepackage[plain]{algorithm}
\usepackage{algpseudocode}
\usepackage{enumitem}
\usepackage{chngcntr}

% Libraries

\usetikzlibrary{automata, positioning, arrows}

%
% Basic Document Settings
%

\topmargin=-0.45in
\evensidemargin=0in
\oddsidemargin=0in
\textwidth=6.5in
\textheight=9.0in
\headsep=0.25in

\linespread{1.1}

\pagestyle{fancy}
\lhead{\hmwkAuthorName}
\chead{}
\rhead{\hmwkClass\ (\hmwkClassInstructor): \hmwkTitle}
\lfoot{\lastxmark}
\cfoot{\thepage}

\renewcommand\headrulewidth{0.4pt}
\renewcommand\footrulewidth{0.4pt}

\setlength\parindent{0pt}
\setcounter{secnumdepth}{0}

\newcommand{\hmwkClass}{MATH 3380 / Analysis 1}        % Class
\newcommand{\hmwkClassInstructor}{Dr. Welsh}           % Instructor
\newcommand{\hmwkAuthorName}{\textbf{Joshua Mitchell}} % Author

%
% Title Page
%

\title{
    \vspace{2in}
    \textmd{\textbf{\hmwkClass:\ \hmwkTitle}}\\
    \normalsize\vspace{0.1in}\small\vspace{0.1in}\large{\textit{\hmwkClassInstructor}}
    \vspace{3in}
}

\author{\hmwkAuthorName}
\date{}

\renewcommand{\part}[1]{\textbf{\large Part \Alph{partCounter}}\stepcounter{partCounter}\\}

% Integral dx
\newcommand{\dx}{\mathrm{d}x}

%
% Various Helper Commands
%

% For derivatives
\newcommand{\deriv}[1]{\frac{\mathrm{d}}{\mathrm{d}x} (#1)}

% For partial derivatives
\newcommand{\pderiv}[2]{\frac{\partial}{\partial #1} (#2)}


% Alias for the Solution section header
\newcommand{\solution}{\textbf{\large Solution}}

% Probability commands: Expectation, Variance, Covariance, Bias
\newcommand{\E}{\mathrm{E}}
\newcommand{\Var}{\mathrm{Var}}
\newcommand{\Cov}{\mathrm{Cov}}
\newcommand{\Bias}{\mathrm{Bias}}

% Formatting commands:

\newcommand{\mt}[1]{\ensuremath{#1}}
\newcommand{\nm}[1]{\textrm{#1}}

\newcommand\bsc[2][\DefaultOpt]{%
  \def\DefaultOpt{#2}%
  \section[#1]{#2}%
}
\newcommand\ssc[2][\DefaultOpt]{%
  \def\DefaultOpt{#2}%
  \subsection[#1]{#2}%
}
\newcommand{\bgpf}{\begin{proof} $ $\newline}

\newcommand{\bgeq}{\begin{equation*}}
\newcommand{\eeq}{\end{equation*}}	

\newcommand{\balist}{\begin{enumerate}[label=\alph*.]}
\newcommand{\elist}{\end{enumerate}}

\newcommand{\bilist}{\begin{enumerate}[label=\roman*)]}	

\newcommand{\bgsp}{\begin{split}}
% \newcommand{\esp}{\end{split}} % doesn't work for some reason.

\newcommand\prs[1]{~~~\textbf{(#1)}}

\newcommand{\lt}[1]{\textbf{Let: } #1}
     							   %  if you're setting it to be true
\newcommand{\supp}[1]{\textbf{Suppose: } #1}
     							   %  Suppose (if it'll end up false)
\newcommand{\wts}[1]{\textbf{Want to show: } #1}
     							   %  Want to show
\newcommand{\as}[1]{\textbf{Assume: } #1}
     							   %  if you think it follows from truth
\newcommand{\bpth}[1]{\textbf{(#1)}}

\newcommand{\step}[2]{\begin{equation}\tag{#2}#1\end{equation}}
\newcommand{\epf}{\end{proof}}

\newcommand{\dbs}[3]{\mt{#1_{#2_#3}}}

\newcommand{\sidenote}[1]{-----------------------------------------------------------------Side Note----------------------------------------------------------------
#1 \

---------------------------------------------------------------------------------------------------------------------------------------------}

% Analysis / Logical commands:

\newcommand{\br}{\mt{\mathbb{R}} }       % |R
\newcommand{\bq}{\mt{\mathbb{Q}} }       % |Q
\newcommand{\bn}{\mt{\mathbb{N}} }       % |N
\newcommand{\bc}{\mt{\mathbb{C}} }       % |C
\newcommand{\bz}{\mt{\mathbb{Z}} }       % |Z

\newcommand{\ep}{\mt{\epsilon} }         % epsilon
\newcommand{\fa}{\mt{\forall} }          % for all
\newcommand{\afa}{\mt{\alpha} }
\newcommand{\bta}{\mt{\beta} }
\newcommand{\mem}{\mt{\in} }
\newcommand{\exs}{\mt{\exists} }

\newcommand{\es}{\mt{\emptyset}}        % empty set
\newcommand{\sbs}{\mt{\subset} }         % subset of
\newcommand{\fs}[2]{\{\uw{#1}{1}, \uw{#1}{2}, ... \uw{#1}{#2}\}}

\newcommand{\lra}{ \mt{\longrightarrow} } % implies ----->
\newcommand{\rar}{ \mt{\Rightarrow} }     % implies -->

\newcommand{\lla}{ \mt{\longleftarrow} }  % implies <-----
\newcommand{\lar}{ \mt{\Leftarrow} }      % implies <--

\newcommand{\eql}{\mt{=} }
\newcommand{\pr}{\mt{^\prime} } 		   % prime (i.e. R')
\newcommand{\uw}[2]{#1\mt{_{#2}}}
\newcommand{\frc}[2]{\mt{\frac{#1}{#2}}}

\newcommand{\bnm}[2]{\mt{#1\setminus{#2}}}
\newcommand{\bnt}[2]{\mt{\textrm{#1}\setminus{\textrm{#2}}}}
\newcommand{\bi}{\bnm{\mathbb{R}}{\mathbb{Q}}}

\newcommand{\urng}[2]{\mt{\bigcup_{#1}^{#2}}}
\newcommand{\nrng}[2]{\mt{\bigcap_{#1}^{#2}}}

\newcommand{\nbho}[3]{\textrm{N(}#1, #2\textrm{) }\cap \textrm{ #3} \neq \emptyset}
     							   %  N(x, eps) intersect S \= emptyset
\newcommand{\nbhe}[3]{\textrm{N(}#1, #2\textrm{) }\cap \textrm{ #3} = \emptyset}
     							   %  N(x, eps) intersect S  = emptyset
\newcommand{\dnbho}[3]{\textrm{N*(}#1, #2\textrm{) }\cap \textrm{ #3} \neq \emptyset}
     							   %  N*(x, eps) intersect S \= emptyset
\newcommand{\dnbhe}[3]{\textrm{N*(}#1, #2\textrm{) }\cap \textrm{ #3} = \emptyset}
     							   %  N*(x, eps) intersect S = emptyset
     							 


% ----------

% ----------

% Packages

\usepackage{fancyhdr}
\usepackage{extramarks}
\usepackage{amsmath}
\usepackage{amssymb}
\usepackage{amsthm}
\usepackage{amsfonts}
\usepackage{tikz}
\usepackage[plain]{algorithm}
\usepackage{algpseudocode}
\usepackage{enumitem}
\usepackage{chngcntr}

% Libraries

\usetikzlibrary{automata, positioning, arrows}

%
% Basic Document Settings
%

\topmargin=-0.45in
\evensidemargin=0in
\oddsidemargin=0in
\textwidth=6.5in
\textheight=9.0in
\headsep=0.25in

\linespread{1.1}

\pagestyle{fancy}
\lhead{\hmwkAuthorName}
\chead{}
\rhead{\hmwkClass\ (\hmwkClassInstructor): \hmwkTitle}
\lfoot{\lastxmark}
\cfoot{\thepage}

\renewcommand\headrulewidth{0.4pt}
\renewcommand\footrulewidth{0.4pt}

\setlength\parindent{0pt}
\setcounter{secnumdepth}{0}

\newcommand{\hmwkClass}{MATH 3380 / Analysis 1}        % Class
\newcommand{\hmwkClassInstructor}{Dr. Welsh}           % Instructor
\newcommand{\hmwkAuthorName}{\textbf{Joshua Mitchell}} % Author

%
% Title Page
%

\title{
    \vspace{2in}
    \textmd{\textbf{\hmwkClass:\ \hmwkTitle}}\\
    \normalsize\vspace{0.1in}\small\vspace{0.1in}\large{\textit{\hmwkClassInstructor}}
    \vspace{3in}
}

\author{\hmwkAuthorName}
\date{}

\renewcommand{\part}[1]{\textbf{\large Part \Alph{partCounter}}\stepcounter{partCounter}\\}

% Integral dx
\newcommand{\dx}{\mathrm{d}x}

%
% Various Helper Commands
%

% For derivatives
\newcommand{\deriv}[1]{\frac{\mathrm{d}}{\mathrm{d}x} (#1)}

% For partial derivatives
\newcommand{\pderiv}[2]{\frac{\partial}{\partial #1} (#2)}


% Alias for the Solution section header
\newcommand{\solution}{\textbf{\large Solution}}

% Probability commands: Expectation, Variance, Covariance, Bias
\newcommand{\E}{\mathrm{E}}
\newcommand{\Var}{\mathrm{Var}}
\newcommand{\Cov}{\mathrm{Cov}}
\newcommand{\Bias}{\mathrm{Bias}}

% Formatting commands:

\newcommand{\mt}[1]{\ensuremath{#1}}
\newcommand{\nm}[1]{\textrm{#1}}

\newcommand\bsc[2][\DefaultOpt]{%
  \def\DefaultOpt{#2}%
  \section[#1]{#2}%
}
\newcommand\ssc[2][\DefaultOpt]{%
  \def\DefaultOpt{#2}%
  \subsection[#1]{#2}%
}
\newcommand{\bgpf}{\begin{proof} $ $\newline}

\newcommand{\bgeq}{\begin{equation*}}
\newcommand{\eeq}{\end{equation*}}	

\newcommand{\balist}{\begin{enumerate}[label=\alph*.]}
\newcommand{\elist}{\end{enumerate}}

\newcommand{\bilist}{\begin{enumerate}[label=\roman*)]}	

\newcommand{\bgsp}{\begin{split}}
% \newcommand{\esp}{\end{split}} % doesn't work for some reason.

\newcommand\prs[1]{~~~\textbf{(#1)}}

\newcommand{\lt}[1]{\textbf{Let: } #1}
     							   %  if you're setting it to be true
\newcommand{\supp}[1]{\textbf{Suppose: } #1}
     							   %  Suppose (if it'll end up false)
\newcommand{\wts}[1]{\textbf{Want to show: } #1}
     							   %  Want to show
\newcommand{\as}[1]{\textbf{Assume: } #1}
     							   %  if you think it follows from truth
\newcommand{\bpth}[1]{\textbf{(#1)}}

\newcommand{\step}[2]{\begin{equation}\tag{#2}#1\end{equation}}
\newcommand{\epf}{\end{proof}}

\newcommand{\dbs}[3]{\mt{#1_{#2_#3}}}

\newcommand{\sidenote}[1]{-----------------------------------------------------------------Side Note----------------------------------------------------------------
#1 \

---------------------------------------------------------------------------------------------------------------------------------------------}

% Analysis / Logical commands:

\newcommand{\br}{\mt{\mathbb{R}} }       % |R
\newcommand{\bq}{\mt{\mathbb{Q}} }       % |Q
\newcommand{\bn}{\mt{\mathbb{N}} }       % |N
\newcommand{\bc}{\mt{\mathbb{C}} }       % |C
\newcommand{\bz}{\mt{\mathbb{Z}} }       % |Z

\newcommand{\ep}{\mt{\epsilon} }         % epsilon
\newcommand{\fa}{\mt{\forall} }          % for all
\newcommand{\afa}{\mt{\alpha} }
\newcommand{\bta}{\mt{\beta} }
\newcommand{\mem}{\mt{\in} }
\newcommand{\exs}{\mt{\exists} }

\newcommand{\es}{\mt{\emptyset} }        % empty set
\newcommand{\sbs}{\mt{\subset} }         % subset of
\newcommand{\fs}[2]{\{\uw{#1}{1}, \uw{#1}{2}, ... \uw{#1}{#2}\}}

\newcommand{\lra}{ \mt{\longrightarrow} } % implies ----->
\newcommand{\rar}{ \mt{\Rightarrow} }     % implies -->

\newcommand{\lla}{ \mt{\longleftarrow} }  % implies <-----
\newcommand{\lar}{ \mt{\Leftarrow} }      % implies <--

\newcommand{\eql}{\mt{=} }
\newcommand{\pr}{\mt{^\prime} } 		   % prime (i.e. R')
\newcommand{\uw}[2]{#1\mt{_{#2}}}
\newcommand{\uf}[2]{#1\mt{^{#2}}}
\newcommand{\frc}[2]{\mt{\frac{#1}{#2}}}
\newcommand{\lmti}[1]{\mt{\displaystyle{\lim_{#1 \to \infty}}}}
\newcommand{\limt}[2]{\mt{\displaystyle{\lim_{#1 \to #2}}}}

\newcommand{\bnm}[2]{\mt{#1\setminus{#2}}}
\newcommand{\bnt}[2]{\mt{\textrm{#1}\setminus{\textrm{#2}}}}
\newcommand{\bi}{\bnm{\mathbb{R}}{\mathbb{Q}}}

\newcommand{\urng}[2]{\mt{\bigcup_{#1}^{#2}}}
\newcommand{\nrng}[2]{\mt{\bigcap_{#1}^{#2}}}
\newcommand{\nck}[2]{\mt{{#1 \choose #2}}}

\newcommand{\nbho}[3]{\textrm{N(}#1, #2\textrm{) }\cap \textrm{ #3} \neq \emptyset}
     							   %  N(x, eps) intersect S \= emptyset
\newcommand{\nbhe}[3]{\textrm{N(}#1, #2\textrm{) }\cap \textrm{ #3} = \emptyset}
     							   %  N(x, eps) intersect S  = emptyset
\newcommand{\dnbho}[3]{\textrm{N*(}#1, #2\textrm{) }\cap \textrm{ #3} \neq \emptyset}
     							   %  N*(x, eps) intersect S \= emptyset
\newcommand{\dnbhe}[3]{\textrm{N*(}#1, #2\textrm{) }\cap \textrm{ #3} = \emptyset}
     							   %  N*(x, eps) intersect S = emptyset
     							 
% ----------

\begin{document}

Homework Due 10/5/17: (7 problems) Section 4.1 pages 169 - 170; 1, 6(b), 7(f), 9(a), 11, 12, 15

Homework Due 10/12/17: (13 problems) Section 4.2 pages 177 - 178; 1, 2, 4, 5(a)(c)(e)(g)(i)(k), 9, 10, 17, 18

\bsc{Theorem 4.1.13}{

Every convergent sequence is bounded.

\bgpf

\as{\uw{S}{n} \lra S as r \lra $\infty$}

Then for \ep \eql 1, \exs N(\ep) \mem \bn st, \fa n $\geq$ N,

$|$\uw{S}{n}$|$ $-$ $|$S$|$ $\leq$ $|$ $|$\uw{S}{n}$|$ $-$ $|$S$|$ $|$ $\leq$ $|$\uw{S}{n} $-$ S$|$ $<$ 1  (page 121, Ex 61(a))

\sidenote{
x $\leq$ $|$x$|$ \fa x \mem \br
}

So $|$\uw{S}{n}$|$ $<$ 1 $+$ $|$S$|$, \fa n $\geq$ N

$|$\uw{S}{n}$|$ $\leq$ $|$ $|$\uw{S}{n} $-$ S $+$ S$|$ $\leq$ $|$\uw{S}{n} $-$ S$|$ $+$ $|$S$|$ $<$ 1 $+$ $|$S$|$ \fa n $\geq$ N

Then,

\lt{m \eql max \{$|$\uw{S}{1}$|$, $|$\uw{S}{2}$|$, ... $|$\uw{S}{N - 1}$|$, $|$S$|$}\}

Then $|$\uw{S}{n}$|$ $\leq$ m, \fa n \mem \bn

Hence, \{\uw{S}{n}\} is bounded.

\epf

}

\bsc{Theorem 4.1.14}{

If a sequence converges, then its limit is unique.

\bgpf

\as{\{\uw{S}{n}\} is a sequence and \uw{S}{n} \lra S as n \lra $\infty$ and \uw{S}{n} \lra t as n \lra $\infty$}

Then \fa \ep $>$ 0, \exs N,(\ep) st

$|$\uw{S}{n} $-$ S$|$ $<$ \frc{\epsilon}{2}, \fa n $\geq$ N \bpth{1}

Also,

\exs \uw{N}{2}(\ep) \mem \bn st

$|$\uw{S}{n} $-$ t$|$ $<$ \frc{\ep}{2}, \fa n $\geq$ \uw{N}{2} \bpth{2}

Set N \eql max \{\uw{N}{1}, \uw{N}{2}\}

From \bpth{1}, \bpth{2}

$|$S $-$ t$|$ \eql $|$(S $-$ \uw{S}{n}) $+$ (\uw{S}{n} $-$ t)$|$ $\leq$ $|$S $-$ \uw{S}{n}$|$ $+$ $|$\uw{S}{n} $-$ t$|$ $<$ \frc{\epsilon}{2} $+$ \frc{\epsilon}{2} \eql \ep, \fa n $\geq$ N

Hence,

s \eql t

\epf

}

\bsc{4.2 Limit Theorems}{

\ssc{Theorem 4.2.1}{

Suppose that \{\uw{S}{n}\} and \{\uw{t}{n}\} are convergent sequences with \lmti{n} \uw{S}{n} \eql S and \lmti{n} \uw{t}{n} \eql t.

Then,

\balist
\item \lmti{n} (\uw{S}{n} $+$ \uw{t}{n}) \eql s $+$ t
\item \lmti{n} k\uw{S}{n} \eql ks and \lmti{n}(k $+$ \uw{S}{n}) \eql k $+$ s, for any k \mem \br
\item \lmti{n}(\uw{S}{n}\uw{t}{n}) \eql st
\item \lmti{n}(\frc{\uw{S}{n}}{\uw{t}{n}}) \eql \frc{s}{t}, provided that \uw{t}{n} $\neq$ 0 \fa n \mem \bn and t $\neq$ 0
\elist

\bgpf

\bpth{a}

$|$t $+$ s $-$ (\uw{s}{n} $+$ \uw{t}{n})$|$ \eql 

$|$(t $-$ \uw{t}{n}) $+$ (s $-$ \uw{s}{n})$|$ $\leq$ $|$t $-$ \uw{t}{n}$|$ $+$ $|$s - \uw{s}{n}$|$ \bpth{1}

\fa \ep $>$ 0, \exs \uw{N}{1}(\ep), \uw{N}{2}(\ep) st

\step{|t - \uw{t}{n}| < \frac{\epsilon}{2} \textrm{  } \fa n \geq \uw{N}{1}}{2}
and
\step{|S - \uw{S}{n}| < \frac{\epsilon}{2}, \textrm{  } \fa n \geq \uw{N}{2}}{3}

\lt{N \eql max \{\uw{N}{1}, \uw{N}{2}\}}

From \bpth{1} - \bpth{3},

$|$S $+$ t - (\uw{s}{n} $+$ \uw{t}{n})$|$ $<$ \frc{\ep}{2} $+$ \frc{\ep}{2} \eql \ep \fa n $\geq$ N

Hence, result. \

\

\bpth{c}

$|$st $-$ \uw{s}{n}\uw{t}{n}$|$ \eql $|$(st $-$ \uw{s}{n}t) $+$ (\uw{s}{n}t $-$ \uw{s}{n}\uw{t}{n})$|$ $\leq$ $|$st $-$ \uw{s}{n}t$|$ $+$ $|$\uw{s}{n}t $-$ \uw{s}{n}\uw{t}{n}$|$ \eql $|$s $-$ \uw{s}{n}$|$$|$t$|$ $+$ $|$\uw{s}{n}$|$$|$t $-$ \uw{t}{n}$|$

By theorem 4.1.3, \exs m $>$ 0 st $|$\uw{s}{n}$|$ $\leq$ \uw{M}{1} \fa n \mem \bn

So,

$|$st $-$ \uw{s}{n}\uw{t}{n}$|$ $\leq$ $|$t$|$$|$s $-$ \uw{s}{n}$|$ $+$ M$|$t $-$ \uw{t}{n}$|$

\lt{\ep $>$ 0}

Then \exs \uw{N}{1}(\ep), \uw{N}{2}(\ep) \mem \bn st

\fa $|$s - \uw{s}{n}$|$ $<$ $\frac{|t|\epsilon}{|t| + M}$, \fa n $\geq$ \uw{N}{1}

and

M $|$t - \uw{t}{n}$|$ $<$ $\frac{M\epsilon}{|t| + M}$, \fa n $\geq$ \uw{N}{2}

Set N \eql max \{\uw{N}{1}, \uw{N}{2}\}

Then

$|$st $-$ \uw{s}{n}\uw{t}{n}$|$ $<$ $|$t$|$ $\frac{\epsilon}{|t| + M}$ $+$ $\frac{M \ep}{|t| + M}$ \eql \ep ($\frac{|t| + M}{|t| + M}$) \eql \ep \fa n $\geq$ N

Hence, result. \

\

\bpth{d}

Since $\frac{\uw{s}{n}}{\uw{t}{n}}$ \eql (\frc{1}{\uw{t}{n}})(\uw{s}{n}), the proof follows from \bpth{c} if we can prove that \lmti{n} \frc{1}{\uw{t}{n}} \eql \frc{1}{t}

Now:

$|$\frc{1}{t} $+$ \frc{1}{\uw{t}{n}}$|$ \eql $|$\frc{\uw{t}{n} - t}{t\uw{t}{n}}$|$ \eql \frc{|\uw{t}{n} - t|}{|t||\uw{t}{n}|} \bpth{1}

\sidenote{
$|$\uw{t}{n}$|$ $>$ 1

$|$\uw{t}{n}$|$ $\geq$ M
}

Recall that:

$|$t$|$ $-$ $|$\uw{t}{n}$|$ $\leq$ $|$ $|$t$|$ $-$ $|$\uw{t}{n}$|$ $|$ $\leq$ $|$t $-$ \uw{t}{n}$|$ from page 121, example 6(a)

For \ep \eql $|$t$|$ $>$ 0, \exs \uw{N}{1}(\ep) \mem \bn st

$|$t $-$ \uw{t}{n}$|$ $<$ \frc{|t|}{2}, \fa n $\geq$ \uw{N}{1}

Now $|$t$|$ $-$ $|$\uw{t}{n}$|$ $\leq$ $|$ $|$t$|$ $-$ $|$\uw{t}{n}$|$ $|$ $\leq$ $|$t $-$ \uw{t}{n}$|$ $<$ $\frac{|t|}{2}$ \fa n $\geq$ \uw{N}{1}

So $|$\uw{t}{n}$|$ $>$ $\frac{|t|}{2}$

Equivalently, $\frac{|t - \uw{t}{n}|}{|t||\uw{t}{n}|}$ $<$ $\frac{2|t - \uw{t}{n}|}{|t||t|}$ \bpth{2}

From \bpth{1} and \bpth{2}

$|$\frc{1}{n} $-$ \frc{1}{\uw{t}{n}}$|$ $<$ $\frac{2|\uw{t}{n} - t|}{|t|^2}$, \fa n $\geq$ \uw{N}{1} \bpth{3}

Also,

\exs \uw{N}{2}(\ep) \mem \bn st

$|$\uw{t}{n} $-$ t$|$ $<$ $\frac{\epsilon|t|^2}{2}$, \fa n $\geq$ \uw{N}{2} \bpth{4}

\lt{N \eql max \{\uw{N}{1}, \uw{N}{2}\}}

Then from \bpth{3} and \bpth{4},

$|$\frc{1}{t} $-$ \frc{1}{\uw{t}{n}}$|$ $<$ $\frac{2}{|t|^2}$ $\frac{\epsilon|t|^2}{2}$ \eql \ep \fa n $\geq$ N

Hence, result.

\epf

}

\bsc{Example 4.2.2}{

Find

\lmti{n}$\frac{(4n^2 - 3)}{(5n^2 - 2n)}$

\eql

\lmti{n} $\frac{n^{2}(4 - \frac{3}{n^2}}{n^{2}(5 - \frac{2}{n})}$

Now, 

\lmti{n} \frc{3}{n^2} \eql 0 \eql \lmti{n} \frc{2}{n}

By Theorem 4.2.1, \bpth{b} 

\lmti{n}(4 $-$ \frc{3}{n^2}) \eql 4

and

\lmti{n} (5 $-$ \frc{2}{n}) \eql 5

By Theorem 4.2.11 (d), 

\lmti{n} $\frac{(4n^2 - 3)}{(5n^2 - 2n)}$ \eql \frc{4}{5}

}

}

\bsc{Theorem 4.2.4}{

Assume that

\lmti{n} \uw{s}{n} \eql s

and

\lmti{n} \uw{t}{n} \eql t

If \uw{s}{n} $\leq$ \uw{t}{n} \fa n \mem \bn

then s $\leq$ t

\bgpf

Assume s $>$ t

Then s $-$ t \eql 0

\exs \uw{N}{1}(s $-$ t), \uw{N}{2}(s $-$ t) \mem \bn st

$|$s $-$ \uw{s}{n}$|$

$|$\uw{s}{n} $-$ s$|$ $<$ \frc{s - t}{2}, \fa n $\geq$ \uw{N}{1} \bpth{1}

and

$|$\uw{t}{n} $-$ t$|$ $<$ \frc{s - t}{2}, \fa n $\geq$ \uw{N}{2} \bpth{2}

\lt{N \eql max \{\uw{N}{1}, \uw{N}{2}\}}

From \bpth{1}

From \bpth{2}


\epf

}


\end{document}