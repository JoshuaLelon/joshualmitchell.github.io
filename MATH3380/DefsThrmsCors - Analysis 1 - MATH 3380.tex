% Thank you Josh Davis for this template!
% https://github.com/jdavis/latex-homework-template/blob/master/homework.tex

\documentclass{article}

\newcommand{\hmwkTitle}{Defs/Thrms/Cors List}

% % ----------

% Packages

\usepackage{fancyhdr}
\usepackage{extramarks}
\usepackage{amsmath}
\usepackage{amssymb}
\usepackage{amsthm}
\usepackage{amsfonts}
\usepackage{tikz}
\usepackage[plain]{algorithm}
\usepackage{algpseudocode}
\usepackage{enumitem}
\usepackage{chngcntr}

% Libraries

\usetikzlibrary{automata, positioning, arrows}

%
% Basic Document Settings
%

\topmargin=-0.45in
\evensidemargin=0in
\oddsidemargin=0in
\textwidth=6.5in
\textheight=9.0in
\headsep=0.25in

\linespread{1.1}

\pagestyle{fancy}
\lhead{\hmwkAuthorName}
\chead{}
\rhead{\hmwkClass\ (\hmwkClassInstructor): \hmwkTitle}
\lfoot{\lastxmark}
\cfoot{\thepage}

\renewcommand\headrulewidth{0.4pt}
\renewcommand\footrulewidth{0.4pt}

\setlength\parindent{0pt}
\setcounter{secnumdepth}{0}

\newcommand{\hmwkClass}{MATH 3380 / Analysis 1}        % Class
\newcommand{\hmwkClassInstructor}{Dr. Welsh}           % Instructor
\newcommand{\hmwkAuthorName}{\textbf{Joshua Mitchell}} % Author

%
% Title Page
%

\title{
    \vspace{2in}
    \textmd{\textbf{\hmwkClass:\ \hmwkTitle}}\\
    \normalsize\vspace{0.1in}\small\vspace{0.1in}\large{\textit{\hmwkClassInstructor}}
    \vspace{3in}
}

\author{\hmwkAuthorName}
\date{}

\renewcommand{\part}[1]{\textbf{\large Part \Alph{partCounter}}\stepcounter{partCounter}\\}

% Integral dx
\newcommand{\dx}{\mathrm{d}x}

%
% Various Helper Commands
%

% For derivatives
\newcommand{\deriv}[1]{\frac{\mathrm{d}}{\mathrm{d}x} (#1)}

% For partial derivatives
\newcommand{\pderiv}[2]{\frac{\partial}{\partial #1} (#2)}


% Alias for the Solution section header
\newcommand{\solution}{\textbf{\large Solution}}

% Probability commands: Expectation, Variance, Covariance, Bias
\newcommand{\E}{\mathrm{E}}
\newcommand{\Var}{\mathrm{Var}}
\newcommand{\Cov}{\mathrm{Cov}}
\newcommand{\Bias}{\mathrm{Bias}}

% Formatting commands:

\newcommand{\mt}[1]{\ensuremath{#1}}
\newcommand{\nm}[1]{\textrm{#1}}

\newcommand\bsc[2][\DefaultOpt]{%
  \def\DefaultOpt{#2}%
  \section[#1]{#2}%
}
\newcommand\ssc[2][\DefaultOpt]{%
  \def\DefaultOpt{#2}%
  \subsection[#1]{#2}%
}
\newcommand{\bgpf}{\begin{proof} $ $\newline}

\newcommand{\bgeq}{\begin{equation*}}
\newcommand{\eeq}{\end{equation*}}	

\newcommand{\balist}{\begin{enumerate}[label=\alph*.]}
\newcommand{\elist}{\end{enumerate}}

\newcommand{\bilist}{\begin{enumerate}[label=\roman*)]}	

\newcommand{\bgsp}{\begin{split}}
% \newcommand{\esp}{\end{split}} % doesn't work for some reason.

\newcommand\prs[1]{~~~\textbf{(#1)}}

\newcommand{\lt}[1]{\textbf{Let: } #1}
     							   %  if you're setting it to be true
\newcommand{\supp}[1]{\textbf{Suppose: } #1}
     							   %  Suppose (if it'll end up false)
\newcommand{\wts}[1]{\textbf{Want to show: } #1}
     							   %  Want to show
\newcommand{\as}[1]{\textbf{Assume: } #1}
     							   %  if you think it follows from truth
\newcommand{\bpth}[1]{\textbf{(#1)}}

\newcommand{\step}[2]{\begin{equation}\tag{#2}#1\end{equation}}
\newcommand{\epf}{\end{proof}}

\newcommand{\dbs}[3]{\mt{#1_{#2_#3}}}

\newcommand{\sidenote}[1]{-----------------------------------------------------------------Side Note----------------------------------------------------------------
#1 \

---------------------------------------------------------------------------------------------------------------------------------------------}

% Analysis / Logical commands:

\newcommand{\br}{\mt{\mathbb{R}} }       % |R
\newcommand{\bq}{\mt{\mathbb{Q}} }       % |Q
\newcommand{\bn}{\mt{\mathbb{N}} }       % |N
\newcommand{\bc}{\mt{\mathbb{C}} }       % |C
\newcommand{\bz}{\mt{\mathbb{Z}} }       % |Z

\newcommand{\ep}{\mt{\epsilon} }         % epsilon
\newcommand{\fa}{\mt{\forall} }          % for all
\newcommand{\afa}{\mt{\alpha} }
\newcommand{\bta}{\mt{\beta} }
\newcommand{\mem}{\mt{\in} }
\newcommand{\exs}{\mt{\exists} }

\newcommand{\es}{\mt{\emptyset}}        % empty set
\newcommand{\sbs}{\mt{\subset} }         % subset of
\newcommand{\fs}[2]{\{\uw{#1}{1}, \uw{#1}{2}, ... \uw{#1}{#2}\}}

\newcommand{\lra}{ \mt{\longrightarrow} } % implies ----->
\newcommand{\rar}{ \mt{\Rightarrow} }     % implies -->

\newcommand{\lla}{ \mt{\longleftarrow} }  % implies <-----
\newcommand{\lar}{ \mt{\Leftarrow} }      % implies <--

\newcommand{\eql}{\mt{=} }
\newcommand{\pr}{\mt{^\prime} } 		   % prime (i.e. R')
\newcommand{\uw}[2]{#1\mt{_{#2}}}
\newcommand{\frc}[2]{\mt{\frac{#1}{#2}}}

\newcommand{\bnm}[2]{\mt{#1\setminus{#2}}}
\newcommand{\bnt}[2]{\mt{\textrm{#1}\setminus{\textrm{#2}}}}
\newcommand{\bi}{\bnm{\mathbb{R}}{\mathbb{Q}}}

\newcommand{\urng}[2]{\mt{\bigcup_{#1}^{#2}}}
\newcommand{\nrng}[2]{\mt{\bigcap_{#1}^{#2}}}

\newcommand{\nbho}[3]{\textrm{N(}#1, #2\textrm{) }\cap \textrm{ #3} \neq \emptyset}
     							   %  N(x, eps) intersect S \= emptyset
\newcommand{\nbhe}[3]{\textrm{N(}#1, #2\textrm{) }\cap \textrm{ #3} = \emptyset}
     							   %  N(x, eps) intersect S  = emptyset
\newcommand{\dnbho}[3]{\textrm{N*(}#1, #2\textrm{) }\cap \textrm{ #3} \neq \emptyset}
     							   %  N*(x, eps) intersect S \= emptyset
\newcommand{\dnbhe}[3]{\textrm{N*(}#1, #2\textrm{) }\cap \textrm{ #3} = \emptyset}
     							   %  N*(x, eps) intersect S = emptyset
     							 


% ----------

% ----------

% Packages

\usepackage{fancyhdr}
\usepackage{extramarks}
\usepackage{amsmath}
\usepackage{amssymb}
\usepackage{amsthm}
\usepackage{amsfonts}
\usepackage{tikz}
\usepackage[plain]{algorithm}
\usepackage{algpseudocode}
\usepackage{enumitem}
\usepackage{chngcntr}

% Libraries

\usetikzlibrary{automata, positioning, arrows}

%
% Basic Document Settings
%

\topmargin=-0.45in
\evensidemargin=0in
\oddsidemargin=0in
\textwidth=6.5in
\textheight=9.0in
\headsep=0.25in

\linespread{1.1}

\pagestyle{fancy}
\lhead{\hmwkAuthorName}
\chead{}
\rhead{\hmwkClass\ (\hmwkClassInstructor): \hmwkTitle}
\lfoot{\lastxmark}
\cfoot{\thepage}

\renewcommand\headrulewidth{0.4pt}
\renewcommand\footrulewidth{0.4pt}

\setlength\parindent{0pt}
\setcounter{secnumdepth}{0}

\newcommand{\hmwkClass}{MATH 3380 / Analysis 1}        % Class
\newcommand{\hmwkClassInstructor}{Dr. Welsh}           % Instructor
\newcommand{\hmwkAuthorName}{\textbf{Joshua Mitchell}} % Author

%
% Title Page
%

\title{
    \vspace{2in}
    \textmd{\textbf{\hmwkClass:\ \hmwkTitle}}\\
    \normalsize\vspace{0.1in}\small\vspace{0.1in}\large{\textit{\hmwkClassInstructor}}
    \vspace{3in}
}

\author{\hmwkAuthorName}
\date{}

\renewcommand{\part}[1]{\textbf{\large Part \Alph{partCounter}}\stepcounter{partCounter}\\}

% Integral dx
\newcommand{\dx}{\mathrm{d}x}

%
% Various Helper Commands
%

% For derivatives
\newcommand{\deriv}[1]{\frac{\mathrm{d}}{\mathrm{d}x} (#1)}

% For partial derivatives
\newcommand{\pderiv}[2]{\frac{\partial}{\partial #1} (#2)}


% Alias for the Solution section header
\newcommand{\solution}{\textbf{\large Solution}}

% Probability commands: Expectation, Variance, Covariance, Bias
\newcommand{\E}{\mathrm{E}}
\newcommand{\Var}{\mathrm{Var}}
\newcommand{\Cov}{\mathrm{Cov}}
\newcommand{\Bias}{\mathrm{Bias}}

% Formatting commands:

\newcommand{\mt}[1]{\ensuremath{#1}}
\newcommand{\nm}[1]{\textrm{#1}}

\newcommand\bsc[2][\DefaultOpt]{%
  \def\DefaultOpt{#2}%
  \section[#1]{#2}%
}
\newcommand\ssc[2][\DefaultOpt]{%
  \def\DefaultOpt{#2}%
  \subsection[#1]{#2}%
}
\newcommand{\bgpf}{\begin{proof} $ $\newline}

\newcommand{\bgeq}{\begin{equation*}}
\newcommand{\eeq}{\end{equation*}}	

\newcommand{\balist}{\begin{enumerate}[label=\alph*.]}
\newcommand{\elist}{\end{enumerate}}

\newcommand{\bilist}{\begin{enumerate}[label=\roman*)]}	

\newcommand{\bgsp}{\begin{split}}
% \newcommand{\esp}{\end{split}} % doesn't work for some reason.

\newcommand\prs[1]{~~~\textbf{(#1)}}

\newcommand{\lt}[1]{\textbf{Let: } #1}
     							   %  if you're setting it to be true
\newcommand{\supp}[1]{\textbf{Suppose: } #1}
     							   %  Suppose (if it'll end up false)
\newcommand{\wts}[1]{\textbf{Want to show: } #1}
     							   %  Want to show
\newcommand{\as}[1]{\textbf{Assume: } #1}
     							   %  if you think it follows from truth
\newcommand{\bpth}[1]{\textbf{(#1)}}

\newcommand{\step}[2]{\begin{equation}\tag{#2}#1\end{equation}}
\newcommand{\epf}{\end{proof}}

\newcommand{\dbs}[3]{\mt{#1_{#2_#3}}}

\newcommand{\sidenote}[1]{-----------------------------------------------------------------Side Note----------------------------------------------------------------
#1 \

---------------------------------------------------------------------------------------------------------------------------------------------}

% Analysis / Logical commands:

\newcommand{\br}{\mt{\mathbb{R}} }       % |R
\newcommand{\bq}{\mt{\mathbb{Q}} }       % |Q
\newcommand{\bn}{\mt{\mathbb{N}} }       % |N
\newcommand{\bc}{\mt{\mathbb{C}} }       % |C
\newcommand{\bz}{\mt{\mathbb{Z}} }       % |Z

\newcommand{\ep}{\mt{\epsilon} }         % epsilon
\newcommand{\fa}{\mt{\forall} }          % for all
\newcommand{\afa}{\mt{\alpha} }
\newcommand{\bta}{\mt{\beta} }
\newcommand{\mem}{\mt{\in} }
\newcommand{\exs}{\mt{\exists} }

\newcommand{\es}{\mt{\emptyset} }        % empty set
\newcommand{\sbs}{\mt{\subset} }         % subset of
\newcommand{\fs}[2]{\{\uw{#1}{1}, \uw{#1}{2}, ... \uw{#1}{#2}\}}

\newcommand{\lra}{ \mt{\longrightarrow} } % implies ----->
\newcommand{\rar}{ \mt{\Rightarrow} }     % implies -->

\newcommand{\lla}{ \mt{\longleftarrow} }  % implies <-----
\newcommand{\lar}{ \mt{\Leftarrow} }      % implies <--

\newcommand{\eql}{\mt{=} }
\newcommand{\pr}{\mt{^\prime} } 		   % prime (i.e. R')
\newcommand{\uw}[2]{#1\mt{_{#2}}}
\newcommand{\frc}[2]{\mt{\frac{#1}{#2}}}

\newcommand{\bnm}[2]{\mt{#1\setminus{#2}}}
\newcommand{\bnt}[2]{\mt{\textrm{#1}\setminus{\textrm{#2}}}}
\newcommand{\bi}{\bnm{\mathbb{R}}{\mathbb{Q}}}

\newcommand{\urng}[2]{\mt{\bigcup_{#1}^{#2}}}
\newcommand{\nrng}[2]{\mt{\bigcap_{#1}^{#2}}}

\newcommand{\nbho}[3]{\textrm{N(}#1, #2\textrm{) }\cap \textrm{ #3} \neq \emptyset}
     							   %  N(x, eps) intersect S \= emptyset
\newcommand{\nbhe}[3]{\textrm{N(}#1, #2\textrm{) }\cap \textrm{ #3} = \emptyset}
     							   %  N(x, eps) intersect S  = emptyset
\newcommand{\dnbho}[3]{\textrm{N*(}#1, #2\textrm{) }\cap \textrm{ #3} \neq \emptyset}
     							   %  N*(x, eps) intersect S \= emptyset
\newcommand{\dnbhe}[3]{\textrm{N*(}#1, #2\textrm{) }\cap \textrm{ #3} = \emptyset}
     							   %  N*(x, eps) intersect S = emptyset
     							
% ----------

\begin{document}

\bsc{Lecture 1}{

\ssc{Theorem 3.2.8 - pg 118}{
Let x, y $\in$ $\mathbb{R}$

\balist
\item If x $\leq$ y + $\ep$ $\fa$ $\ep$ $>$ $0$, then x $\leq$ y
\item If $|x - y|$ $\leq$ $\ep$ $\fa$ $\ep$ $>$ $0$, then $|x - y|$ $=$ $0$ or, evidently, $x = y$
\elist

}

\ssc{Definition 3.2.9} {

If x $\in \br$,

\bgeq
  |x|=\begin{cases}
    x, & \text{if $x \geq 0$}.\\
    -x, & \text{if $x < 0$}.
  \end{cases}
\eeq
}

\ssc{Theorem 3.2.10} {
Let x, y $\in \br$ and a $\geq 0$ \\
\\
Then
\balist
\item $|x| \geq 0$
\item $|x| \leq a$ iff $-a \leq x \leq a$
\item $|xy| = |x||y|$
\item $|x + y| \leq |x| + |y|$ (equality holds only if signs are the same)
\elist
}

\newpage

\bsc{Lecture 2}{

\ssc{Theorem 3.3: The Completeness Axiom}{
Recall the Fundamental Theorem of Arithmetic: \

if n $\in \bn$ with n $\geq$ 2, then n may be expressed as the product of prime numbers (the prime factorization (PF)). \

The PF is unique with respect to (WRT) order. \

Ex: $12 = 2 * 2 * 2 * 3$
}

\ssc{Theorem 3.3.1}{
\lt{p be a prime number} \

Then $\sqrt{p} \in $ \bnt{\br}{\bq}
}

\ssc{Definition 3.3.7}{
Let S $\sbs \br$. If $\exists$ m $\in \br$ st s $\leq$ m $\fa$ s $\in S$, \

then m is an upper bound of S and we say that S is \textbf{bounded above}. \

Similarly, we can define \textbf{bounded below}. \

If S is bounded above and below, then S is said to be \textbf{bounded}.

S can be open or closed. The example below is closed.

\

\begin{tikzpicture}

\draw[latex-latex] (-3.5,0) -- (3.5,0) ; %edit here for the axis
\foreach \x in  {-3,-2,-1,0,1,2,3} % edit here for the vertical lines
\draw[shift={(\x,0)},color=black] (0pt,3pt) -- (0pt,-3pt);
\foreach \x in {,,,,,,} % edit here for the numbers
\draw[shift={(\x,0)},color=black] (0pt,0pt) -- (0pt,-3pt) node[below] 
{$\x$};
\draw[*-*] (0.92,0) -- (2.08,0);
\draw[very thick] (0.92,0) -- (2.08,0);

\node [below] at (1.5,-0.3) {S};
\node [below] at (0, -0.2) {n};
\node [below] at (3, -0.2) {m};

\end{tikzpicture}

If an upper bound m of S is a member of S, then m is called the maximum (or largest element) of S, and we say that m = \textbf{max S}. Similarly, we may decline \textbf{minimum} of S \textbf{(n = min S)}.

}

\ssc{Theorem 1}{
If a set S $\sbs \br$ possesses a max element, then it is unique. A similar result holds for a minimum element.
}

\ssc{Definition 3.3.5 (supremum defined)}{
Let \es $\neq$ S \sbs \br if S is bounded above, \

then the \textbf{least upper bound} of S is called the \textbf{supremum} of S, denoted by sup S \mem \br \

iff: \

\balist
\item s $\leq$ sup S \fa s \mem S
\item \exs s' \mem S st sup S $-$ \ep $<$ s' \fa \ep $>$ 0
\elist

}

\ssc{Axiom of Completeness of the set of Real Numbers: \br}{
Every \es $\neq$ S \sbs \br that is bounded above has a least upper bound (i.e. sup S \mem \br exists). \

A similar statement can be made about inf S.
}

Remark: In practice 3.3.4, the set T \eql \{q \mem \bq : $0 \leq q \leq \sqrt{2}$\} is bounded.

\

\begin{tikzpicture}

\draw[latex-latex] (-3.5,0) -- (3.5,0) ; %edit here for the axis
\foreach \x in  {-3,-2,-1,0,1,2,3} % edit here for the vertical lines
\draw[shift={(\x,0)},color=black] (0pt,3pt) -- (0pt,-3pt);
\foreach \x in {,,,,,,} % edit here for the numbers
\draw[shift={(\x,0)},color=black] (0pt,0pt) -- (0pt,-3pt) node[below] 
{$\x$};
\draw[*-o] (0.92,0) -- (2.08,0);
\draw[very thick] (0.92,0) -- (1.92,0);

\node [below] at (1.5,-0.3) {T};
\node [below] at (1, -0.2) {$0$};
\node [below] at (2, -0.12) {$\sqrt{2}$};

\end{tikzpicture}

But $\sqrt{2}$ is not rational, so the set wouldn't have a least upper bound.

We need to fill in the gaps to make analysis work.
}
\newpage

\bsc{Lecture 3}{

\ssc{Theorem 1 (infinum definition)}{

\lt{\es $\neq$ S \sbs \br, S is bounded below.}

Then S possesses a greatest lower bound denoted by \textbf{inf S} (the \textbf{infinum} of S), where inf S \mem \br, satisfying:

\bilist
\item inf S $\leq$ s \fa s \mem S
\item \fa \ep $>$ 0, \exs \uw{s}{1}  st  inf S $+$ \ep $>$ \uw{s}{1}
\elist
}

\ssc{Theorem 3.3.7}{

Given nonempty subsets of A, B (A, B \sbs \br),

\lt{C \eql \{x $+$ y: x \mem A, y \mem B\}}

If A and B have suprema, then C has a supremum: sup C \eql sup A $+$ sup B
}

\ssc{Theorem 3.3.8}{

Suppose \es $\neq$ D \sbs \br and 

f : D \lra \br

g : D \lra \br

f(D) \eql \{f(x) : x \mem D\}

If \fa x, y \mem D, f(x) $\leq$ g(y), then
 
f(D) is bounded above and g(D) is bounded below.

Furthermore, sup(f(D)) $\leq$ inf(g(D))
}

\ssc{Theorem 3.3.9: Archimedian Property / Principle of \br (AP)}{

The set \bn \eql \{1, 2, 3...\} is unbounded above in \br
}

}

\newpage

\bsc{Lecture 4}{

\ssc{Theorem 3.3.10}{

Each of the following is equivalent to the AP:

\balist
\item \fa z \mem \br, \exs n \mem \bn st n $>$ z
\item \fa x $>$ 0, y \mem \br, \exs n \mem \bn st nx $>$ y
\item \fa x $>$ 0, \exs n \mem \bn st 0 $<$ \frc{1}{n} $<$ x
\elist
}

\ssc{Theorems 3.3.13 and 3.3.15}{

\lt{x, y \mem \br st x $<$ y}

Then:

\balist
\item \exs r \mem \bq st x $<$ r $<$ y
\item \exs z \mem \bnt{\br}{\bq} st x $<$ z $<$ y
\elist

}

\ssc{Section 3.4: Topology of \br}{}

\ssc{Definitions 3.4.1 and 3.4.2}{

Let x \mem \br and \ep $>$ 0. \

\

\bpth{a} An \ep-neighborhood of x is: N(x, \ep) \eql \{y \mem \br: $|y - x|$ $<$ \ep\}

\

\bpth{b} A deleted \ep-neighborhood of x is: N*(x, \ep) \eql \{y \mem \br: 0 $<$ $|y - x|$ $<$ \ep\} \
}

\ssc{Open Set Topology: Definition 3.4.3 (interior / boundary point)}{

\lt{S \sbs \br}

A point x \mem \br is an \textbf{interior point} of S if \exs \ep $>$ 0 st N(x, \ep) \sbs S. \

\

If, \fa \ep $>$ 0, $\nbho{x}{\ep}{S}$ and $\nbho{x}{\ep}{\bnt{\br}{S	}}$

Then x is a \textbf{boundary point} of S.

The set of all interior points is denoted by \textbf{int S}.

The set of all boundary points is denoted by \textbf{bd S}.

\textbf{Nota Bene (N.B.)}:

int S \sbs S and bd S \eql bd (\bnt{\br}{S})

}

\ssc{Theorem 1}{

\lt{x \mem S \sbs \br}

Then either x \mem int S, or x \mem bd S.

}

}

\newpage

\bsc{Lecture 5}{

\ssc{Definition 3.4.6 - Def of Open/Closed Set}{

\lt{S \sbs \br} \

\

	if bd S \sbs S, then S is closed.
	
	if bd S \sbs (\bnt{\br}{S}), then S is open.

}


\ssc{Theorem 3.4.7}{

\balist
\item A set S is open iff S \eql int S; i.e. iff \fa s \mem S, s is an \textbf{interior point}.
\item A set S is closed iff its compliment, \bnt{\br}{S} is open.

	Equivalently, a set s is open iff \bnt{\br}{S} is closed.
	
\elist
}

\ssc{Theorem 2 (not in book)}{

\lt{x \mem \br, \ep $>$ 0}

Then N(x, \ep), N*(x, \ep) are open sets.
}

\ssc{Theorem 3.4.10}{

\lt{I be an index set. I \sbs \bn \sbs \br}

\supp{\uw{G}{\afa} \sbs \br is an open set \fa \afa \mem I}

Then, 

\balist
\item \urng{\afa \mem I}{} \uw{G}{\afa} is an open set.
\item If \uw{G}{i} \sbs \br is open \fa i \eql 1, 2, ... n \mem \bn, then \nrng{i = 1}{n} \uw{G}{i} is open.
\elist

}

\ssc{Corollary 3.4.11}{

\balist
\item Let \uw{F}{\afa} be closed \fa \afa \mem I, I is an index set.

	Then \nrng{\afa \mem I}{} \uw{F}{\afa} is closed.
\item Let \uw{F}{i} be closed \fa i from 1 to n.
	
	Then ( \urng{i = 1}{n} \uw{F}{i} ) is closed.
\elist
}

\ssc{Accumulation (or Limit) Points; Definition 3.4.14}{

\lt{S \sbs \br}

If \fa \ep $>$ 0, $\dnbho{x}{\ep}{S}$,

Then x \mem \br is an \textbf{accumulation} or \textbf{limit} point. (The set of all accumulation points of S is denoted by \textbf{S\pr})

If x \mem \bnt{S}{S\pr},

then x is an \textbf{isolated point},

in which case, \exs \ep $>$ 0 st N(x, \ep) $\cap$ S \eql \{x\}

}

\ssc{Definition 3.4.16 - Closures}{

\lt{S \sbs \br}

Then the \textbf{closure} of S, denoted by \textbf{cl S}, is defined to be:

\textbf{cl S} \eql S $\cup$ S\pr \
}

}

\newpage

\bsc{Lecture 6}{

\ssc{Theorem 3.4.17 - pg 118}

\lt{$S \sbs \br$} 

Then

\balist
\item S is closed iff S$\pr \sbs$ S
\item cl S is a closed set
\item S is closed iff S = cl S
\item cl S = S U S$\pr$ = S $\cup$ bd S
\elist
}

}

\newpage

\bsc{Lecture 7}{

\ssc{Section 3.5: Compact Sets}{
Three big areas of analysis: compactedness, continuity, and connectedness.
}

\ssc{Definition: Open Cover / Subcover}{

\lt{S \sbs \br}

An \textbf{open cover} of S is a collection C of open sets st S \sbs $\cup$ C. The collection C of open sets is said to \textbf{cover} the set S.

A subset of sets from the collection C that still covers the set is is called a \textbf{subcover} of S.

}

\ssc{Definition 3.5.1}{A set s \sbs \br  is said to be compact if every \textbf{open cover} has a finite \textbf{subcover} \

(i.e. if S \sbs \urng{\alpha \in I}{} \uw{G}{\afa}) , \

where \uw{G}{\afa} is open \fa \afa \mem I; then \exs n \mem \bn and \exs \fs{n}{k} \sbs I \

st S \sbs \urng{i=1}{n} \dbs{G}{\alpha}{i}
}

\ssc{Lemma 3.5.4}{
If $\es \neq S \sbs \br$ and S is \textbf{closed} and \textbf{bounded}, then S has a maximum and a minimum. 

In fact, in this, max S $=$ sup S, and min S $=$ inf S.
}

\ssc{Theorem 3.5.5 (Heine-Borel)}{

A subset $\es \neq$ S $\sbs \br$ is compact iff S is closed and bounded.
}


}

\newpage

\bsc{Lecture 8}{

\ssc{Theorem 3.5.5 (Heine-Borel)}{

A subset $\es \neq$ S $\sbs \br$ is compact iff S is closed and bounded.
}

\ssc{Theorem 3.5.6: Bolzano$-$Weierstrass Theorem}{

If a bounded set S \sbs \br contains an infinite number of points, then \exs at least one point in \br that is an accumulation point of S.
}

}

\ssc{Theorem 3.5.7 (F.I.P.)}{

\lt{\uw{\{\uw{K}{\afa}\}}{\afa \mem I} be a family of compact sets, where I is an index.}

Suppose that the intersection of any finite subfamily of the \uw{K}{\afa}'s has a nonempty intersection.

Then \nrng{\afa \mem I}{} \uw{K}{\afa} $\neq$ \es
}

\ssc{Corollary 3.5.8 Nested Intervals Theorem}{

\lt{\{\uw{A}{n}\}$^\infty_{n = 1}$ be a family of nonempty closed bounded intervals in \br st A$_{n + 1}$ \sbs A$_n$ \fa n \mem \bn}

Then:

\nrng{n = 1}{\infty} \uw{A}{n} $\neq$ \es
}

\end{document}