% Thank you Josh Davis for this template!
% https://github.com/jdavis/latex-homework-template/blob/master/homework.tex

\documentclass{article}

\newcommand{\hmwkTitle}{Lec\ \#21}

% % ----------

% Packages

\usepackage{fancyhdr}
\usepackage{extramarks}
\usepackage{amsmath}
\usepackage{amssymb}
\usepackage{amsthm}
\usepackage{amsfonts}
\usepackage{tikz}
\usepackage[plain]{algorithm}
\usepackage{algpseudocode}
\usepackage{enumitem}
\usepackage{chngcntr}

% Libraries

\usetikzlibrary{automata, positioning, arrows}

%
% Basic Document Settings
%

\topmargin=-0.45in
\evensidemargin=0in
\oddsidemargin=0in
\textwidth=6.5in
\textheight=9.0in
\headsep=0.25in

\linespread{1.1}

\pagestyle{fancy}
\lhead{\hmwkAuthorName}
\chead{}
\rhead{\hmwkClass\ (\hmwkClassInstructor): \hmwkTitle}
\lfoot{\lastxmark}
\cfoot{\thepage}

\renewcommand\headrulewidth{0.4pt}
\renewcommand\footrulewidth{0.4pt}

\setlength\parindent{0pt}
\setcounter{secnumdepth}{0}

\newcommand{\hmwkClass}{MATH 3380 / Analysis 1}        % Class
\newcommand{\hmwkClassInstructor}{Dr. Welsh}           % Instructor
\newcommand{\hmwkAuthorName}{\textbf{Joshua Mitchell}} % Author

%
% Title Page
%

\title{
    \vspace{2in}
    \textmd{\textbf{\hmwkClass:\ \hmwkTitle}}\\
    \normalsize\vspace{0.1in}\small\vspace{0.1in}\large{\textit{\hmwkClassInstructor}}
    \vspace{3in}
}

\author{\hmwkAuthorName}
\date{}

\renewcommand{\part}[1]{\textbf{\large Part \Alph{partCounter}}\stepcounter{partCounter}\\}

% Integral dx
\newcommand{\dx}{\mathrm{d}x}

%
% Various Helper Commands
%

% For derivatives
\newcommand{\deriv}[1]{\frac{\mathrm{d}}{\mathrm{d}x} (#1)}

% For partial derivatives
\newcommand{\pderiv}[2]{\frac{\partial}{\partial #1} (#2)}


% Alias for the Solution section header
\newcommand{\solution}{\textbf{\large Solution}}

% Probability commands: Expectation, Variance, Covariance, Bias
\newcommand{\E}{\mathrm{E}}
\newcommand{\Var}{\mathrm{Var}}
\newcommand{\Cov}{\mathrm{Cov}}
\newcommand{\Bias}{\mathrm{Bias}}

% Formatting commands:

\newcommand{\mt}[1]{\ensuremath{#1}}
\newcommand{\nm}[1]{\textrm{#1}}

\newcommand\bsc[2][\DefaultOpt]{%
  \def\DefaultOpt{#2}%
  \section[#1]{#2}%
}
\newcommand\ssc[2][\DefaultOpt]{%
  \def\DefaultOpt{#2}%
  \subsection[#1]{#2}%
}
\newcommand{\bgpf}{\begin{proof} $ $\newline}

\newcommand{\bgeq}{\begin{equation*}}
\newcommand{\eeq}{\end{equation*}}	

\newcommand{\balist}{\begin{enumerate}[label=\alph*.]}
\newcommand{\elist}{\end{enumerate}}

\newcommand{\bilist}{\begin{enumerate}[label=\roman*)]}	

\newcommand{\bgsp}{\begin{split}}
% \newcommand{\esp}{\end{split}} % doesn't work for some reason.

\newcommand\prs[1]{~~~\textbf{(#1)}}

\newcommand{\lt}[1]{\textbf{Let: } #1}
     							   %  if you're setting it to be true
\newcommand{\supp}[1]{\textbf{Suppose: } #1}
     							   %  Suppose (if it'll end up false)
\newcommand{\wts}[1]{\textbf{Want to show: } #1}
     							   %  Want to show
\newcommand{\as}[1]{\textbf{Assume: } #1}
     							   %  if you think it follows from truth
\newcommand{\bpth}[1]{\textbf{(#1)}}

\newcommand{\step}[2]{\begin{equation}\tag{#2}#1\end{equation}}
\newcommand{\epf}{\end{proof}}

\newcommand{\dbs}[3]{\mt{#1_{#2_#3}}}

\newcommand{\sidenote}[1]{-----------------------------------------------------------------Side Note----------------------------------------------------------------
#1 \

---------------------------------------------------------------------------------------------------------------------------------------------}

% Analysis / Logical commands:

\newcommand{\br}{\mt{\mathbb{R}} }       % |R
\newcommand{\bq}{\mt{\mathbb{Q}} }       % |Q
\newcommand{\bn}{\mt{\mathbb{N}} }       % |N
\newcommand{\bc}{\mt{\mathbb{C}} }       % |C
\newcommand{\bz}{\mt{\mathbb{Z}} }       % |Z

\newcommand{\ep}{\mt{\epsilon} }         % epsilon
\newcommand{\fa}{\mt{\forall} }          % for all
\newcommand{\afa}{\mt{\alpha} }
\newcommand{\bta}{\mt{\beta} }
\newcommand{\mem}{\mt{\in} }
\newcommand{\exs}{\mt{\exists} }

\newcommand{\es}{\mt{\emptyset}}        % empty set
\newcommand{\sbs}{\mt{\subset} }         % subset of
\newcommand{\fs}[2]{\{\uw{#1}{1}, \uw{#1}{2}, ... \uw{#1}{#2}\}}

\newcommand{\lra}{ \mt{\longrightarrow} } % implies ----->
\newcommand{\rar}{ \mt{\Rightarrow} }     % implies -->

\newcommand{\lla}{ \mt{\longleftarrow} }  % implies <-----
\newcommand{\lar}{ \mt{\Leftarrow} }      % implies <--

\newcommand{\eql}{\mt{=} }
\newcommand{\pr}{\mt{^\prime} } 		   % prime (i.e. R')
\newcommand{\uw}[2]{#1\mt{_{#2}}}
\newcommand{\frc}[2]{\mt{\frac{#1}{#2}}}

\newcommand{\bnm}[2]{\mt{#1\setminus{#2}}}
\newcommand{\bnt}[2]{\mt{\textrm{#1}\setminus{\textrm{#2}}}}
\newcommand{\bi}{\bnm{\mathbb{R}}{\mathbb{Q}}}

\newcommand{\urng}[2]{\mt{\bigcup_{#1}^{#2}}}
\newcommand{\nrng}[2]{\mt{\bigcap_{#1}^{#2}}}

\newcommand{\nbho}[3]{\textrm{N(}#1, #2\textrm{) }\cap \textrm{ #3} \neq \emptyset}
     							   %  N(x, eps) intersect S \= emptyset
\newcommand{\nbhe}[3]{\textrm{N(}#1, #2\textrm{) }\cap \textrm{ #3} = \emptyset}
     							   %  N(x, eps) intersect S  = emptyset
\newcommand{\dnbho}[3]{\textrm{N*(}#1, #2\textrm{) }\cap \textrm{ #3} \neq \emptyset}
     							   %  N*(x, eps) intersect S \= emptyset
\newcommand{\dnbhe}[3]{\textrm{N*(}#1, #2\textrm{) }\cap \textrm{ #3} = \emptyset}
     							   %  N*(x, eps) intersect S = emptyset
     							 


% ----------

% ----------

% Packages

\usepackage{fancyhdr}
\usepackage{extramarks}
\usepackage{amsmath}
\usepackage{amssymb}
\usepackage{amsthm}
\usepackage{amsfonts}
\usepackage{tikz}
\usepackage[plain]{algorithm}
\usepackage{algpseudocode}
\usepackage{enumitem}
\usepackage{chngcntr}

% Libraries

\graphicspath{{/Users/jm/iclouddrive/3380pics/}}

\usetikzlibrary{automata, positioning, arrows}

%
% Basic Document Settings
%

\topmargin=-0.45in
\evensidemargin=0in
\oddsidemargin=0in
\textwidth=6.5in
\textheight=9.0in
\headsep=0.25in

\linespread{1.1}

\pagestyle{fancy}
\lhead{\hmwkAuthorName}
\chead{}
\rhead{\hmwkClass\ (\hmwkClassInstructor): \hmwkTitle}
\lfoot{\lastxmark}
\cfoot{\thepage}

\renewcommand\headrulewidth{0.4pt}
\renewcommand\footrulewidth{0.4pt}

\setlength\parindent{0pt}
\setcounter{secnumdepth}{0}

\newcommand{\hmwkClass}{MATH 3380 / Analysis 1}        % Class
\newcommand{\hmwkClassInstructor}{Dr. Welsh}           % Instructor
\newcommand{\hmwkAuthorName}{\textbf{Joshua Mitchell}} % Author

%
% Title Page
%

\title{
    \vspace{2in}
    \textmd{\textbf{\hmwkClass:\ \hmwkTitle}}\\
    \normalsize\vspace{0.1in}\small\vspace{0.1in}\large{\textit{\hmwkClassInstructor}}
    \vspace{3in}
}

\author{\hmwkAuthorName}
\date{}

\renewcommand{\part}[1]{\textbf{\large Part \Alph{partCounter}}\stepcounter{partCounter}\\}

% Integral dx
\newcommand{\dx}{\mathrm{d}x}

%
% Various Helper Commands
%

% For derivatives
\newcommand{\deriv}[1]{\frac{\mathrm{d}}{\mathrm{d}x} (#1)}

% For partial derivatives
\newcommand{\pderiv}[2]{\frac{\partial}{\partial #1} (#2)}


% Alias for the Solution section header
\newcommand{\solution}{\textbf{\large Solution}}

% Probability commands: Expectation, Variance, Covariance, Bias
\newcommand{\E}{\mathrm{E}}
\newcommand{\Var}{\mathrm{Var}}
\newcommand{\Cov}{\mathrm{Cov}}
\newcommand{\Bias}{\mathrm{Bias}}

% Formatting commands:

\newcommand{\mt}[1]{\ensuremath{#1}}
\newcommand{\nm}[1]{\textrm{#1}}

\newcommand\bsc[2][\DefaultOpt]{%
  \def\DefaultOpt{#2}%
  \section[#1]{#2}%
}
\newcommand\ssc[2][\DefaultOpt]{%
  \def\DefaultOpt{#2}%
  \subsection[#1]{#2}%
}
\newcommand{\bgpf}{\begin{proof} $ $\newline}

\newcommand{\bgeq}{\begin{equation*}}
\newcommand{\eeq}{\end{equation*}}	

\newcommand{\balist}{\begin{enumerate}[label=\alph*.]}
\newcommand{\elist}{\end{enumerate}}

\newcommand{\bilist}{\begin{enumerate}[label=\roman*)]}	

\newcommand{\bgsp}{\begin{split}}
% \newcommand{\esp}{\end{split}} % doesn't work for some reason.

\newcommand\prs[1]{~~~\textbf{(#1)}}

\newcommand{\lt}[1]{\textbf{Let: } #1}
     							   %  if you're setting it to be true
\newcommand{\supp}[1]{\textbf{Suppose: } #1}
     							   %  Suppose (if it'll end up false)
\newcommand{\wts}[1]{\textbf{Want to show: } #1}
     							   %  Want to show
\newcommand{\as}[1]{\textbf{Assume: } #1}
     							   %  if you think it follows from truth
\newcommand{\bpth}[1]{\textbf{(#1)}}

\newcommand{\step}[2]{\begin{equation}\tag{#2}#1\end{equation}}
\newcommand{\epf}{\end{proof}}

\newcommand{\dbs}[3]{\mt{#1_{#2_#3}}}

\newcommand{\sidenote}[1]{-----------------------------------------------------------------Side Note----------------------------------------------------------------
#1 \

---------------------------------------------------------------------------------------------------------------------------------------------}

% Analysis / Logical commands:

\newcommand{\br}{\mt{\mathbb{R}} }       % |R
\newcommand{\bq}{\mt{\mathbb{Q}} }       % |Q
\newcommand{\bn}{\mt{\mathbb{N}} }       % |N
\newcommand{\bc}{\mt{\mathbb{C}} }       % |C
\newcommand{\bz}{\mt{\mathbb{Z}} }       % |Z

\newcommand{\ep}{\mt{\epsilon} }         % epsilon
\newcommand{\fa}{\mt{\forall} }          % for all
\newcommand{\afa}{\mt{\alpha} }
\newcommand{\bta}{\mt{\beta} }
\newcommand{\dta}{\mt{\delta} }
\newcommand{\mem}{\mt{\in} }
\newcommand{\exs}{\mt{\exists} }

\newcommand{\es}{\mt{\emptyset} }        % empty set
\newcommand{\sbs}{\mt{\subset} }         % subset of
\newcommand{\fs}[2]{\{\uw{#1}{1}, \uw{#1}{2}, ... \uw{#1}{#2}\}}

\newcommand{\lra}{ \mt{\longrightarrow} } % implies ----->
\newcommand{\rar}{ \mt{\Rightarrow} }     % implies -->

\newcommand{\lla}{ \mt{\longleftarrow} }  % implies <-----
\newcommand{\lar}{ \mt{\Leftarrow} }      % implies <--

\newcommand{\av}[1]{\mt{|}#1\mt{|}}  % absolute value

\newcommand{\prn}[1]{(#1)}
\newcommand{\bk}[1]{\{#1\}}

\newcommand{\ps}{\mt{+} }
\newcommand{\ms}{\mt{-} }

\newcommand{\ls}{\mt{<} }
\newcommand{\gr}{\mt{>} }

\newcommand{\lse}{\mt{\leq} }
\newcommand{\gre}{\mt{\geq} }

\newcommand{\eql}{\mt{=} }

\newcommand{\pr}{\mt{^\prime} } 		   % prime (i.e. R')
\newcommand{\uw}[2]{#1\mt{_{#2}}}
\newcommand{\uf}[2]{#1\mt{^{#2}}}
\newcommand{\frc}[2]{\mt{\frac{#1}{#2}}}
\newcommand{\lmti}[1]{\mt{\displaystyle{\lim_{#1 \to \infty}}}}
\newcommand{\limt}[2]{\mt{\displaystyle{\lim_{#1 \to #2}}}}

\newcommand{\bnm}[2]{\mt{#1\setminus{#2}}}
\newcommand{\bnt}[2]{\mt{\textrm{#1}\setminus{\textrm{#2}}}}
\newcommand{\bi}{\bnm{\mathbb{R}}{\mathbb{Q}}}

\newcommand{\urng}[2]{\mt{\bigcup_{#1}^{#2}}}
\newcommand{\nrng}[2]{\mt{\bigcap_{#1}^{#2}}}
\newcommand{\nck}[2]{\mt{{#1 \choose #2}}}

\newcommand{\nbho}[3]{\textrm{N(}#1, #2\textrm{) }\cap \textrm{ #3} \neq \emptyset}
     							   %  N(x, eps) intersect S \= emptyset
\newcommand{\nbhe}[3]{\textrm{N(}#1, #2\textrm{) }\cap \textrm{ #3} = \emptyset}
     							   %  N(x, eps) intersect S  = emptyset
\newcommand{\dnbho}[3]{\textrm{N*(}#1, #2\textrm{) }\cap \textrm{ #3} \neq \emptyset}
     							   %  N*(x, eps) intersect S \= emptyset
\newcommand{\dnbhe}[3]{\textrm{N*(}#1, #2\textrm{) }\cap \textrm{ #3} = \emptyset}
     							   %  N*(x, eps) intersect S = emptyset
     							   
\newcommand{\eqn}[1]{\[#1\]}
\newcommand{\splt}[1]{\begin{split}#1\end{split}}

\newcommand{\infy}{\mt{\infty} }
\newcommand{\unn}{\mt{\cup} }
\newcommand{\inn}{\mt{\cap} }
\newcommand\tab[1][1cm]{\hspace*{#1}}

     							 
% ----------

\begin{document}

Do 20 if you finish 19 on the HW

\ssc{Theorem 5.1.13 (as seen in Lec 20)}{

Let f : D \lra \br, g : D \lra \br and let c \mem D\pr

If \limt{x}{c} f(x) \eql L and \limt{x}{c} g(x) \eql M, then

\balist
\item \limt{x}{c} (f \ps g) \eql L \ps M
\item Let k \mem \br, \limt{x}{c} kf \eql kL
\item \limt{x}{c} (fg) \eql LM
\item \limt{x}{c} (\frc{f}{g}) \eql \frc{L}{M}, provided that M $\neq$ 0
\elist

\bgpf
\bpth{a} through \bpth{c} are similar to \bpth{d}.

\bpth{d}: Let \bk{\uw{s}{n}} be a sequence in D st \uw{s}{n} $\neq$ c \fa n \mem \bn and \lmti{n} \uw{s}{n} \eql c.

Then, by Theorem 5.1.8, \lmti{n} f(\uw{s}{n}) \eql L.

Now, \lmti{n} g(x) \eql M $\neq$ 0.

So \exs N \mem \bn st
\eqn{g(s_n) \neq 0 \textrm{ for } n \gre N}
(ask why? next time)

Then, \lmti{n} (\frc{f}{g})(\uw{s}{n}) \eql \lmti{n} \frc{f(s_n)}{g(s_n)} \eql \frc{\lmti{n} f(s_n)}{\lmti{n} g(s_n)} (by Theorem 4.2.11d) \eql \frc{L}{M}

Recall:

\av{x} \ms \av{y} \lse \av{\av{x} \ms \av{y}} \lse \av{x \ms y}

\av{y} \gre \av{x} \ms \av{x \ms y}

So,

\av{g(\uw{s}{n})} \gre \av{M} \ms \av{M \ms g(\uw{s}{n})}

and since,

\lmti{n} g(\uw{s}{n}) \eql M $\neq$ 0

\av{g(\uw{s}{n}) \ms M} \ls \frc{\av{M}}{2}

$-$\av{g(\uw{s}{n}) \ms M} \gr \frc{-\av{M}}{2}

for n \gre N

So,

\av{g(\uw{s}{n})} \gr \av{M} \ms \frc{\av{M}}{2} \eql \frc{\av{M}}{2} \gr 0 for n \gre N 
\epf 
}

\

Also, for the homework:

\limt{x}{c} P(x) \eql P(c) where P is a polynomial.

\ssc{Example 5.1.14}{

Since \limt{x}{c} \uf{x}{1} \eql c (Example 5.1.3),

then it follows by induction that \limt{x}{c} \uf{x}{n} \eql \uf{c}{n} \fa n \mem \bn

\as{for k \mem \bn, \limt{x}{c} \uf{x}{k} \eql \uf{c}{k}}

\wts{\limt{x}{c} \uf{x}{k + 1} \eql \uf{c}{k + 1}}

Define: f(x) \eql \uf{x}{k}, g(x) \eql x

Then \limt{x}{c} f(x) \eql \uf{c}{k}, \limt{x}{c} g(x) \eql c

So,

\eqn{\limt{x}{c} \uf{x}{k + 1} \eql \limt{x}{c} (\uf{x}{k} \uf{x}{1}) = \limt{x}{c} (f(x) g(x)) = \limt{x}{c} f(x) \limt{x}{c} g(x) = c^k c = c^{k + 1}}
Hence, \limt{x}{c} \uf{x}{n} \eql \uf{c}{n}, \fa n \mem \bn by induction.

Combining the result with Theorem 5.1.13, we see that if P is a polynomial and c \mem \br, then \limt{x}{c} P(x) \eql P(c).

To see this,

\lt{P(x) \eql \uw{a}{n}\uf{x}{n} \ps \uw{a}{n - 1}\uf{x}{n - 1} \ps ... \uw{a}{1}x \ps \uw{a}{0} where \uw{a}{i} \mem \br for i \eql 0, 1, 2, ... n}

\ssc{Example 5.1.5}{

Find \limt{x}{1} \frc{2x^2 - 3x + 1}{x - 1}

We need the 0 \ls part of the limit definition so that the limit can exist even if the function is undefined at that point.

Notice that for x $\neq$ 1,

\eqn{\frac{2x^2 - 3x + 1}{x - 1} = \frac{(x - 1)(2x - 1)}{(x - 1)}}

So,

\eqn{\limt{x}{1} \frac{2x^2 - 3x + 1}{x - 1} = \limt{x}{1} 2x - 1 = 2 - 1 = 1}

If we let q \eql f(x) \eql \frc{2x^2 - 3x + 1}{x - 1}, then f(x) \eql 2x \ms 1 with a hole at x \eql 1.
}

\ssc{One Sided Limits}{

\lt{f : D \lra \br and let c \mem D\pr}

Then,

\bilist
\item We write \limt{x}{c-} f(x) \eql L iff, for \ep \gr 0, \exs \dta \gr 0 st \av{f(x) \ms L} \ls \ep whenever c \ms \dta \ls x \ls c and x \mem D
\item We write \limt{x}{c+} f(x) \eql L iff, for \ep \gr 0, \exs \dta \gr 0 st \av{f(x) \ms L} \ls \ep whenever c \ls x \ls c \ps \dta and x \mem D
\elist

Of course, if \limt{x}{c} f(x) \eql L iff both \limt{x}{c-} f(x) \eql \limt{x}{c+} f(x) \eql L
}

\bsc{5.2 Continuous Functions}{

\ssc{Definition 5.2.1}{

\lt{f : D \lra \br and c \mem D (we don't know that c is an accumulation point)}

We say that f is \textbf{continuous at c} if

for any \ep \gr 0, \exs \dta \gr 0 st
\eqn{|f(x) - f(c)| < \epsilon \textrm{ whenever } |x - c| < \delta \textrm{ and } x \mem D}

\textbf{N.B.}: f(c) must be defined in order for f(x) to be continuous at x \eql c
}

\ssc{Theorem 5.2.2}{

\lt{f : D \lra \br and c \mem D}

Then the following are equivalent:

\balist
\item f is continuous at c
\item If \bk{\uw{x}{n}} is any sequence in D st \uw{x}{n} \lra c as n \lra \infy (\uw{x}{n} can actually be c),
	
	then \lmti{n} f(\uw{x}{n}) \eql f(c)
\item For every neighborhood V of f(c), \exs a neighborhood U of c st f(U \inn D) \sbs V
	
	Furthermore, if c \mem D\pr, then the above are all equivalent to d)
\item f has a limit at c and \limt{x}{c} f(x) \eql f(c)
\elist

\bgpf

Case:
\bilist
\item c \mem \bnt{D}{D\pr} (i.e. c is an isolated point)
	
	Thus, \exs a neighborhood U \sbs \br of c st
	\eqn{U \cap D \eql \bk{c}}
	(i.e. U \eql (c \ms \dta, c \ps \dta) \eql \bk{c})
	
	\bpth{a}
	
	\wts{f is continuous at x \eql c}
	
	For \ep \gr 0, \exs \dta \gr 0 st (c \ms \dta, c \ps \dta) \sbs U.
	
	This follows since a neighborhood is open. Thus,
	\eqn{|f(x) - f(c)| = 0 < \epsilon \textrm{ whenever} |x - c| < \delta \textrm{ and } x \mem D}
	This means by definition that f(x) is continuous at x \eql c.
	
	\bpth{b}
	
	\lt{\bk{\uw{x}{n}} \sbs D st \uw{x}{n} \lra c as n \lra \infy}
	
	and
	
	For \ep \gr 0, \exs \dta \gr 0 st (c \ms \dta, c \ps \dta) \sbs U
		
	\wts{\lmti{n} f(\uw{x}{n}) \eql f(c).}
	
	Since U is open, \exs N \mem \bn st
	\eqn{|x_n - c| < \delta \textrm{ for } n \gre N}
	Thus, \uw{x}{n} \mem U for n \gre N
	
	We see that
	\eqn{|f(x_n) - f(c)| = 0 < \epsilon \textrm{ for } n \gre N}
	Hence,
	\lmti{n} f(\uw{x}{n}) \eql f(c)
	
	\bpth{c}
	
	Now,
	
	\lt{V be a neighborhood of f(c)}
	
	Then, using U as defined prior to (a):
	\eqn{f(U \cap D) \sbs V}
	Hence, a, b, and c, are equivalent if c \mem \bnt{D}{D\pr}

\item c \mem D \inn D\pr (i.e. c is an accumulation point)
\elist

\epf

}

}

}


\end{document}