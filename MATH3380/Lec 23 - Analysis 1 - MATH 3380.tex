% Thank you Josh Davis for this template!
% https://github.com/jdavis/latex-homework-template/blob/master/homework.tex

\documentclass{article}

\newcommand{\hmwkTitle}{Lec\ \#23}

% % ----------

% Packages

\usepackage{fancyhdr}
\usepackage{extramarks}
\usepackage{amsmath}
\usepackage{amssymb}
\usepackage{amsthm}
\usepackage{amsfonts}
\usepackage{tikz}
\usepackage[plain]{algorithm}
\usepackage{algpseudocode}
\usepackage{enumitem}
\usepackage{chngcntr}

% Libraries

\usetikzlibrary{automata, positioning, arrows}

%
% Basic Document Settings
%

\topmargin=-0.45in
\evensidemargin=0in
\oddsidemargin=0in
\textwidth=6.5in
\textheight=9.0in
\headsep=0.25in

\linespread{1.1}

\pagestyle{fancy}
\lhead{\hmwkAuthorName}
\chead{}
\rhead{\hmwkClass\ (\hmwkClassInstructor): \hmwkTitle}
\lfoot{\lastxmark}
\cfoot{\thepage}

\renewcommand\headrulewidth{0.4pt}
\renewcommand\footrulewidth{0.4pt}

\setlength\parindent{0pt}
\setcounter{secnumdepth}{0}

\newcommand{\hmwkClass}{MATH 3380 / Analysis 1}        % Class
\newcommand{\hmwkClassInstructor}{Dr. Welsh}           % Instructor
\newcommand{\hmwkAuthorName}{\textbf{Joshua Mitchell}} % Author

%
% Title Page
%

\title{
    \vspace{2in}
    \textmd{\textbf{\hmwkClass:\ \hmwkTitle}}\\
    \normalsize\vspace{0.1in}\small\vspace{0.1in}\large{\textit{\hmwkClassInstructor}}
    \vspace{3in}
}

\author{\hmwkAuthorName}
\date{}

\renewcommand{\part}[1]{\textbf{\large Part \Alph{partCounter}}\stepcounter{partCounter}\\}

% Integral dx
\newcommand{\dx}{\mathrm{d}x}

%
% Various Helper Commands
%

% For derivatives
\newcommand{\deriv}[1]{\frac{\mathrm{d}}{\mathrm{d}x} (#1)}

% For partial derivatives
\newcommand{\pderiv}[2]{\frac{\partial}{\partial #1} (#2)}


% Alias for the Solution section header
\newcommand{\solution}{\textbf{\large Solution}}

% Probability commands: Expectation, Variance, Covariance, Bias
\newcommand{\E}{\mathrm{E}}
\newcommand{\Var}{\mathrm{Var}}
\newcommand{\Cov}{\mathrm{Cov}}
\newcommand{\Bias}{\mathrm{Bias}}

% Formatting commands:

\newcommand{\mt}[1]{\ensuremath{#1}}
\newcommand{\nm}[1]{\textrm{#1}}

\newcommand\bsc[2][\DefaultOpt]{%
  \def\DefaultOpt{#2}%
  \section[#1]{#2}%
}
\newcommand\ssc[2][\DefaultOpt]{%
  \def\DefaultOpt{#2}%
  \subsection[#1]{#2}%
}
\newcommand{\bgpf}{\begin{proof} $ $\newline}

\newcommand{\bgeq}{\begin{equation*}}
\newcommand{\eeq}{\end{equation*}}	

\newcommand{\balist}{\begin{enumerate}[label=\alph*.]}
\newcommand{\elist}{\end{enumerate}}

\newcommand{\bilist}{\begin{enumerate}[label=\roman*)]}	

\newcommand{\bgsp}{\begin{split}}
% \newcommand{\esp}{\end{split}} % doesn't work for some reason.

\newcommand\prs[1]{~~~\textbf{(#1)}}

\newcommand{\lt}[1]{\textbf{Let: } #1}
     							   %  if you're setting it to be true
\newcommand{\supp}[1]{\textbf{Suppose: } #1}
     							   %  Suppose (if it'll end up false)
\newcommand{\wts}[1]{\textbf{Want to show: } #1}
     							   %  Want to show
\newcommand{\as}[1]{\textbf{Assume: } #1}
     							   %  if you think it follows from truth
\newcommand{\bpth}[1]{\textbf{(#1)}}

\newcommand{\step}[2]{\begin{equation}\tag{#2}#1\end{equation}}
\newcommand{\epf}{\end{proof}}

\newcommand{\dbs}[3]{\mt{#1_{#2_#3}}}

\newcommand{\sidenote}[1]{-----------------------------------------------------------------Side Note----------------------------------------------------------------
#1 \

---------------------------------------------------------------------------------------------------------------------------------------------}

% Analysis / Logical commands:

\newcommand{\br}{\mt{\mathbb{R}} }       % |R
\newcommand{\bq}{\mt{\mathbb{Q}} }       % |Q
\newcommand{\bn}{\mt{\mathbb{N}} }       % |N
\newcommand{\bc}{\mt{\mathbb{C}} }       % |C
\newcommand{\bz}{\mt{\mathbb{Z}} }       % |Z

\newcommand{\ep}{\mt{\epsilon} }         % epsilon
\newcommand{\fa}{\mt{\forall} }          % for all
\newcommand{\afa}{\mt{\alpha} }
\newcommand{\bta}{\mt{\beta} }
\newcommand{\mem}{\mt{\in} }
\newcommand{\exs}{\mt{\exists} }

\newcommand{\es}{\mt{\emptyset}}        % empty set
\newcommand{\sbs}{\mt{\subset} }         % subset of
\newcommand{\fs}[2]{\{\uw{#1}{1}, \uw{#1}{2}, ... \uw{#1}{#2}\}}

\newcommand{\lra}{ \mt{\longrightarrow} } % implies ----->
\newcommand{\rar}{ \mt{\Rightarrow} }     % implies -->

\newcommand{\lla}{ \mt{\longleftarrow} }  % implies <-----
\newcommand{\lar}{ \mt{\Leftarrow} }      % implies <--

\newcommand{\eql}{\mt{=} }
\newcommand{\pr}{\mt{^\prime} } 		   % prime (i.e. R')
\newcommand{\uw}[2]{#1\mt{_{#2}}}
\newcommand{\frc}[2]{\mt{\frac{#1}{#2}}}

\newcommand{\bnm}[2]{\mt{#1\setminus{#2}}}
\newcommand{\bnt}[2]{\mt{\textrm{#1}\setminus{\textrm{#2}}}}
\newcommand{\bi}{\bnm{\mathbb{R}}{\mathbb{Q}}}

\newcommand{\urng}[2]{\mt{\bigcup_{#1}^{#2}}}
\newcommand{\nrng}[2]{\mt{\bigcap_{#1}^{#2}}}

\newcommand{\nbho}[3]{\textrm{N(}#1, #2\textrm{) }\cap \textrm{ #3} \neq \emptyset}
     							   %  N(x, eps) intersect S \= emptyset
\newcommand{\nbhe}[3]{\textrm{N(}#1, #2\textrm{) }\cap \textrm{ #3} = \emptyset}
     							   %  N(x, eps) intersect S  = emptyset
\newcommand{\dnbho}[3]{\textrm{N*(}#1, #2\textrm{) }\cap \textrm{ #3} \neq \emptyset}
     							   %  N*(x, eps) intersect S \= emptyset
\newcommand{\dnbhe}[3]{\textrm{N*(}#1, #2\textrm{) }\cap \textrm{ #3} = \emptyset}
     							   %  N*(x, eps) intersect S = emptyset
     							 


% ----------

% ----------

% Packages

\usepackage{fancyhdr}
\usepackage{extramarks}
\usepackage{amsmath}
\usepackage{amssymb}
\usepackage{amsthm}
\usepackage{amsfonts}
\usepackage{tikz}
\usepackage[plain]{algorithm}
\usepackage{algpseudocode}
\usepackage{enumitem}
\usepackage{chngcntr}

% Libraries

\graphicspath{{/Users/jm/iclouddrive/3380pics/}}

\usetikzlibrary{automata, positioning, arrows}

%
% Basic Document Settings
%

\topmargin=-0.45in
\evensidemargin=0in
\oddsidemargin=0in
\textwidth=6.5in
\textheight=9.0in
\headsep=0.25in

\linespread{1.1}

\pagestyle{fancy}
\lhead{\hmwkAuthorName}
\chead{}
\rhead{\hmwkClass\ (\hmwkClassInstructor): \hmwkTitle}
\lfoot{\lastxmark}
\cfoot{\thepage}

\renewcommand\headrulewidth{0.4pt}
\renewcommand\footrulewidth{0.4pt}

\setlength\parindent{0pt}
\setcounter{secnumdepth}{0}

\newcommand{\hmwkClass}{MATH 3380 / Analysis 1}        % Class
\newcommand{\hmwkClassInstructor}{Dr. Welsh}           % Instructor
\newcommand{\hmwkAuthorName}{\textbf{Joshua Mitchell}} % Author

%
% Title Page
%

\title{
    \vspace{2in}
    \textmd{\textbf{\hmwkClass:\ \hmwkTitle}}\\
    \normalsize\vspace{0.1in}\small\vspace{0.1in}\large{\textit{\hmwkClassInstructor}}
    \vspace{3in}
}

\author{\hmwkAuthorName}
\date{}

\renewcommand{\part}[1]{\textbf{\large Part \Alph{partCounter}}\stepcounter{partCounter}\\}

% Integral dx
\newcommand{\dx}{\mathrm{d}x}

%
% Various Helper Commands
%

% For derivatives
\newcommand{\deriv}[1]{\frac{\mathrm{d}}{\mathrm{d}x} (#1)}

% For partial derivatives
\newcommand{\pderiv}[2]{\frac{\partial}{\partial #1} (#2)}


% Alias for the Solution section header
\newcommand{\solution}{\textbf{\large Solution}}

% Probability commands: Expectation, Variance, Covariance, Bias
\newcommand{\E}{\mathrm{E}}
\newcommand{\Var}{\mathrm{Var}}
\newcommand{\Cov}{\mathrm{Cov}}
\newcommand{\Bias}{\mathrm{Bias}}

% Formatting commands:

\newcommand{\mt}[1]{\ensuremath{#1}}
\newcommand{\nm}[1]{\textrm{#1}}

\newcommand\bsc[2][\DefaultOpt]{%
  \def\DefaultOpt{#2}%
  \section[#1]{#2}%
}
\newcommand\ssc[2][\DefaultOpt]{%
  \def\DefaultOpt{#2}%
  \subsection[#1]{#2}%
}
\newcommand{\bgpf}{\begin{proof} $ $\newline}

\newcommand{\bgeq}{\begin{equation*}}
\newcommand{\eeq}{\end{equation*}}	

\newcommand{\balist}{\begin{enumerate}[label=\alph*.]}
\newcommand{\elist}{\end{enumerate}}

\newcommand{\bilist}{\begin{enumerate}[label=\roman*)]}	

\newcommand{\bgsp}{\begin{split}}
% \newcommand{\esp}{\end{split}} % doesn't work for some reason.

\newcommand\prs[1]{~~~\textbf{(#1)}}

\newcommand{\lt}[1]{\textbf{Let: } #1}
     							   %  if you're setting it to be true
\newcommand{\supp}[1]{\textbf{Suppose: } #1}
     							   %  Suppose (if it'll end up false)
\newcommand{\wts}[1]{\textbf{Want to show: } #1}
     							   %  Want to show
\newcommand{\as}[1]{\textbf{Assume: } #1}
     							   %  if you think it follows from truth
\newcommand{\bpth}[1]{\textbf{(#1)}}

\newcommand{\step}[2]{\begin{equation}\tag{#2}#1\end{equation}}
\newcommand{\epf}{\end{proof}}

\newcommand{\dbs}[3]{\mt{#1_{#2_#3}}}

\newcommand{\sidenote}[1]{-----------------------------------------------------------------Side Note----------------------------------------------------------------
#1 \

---------------------------------------------------------------------------------------------------------------------------------------------}

% Analysis / Logical commands:

\newcommand{\br}{\mt{\mathbb{R}} }       % |R
\newcommand{\bq}{\mt{\mathbb{Q}} }       % |Q
\newcommand{\bn}{\mt{\mathbb{N}} }       % |N
\newcommand{\bc}{\mt{\mathbb{C}} }       % |C
\newcommand{\bz}{\mt{\mathbb{Z}} }       % |Z

\newcommand{\ep}{\mt{\epsilon} }         % epsilon
\newcommand{\fa}{\mt{\forall} }          % for all
\newcommand{\afa}{\mt{\alpha} }
\newcommand{\bta}{\mt{\beta} }
\newcommand{\dta}{\mt{\delta} }
\newcommand{\mem}{\mt{\in} }
\newcommand{\exs}{\mt{\exists} }

\newcommand{\es}{\mt{\emptyset} }        % empty set
\newcommand{\sbs}{\mt{\subset} }         % subset of
\newcommand{\fs}[2]{\{\uw{#1}{1}, \uw{#1}{2}, ... \uw{#1}{#2}\}}

\newcommand{\lra}{ \mt{\longrightarrow} } % implies ----->
\newcommand{\rar}{ \mt{\Rightarrow} }     % implies -->

\newcommand{\lla}{ \mt{\longleftarrow} }  % implies <-----
\newcommand{\lar}{ \mt{\Leftarrow} }      % implies <--

\newcommand{\av}[1]{\mt{|}#1\mt{|}}  % absolute value

\newcommand{\prn}[1]{(#1)}
\newcommand{\bk}[1]{\{#1\}}

\newcommand{\ps}{\mt{+} }
\newcommand{\ms}{\mt{-} }

\newcommand{\ls}{\mt{<} }
\newcommand{\gr}{\mt{>} }

\newcommand{\lse}{\mt{\leq} }
\newcommand{\gre}{\mt{\geq} }

\newcommand{\eql}{\mt{=} }

\newcommand{\pr}{\mt{^\prime} } 		   % prime (i.e. R')
\newcommand{\uw}[2]{#1\mt{_{#2}}}
\newcommand{\uf}[2]{#1\mt{^{#2}}}
\newcommand{\frc}[2]{\mt{\frac{#1}{#2}}}
\newcommand{\lmti}[1]{\mt{\displaystyle{\lim_{#1 \to \infty}}}}
\newcommand{\limt}[2]{\mt{\displaystyle{\lim_{#1 \to #2}}}}

\newcommand{\bnm}[2]{\mt{#1\setminus{#2}}}
\newcommand{\bnt}[2]{\mt{\textrm{#1}\setminus{\textrm{#2}}}}
\newcommand{\bi}{\bnm{\mathbb{R}}{\mathbb{Q}}}

\newcommand{\urng}[2]{\mt{\bigcup_{#1}^{#2}}}
\newcommand{\nrng}[2]{\mt{\bigcap_{#1}^{#2}}}
\newcommand{\nck}[2]{\mt{{#1 \choose #2}}}

\newcommand{\nbho}[3]{\textrm{N(}#1, #2\textrm{) }\cap \textrm{ #3} \neq \emptyset}
     							   %  N(x, eps) intersect S \= emptyset
\newcommand{\nbhe}[3]{\textrm{N(}#1, #2\textrm{) }\cap \textrm{ #3} = \emptyset}
     							   %  N(x, eps) intersect S  = emptyset
\newcommand{\dnbho}[3]{\textrm{N*(}#1, #2\textrm{) }\cap \textrm{ #3} \neq \emptyset}
     							   %  N*(x, eps) intersect S \= emptyset
\newcommand{\dnbhe}[3]{\textrm{N*(}#1, #2\textrm{) }\cap \textrm{ #3} = \emptyset}
     							   %  N*(x, eps) intersect S = emptyset
     							   
\newcommand{\eqn}[1]{\[#1\]}
\newcommand{\splt}[1]{\begin{split}#1\end{split}}

\newcommand{\infy}{\mt{\infty} }
\newcommand{\unn}{\mt{\cup} }
\newcommand{\inn}{\mt{\cap} }
\newcommand\tab[1][1cm]{\hspace*{#1}}

     							 
% ----------

\begin{document}

\bsc{Section 5.2 Continued}{
\ssc{Theorem 5.2.14}{

A function f : D \lra \br is continuous on D iff for every open set G in \br \exs an open set H in \br st H \inn D \eql \uf{f}{-1}(G)

\bgpf

\lra 

\lt{G be an open subset of \br}

\as{f is continuous on D}

If c \mem \uf{f}{-1}(G), then f(c) \mem G.

Since G is open, \exs a neighborhood V of f(c) such that v \sbs G

By Theorem 5.2.2(c), \exs a neighborhood U(c) of c, such that 

f(U(c) \inn D) \sbs V

Now, let H \eql \urng{c \mem \uf{f}{-1}(G)}{} U(G)

Since each neighborhood U(c) is open, it follows that H is open and that H \inn D \eql \uf{f}{-1}(G)

\lla

\lt{V be a neighborhood of f(c) since c \mem D}


Since V is an open set, our hypothesis implies that \exs an open set H \sbs \br st H \inn D \eql \uf{f}{-1}(V)

Since f(c) \mem V, we have c \mem H

But, H is an open set, so \exs a neighborhood U of c st U \sbs H.

Thus, f(U \inn D) \sbs f(H \inn D) \sbs V

From Theorem 5.2.2, f is continuous on D.

\epf

}

\ssc{Corollary 5.2.15}{

A function f : \br \lra \br is continuous iff \uf{f}{-1}(G) is open in \br whenever G is open in \br
}

\ssc{Example 5.2.16}{

Define f : \br \lra \br

f(x) \eql \bk{x if x \lse 2, 4 if x \gr 2}

If G \eql (1, 3), then \uf{f}{-1}(G) \eql (1, 2]

}

}

\bsc{Section 5.3: Properties of Continuous Functions}{

\ssc{Definition 5.3.1: Boundedness of Function}{

A function f: D \lra \br is said to be bounded if the range f(D) is a bounded subset of \br (i.e. f is bounded if there exists M \mem \br such that \av{f(x)} \lse M for all x \mem D).

Note: A continuous function may not be bounded even when the domain is bounded.
}

\ssc{Theorem 5.3.2}{

If D \sbs \br is compact and f : D \lra \br is continuous, then

f(D) is compact

\bgpf

\lt{J \eql \bk{\uw{G}{\afa}} be an open cover of f(D)}

\wts{J has a finite subcover.}

Since f is continuous on D, 

Theorem 5.2.14 implies that for each open set \uw{G}{\afa} in J, \exs an open set \uw{H}{\afa} st \uw{H}{\afa} \inn D \eql \uf{f}{-1}(\uw{G}{\afa})

Moreover, since f(D) \sbs \urng{}{}\uw{G}{\afa},

it follows that D \sbs \urng{}{} \uf{f}{-1}(\uw{G}{\afa}) \sbs \urng{}{}\uw{H}{\afa},

Thus, the collection \bk{\uw{H}{\afa}} is an open cover of D.

Since D is compact, \exs finitely many sets \dbs{H}{\afa}{1}, \dbs{H}{\afa}{2}, .. \dbs{H}{\afa}{n} such that

D \sbs \dbs{H}{\afa}{1} \unn \dbs{H}{\afa}{2} ... \dbs{H}{\afa}{n}

But then D \sbs (\dbs{H}{\afa}{1} \inn D) \unn (\dbs{H}{\afa}{2} \inn D) ... (\dbs{H}{\afa}{n} \inn D)

f(D) \sbs \dbs{G}{\afa}{1} \unn \dbs{G}{\afa}{2} ... \dbs{G}{\afa}{n}

Therefore,

\bk{\dbs{G}{\afa}{1}, ... \dbs{G}{\afa}{n}} is a finite subcover of J for f(D)

Therefore, f(D) is compact.

\epf

}

\ssc{Corollary 5.3.5}{

\lt{D \sbs \br be compact}

f : D \lra \br is continuous implies f assumes min and max values on D.

That is to say: \exs points \uw{x}{1}, \uw{x}{2} \mem D such that f(\uw{x}{1}) \lse f(x) \lse f(\uw{x}{2}) \fa x \mem D

\bgpf

By Theorem 5.3.2, f(D) is compact.

From Lemma 3.5.4, f(D) has both a minimum, \uw{y}{1}, and a maximum, \uw{y}{2}.

Since \uw{y}{1}, \uw{y}{2} \mem f(D), there exists \uw{x}{1}, \uw{x}{2} \mem D st f(\uw{x}{1}) \eql \uw{y}{1} and f(\uw{x}{2}) \eql \uw{y}{2}

Thus, 

f(\uw{x}{1}) \lse f(x) \lse f(\uw{x}{2}) \fa x \mem D

\epf
}

\ssc{Lemma 5.3.5}{

\lt{f : [a, b] \lra \br be continuous}

f(a) \ls 0 \ls f(b) \rar \exs c \mem (a, b) st f(c) \eql 0

\bgpf

\lt{c \eql max\bk{x : f(x) \lse 0} and S \eql \bk{x \mem [a, b] : f(x) \lse 0}}

Since a \mem S, S is nonempty.

Notice that S is bounded above by b, so c \eql sup S exists as a real number in [a, b]

\wts{f(c) \eql 0}

\supp{f(c) \ls 0}

Then \exs a neighborhood U of c such that f(x) \ls 0 for all x \mem U \inn [a, b]

(This comes from Exercise 5.2.13)

Now c $\neq$ b, since f(c) \ls 0 \ls f(b)

Thus, U contains an in between point p st c \ls p \ls b

But f(p) \ls 0 since p \mem U

Therefore, p \mem S

This contradicts c being an upper bound for S.

\supp{f(c) \gr 0}

Similarly,

If f(c) \gr 0, then \exs a neighborhood U of c such that f(x) \gr 0 for all x \mem U \inn [a, b]

Now, c $\neq$ a, since f(c) \gr 0 \gr f(a)

Thus, U contains a point p st a \ls p \ls c

Since f(x) \gr 0 \fa x \mem U, no points of S are in [p, c]

That is to say, p is an upper bound for S.

This contradicts c being the least upper bound (supremum) of S.

Hence, f(c) \eql 0

Since f(a) \ls 0 \ls f(b) and f(c) \eql 0,

\exs c \mem (a, b)
\epf
}

\ssc{Theorem 5.3.6 - Intermediate Value Theorem}{

\as{f : [a, b] \lra \br is continuous}

Then f has the intermediate value property on [a, b].

That is, if k is any value between f(a) and f(b), 

i.e. f(a) \ls k \ls f(b) or f(b) \ls k \ls f(a), 

then \exs c \mem (a, b) st f(c) \eql k


\bgpf

\lt{k be between f(a) and f(b)}

If f(a) \ls f(b), from Lemma 5.3.5, consider the continuous function:

g : [a, b] \lra \br given by g(x) \eql f(x) \ms k

Then, 

g(a) \eql f(a) \ms k \ls 0 

and 

g(b) \eql f(b) \ms k \gr 0

From Lemma 5.3.5, \exs c \mem (a, b) 

st 

g(c) \eql 0 \eql f(c) \ms k \rar f(c) \eql k

Similarly, we can prove when f(a) \gr f(b)

\epf

}

\ssc{Exercise 5.3.7}{

Using the intermediate value theorem, we can show that every positive number has a positive nth root.

\as{k \gr 0, n \mem \bn}

\lt{f(x) \eql \uf{x}{n}}

Notice that f(0) \eql 0 \ls k

if b \eql k \ps 1, then from Bernolli's inequality (Exercise 3.1.24),
\eqn{b^n = (k + 1)^n \gre 1 + kn > k}
\eqn{f(b) = b^n = (k + 1)^n \gre 1 + kn > k}
Since f is continuous, 

\exs c \mem (0, b) st f(c) \eql k \eql \uf{c}{n}, where c is the nth root of k
}

\ssc{Theorem 5.3.10}{

\lt{I be a compact interval}

\as{f : I \lra \br is a continuous function}

Then, the set f(I) is a compact interval.

\bgpf

From Corollary 5.3.3,

\exs \uw{x}{1}, \uw{x}{2} \mem I st f(\uw{x}{1}) \lse f(x) \lse f(\uw{x}{2}), for all x \mem I

\lt{\uw{m}{1} \eql f(\uw{x}{1}), \uw{m}{2} \eql f(\uw{x}{2}), and f(I) \sbs \sbs [\uw{m}{1}, \uw{m}{2}]}

If \uw{m}{1} \eql \uw{m}{2}, then f(I) \eql \bk{\uw{m}{1}} \eql [\uw{m}{1}, \uw{m}{2}], and we're done.

If \uw{m}{1} \ls \uw{m}{2} and k \mem (\uw{m}{1}, \uw{m}{2}), then by Theorem 5.3.6, we have

k \eql f(c), c \mem (\uw{x}{1}, \uw{x}{2}) and (\uw{m}{1}, \uw{m}{2}) \sbs f(I).

\uw{m}{1}, \uw{m}{2} \mem f(I), [\uw{m}{1}, \uw{m}{2}] \sbs f(I), f(I) is the compact interval [\uw{m}{1}, \uw{m}{2}], and we are done.

\epf

}

}
\end{document}