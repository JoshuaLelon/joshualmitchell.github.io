% Thank you Josh Davis for this template!
% https://github.com/jdavis/latex-homework-template/blob/master/homework.tex

\documentclass{article}

\newcommand{\hmwkTitle}{Lec\ \#14}

% % ----------

% Packages

\usepackage{fancyhdr}
\usepackage{extramarks}
\usepackage{amsmath}
\usepackage{amssymb}
\usepackage{amsthm}
\usepackage{amsfonts}
\usepackage{tikz}
\usepackage[plain]{algorithm}
\usepackage{algpseudocode}
\usepackage{enumitem}
\usepackage{chngcntr}

% Libraries

\usetikzlibrary{automata, positioning, arrows}

%
% Basic Document Settings
%

\topmargin=-0.45in
\evensidemargin=0in
\oddsidemargin=0in
\textwidth=6.5in
\textheight=9.0in
\headsep=0.25in

\linespread{1.1}

\pagestyle{fancy}
\lhead{\hmwkAuthorName}
\chead{}
\rhead{\hmwkClass\ (\hmwkClassInstructor): \hmwkTitle}
\lfoot{\lastxmark}
\cfoot{\thepage}

\renewcommand\headrulewidth{0.4pt}
\renewcommand\footrulewidth{0.4pt}

\setlength\parindent{0pt}
\setcounter{secnumdepth}{0}

\newcommand{\hmwkClass}{MATH 3380 / Analysis 1}        % Class
\newcommand{\hmwkClassInstructor}{Dr. Welsh}           % Instructor
\newcommand{\hmwkAuthorName}{\textbf{Joshua Mitchell}} % Author

%
% Title Page
%

\title{
    \vspace{2in}
    \textmd{\textbf{\hmwkClass:\ \hmwkTitle}}\\
    \normalsize\vspace{0.1in}\small\vspace{0.1in}\large{\textit{\hmwkClassInstructor}}
    \vspace{3in}
}

\author{\hmwkAuthorName}
\date{}

\renewcommand{\part}[1]{\textbf{\large Part \Alph{partCounter}}\stepcounter{partCounter}\\}

% Integral dx
\newcommand{\dx}{\mathrm{d}x}

%
% Various Helper Commands
%

% For derivatives
\newcommand{\deriv}[1]{\frac{\mathrm{d}}{\mathrm{d}x} (#1)}

% For partial derivatives
\newcommand{\pderiv}[2]{\frac{\partial}{\partial #1} (#2)}


% Alias for the Solution section header
\newcommand{\solution}{\textbf{\large Solution}}

% Probability commands: Expectation, Variance, Covariance, Bias
\newcommand{\E}{\mathrm{E}}
\newcommand{\Var}{\mathrm{Var}}
\newcommand{\Cov}{\mathrm{Cov}}
\newcommand{\Bias}{\mathrm{Bias}}

% Formatting commands:

\newcommand{\mt}[1]{\ensuremath{#1}}
\newcommand{\nm}[1]{\textrm{#1}}

\newcommand\bsc[2][\DefaultOpt]{%
  \def\DefaultOpt{#2}%
  \section[#1]{#2}%
}
\newcommand\ssc[2][\DefaultOpt]{%
  \def\DefaultOpt{#2}%
  \subsection[#1]{#2}%
}
\newcommand{\bgpf}{\begin{proof} $ $\newline}

\newcommand{\bgeq}{\begin{equation*}}
\newcommand{\eeq}{\end{equation*}}	

\newcommand{\balist}{\begin{enumerate}[label=\alph*.]}
\newcommand{\elist}{\end{enumerate}}

\newcommand{\bilist}{\begin{enumerate}[label=\roman*)]}	

\newcommand{\bgsp}{\begin{split}}
% \newcommand{\esp}{\end{split}} % doesn't work for some reason.

\newcommand\prs[1]{~~~\textbf{(#1)}}

\newcommand{\lt}[1]{\textbf{Let: } #1}
     							   %  if you're setting it to be true
\newcommand{\supp}[1]{\textbf{Suppose: } #1}
     							   %  Suppose (if it'll end up false)
\newcommand{\wts}[1]{\textbf{Want to show: } #1}
     							   %  Want to show
\newcommand{\as}[1]{\textbf{Assume: } #1}
     							   %  if you think it follows from truth
\newcommand{\bpth}[1]{\textbf{(#1)}}

\newcommand{\step}[2]{\begin{equation}\tag{#2}#1\end{equation}}
\newcommand{\epf}{\end{proof}}

\newcommand{\dbs}[3]{\mt{#1_{#2_#3}}}

\newcommand{\sidenote}[1]{-----------------------------------------------------------------Side Note----------------------------------------------------------------
#1 \

---------------------------------------------------------------------------------------------------------------------------------------------}

% Analysis / Logical commands:

\newcommand{\br}{\mt{\mathbb{R}} }       % |R
\newcommand{\bq}{\mt{\mathbb{Q}} }       % |Q
\newcommand{\bn}{\mt{\mathbb{N}} }       % |N
\newcommand{\bc}{\mt{\mathbb{C}} }       % |C
\newcommand{\bz}{\mt{\mathbb{Z}} }       % |Z

\newcommand{\ep}{\mt{\epsilon} }         % epsilon
\newcommand{\fa}{\mt{\forall} }          % for all
\newcommand{\afa}{\mt{\alpha} }
\newcommand{\bta}{\mt{\beta} }
\newcommand{\mem}{\mt{\in} }
\newcommand{\exs}{\mt{\exists} }

\newcommand{\es}{\mt{\emptyset}}        % empty set
\newcommand{\sbs}{\mt{\subset} }         % subset of
\newcommand{\fs}[2]{\{\uw{#1}{1}, \uw{#1}{2}, ... \uw{#1}{#2}\}}

\newcommand{\lra}{ \mt{\longrightarrow} } % implies ----->
\newcommand{\rar}{ \mt{\Rightarrow} }     % implies -->

\newcommand{\lla}{ \mt{\longleftarrow} }  % implies <-----
\newcommand{\lar}{ \mt{\Leftarrow} }      % implies <--

\newcommand{\eql}{\mt{=} }
\newcommand{\pr}{\mt{^\prime} } 		   % prime (i.e. R')
\newcommand{\uw}[2]{#1\mt{_{#2}}}
\newcommand{\frc}[2]{\mt{\frac{#1}{#2}}}

\newcommand{\bnm}[2]{\mt{#1\setminus{#2}}}
\newcommand{\bnt}[2]{\mt{\textrm{#1}\setminus{\textrm{#2}}}}
\newcommand{\bi}{\bnm{\mathbb{R}}{\mathbb{Q}}}

\newcommand{\urng}[2]{\mt{\bigcup_{#1}^{#2}}}
\newcommand{\nrng}[2]{\mt{\bigcap_{#1}^{#2}}}

\newcommand{\nbho}[3]{\textrm{N(}#1, #2\textrm{) }\cap \textrm{ #3} \neq \emptyset}
     							   %  N(x, eps) intersect S \= emptyset
\newcommand{\nbhe}[3]{\textrm{N(}#1, #2\textrm{) }\cap \textrm{ #3} = \emptyset}
     							   %  N(x, eps) intersect S  = emptyset
\newcommand{\dnbho}[3]{\textrm{N*(}#1, #2\textrm{) }\cap \textrm{ #3} \neq \emptyset}
     							   %  N*(x, eps) intersect S \= emptyset
\newcommand{\dnbhe}[3]{\textrm{N*(}#1, #2\textrm{) }\cap \textrm{ #3} = \emptyset}
     							   %  N*(x, eps) intersect S = emptyset
     							 


% ----------

% ----------

% Packages

\usepackage{fancyhdr}
\usepackage{extramarks}
\usepackage{amsmath}
\usepackage{amssymb}
\usepackage{amsthm}
\usepackage{amsfonts}
\usepackage{tikz}
\usepackage[plain]{algorithm}
\usepackage{algpseudocode}
\usepackage{enumitem}
\usepackage{chngcntr}

% Libraries

\usetikzlibrary{automata, positioning, arrows}

%
% Basic Document Settings
%

\topmargin=-0.45in
\evensidemargin=0in
\oddsidemargin=0in
\textwidth=6.5in
\textheight=9.0in
\headsep=0.25in

\linespread{1.1}

\pagestyle{fancy}
\lhead{\hmwkAuthorName}
\chead{}
\rhead{\hmwkClass\ (\hmwkClassInstructor): \hmwkTitle}
\lfoot{\lastxmark}
\cfoot{\thepage}

\renewcommand\headrulewidth{0.4pt}
\renewcommand\footrulewidth{0.4pt}

\setlength\parindent{0pt}
\setcounter{secnumdepth}{0}

\newcommand{\hmwkClass}{MATH 3380 / Analysis 1}        % Class
\newcommand{\hmwkClassInstructor}{Dr. Welsh}           % Instructor
\newcommand{\hmwkAuthorName}{\textbf{Joshua Mitchell}} % Author

%
% Title Page
%

\title{
    \vspace{2in}
    \textmd{\textbf{\hmwkClass:\ \hmwkTitle}}\\
    \normalsize\vspace{0.1in}\small\vspace{0.1in}\large{\textit{\hmwkClassInstructor}}
    \vspace{3in}
}

\author{\hmwkAuthorName}
\date{}

\renewcommand{\part}[1]{\textbf{\large Part \Alph{partCounter}}\stepcounter{partCounter}\\}

% Integral dx
\newcommand{\dx}{\mathrm{d}x}

%
% Various Helper Commands
%

% For derivatives
\newcommand{\deriv}[1]{\frac{\mathrm{d}}{\mathrm{d}x} (#1)}

% For partial derivatives
\newcommand{\pderiv}[2]{\frac{\partial}{\partial #1} (#2)}


% Alias for the Solution section header
\newcommand{\solution}{\textbf{\large Solution}}

% Probability commands: Expectation, Variance, Covariance, Bias
\newcommand{\E}{\mathrm{E}}
\newcommand{\Var}{\mathrm{Var}}
\newcommand{\Cov}{\mathrm{Cov}}
\newcommand{\Bias}{\mathrm{Bias}}

% Formatting commands:

\newcommand{\mt}[1]{\ensuremath{#1}}
\newcommand{\nm}[1]{\textrm{#1}}

\newcommand\bsc[2][\DefaultOpt]{%
  \def\DefaultOpt{#2}%
  \section[#1]{#2}%
}
\newcommand\ssc[2][\DefaultOpt]{%
  \def\DefaultOpt{#2}%
  \subsection[#1]{#2}%
}
\newcommand{\bgpf}{\begin{proof} $ $\newline}

\newcommand{\bgeq}{\begin{equation*}}
\newcommand{\eeq}{\end{equation*}}	

\newcommand{\balist}{\begin{enumerate}[label=\alph*.]}
\newcommand{\elist}{\end{enumerate}}

\newcommand{\bilist}{\begin{enumerate}[label=\roman*)]}	

\newcommand{\bgsp}{\begin{split}}
% \newcommand{\esp}{\end{split}} % doesn't work for some reason.

\newcommand\prs[1]{~~~\textbf{(#1)}}

\newcommand{\lt}[1]{\textbf{Let: } #1}
     							   %  if you're setting it to be true
\newcommand{\supp}[1]{\textbf{Suppose: } #1}
     							   %  Suppose (if it'll end up false)
\newcommand{\wts}[1]{\textbf{Want to show: } #1}
     							   %  Want to show
\newcommand{\as}[1]{\textbf{Assume: } #1}
     							   %  if you think it follows from truth
\newcommand{\bpth}[1]{\textbf{(#1)}}

\newcommand{\step}[2]{\begin{equation}\tag{#2}#1\end{equation}}
\newcommand{\epf}{\end{proof}}

\newcommand{\dbs}[3]{\mt{#1_{#2_#3}}}

\newcommand{\sidenote}[1]{-----------------------------------------------------------------Side Note----------------------------------------------------------------
#1 \

---------------------------------------------------------------------------------------------------------------------------------------------}

% Analysis / Logical commands:

\newcommand{\br}{\mt{\mathbb{R}} }       % |R
\newcommand{\bq}{\mt{\mathbb{Q}} }       % |Q
\newcommand{\bn}{\mt{\mathbb{N}} }       % |N
\newcommand{\bc}{\mt{\mathbb{C}} }       % |C
\newcommand{\bz}{\mt{\mathbb{Z}} }       % |Z

\newcommand{\ep}{\mt{\epsilon} }         % epsilon
\newcommand{\fa}{\mt{\forall} }          % for all
\newcommand{\afa}{\mt{\alpha} }
\newcommand{\bta}{\mt{\beta} }
\newcommand{\mem}{\mt{\in} }
\newcommand{\exs}{\mt{\exists} }

\newcommand{\es}{\mt{\emptyset} }        % empty set
\newcommand{\sbs}{\mt{\subset} }         % subset of
\newcommand{\fs}[2]{\{\uw{#1}{1}, \uw{#1}{2}, ... \uw{#1}{#2}\}}

\newcommand{\lra}{ \mt{\longrightarrow} } % implies ----->
\newcommand{\rar}{ \mt{\Rightarrow} }     % implies -->

\newcommand{\lla}{ \mt{\longleftarrow} }  % implies <-----
\newcommand{\lar}{ \mt{\Leftarrow} }      % implies <--

\newcommand{\av}[1]{\mt{|}#1\mt{|}}  % absolute value

\newcommand{\prn}[1]{(#1)}
\newcommand{\bk}[1]{\{#1\}}

\newcommand{\ps}{\mt{+} }
\newcommand{\ms}{\mt{-} }

\newcommand{\ls}{\mt{<} }
\newcommand{\gr}{\mt{>} }

\newcommand{\lse}{\mt{\leq} }
\newcommand{\gre}{\mt{\geq} }

\newcommand{\eql}{\mt{=} }

\newcommand{\pr}{\mt{^\prime} } 		   % prime (i.e. R')
\newcommand{\uw}[2]{#1\mt{_{#2}}}
\newcommand{\uf}[2]{#1\mt{^{#2}}}
\newcommand{\frc}[2]{\mt{\frac{#1}{#2}}}
\newcommand{\lmti}[1]{\mt{\displaystyle{\lim_{#1 \to \infty}}}}
\newcommand{\limt}[2]{\mt{\displaystyle{\lim_{#1 \to #2}}}}

\newcommand{\bnm}[2]{\mt{#1\setminus{#2}}}
\newcommand{\bnt}[2]{\mt{\textrm{#1}\setminus{\textrm{#2}}}}
\newcommand{\bi}{\bnm{\mathbb{R}}{\mathbb{Q}}}

\newcommand{\urng}[2]{\mt{\bigcup_{#1}^{#2}}}
\newcommand{\nrng}[2]{\mt{\bigcap_{#1}^{#2}}}
\newcommand{\nck}[2]{\mt{{#1 \choose #2}}}

\newcommand{\nbho}[3]{\textrm{N(}#1, #2\textrm{) }\cap \textrm{ #3} \neq \emptyset}
     							   %  N(x, eps) intersect S \= emptyset
\newcommand{\nbhe}[3]{\textrm{N(}#1, #2\textrm{) }\cap \textrm{ #3} = \emptyset}
     							   %  N(x, eps) intersect S  = emptyset
\newcommand{\dnbho}[3]{\textrm{N*(}#1, #2\textrm{) }\cap \textrm{ #3} \neq \emptyset}
     							   %  N*(x, eps) intersect S \= emptyset
\newcommand{\dnbhe}[3]{\textrm{N*(}#1, #2\textrm{) }\cap \textrm{ #3} = \emptyset}
     							   %  N*(x, eps) intersect S = emptyset
     							   
\newcommand{\eqn}[1]{\[#1\]}
\newcommand{\splt}[1]{\begin{split}#1\end{split}}
     							 
% ----------

\begin{document}

\bsc{Theorem 4.3.8}{

\balist
\item If \bk{\uw{s}{n}} is an unbounded increasing sequence, then \lmti{n} \uw{s}{n} \eql $\infty$
\item If \bk{\uw{s}{n}} is an unbounded decreasing sequence, then \lmti{n} \uw{s}{n} \eql $-\infty$
\elist

\bgpf

\bpth{a}

Since \uw{s}{1} \lse \uw{s}{n} \fa n \mem \bn 

Thus, if \bk{\uw{s}{n}} is unbounded, then it must be unbounded above.

Thus, for m \mem \br, \exs N \mem \bn st \uw{s}{N} \gr m

Because it's increasing,

\uw{s}{n} \gre \uw{s}{N} \gr m for n \gre N

This is the definition of

Hence,

\lmti{n} \uw{s}{n} \eql $\infty$ \

\

\bpth{b} is similar.

\epf

}

\bsc{Cauchy Sequences}{

\ssc{Definition 4.3.9}{

A sequence \bk{\uw{s}{n}} is \textbf{Cauchy} if \fa \ep \gr 0, \exs N(\ep) \mem \bn st

\eqn{|s_n - s_m| < \ep\textrm{, for m, n }\gre N}

}

\ssc{Lemma 4.3.10}{

Every convergent sequence is Cauchy.

\bgpf

\lt{\bk{\uw{s}{n}} converge to s.}

\fa \ep \gr 0, \exs N(\ep) \mem \bn st 

\eqn{|s_n - s| < \frac{\epsilon}{2}\textrm{, for n }\gre N}

Then

\eqn{|s_n - s_m| = |(s_n - s) + (s - s_m)| \lse |s_n - s| + |s - s_m| < \frac{\epsilon}{2} + \frac{\epsilon}{2} = \ep\textrm{, for n, m }\gre N}

Hence, \bk{\uw{s}{n}} is Cauchy.
\epf

}

\ssc{Lemma 4.3.11}{

Every Cauchy sequence is bounded (similar to exam question: Every convergent sequence is bounded)

\bgpf

This appeared in a similar context in the HW: Example 13, page 186

\epf

}

}

\bsc{Theorem 4.3.12 - Cauchy Convergence Criterion}{

A sequence of real numbers is convergent iff it is a Cauchy sequence.

\bgpf

\lra

Assume that \bk{\uw{s}{n}} is convergent.

Then, by Lemma 4.3.10, \bk{\uw{s}{n}} is Cauchy.

\lla

Conversely, assume that \bk{\uw{s}{n}} is Cauchy.

\wts{\bk{\uw{s}{n}} converges}

\lt{S \eql \bk{\uw{s}{n} : n \mem \bn} be the range of \bk{\uw{s}{n}}}

\bilist
\item S is finite.
	
	Thus, \exs k \mem \bn and \bk{\uw{n}{1}, \uw{n}{2}, ... \uw{n}{k}} \sbs \bn st
	
	\eqn{S = \bk{\dbs{s}{n}{1}, \dbs{s}{n}{2}, ... \dbs{s}{n}{k}}}
	
	Define m:
	
	\eqn{m = \bk{|\dbs{s}{n}{i} - \dbs{s}{n}{j}| : 1 \lse i \lse j \lse k}}
	\eqn{m = \bk{|\dbs{s}{n}{i} - \dbs{s}{n}{j}| : i, j, \mem \bk{n, k}\textrm{ and }i \neq j}}
	
	Now, for \ep \eql \frc{m}{2}, \exs N(\ep) \mem \bn st
	
	\eqn{|s_n - s_m| < \frac{m}{2}\textrm{, for n, m }\gre N}
	
	In particular,
	
	\step{|s_n - s_N| < \frac{m}{2}\textrm{ for } n \gre N}{1}
	
	Now, \exs l \mem \bk{1, 2, ... k} st \uw{s}{N} \eql \dbs{s}{n}{l}
	
	Thus, \bpth{1} implies that
	
	\eqn{|\uw{s}{n} - \dbs{s}{n}{l}| < \frac{m}{2}\textrm{ for }n \gre N}
	
	Thus, \uw{s}{n} \eql \dbs{s}{n}{l} \fa n \gre N
	
	Hence, \lmti{n} \uw{s}{n} \eql \dbs{s}{n}{l}
\item S is infinite.
	
	Since \bk{\uw{s}{n}} is Cauchy, it follows by Lemma 4.3.11 that S is bounded.
	
	By the Bolzano-Weierstrass theorem, 
	
	\exs s mem \br st
	
	s \mem S\pr (i.e. s is an accumulation point of S)
	
	\wts{\lmti{n} \eql s}
	
	For \ep \gr 0, \exs N(\ep) \mem \bn st
	
	\step{|s_n - s_m| < \frac{\epsilon}{2}\textrm{ for n, m}\gre N}{1}
	
	Also by Exercise 15, Section 1.4, page 142, since s \mem S\pr,
	
	Every deleted neighborhood of s, N*(s,$\epsilon$) contains \textbf{an infinite number} of points from S.
	
	Since there are an infinite number of points in N*(s,$\epsilon$), it's totally reasonable that there are an infinite number of points in N*(s,$\frac{\epsilon}{2}$)
	
	Thus, \exs m \mem \bn with M \gre N st
	
	\uw{s}{m} \mem N(s,$\frac{\epsilon}{2}$)
	
	So,
	
	\step{|s - s_m| < \frac{\epsilon}{2}}{2}
	
	From \bpth{1} and \bpth{2},
	
	\eqn{
		\splt{
			|s - s_n| & = |(s - s_m) + (s_m - s_n)| \\
			& \lse |s - s_m| + |s_m - s_n| \\
			& < \frac{\epsilon}{2} + \frac{\epsilon}{2} = \epsilon \\
			& \fa n \gre N
		}
	}
	
	Hence, \lmti{n} \uw{s}{n} \eql s, which completes the proof.
	
\elist

\epf

}

\sidenote{
Better ratio test:

\bk{\uw{s}{n}} is a sequence.

Test:

\lmti{n} \frc{|s_{n + 1}|}{|s_n|} \eql L \ls 1

If \lmti{n} \av{\uw{s}{n}} \eql 0,

\lmti{n} \av{\uw{s}{n}} \eql 0 \eql \lmti{n} \av{\uw{s}{n} \ms 0} \eql \lmti{n} \uw{s}{n} \eql 0

The reason is because, if you're not careful, you can conclude something like, say, \lmti{n} \uw{s}{n} \eql $(-2)^n$ \eql 0

\uw{s}{n} \eql $(-2)^n$

\frc{\uw{s}{n + 1}}{\uw{s}{n}} \eql $\frac{(-2)^{n + 1}}{(-2)^n}$ \eql $-2$ \ls 1

which would tell you, in theory, that the limit is 0. Which is \textbf{not} true.
}

\bsc{Example 4.3.13}{

Show that the harmonic series \uw{s}{n} \eql $\sum_{n = 1}^\infty$ \frc{1}{n} diverges.

\eqn{\frac{1}{1} + \frac{1}{2} + \frac{1}{3} + ... + \frac{1}{n}}

\textbf{Solution:}

Let n \mem \bn and

\eqn{\uw{s}{n} = 1 + \frac{1}{2} + \frac{1}{3} + ... + \frac{1}{n} + \frac{1}{n + 1} + ... + \frac{1}{m}}

Then, for m \gr n,

\eqn{
	\splt{
	|s_n - s_m| & = s_m - s_n \\
	& = \frac{1}{n + 1} + \frac{1}{n + 2} + ... + \frac{1}{n + (m - n)} \\
	& > \frac{m - n}{n + (m - n)} = \frac{m - n}{m}
	}
}

So, if m \eql 2n, then

\eqn{|s_n - s_{2n}| > \frac{2n - n}{2n} = \frac{n}{2n} = \frac{1}{2}\textrm{, }\fa n \mem \bn}

This tells me that \bk{\uw{s}{n}} is \textbf{not} Cauchy.

Hence, by the Cauchy Convergence Criterion, \av{\uw{s}{n}} diverges.

Notice that \bk{\uw{s}{n}} is a monotonically increasing sequence that is unbounded.

So, by Theorem 4.3.8(a), \lmti{n} \uw{s}{n} \eql $\infty$

}

\bsc{4.4.1 Subsequences}{

\ssc{Definition 4.4.1}{

\lt{\bk{\uw{s}{n}}$^\infty_{n = 1}$ be a sequence}

Also, let \bk{\uw{n}{k}} be a sequence \mem \bn st

\eqn{n_1 < n_2 < n_3 ...}

The sequence \bk{\dbs{s}{n}{k}}$^\infty_{k = 1}$ is called a \textbf{subsequence} of \bk{\uw{s}{n}}.

Notice that, in this case, \uw{n}{k} \gre k (i.e. k \lse \uw{t}{k}) \fa k \mem \bn

Thus, \lmti{n} \uw{n}{k} \eql $\infty$

\sidenote{
If \uw{s}{n} \lse \uw{t}{n} \fa n \mem \bn, and

if \lmti{n} \uw{s}{n} \eql $\infty$,

then \lmti{n} \uw{t}{n} \eql $\infty$
}

}

\ssc{Practice 4.4.3}{
Let \bk{\uw{n}{k}} be a sequence in \bn such that \uw{n}{k} \ls \uw{n}{k + 1} \fa k \mem \bn.

Use induction to prove that \uw{n}{k} \gre k, \fa k \mem \bn 

\textbf{Solution:}

Notice that 1 \lse \uw{n}{1}

For l \mem \bn, 

assume that l \lse \uw{n}{l}

Now, consider l \ps 1 \lse \uw{n}{l} \ps 1

\sidenote{
\uw{n}{k} \ls \uw{n}{k + 1}

\uw{n}{k} \ps 1 \lse \uw{n}{k + 1}
}

So, l \ps 1 \lse \uw{n}{l} \ps 1 \lse \uw{n}{l + 1}

Hence,

k \lse \uw{n}{k} \fa k \mem \bn 
}
}
\end{document}