% Thank you Josh Davis for this template!
% https://github.com/jdavis/latex-homework-template/blob/master/homework.tex

\documentclass{article}

\newcommand{\hmwkTitle}{HW\ \#10}

% % ----------

% Packages

\usepackage{fancyhdr}
\usepackage{extramarks}
\usepackage{amsmath}
\usepackage{amssymb}
\usepackage{amsthm}
\usepackage{amsfonts}
\usepackage{tikz}
\usepackage[plain]{algorithm}
\usepackage{algpseudocode}
\usepackage{enumitem}
\usepackage{chngcntr}

% Libraries

\usetikzlibrary{automata, positioning, arrows}

%
% Basic Document Settings
%

\topmargin=-0.45in
\evensidemargin=0in
\oddsidemargin=0in
\textwidth=6.5in
\textheight=9.0in
\headsep=0.25in

\linespread{1.1}

\pagestyle{fancy}
\lhead{\hmwkAuthorName}
\chead{}
\rhead{\hmwkClass\ (\hmwkClassInstructor): \hmwkTitle}
\lfoot{\lastxmark}
\cfoot{\thepage}

\renewcommand\headrulewidth{0.4pt}
\renewcommand\footrulewidth{0.4pt}

\setlength\parindent{0pt}
\setcounter{secnumdepth}{0}

\newcommand{\hmwkClass}{MATH 3380 / Analysis 1}        % Class
\newcommand{\hmwkClassInstructor}{Dr. Welsh}           % Instructor
\newcommand{\hmwkAuthorName}{\textbf{Joshua Mitchell}} % Author

%
% Title Page
%

\title{
    \vspace{2in}
    \textmd{\textbf{\hmwkClass:\ \hmwkTitle}}\\
    \normalsize\vspace{0.1in}\small\vspace{0.1in}\large{\textit{\hmwkClassInstructor}}
    \vspace{3in}
}

\author{\hmwkAuthorName}
\date{}

\renewcommand{\part}[1]{\textbf{\large Part \Alph{partCounter}}\stepcounter{partCounter}\\}

% Integral dx
\newcommand{\dx}{\mathrm{d}x}

%
% Various Helper Commands
%

% For derivatives
\newcommand{\deriv}[1]{\frac{\mathrm{d}}{\mathrm{d}x} (#1)}

% For partial derivatives
\newcommand{\pderiv}[2]{\frac{\partial}{\partial #1} (#2)}


% Alias for the Solution section header
\newcommand{\solution}{\textbf{\large Solution}}

% Probability commands: Expectation, Variance, Covariance, Bias
\newcommand{\E}{\mathrm{E}}
\newcommand{\Var}{\mathrm{Var}}
\newcommand{\Cov}{\mathrm{Cov}}
\newcommand{\Bias}{\mathrm{Bias}}

% Formatting commands:

\newcommand{\mt}[1]{\ensuremath{#1}}
\newcommand{\nm}[1]{\textrm{#1}}

\newcommand\bsc[2][\DefaultOpt]{%
  \def\DefaultOpt{#2}%
  \section[#1]{#2}%
}
\newcommand\ssc[2][\DefaultOpt]{%
  \def\DefaultOpt{#2}%
  \subsection[#1]{#2}%
}
\newcommand{\bgpf}{\begin{proof} $ $\newline}

\newcommand{\bgeq}{\begin{equation*}}
\newcommand{\eeq}{\end{equation*}}	

\newcommand{\balist}{\begin{enumerate}[label=\alph*.]}
\newcommand{\elist}{\end{enumerate}}

\newcommand{\bilist}{\begin{enumerate}[label=\roman*)]}	

\newcommand{\bgsp}{\begin{split}}
% \newcommand{\esp}{\end{split}} % doesn't work for some reason.

\newcommand\prs[1]{~~~\textbf{(#1)}}

\newcommand{\lt}[1]{\textbf{Let: } #1}
     							   %  if you're setting it to be true
\newcommand{\supp}[1]{\textbf{Suppose: } #1}
     							   %  Suppose (if it'll end up false)
\newcommand{\wts}[1]{\textbf{Want to show: } #1}
     							   %  Want to show
\newcommand{\as}[1]{\textbf{Assume: } #1}
     							   %  if you think it follows from truth
\newcommand{\bpth}[1]{\textbf{(#1)}}

\newcommand{\step}[2]{\begin{equation}\tag{#2}#1\end{equation}}
\newcommand{\epf}{\end{proof}}

\newcommand{\dbs}[3]{\mt{#1_{#2_#3}}}

\newcommand{\sidenote}[1]{-----------------------------------------------------------------Side Note----------------------------------------------------------------
#1 \

---------------------------------------------------------------------------------------------------------------------------------------------}

% Analysis / Logical commands:

\newcommand{\br}{\mt{\mathbb{R}} }       % |R
\newcommand{\bq}{\mt{\mathbb{Q}} }       % |Q
\newcommand{\bn}{\mt{\mathbb{N}} }       % |N
\newcommand{\bc}{\mt{\mathbb{C}} }       % |C
\newcommand{\bz}{\mt{\mathbb{Z}} }       % |Z

\newcommand{\ep}{\mt{\epsilon} }         % epsilon
\newcommand{\fa}{\mt{\forall} }          % for all
\newcommand{\afa}{\mt{\alpha} }
\newcommand{\bta}{\mt{\beta} }
\newcommand{\mem}{\mt{\in} }
\newcommand{\exs}{\mt{\exists} }

\newcommand{\es}{\mt{\emptyset}}        % empty set
\newcommand{\sbs}{\mt{\subset} }         % subset of
\newcommand{\fs}[2]{\{\uw{#1}{1}, \uw{#1}{2}, ... \uw{#1}{#2}\}}

\newcommand{\lra}{ \mt{\longrightarrow} } % implies ----->
\newcommand{\rar}{ \mt{\Rightarrow} }     % implies -->

\newcommand{\lla}{ \mt{\longleftarrow} }  % implies <-----
\newcommand{\lar}{ \mt{\Leftarrow} }      % implies <--

\newcommand{\eql}{\mt{=} }
\newcommand{\pr}{\mt{^\prime} } 		   % prime (i.e. R')
\newcommand{\uw}[2]{#1\mt{_{#2}}}
\newcommand{\frc}[2]{\mt{\frac{#1}{#2}}}

\newcommand{\bnm}[2]{\mt{#1\setminus{#2}}}
\newcommand{\bnt}[2]{\mt{\textrm{#1}\setminus{\textrm{#2}}}}
\newcommand{\bi}{\bnm{\mathbb{R}}{\mathbb{Q}}}

\newcommand{\urng}[2]{\mt{\bigcup_{#1}^{#2}}}
\newcommand{\nrng}[2]{\mt{\bigcap_{#1}^{#2}}}

\newcommand{\nbho}[3]{\textrm{N(}#1, #2\textrm{) }\cap \textrm{ #3} \neq \emptyset}
     							   %  N(x, eps) intersect S \= emptyset
\newcommand{\nbhe}[3]{\textrm{N(}#1, #2\textrm{) }\cap \textrm{ #3} = \emptyset}
     							   %  N(x, eps) intersect S  = emptyset
\newcommand{\dnbho}[3]{\textrm{N*(}#1, #2\textrm{) }\cap \textrm{ #3} \neq \emptyset}
     							   %  N*(x, eps) intersect S \= emptyset
\newcommand{\dnbhe}[3]{\textrm{N*(}#1, #2\textrm{) }\cap \textrm{ #3} = \emptyset}
     							   %  N*(x, eps) intersect S = emptyset
     							 


% ----------

% ----------

% Packages

\usepackage{fancyhdr}
\usepackage{extramarks}
\usepackage{amsmath}
\usepackage{amssymb}
\usepackage{amsthm}
\usepackage{amsfonts}
\usepackage{tikz}
\usepackage[plain]{algorithm}
\usepackage{algpseudocode}
\usepackage{enumitem}
\usepackage{chngcntr}

% Libraries

\graphicspath{{/Users/jm/iclouddrive/3380pics/}}

\usetikzlibrary{automata, positioning, arrows}

%
% Basic Document Settings
%

\topmargin=-0.45in
\evensidemargin=0in
\oddsidemargin=0in
\textwidth=6.5in
\textheight=9.0in
\headsep=0.25in

\linespread{1.1}

\pagestyle{fancy}
\lhead{\hmwkAuthorName}
\chead{}
\rhead{\hmwkClass\ (\hmwkClassInstructor): \hmwkTitle}
\lfoot{\lastxmark}
\cfoot{\thepage}

\renewcommand\headrulewidth{0.4pt}
\renewcommand\footrulewidth{0.4pt}

\setlength\parindent{0pt}
\setcounter{secnumdepth}{0}

\newcommand{\hmwkClass}{MATH 3380 / Analysis 1}        % Class
\newcommand{\hmwkClassInstructor}{Dr. Welsh}           % Instructor
\newcommand{\hmwkAuthorName}{\textbf{Joshua Mitchell}} % Author

%
% Title Page
%

\title{
    \vspace{2in}
    \textmd{\textbf{\hmwkClass:\ \hmwkTitle}}\\
    \normalsize\vspace{0.1in}\small\vspace{0.1in}\large{\textit{\hmwkClassInstructor}}
    \vspace{3in}
}

\author{\hmwkAuthorName}
\date{}

\renewcommand{\part}[1]{\textbf{\large Part \Alph{partCounter}}\stepcounter{partCounter}\\}

% Integral dx
\newcommand{\dx}{\mathrm{d}x}

%
% Various Helper Commands
%

% For derivatives
\newcommand{\deriv}[1]{\frac{\mathrm{d}}{\mathrm{d}x} (#1)}

% For partial derivatives
\newcommand{\pderiv}[2]{\frac{\partial}{\partial #1} (#2)}


% Alias for the Solution section header
\newcommand{\solution}{\textbf{\large Solution}}

% Probability commands: Expectation, Variance, Covariance, Bias
\newcommand{\E}{\mathrm{E}}
\newcommand{\Var}{\mathrm{Var}}
\newcommand{\Cov}{\mathrm{Cov}}
\newcommand{\Bias}{\mathrm{Bias}}

% Formatting commands:

\newcommand{\mt}[1]{\ensuremath{#1}}
\newcommand{\nm}[1]{\textrm{#1}}

\newcommand\bsc[2][\DefaultOpt]{%
  \def\DefaultOpt{#2}%
  \section[#1]{#2}%
}
\newcommand\ssc[2][\DefaultOpt]{%
  \def\DefaultOpt{#2}%
  \subsection[#1]{#2}%
}
\newcommand{\bgpf}{\begin{proof} $ $\newline}

\newcommand{\bgeq}{\begin{equation*}}
\newcommand{\eeq}{\end{equation*}}	

\newcommand{\balist}{\begin{enumerate}[label=\alph*.]}
\newcommand{\elist}{\end{enumerate}}

\newcommand{\bilist}{\begin{enumerate}[label=\roman*)]}	

\newcommand{\bgsp}{\begin{split}}
% \newcommand{\esp}{\end{split}} % doesn't work for some reason.

\newcommand\prs[1]{~~~\textbf{(#1)}}

\newcommand{\lt}[1]{\textbf{Let: } #1}
     							   %  if you're setting it to be true
\newcommand{\supp}[1]{\textbf{Suppose: } #1}
     							   %  Suppose (if it'll end up false)
\newcommand{\wts}[1]{\textbf{Want to show: } #1}
     							   %  Want to show
\newcommand{\as}[1]{\textbf{Assume: } #1}
     							   %  if you think it follows from truth
\newcommand{\bpth}[1]{\textbf{(#1)}}

\newcommand{\step}[2]{\begin{equation}\tag{#2}#1\end{equation}}
\newcommand{\epf}{\end{proof}}

\newcommand{\dbs}[3]{\mt{#1_{#2_#3}}}

\newcommand{\sidenote}[1]{-----------------------------------------------------------------Side Note----------------------------------------------------------------
#1 \

---------------------------------------------------------------------------------------------------------------------------------------------}

% Analysis / Logical commands:

\newcommand{\br}{\mt{\mathbb{R}} }       % |R
\newcommand{\bq}{\mt{\mathbb{Q}} }       % |Q
\newcommand{\bn}{\mt{\mathbb{N}} }       % |N
\newcommand{\bc}{\mt{\mathbb{C}} }       % |C
\newcommand{\bz}{\mt{\mathbb{Z}} }       % |Z

\newcommand{\ep}{\mt{\epsilon} }         % epsilon
\newcommand{\fa}{\mt{\forall} }          % for all
\newcommand{\afa}{\mt{\alpha} }
\newcommand{\bta}{\mt{\beta} }
\newcommand{\dta}{\mt{\delta} }
\newcommand{\mem}{\mt{\in} }
\newcommand{\exs}{\mt{\exists} }

\newcommand{\es}{\mt{\emptyset} }        % empty set
\newcommand{\sbs}{\mt{\subset} }         % subset of
\newcommand{\fs}[2]{\{\uw{#1}{1}, \uw{#1}{2}, ... \uw{#1}{#2}\}}

\newcommand{\lra}{ \mt{\longrightarrow} } % implies ----->
\newcommand{\rar}{ \mt{\Rightarrow} }     % implies -->

\newcommand{\lla}{ \mt{\longleftarrow} }  % implies <-----
\newcommand{\lar}{ \mt{\Leftarrow} }      % implies <--

\newcommand{\av}[1]{\mt{|}#1\mt{|}}  % absolute value

\newcommand{\prn}[1]{(#1)}
\newcommand{\bk}[1]{\{#1\}}

\newcommand{\ps}{\mt{+} }
\newcommand{\ms}{\mt{-} }

\newcommand{\ls}{\mt{<} }
\newcommand{\gr}{\mt{>} }

\newcommand{\lse}{\mt{\leq} }
\newcommand{\gre}{\mt{\geq} }

\newcommand{\eql}{\mt{=} }

\newcommand{\pr}{\mt{^\prime} } 		   % prime (i.e. R')
\newcommand{\uw}[2]{#1\mt{_{#2}}}
\newcommand{\uf}[2]{#1\mt{^{#2}}}
\newcommand{\frc}[2]{\mt{\frac{#1}{#2}}}
\newcommand{\lmti}[1]{\mt{\displaystyle{\lim_{#1 \to \infty}}}}
\newcommand{\limt}[2]{\mt{\displaystyle{\lim_{#1 \to #2}}}}

\newcommand{\bnm}[2]{\mt{#1\setminus{#2}}}
\newcommand{\bnt}[2]{\mt{\textrm{#1}\setminus{\textrm{#2}}}}
\newcommand{\bi}{\bnm{\mathbb{R}}{\mathbb{Q}}}

\newcommand{\urng}[2]{\mt{\bigcup_{#1}^{#2}}}
\newcommand{\nrng}[2]{\mt{\bigcap_{#1}^{#2}}}
\newcommand{\nck}[2]{\mt{{#1 \choose #2}}}

\newcommand{\nbho}[3]{\textrm{N(}#1, #2\textrm{) }\cap \textrm{ #3} \neq \emptyset}
     							   %  N(x, eps) intersect S \= emptyset
\newcommand{\nbhe}[3]{\textrm{N(}#1, #2\textrm{) }\cap \textrm{ #3} = \emptyset}
     							   %  N(x, eps) intersect S  = emptyset
\newcommand{\dnbho}[3]{\textrm{N*(}#1, #2\textrm{) }\cap \textrm{ #3} \neq \emptyset}
     							   %  N*(x, eps) intersect S \= emptyset
\newcommand{\dnbhe}[3]{\textrm{N*(}#1, #2\textrm{) }\cap \textrm{ #3} = \emptyset}
     							   %  N*(x, eps) intersect S = emptyset
     							   
\newcommand{\eqn}[1]{\[#1\]}
\newcommand{\splt}[1]{\begin{split}#1\end{split}}

\newcommand{\infy}{\mt{\infty} }
\newcommand{\unn}{\mt{\cup} }
\newcommand{\inn}{\mt{\cap} }
\newcommand\tab[1][1cm]{\hspace*{#1}}

     							 
% ----------

\begin{document}

\bsc{Exercise 1}{

Let f : D \lra \br and let c \mem D. Mark each statement True or False. Justify each answer.

\balist
\item f is continuous at c iff \fa \ep \gr 0, \exs a \dta \gr 0 such that \av{f(x) \ms f(c)} \ls \ep whenever \av{x \ms c} \ls \dta and x \mem D
	
	\textbf{True.} By definition of continuous.
\item if f(D) is a bounded set, then f is continuous on D
	
	\textbf{False.}
	
	\lt{f : D \lra \br be defined by D = \bk{\br} and f \eql \bk{1 if x $\neq$ 0, 0 otherwise.}}
	
	Pick \ep \eql 0.5. Notice that there is no \dta such that \av{f(x) \ms f(0)} \ls 0.5
\item if c is an isolated point of D, then f is continuous at c
	
	\textbf{True.} If you just pick a \dta st only x \eql c fits in \av{x \ms c} \ls \dta (which is possible since it's an isolated point), then that works for any \ep \gr 0 since \av{f(x) \ms f(c)} will always be 0.
\item if f is continuous at c and \prn{\uw{x}{n}} is a sequence in D, then \uw{x}{n} \lra c whenever f(\uw{x}{n}) \lra f(c)

	\textbf{True.}
	
	So, what we are asking is:
	
	\bpth{1} f is continuous at c, \bpth{2} \prn{\uw{x}{n}} is a sequence in D, and \bpth{3} f(\uw{x}{n}) \lra f(c) implies \uw{x}{n} \lra c 
	
	True or false?
	
	\wts{\fa \ep \gr 0, \exs N \mem \bn st n \gre N implies \av{\uw{x}{n} \ms c} \ls \ep}
	
	\fa \ep \gr 0, \exs \dta \gr 0 st x \mem D and \av{x \ms c} \ls \dta implies \av{f(x) \ms f(c)} \ls \ep
	
	So, if we let x \eql \uw{x}{n} (since \uw{x}{n} \mem D),
	
	For \ep \gr 0, \exs \dta \gr 0 st \uw{x}{n} \mem D and \av{\uw{x}{n} \ms c} \ls \dta implies \av{f(\uw{x}{n}) \ms f(c)} \ls \ep \bpth{4}
	
	We also know that, by \bpth{3},
	
	\fa \ep \gr 0, \exs N \mem \bn st n \gre N implies \av{f(\uw{x}{n}) \ms f(c)} \ls \ep
	
	So, by \bpth{1} and \bpth{3},
	
	For \ep \gr 0, \exs \dta \gr 0 st \uw{x}{n} \mem D and \av{\uw{x}{n} \ms c} \ls \dta implies \av{f(\uw{x}{n}) \ms f(c)} \ls \ep
	
	and for this same \ep, there is an N \mem \bn st n \gre N implies the same.
	
	\textbf{I'm not sure at this point, but I think that \exs N \mem n st n \gre N implies \av{f(\uw{x}{n}) \ms f(c)} \ls \ep implies that \av{\uw{x}{n} \ms c} goes to 0 as well.}
	
\item if f is continuous at c, then for every neighborhood V of f(c), there exists a neighborhood U of c such that f(U \inn D) \eql V
	
	\textbf{True. By Theorem 5.2.2 (c).}
\elist

}

\newpage

\bsc{Exercise 2 (omit d)}{

Let f : D \lra R and let c \mem D. Mark each statement True or False. Justify each answer.

\balist
\item if f is continuous at c and c is an accumulation point of D, then \limt{x}{c} f(x) \eql f(c)
	
	\textbf{True, by Theorem 5.2.2 (d).}
\item Every polynomial is continuous at each point in \br
	
	\textbf{True, by Theorem 5.1.13. Since every polynomial can be obtained by Theorem 5.1.13's operations on continuous functions, every polynomial is continuous as well.}
\item if \bk{(\uw{x}{n})} is a Cauchy sequence in D, then \bk{f(\uw{x}{n})} is convergent.
	
	\textbf{False.}
	
	It is true that all Cauchy sequences are convergent, but just because \bk{\uw{x}{n}} is a sequence in D doesn't mean that \uw{x}{n} converges to L st L \mem D. 
	
	If L $\not\in$ D, then f(L) won't be defined.
	
	In order for \bk{f(\uw{x}{n})} to be convergent, that has to be the case. It can't converge to an undefined number.
\item if f : \br \lra \br and g : \br \lra \br are both continuous on \br, then f o g and g o f are both continuous on \br
	
	\textbf{True. Since f and g are defined for all real numbers, f o g and g o f are also defined for all real numbers. Since both functions are continuous everywhere, there are no discontinuities no matter what input either function takes (even if the input is the output of another function, i.e. a composition).}
	
	\textbf{Also true by Theorem 5.2.12.}
\elist
}

\bsc{Exercise 3}{

\lt{f(x) \eql \prn{\uf{x}{2} \ps 4x \ms 21}/\prn{x \ms 3} for x $\neq$ 3.}

How should f(3) be defined so that f will be continuous at 3?
\eqn{f(x) = \frac{x^2 + 4x - 21}{x - 3} = \frac{(x + 7)(x - 3)}{(x - 3)}}
\textbf{f(3) should be defined as 10.}

}

\newpage

\bsc{Exercise 5 (prove the result)}{

Find an example of a function f : \br \lra \br that is continuous at exactly one point.

\lt{f(x) \eql 1 and D \eql \bk{0}}

In order for f(x) to be continuous, it must satisfy this definition:

\fa \ep \gr 0, \exs \dta \gr 0 st x \mem D and \av{x \ms c} \ls \dta implies \av{f(x) \ms f(c)} \ls \ep

If we let c \eql 0, notice that f(0) \eql 1, 0 is in the domain, and \av{f(x) \ms f(c)} \ls \ep for any epsilon

(since f(x) and f(c) will always be the same number under our definitions).

Notice also that f(x) is only defined at x \eql 0.

Since f(x) is continuous, and f(x) is only defined at one point, f(x) is only continuous at one point.
}

\bsc{Exercise 10}{

\balist
\item Let f : D \lra \br and define \av{f} : D \lra \br by \av{f}(x) \eql \av{f(x)}. Suppose that f(x) is continuous at c \mem D. Prove that \av{f} is continuous at c.
	
	\wts{\fa \ep \gr 0, \exs \dta \gr 0 st x \mem D and \av{x \ms c} \ls \dta implies \av{\av	{f}(x) \ms \av{f}(c)} \ls \ep}
	
	Since f(x) is continuous at c \mem D,
	
	\fa \ep \gr 0, \exs \dta \gr 0 st x \mem D and \av{x \ms c} \ls \dta implies \av{f(x) \ms f(c)} \ls \ep
	
	By the triangle inequality,
	
	\av{\av{f(x)} \ms \av{f(c)}} \lse \av{f(x) \ms f(c)}
	
	Since \av{f}(x) \eql \av{f(x)},
	
	\av{\av{f(x)} \ms \av{f(c)}} \lse \av{f(x) \ms f(c)} \ls \ep
	
	\av{\av{f(x)} \ms \av{f(c)}} \ls \ep
	
	\av{\av{f}(x) \ms \av{f}(c)} \ls \ep
	
	Hence, \av{f} is continuous at c.
	
	
\item if \av{f} is continuous at c, does it follow that f is continuous at c? Justify your answer.
	
	\textbf{No.}
	
	For example, if f(x) \eql \bk{1 for x $\neq$ 0, $-$1 otherwise}, \av{f}(x) is defined as 1 for all x \mem \br, making it continuous everywhere. However, f(x) is not continuous at x \eql 0.
\elist

}

\newpage

\bsc{Exercise 11 (just prove the "max" result)}{

Define max(f, g) and min(f, g) as in Example 2.11. 

\textbf{Example 2.11}

max(f, g)(x) \eql max \bk{f(x), g(x)}

\

Show that:

max(f, g) \eql \frc{1}{2}(f \ps g) \ps \frc{1}{2}\av{f \ms g}

\

\lt{h \eql \frc{1}{2}(f \ps g) \ps \frc{1}{2}\av{f \ms g}}

h \ms \frc{1}{2}(f \ps g) \eql \frc{1}{2}\av{f \ms g}

2h \ms (f \ps g) \eql \av{f \ms g}

So,

Case
\bilist
\item 2h \ms (f \ps g) \eql f \ms g (i.e. f \ms g is non-negative and f \gre g)

	2h \eql f \ms g \ps f \ps g
	
	2h \eql 2f
	
	h \eql f
\item $-$(2h \ms (f \ps g)) \eql f \ms g (i.e. f \ms g is negative and g \gr f)
	
	2h \ms (f \ps g) \eql g \ms f
	
	2h \eql g \ms f \ps f \ps g
	
	2h \eql 2g
	
	h \eql g
\elist

Hence, result.

}

\bsc{Exercise 13}{

\lt{f : D \lra \br be continuous at c \mem D}

\as{f(c) \gr 0}

Prove that \exs \afa \gr 0 and a neighborhood U of c st f(x) \gr \afa, \fa x \mem U \inn D

\wts{\exs a neighborhood V of f(c) st \fa v \mem V, v \gr 0}

Since f(c) \gr 0, if we let \ep \eql \frc{f(c) \ms 0}{4} and V \eql N(f(c), $\epsilon$), then we can see that \fa v \mem V, v \gr 0

(which we can do since f is continuous at c)

By Theorem 5.2.2 (c), for this neighborhood V, \exs a neighborhood U of c st f(U \inn D) \sbs V

\lt{\afa \eql \frc{f(c)}{2} \gr 0}

Notice that \fa v \mem V, \afa \ls v

Notice also that f(U \inn D) \sbs V

Thus,

f(x) \gr \afa \fa x \mem U \inn D

}

\newpage

\bsc{Exercise 16}{

\textbf{I got super lost on this problem..}

(First prove that for any H \sbs \br, \uf{f}{-1}(\bnt{\br}{H}) \eql \bnt{\br}{\uf{f}{-1}(H)}, use this in conjunction with Theorem 5.2.14)

\lt{f : \br \lra \br}

Prove that f is continuous on \br iff \uf{f}{-1}(H) is a closed set whenever H is a closed set.

(vs H is a closed set whenever f(H) is a closed set?)

\bgpf
\lra

\lt{f : \br \lra \br be continuous on \br}

\wts{H is a closed set \rar \uf{f}{-1}(H) is a closed set}

\textbf{Theorem 5.2.14} (since D \eql \br):

for every open set H \sbs \br, \exs an open set G \sbs \br st \uf{f}{-1}(H) \eql G

\textbf{Corollary 5.2.15:}

A function f : \br \lra \br is continuous iff \uf{f}{-1}(G) is open in \br whenever G is open in \br

\

So at this point we know that:

G is open in \br \rar \uf{f}{-1}(G) is open in \br

\

\lt{H \sbs \br be open.}

\wts{for any H \sbs \br, \uf{f}{-1}(\bnt{\br}{H}) \eql \bnt{\br}{\uf{f}{-1}(H)}}

\lt{x \mem \uf{f}{-1}(\bnt{\br}{H}}

so, f(x) \mem \bnt{\br}{H}

\supp{x \mem \uf{f}{-1}(H)}

Then f(x) \mem H.

So, x $\not\in$ \uf{f}{-1}(H) or, equivalently, x \mem \bnt{\br}{\uf{f}{-1}(H)}



***Somehow, from here, we show:

\step{\textrm{for } H \sbs \br, \tab \uf{f}{-1}(\bnt{\br}{H}) \eql \bnt{\br}{\uf{f}{-1}(H)}}{1}

\supp{\uf{f}{-1}(H) is closed}

Then, \bnt{\br}{\uf{f}{-1}(H)} must be open.

Which means that \uf{f}{-1}(\bnt{\br}{H}) is open.

We also know that \bnt{\br}{H} is closed since H is open.

However, by Corollary 5.2.15, 

if \bnt{\br}{H} is open then \uf{f}{-1}(\bnt{\br}{H}) must be open,

a contradiction.

So, \uf{f}{-1}(H) must be open.

\

\lla

\lt{\uf{f}{-1}(H) be a closed set whenever H is a closed set}

So,

H is an open set whenever \uf{f}{-1}(H) is an open set.

\wts{for any H \sbs \br, \uf{f}{-1}(\bnt{\br}{H}) \eql \bnt{\br}{\uf{f}{-1}(H)}}

Then,

\wts{for every open set G \sbs \br, \exs an open set H \sbs \br st \uf{f}{-1}(G) \eql H \inn D (which implies continuity on \br)}

\epf

}

\end{document}