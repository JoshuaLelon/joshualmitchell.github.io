% Thank you Josh Davis for this template!
% https://github.com/jdavis/latex-homework-template/blob/master/homework.tex

\documentclass{article}

\newcommand{\hmwkTitle}{Lec\ \#15}

% % ----------

% Packages

\usepackage{fancyhdr}
\usepackage{extramarks}
\usepackage{amsmath}
\usepackage{amssymb}
\usepackage{amsthm}
\usepackage{amsfonts}
\usepackage{tikz}
\usepackage[plain]{algorithm}
\usepackage{algpseudocode}
\usepackage{enumitem}
\usepackage{chngcntr}

% Libraries

\usetikzlibrary{automata, positioning, arrows}

%
% Basic Document Settings
%

\topmargin=-0.45in
\evensidemargin=0in
\oddsidemargin=0in
\textwidth=6.5in
\textheight=9.0in
\headsep=0.25in

\linespread{1.1}

\pagestyle{fancy}
\lhead{\hmwkAuthorName}
\chead{}
\rhead{\hmwkClass\ (\hmwkClassInstructor): \hmwkTitle}
\lfoot{\lastxmark}
\cfoot{\thepage}

\renewcommand\headrulewidth{0.4pt}
\renewcommand\footrulewidth{0.4pt}

\setlength\parindent{0pt}
\setcounter{secnumdepth}{0}

\newcommand{\hmwkClass}{MATH 3380 / Analysis 1}        % Class
\newcommand{\hmwkClassInstructor}{Dr. Welsh}           % Instructor
\newcommand{\hmwkAuthorName}{\textbf{Joshua Mitchell}} % Author

%
% Title Page
%

\title{
    \vspace{2in}
    \textmd{\textbf{\hmwkClass:\ \hmwkTitle}}\\
    \normalsize\vspace{0.1in}\small\vspace{0.1in}\large{\textit{\hmwkClassInstructor}}
    \vspace{3in}
}

\author{\hmwkAuthorName}
\date{}

\renewcommand{\part}[1]{\textbf{\large Part \Alph{partCounter}}\stepcounter{partCounter}\\}

% Integral dx
\newcommand{\dx}{\mathrm{d}x}

%
% Various Helper Commands
%

% For derivatives
\newcommand{\deriv}[1]{\frac{\mathrm{d}}{\mathrm{d}x} (#1)}

% For partial derivatives
\newcommand{\pderiv}[2]{\frac{\partial}{\partial #1} (#2)}


% Alias for the Solution section header
\newcommand{\solution}{\textbf{\large Solution}}

% Probability commands: Expectation, Variance, Covariance, Bias
\newcommand{\E}{\mathrm{E}}
\newcommand{\Var}{\mathrm{Var}}
\newcommand{\Cov}{\mathrm{Cov}}
\newcommand{\Bias}{\mathrm{Bias}}

% Formatting commands:

\newcommand{\mt}[1]{\ensuremath{#1}}
\newcommand{\nm}[1]{\textrm{#1}}

\newcommand\bsc[2][\DefaultOpt]{%
  \def\DefaultOpt{#2}%
  \section[#1]{#2}%
}
\newcommand\ssc[2][\DefaultOpt]{%
  \def\DefaultOpt{#2}%
  \subsection[#1]{#2}%
}
\newcommand{\bgpf}{\begin{proof} $ $\newline}

\newcommand{\bgeq}{\begin{equation*}}
\newcommand{\eeq}{\end{equation*}}	

\newcommand{\balist}{\begin{enumerate}[label=\alph*.]}
\newcommand{\elist}{\end{enumerate}}

\newcommand{\bilist}{\begin{enumerate}[label=\roman*)]}	

\newcommand{\bgsp}{\begin{split}}
% \newcommand{\esp}{\end{split}} % doesn't work for some reason.

\newcommand\prs[1]{~~~\textbf{(#1)}}

\newcommand{\lt}[1]{\textbf{Let: } #1}
     							   %  if you're setting it to be true
\newcommand{\supp}[1]{\textbf{Suppose: } #1}
     							   %  Suppose (if it'll end up false)
\newcommand{\wts}[1]{\textbf{Want to show: } #1}
     							   %  Want to show
\newcommand{\as}[1]{\textbf{Assume: } #1}
     							   %  if you think it follows from truth
\newcommand{\bpth}[1]{\textbf{(#1)}}

\newcommand{\step}[2]{\begin{equation}\tag{#2}#1\end{equation}}
\newcommand{\epf}{\end{proof}}

\newcommand{\dbs}[3]{\mt{#1_{#2_#3}}}

\newcommand{\sidenote}[1]{-----------------------------------------------------------------Side Note----------------------------------------------------------------
#1 \

---------------------------------------------------------------------------------------------------------------------------------------------}

% Analysis / Logical commands:

\newcommand{\br}{\mt{\mathbb{R}} }       % |R
\newcommand{\bq}{\mt{\mathbb{Q}} }       % |Q
\newcommand{\bn}{\mt{\mathbb{N}} }       % |N
\newcommand{\bc}{\mt{\mathbb{C}} }       % |C
\newcommand{\bz}{\mt{\mathbb{Z}} }       % |Z

\newcommand{\ep}{\mt{\epsilon} }         % epsilon
\newcommand{\fa}{\mt{\forall} }          % for all
\newcommand{\afa}{\mt{\alpha} }
\newcommand{\bta}{\mt{\beta} }
\newcommand{\mem}{\mt{\in} }
\newcommand{\exs}{\mt{\exists} }

\newcommand{\es}{\mt{\emptyset}}        % empty set
\newcommand{\sbs}{\mt{\subset} }         % subset of
\newcommand{\fs}[2]{\{\uw{#1}{1}, \uw{#1}{2}, ... \uw{#1}{#2}\}}

\newcommand{\lra}{ \mt{\longrightarrow} } % implies ----->
\newcommand{\rar}{ \mt{\Rightarrow} }     % implies -->

\newcommand{\lla}{ \mt{\longleftarrow} }  % implies <-----
\newcommand{\lar}{ \mt{\Leftarrow} }      % implies <--

\newcommand{\eql}{\mt{=} }
\newcommand{\pr}{\mt{^\prime} } 		   % prime (i.e. R')
\newcommand{\uw}[2]{#1\mt{_{#2}}}
\newcommand{\frc}[2]{\mt{\frac{#1}{#2}}}

\newcommand{\bnm}[2]{\mt{#1\setminus{#2}}}
\newcommand{\bnt}[2]{\mt{\textrm{#1}\setminus{\textrm{#2}}}}
\newcommand{\bi}{\bnm{\mathbb{R}}{\mathbb{Q}}}

\newcommand{\urng}[2]{\mt{\bigcup_{#1}^{#2}}}
\newcommand{\nrng}[2]{\mt{\bigcap_{#1}^{#2}}}

\newcommand{\nbho}[3]{\textrm{N(}#1, #2\textrm{) }\cap \textrm{ #3} \neq \emptyset}
     							   %  N(x, eps) intersect S \= emptyset
\newcommand{\nbhe}[3]{\textrm{N(}#1, #2\textrm{) }\cap \textrm{ #3} = \emptyset}
     							   %  N(x, eps) intersect S  = emptyset
\newcommand{\dnbho}[3]{\textrm{N*(}#1, #2\textrm{) }\cap \textrm{ #3} \neq \emptyset}
     							   %  N*(x, eps) intersect S \= emptyset
\newcommand{\dnbhe}[3]{\textrm{N*(}#1, #2\textrm{) }\cap \textrm{ #3} = \emptyset}
     							   %  N*(x, eps) intersect S = emptyset
     							 


% ----------

% ----------

% Packages

\usepackage{fancyhdr}
\usepackage{extramarks}
\usepackage{amsmath}
\usepackage{amssymb}
\usepackage{amsthm}
\usepackage{amsfonts}
\usepackage{tikz}
\usepackage[plain]{algorithm}
\usepackage{algpseudocode}
\usepackage{enumitem}
\usepackage{chngcntr}

% Libraries

\graphicspath{{/Users/jm/iclouddrive/3380pics/}}

\usetikzlibrary{automata, positioning, arrows}

%
% Basic Document Settings
%

\topmargin=-0.45in
\evensidemargin=0in
\oddsidemargin=0in
\textwidth=6.5in
\textheight=9.0in
\headsep=0.25in

\linespread{1.1}

\pagestyle{fancy}
\lhead{\hmwkAuthorName}
\chead{}
\rhead{\hmwkClass\ (\hmwkClassInstructor): \hmwkTitle}
\lfoot{\lastxmark}
\cfoot{\thepage}

\renewcommand\headrulewidth{0.4pt}
\renewcommand\footrulewidth{0.4pt}

\setlength\parindent{0pt}
\setcounter{secnumdepth}{0}

\newcommand{\hmwkClass}{MATH 3380 / Analysis 1}        % Class
\newcommand{\hmwkClassInstructor}{Dr. Welsh}           % Instructor
\newcommand{\hmwkAuthorName}{\textbf{Joshua Mitchell}} % Author

%
% Title Page
%

\title{
    \vspace{2in}
    \textmd{\textbf{\hmwkClass:\ \hmwkTitle}}\\
    \normalsize\vspace{0.1in}\small\vspace{0.1in}\large{\textit{\hmwkClassInstructor}}
    \vspace{3in}
}

\author{\hmwkAuthorName}
\date{}

\renewcommand{\part}[1]{\textbf{\large Part \Alph{partCounter}}\stepcounter{partCounter}\\}

% Integral dx
\newcommand{\dx}{\mathrm{d}x}

%
% Various Helper Commands
%

% For derivatives
\newcommand{\deriv}[1]{\frac{\mathrm{d}}{\mathrm{d}x} (#1)}

% For partial derivatives
\newcommand{\pderiv}[2]{\frac{\partial}{\partial #1} (#2)}


% Alias for the Solution section header
\newcommand{\solution}{\textbf{\large Solution}}

% Probability commands: Expectation, Variance, Covariance, Bias
\newcommand{\E}{\mathrm{E}}
\newcommand{\Var}{\mathrm{Var}}
\newcommand{\Cov}{\mathrm{Cov}}
\newcommand{\Bias}{\mathrm{Bias}}

% Formatting commands:

\newcommand{\mt}[1]{\ensuremath{#1}}
\newcommand{\nm}[1]{\textrm{#1}}

\newcommand\bsc[2][\DefaultOpt]{%
  \def\DefaultOpt{#2}%
  \section[#1]{#2}%
}
\newcommand\ssc[2][\DefaultOpt]{%
  \def\DefaultOpt{#2}%
  \subsection[#1]{#2}%
}
\newcommand{\bgpf}{\begin{proof} $ $\newline}

\newcommand{\bgeq}{\begin{equation*}}
\newcommand{\eeq}{\end{equation*}}	

\newcommand{\balist}{\begin{enumerate}[label=\alph*.]}
\newcommand{\elist}{\end{enumerate}}

\newcommand{\bilist}{\begin{enumerate}[label=\roman*)]}	

\newcommand{\bgsp}{\begin{split}}
% \newcommand{\esp}{\end{split}} % doesn't work for some reason.

\newcommand\prs[1]{~~~\textbf{(#1)}}

\newcommand{\lt}[1]{\textbf{Let: } #1}
     							   %  if you're setting it to be true
\newcommand{\supp}[1]{\textbf{Suppose: } #1}
     							   %  Suppose (if it'll end up false)
\newcommand{\wts}[1]{\textbf{Want to show: } #1}
     							   %  Want to show
\newcommand{\as}[1]{\textbf{Assume: } #1}
     							   %  if you think it follows from truth
\newcommand{\bpth}[1]{\textbf{(#1)}}

\newcommand{\step}[2]{\begin{equation}\tag{#2}#1\end{equation}}
\newcommand{\epf}{\end{proof}}

\newcommand{\dbs}[3]{\mt{#1_{#2_#3}}}

\newcommand{\sidenote}[1]{-----------------------------------------------------------------Side Note----------------------------------------------------------------
#1 \

---------------------------------------------------------------------------------------------------------------------------------------------}

% Analysis / Logical commands:

\newcommand{\br}{\mt{\mathbb{R}} }       % |R
\newcommand{\bq}{\mt{\mathbb{Q}} }       % |Q
\newcommand{\bn}{\mt{\mathbb{N}} }       % |N
\newcommand{\bc}{\mt{\mathbb{C}} }       % |C
\newcommand{\bz}{\mt{\mathbb{Z}} }       % |Z

\newcommand{\ep}{\mt{\epsilon} }         % epsilon
\newcommand{\fa}{\mt{\forall} }          % for all
\newcommand{\afa}{\mt{\alpha} }
\newcommand{\bta}{\mt{\beta} }
\newcommand{\mem}{\mt{\in} }
\newcommand{\exs}{\mt{\exists} }

\newcommand{\es}{\mt{\emptyset} }        % empty set
\newcommand{\sbs}{\mt{\subset} }         % subset of
\newcommand{\fs}[2]{\{\uw{#1}{1}, \uw{#1}{2}, ... \uw{#1}{#2}\}}

\newcommand{\lra}{ \mt{\longrightarrow} } % implies ----->
\newcommand{\rar}{ \mt{\Rightarrow} }     % implies -->

\newcommand{\lla}{ \mt{\longleftarrow} }  % implies <-----
\newcommand{\lar}{ \mt{\Leftarrow} }      % implies <--

\newcommand{\av}[1]{\mt{|}#1\mt{|}}  % absolute value

\newcommand{\prn}[1]{(#1)}
\newcommand{\bk}[1]{\{#1\}}

\newcommand{\ps}{\mt{+} }
\newcommand{\ms}{\mt{-} }

\newcommand{\ls}{\mt{<} }
\newcommand{\gr}{\mt{>} }

\newcommand{\lse}{\mt{\leq} }
\newcommand{\gre}{\mt{\geq} }

\newcommand{\eql}{\mt{=} }

\newcommand{\pr}{\mt{^\prime} } 		   % prime (i.e. R')
\newcommand{\uw}[2]{#1\mt{_{#2}}}
\newcommand{\uf}[2]{#1\mt{^{#2}}}
\newcommand{\frc}[2]{\mt{\frac{#1}{#2}}}
\newcommand{\lmti}[1]{\mt{\displaystyle{\lim_{#1 \to \infty}}}}
\newcommand{\limt}[2]{\mt{\displaystyle{\lim_{#1 \to #2}}}}

\newcommand{\bnm}[2]{\mt{#1\setminus{#2}}}
\newcommand{\bnt}[2]{\mt{\textrm{#1}\setminus{\textrm{#2}}}}
\newcommand{\bi}{\bnm{\mathbb{R}}{\mathbb{Q}}}

\newcommand{\urng}[2]{\mt{\bigcup_{#1}^{#2}}}
\newcommand{\nrng}[2]{\mt{\bigcap_{#1}^{#2}}}
\newcommand{\nck}[2]{\mt{{#1 \choose #2}}}

\newcommand{\nbho}[3]{\textrm{N(}#1, #2\textrm{) }\cap \textrm{ #3} \neq \emptyset}
     							   %  N(x, eps) intersect S \= emptyset
\newcommand{\nbhe}[3]{\textrm{N(}#1, #2\textrm{) }\cap \textrm{ #3} = \emptyset}
     							   %  N(x, eps) intersect S  = emptyset
\newcommand{\dnbho}[3]{\textrm{N*(}#1, #2\textrm{) }\cap \textrm{ #3} \neq \emptyset}
     							   %  N*(x, eps) intersect S \= emptyset
\newcommand{\dnbhe}[3]{\textrm{N*(}#1, #2\textrm{) }\cap \textrm{ #3} = \emptyset}
     							   %  N*(x, eps) intersect S = emptyset
     							   
\newcommand{\eqn}[1]{\[#1\]}
\newcommand{\splt}[1]{\begin{split}#1\end{split}}

\newcommand{\infy}{\mt{\infty} }
     							 
% ----------

\begin{document}
HW 8: pages 193, \#1, 2, 3, 5, 9, 10, 17

For 2(c), see Theorem 1 and Example 9 below

Make sure when you do these problems, justify the answer by either writing down the theorem name or providing a counter example.

\bsc{Section 4.3 Continued}{

\ssc{Theorem 4.4.4}{

If a sequence \uw{s}{n} converges to s \mem \br, then every subsequence of \bk{\uw{s}{n}} converges to s as well.

\bgpf

Assume \bk{\uw{s}{n}} converges to s.

\fa \ep \gr 0, \exs N(\ep) \mem \bn st \av{\uw{s}{n} \ms s} \ls \ep for n \gre N \bpth{1} \

\

\lt{\bk{\dbs{s}{n}{k}}$_{k = 1}^\infty$ be a subsequence of \bk{\uw{s}{n}}}

\sidenote{
\fa k \mem \bn, 

if \uw{n}{k} \gre k and k diverges to $\infty$, then \uw{n}{k} diverges to $\infty$

So, for N \mem \bn, \exs k \mem \bn st \uw{n}{k} \gr N for k \gre K
}

By practice 4.4.3,

\lmti{k} \uw{n}{k} \eql \infy

Thus,

\exs K \mem \bn st \uw{n}{k} \gr N for k \gre K \bpth{2}

From \bpth{1} and \bpth{2},
\eqn{|\dbs{s}{n}{k} - s| < \epsilon \textrm{ for }k \gre K}
Hence,

\lmti{k} \dbs{s}{n}{k} \eql s
\epf
}

\ssc{Example 4.4.5 (see page 170, Ex 7(f) for a similar example) (can use this for hw)}{

Prove that if 0 \ls x \ls 1, then \lmti{n} \uf{x}{\frc{1}{n}} \eql 1

\bgpf

\lt{x \mem \br, 0 \ls x \ls 1}

Define \uw{s}{n} \eql \uf{x}{\frc{1}{n}} for n \mem \bn 

We shall prove that \bk{\uw{s}{n}} is an increasing sequence that is bounded above.

Notice that for n \mem \bn,
\eqn{
	\splt{
		x^{\frac{1}{n + 1}} - x^{\frac{1}{n}} & = x^{\frac{1}{n + 1}}( 1 - x^{\frac{1}{n} - \frac{1}{n + 1}}  ) \\
		& = x^{\frac{1}{n + 1}}(1 - x^{n(n + 1)}) \\
		& = \frac{1}{n} - \frac{1}{n + 1} \\
		& = \frac{n + 1 - n}{n(n + 1)} \\
		& = \frac{1}{n(n + 1)} \gr 0
		}
	}
So, \uw{s}{n + 1} \gre \uw{s}{n}, \fa n \mem \bn, i.e. \bk{\uw{s}{n}} is increasing and \uw{s}{n} \eql \uf{x}{\frc{1}{n}} \ls 1 \fa n \mem \bn.

By the Monotone Convergence Theorem (4.3.3?),

\exs s \mem \br st
\eqn{\lmti{n} \uw{s}{n} = s}
Now \bk{\uw{s}{2k}} is a subsequence of \uw{s}{n}.

By Theorem 4.4.4, 

\lmti{k} \uw{s}{2k} \eql s

\sidenote{This is \bk{\dbs{s}{n}{k}} where \uw{n}{k} \eql 2k \fa k \mem \bn}

However,

\uw{s}{2k} \eql \uf{x}{\frc{1}{2k}} \eql \uf{\prn{\uf{x}{\frc{1}{k}}}}{\frc{1}{2}} \eql $\sqrt{\uw{s}{k}}$

By Exercise 4.2.6,

\lmti{k} \uw{s}{2k} \eql $\sqrt{s}$

Thus, s \eql $\sqrt{s}$

But, if a sequence converges, then the limit is unique.

So,

\eqn{s^2 = s}
\eqn{s(s - 1) = 0}
\eqn{s = 0, 1}

However, we know that one of those must be wrong.

Since \uw{s}{1} \eql \uf{x}{\frc{1}{1}} \eql x \gr 0 and \uw{s}{n} \gre s \fa n \mem \bn,

we see that s $\neq$ 0.

Hence, s \eql 1

\epf

}

\ssc{Exercise 4.4.6}{
If \uw{s}{n} \eql \mt{(-1)^n} \fa n \mem \bn, prove that \bk{\uw{s}{n}} diverges.

Notice that

\uw{s}{2k} \eql 1 \fa k \mem \bn

while

\uw{s}{2k - 1} \eql $-1$, \fa k \mem \bn

Thus, we have subsequences \bk{\uw{s}{2k}}, \bk{\uw{s}{2k - 1}} st 

\eqn{\lmti{k} s_{2k} = 1 \textrm{ and } \lmti{k} \uw{s}{2k - 1} = -1}

Hence, \bk{\uw{s}{n}} diverges.
}

\ssc{Theorem 4.4.7}{

Every bounded sequence has a convergent subsequence.

\bgpf

\lt{\bk{\uw{s}{n}} be a bounded sequence}

Denote S as the range of \bk{\uw{s}{n}}: S \eql \bk{\uw{s}{n}: n \mem \bn}

\bilist
\item S is finite.
	
	\exs k \mem \bn st S \eql \bk{\uw{s}{1}, \uw{s}{2}, ... \uw{s}{k}}
	
	Then there is at least one element s \mem S st s is equal to an infinite number of terms of \bk{\uw{s}{n}}. (i.e. if the range has a finite number of elements, then that means \uw{s}{n} jumps between each of those elements an infinite number of times. Think of 1, $-1$, 1, $-1$...)
	
	Thus,
	
	Choose \uw{n}{1} such that \dbs{s}{n}{1} \eql s.
	
	Then,
	
	Choose \uw{n}{2} \gr \uw{n}{1} such that \dbs{s}{n}{2} \eql s.
	
	Inductively, \exs \dbs{s}{n}{k} \mem S such that \dbs{s}{n}{k} \eql s and \uw{n}{1} \ls \uw{n}{2} \ls ... \ls \uw{n}{k}
	
	Hence, \lmti{k} \dbs{s}{n}{k} \eql s
\item S is infinite.
	
	Since \bk{\uw{s}{n}} is bounded, S (our set described above) is also bounded.
	
	By the Bolzano-Weierstrass Theorem, \exs s \mem S\pr (i.e. an accumulation or limit point: s)
	
	By HW Exercise 15, page 142 (section 3.4), if x \mem S\pr, then N(x, $\epsilon$) contains an infinite number of points in S.
	
	Thus, \exs \dbs{s}{n}{1} \mem S st \dbs{s}{n}{1} \mem N(s, 1) (i.e. (s \ms 1, s \ps 1)
	
	\av{\dbs{s}{n}{1} \ms s} \ls \frc{1}{1}
	
	Then,
	
	\exs\uw{n}{2} \gr \uw{n}{1} st
	
	\av{\dbs{s}{n}{2} \ms s} \ls \frc{1}{2}
	
	So, inductively, we can keep doing this (i.e. for N(s, \frc{1}{3}), N(s, \frc{1}{4}), etc)
	
	Thus,
	\eqn{|\dbs{s}{n}{k} - s| \ls \frac{1}{k} \textrm{ and } n_1 < n_2 < ... < n_k}
	
	Hence, \lmti{k} \dbs{s}{n}{k} \eql s
\elist

\epf

}

\ssc{Theorem 4.4.8}{

Every unbounded sequence contains a monotonic sequence that diverges to \infy or $-$\infy 

\bgpf

\lt{\bk{\uw{s}{n}} be a sequence that is unbounded above}

Then, for m \mem \br, \exs N \mem \bn st

\uw{s}{n} \gr m if n \gre N

Notice that this implies that there are an infinite number of terms of \uw{s}{n} that are strictly larger than m.

(If there were only a finite number of terms greater than m, then \uw{s}{n} wouldn't be unbounded above. There would be a largest term, which would make it have an upper bound.)

Thus,

\exs \uw{n}{1} \mem \bn st

\dbs{s}{n}{1} \gr 1

Then,

\exs \uw{n}{2} \gr \uw{n}{1} st

\dbs{s}{n}{2} \gr 2

So, inductively, for k \mem \bn, \exs \uw{n}{k} \mem \bn st
\eqn{\dbs{s}{n}{k} \gr k \textrm{ where } n_1 < n_2 < ... < n_k}
Hence, for m \mem \br, the AP guarantees k \mem \bn st \dbs{s}{n}{k} \gr k \gr m, \fa k \gre K.

This implies that \lmti{k} \dbs{s}{n}{k} \eql \infy

If S is bounded above, then S must be unbounded below, a similar method shows that there is a subsequence \bk{\dbs{s}{n}{l}} st \lmti{l} \dbs{s}{n}{l} \eql $-\infty$

\epf
}
}
\end{document}