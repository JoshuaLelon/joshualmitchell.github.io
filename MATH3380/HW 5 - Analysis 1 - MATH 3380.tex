% Thank you Josh Davis for this template!
% https://github.com/jdavis/latex-homework-template/blob/master/homework.tex

\documentclass{article}

\newcommand{\hmwkTitle}{HW\ \#5}

% % ----------

% Packages

\usepackage{fancyhdr}
\usepackage{extramarks}
\usepackage{amsmath}
\usepackage{amssymb}
\usepackage{amsthm}
\usepackage{amsfonts}
\usepackage{tikz}
\usepackage[plain]{algorithm}
\usepackage{algpseudocode}
\usepackage{enumitem}
\usepackage{chngcntr}

% Libraries

\usetikzlibrary{automata, positioning, arrows}

%
% Basic Document Settings
%

\topmargin=-0.45in
\evensidemargin=0in
\oddsidemargin=0in
\textwidth=6.5in
\textheight=9.0in
\headsep=0.25in

\linespread{1.1}

\pagestyle{fancy}
\lhead{\hmwkAuthorName}
\chead{}
\rhead{\hmwkClass\ (\hmwkClassInstructor): \hmwkTitle}
\lfoot{\lastxmark}
\cfoot{\thepage}

\renewcommand\headrulewidth{0.4pt}
\renewcommand\footrulewidth{0.4pt}

\setlength\parindent{0pt}
\setcounter{secnumdepth}{0}

\newcommand{\hmwkClass}{MATH 3380 / Analysis 1}        % Class
\newcommand{\hmwkClassInstructor}{Dr. Welsh}           % Instructor
\newcommand{\hmwkAuthorName}{\textbf{Joshua Mitchell}} % Author

%
% Title Page
%

\title{
    \vspace{2in}
    \textmd{\textbf{\hmwkClass:\ \hmwkTitle}}\\
    \normalsize\vspace{0.1in}\small\vspace{0.1in}\large{\textit{\hmwkClassInstructor}}
    \vspace{3in}
}

\author{\hmwkAuthorName}
\date{}

\renewcommand{\part}[1]{\textbf{\large Part \Alph{partCounter}}\stepcounter{partCounter}\\}

% Integral dx
\newcommand{\dx}{\mathrm{d}x}

%
% Various Helper Commands
%

% For derivatives
\newcommand{\deriv}[1]{\frac{\mathrm{d}}{\mathrm{d}x} (#1)}

% For partial derivatives
\newcommand{\pderiv}[2]{\frac{\partial}{\partial #1} (#2)}


% Alias for the Solution section header
\newcommand{\solution}{\textbf{\large Solution}}

% Probability commands: Expectation, Variance, Covariance, Bias
\newcommand{\E}{\mathrm{E}}
\newcommand{\Var}{\mathrm{Var}}
\newcommand{\Cov}{\mathrm{Cov}}
\newcommand{\Bias}{\mathrm{Bias}}

% Formatting commands:

\newcommand{\mt}[1]{\ensuremath{#1}}
\newcommand{\nm}[1]{\textrm{#1}}

\newcommand\bsc[2][\DefaultOpt]{%
  \def\DefaultOpt{#2}%
  \section[#1]{#2}%
}
\newcommand\ssc[2][\DefaultOpt]{%
  \def\DefaultOpt{#2}%
  \subsection[#1]{#2}%
}
\newcommand{\bgpf}{\begin{proof} $ $\newline}

\newcommand{\bgeq}{\begin{equation*}}
\newcommand{\eeq}{\end{equation*}}	

\newcommand{\balist}{\begin{enumerate}[label=\alph*.]}
\newcommand{\elist}{\end{enumerate}}

\newcommand{\bilist}{\begin{enumerate}[label=\roman*)]}	

\newcommand{\bgsp}{\begin{split}}
% \newcommand{\esp}{\end{split}} % doesn't work for some reason.

\newcommand\prs[1]{~~~\textbf{(#1)}}

\newcommand{\lt}[1]{\textbf{Let: } #1}
     							   %  if you're setting it to be true
\newcommand{\supp}[1]{\textbf{Suppose: } #1}
     							   %  Suppose (if it'll end up false)
\newcommand{\wts}[1]{\textbf{Want to show: } #1}
     							   %  Want to show
\newcommand{\as}[1]{\textbf{Assume: } #1}
     							   %  if you think it follows from truth
\newcommand{\bpth}[1]{\textbf{(#1)}}

\newcommand{\step}[2]{\begin{equation}\tag{#2}#1\end{equation}}
\newcommand{\epf}{\end{proof}}

\newcommand{\dbs}[3]{\mt{#1_{#2_#3}}}

\newcommand{\sidenote}[1]{-----------------------------------------------------------------Side Note----------------------------------------------------------------
#1 \

---------------------------------------------------------------------------------------------------------------------------------------------}

% Analysis / Logical commands:

\newcommand{\br}{\mt{\mathbb{R}} }       % |R
\newcommand{\bq}{\mt{\mathbb{Q}} }       % |Q
\newcommand{\bn}{\mt{\mathbb{N}} }       % |N
\newcommand{\bc}{\mt{\mathbb{C}} }       % |C
\newcommand{\bz}{\mt{\mathbb{Z}} }       % |Z

\newcommand{\ep}{\mt{\epsilon} }         % epsilon
\newcommand{\fa}{\mt{\forall} }          % for all
\newcommand{\afa}{\mt{\alpha} }
\newcommand{\bta}{\mt{\beta} }
\newcommand{\mem}{\mt{\in} }
\newcommand{\exs}{\mt{\exists} }

\newcommand{\es}{\mt{\emptyset}}        % empty set
\newcommand{\sbs}{\mt{\subset} }         % subset of
\newcommand{\fs}[2]{\{\uw{#1}{1}, \uw{#1}{2}, ... \uw{#1}{#2}\}}

\newcommand{\lra}{ \mt{\longrightarrow} } % implies ----->
\newcommand{\rar}{ \mt{\Rightarrow} }     % implies -->

\newcommand{\lla}{ \mt{\longleftarrow} }  % implies <-----
\newcommand{\lar}{ \mt{\Leftarrow} }      % implies <--

\newcommand{\eql}{\mt{=} }
\newcommand{\pr}{\mt{^\prime} } 		   % prime (i.e. R')
\newcommand{\uw}[2]{#1\mt{_{#2}}}
\newcommand{\frc}[2]{\mt{\frac{#1}{#2}}}

\newcommand{\bnm}[2]{\mt{#1\setminus{#2}}}
\newcommand{\bnt}[2]{\mt{\textrm{#1}\setminus{\textrm{#2}}}}
\newcommand{\bi}{\bnm{\mathbb{R}}{\mathbb{Q}}}

\newcommand{\urng}[2]{\mt{\bigcup_{#1}^{#2}}}
\newcommand{\nrng}[2]{\mt{\bigcap_{#1}^{#2}}}

\newcommand{\nbho}[3]{\textrm{N(}#1, #2\textrm{) }\cap \textrm{ #3} \neq \emptyset}
     							   %  N(x, eps) intersect S \= emptyset
\newcommand{\nbhe}[3]{\textrm{N(}#1, #2\textrm{) }\cap \textrm{ #3} = \emptyset}
     							   %  N(x, eps) intersect S  = emptyset
\newcommand{\dnbho}[3]{\textrm{N*(}#1, #2\textrm{) }\cap \textrm{ #3} \neq \emptyset}
     							   %  N*(x, eps) intersect S \= emptyset
\newcommand{\dnbhe}[3]{\textrm{N*(}#1, #2\textrm{) }\cap \textrm{ #3} = \emptyset}
     							   %  N*(x, eps) intersect S = emptyset
     							 


% ----------

% ----------

% Packages

\usepackage{fancyhdr}
\usepackage{extramarks}
\usepackage{amsmath}
\usepackage{amssymb}
\usepackage{amsthm}
\usepackage{amsfonts}
\usepackage{tikz}
\usepackage[plain]{algorithm}
\usepackage{algpseudocode}
\usepackage{enumitem}
\usepackage{chngcntr}

% Libraries

\usetikzlibrary{automata, positioning, arrows}

%
% Basic Document Settings
%

\topmargin=-0.45in
\evensidemargin=0in
\oddsidemargin=0in
\textwidth=6.5in
\textheight=9.0in
\headsep=0.25in

\linespread{1.1}

\pagestyle{fancy}
\lhead{\hmwkAuthorName}
\chead{}
\rhead{\hmwkClass\ (\hmwkClassInstructor): \hmwkTitle}
\lfoot{\lastxmark}
\cfoot{\thepage}

\renewcommand\headrulewidth{0.4pt}
\renewcommand\footrulewidth{0.4pt}

\setlength\parindent{0pt}
\setcounter{secnumdepth}{0}

\newcommand{\hmwkClass}{MATH 3380 / Analysis 1}        % Class
\newcommand{\hmwkClassInstructor}{Dr. Welsh}           % Instructor
\newcommand{\hmwkAuthorName}{\textbf{Joshua Mitchell}} % Author

%
% Title Page
%

\title{
    \vspace{2in}
    \textmd{\textbf{\hmwkClass:\ \hmwkTitle}}\\
    \normalsize\vspace{0.1in}\small\vspace{0.1in}\large{\textit{\hmwkClassInstructor}}
    \vspace{3in}
}

\author{\hmwkAuthorName}
\date{}

\renewcommand{\part}[1]{\textbf{\large Part \Alph{partCounter}}\stepcounter{partCounter}\\}

% Integral dx
\newcommand{\dx}{\mathrm{d}x}

%
% Various Helper Commands
%

% For derivatives
\newcommand{\deriv}[1]{\frac{\mathrm{d}}{\mathrm{d}x} (#1)}

% For partial derivatives
\newcommand{\pderiv}[2]{\frac{\partial}{\partial #1} (#2)}


% Alias for the Solution section header
\newcommand{\solution}{\textbf{\large Solution}}

% Probability commands: Expectation, Variance, Covariance, Bias
\newcommand{\E}{\mathrm{E}}
\newcommand{\Var}{\mathrm{Var}}
\newcommand{\Cov}{\mathrm{Cov}}
\newcommand{\Bias}{\mathrm{Bias}}

% Formatting commands:

\newcommand{\mt}[1]{\ensuremath{#1}}
\newcommand{\nm}[1]{\textrm{#1}}

\newcommand\bsc[2][\DefaultOpt]{%
  \def\DefaultOpt{#2}%
  \section[#1]{#2}%
}
\newcommand\ssc[2][\DefaultOpt]{%
  \def\DefaultOpt{#2}%
  \subsection[#1]{#2}%
}
\newcommand{\bgpf}{\begin{proof} $ $\newline}

\newcommand{\bgeq}{\begin{equation*}}
\newcommand{\eeq}{\end{equation*}}	

\newcommand{\balist}{\begin{enumerate}[label=\alph*.]}
\newcommand{\elist}{\end{enumerate}}

\newcommand{\bilist}{\begin{enumerate}[label=\roman*)]}	

\newcommand{\bgsp}{\begin{split}}
% \newcommand{\esp}{\end{split}} % doesn't work for some reason.

\newcommand\prs[1]{~~~\textbf{(#1)}}

\newcommand{\lt}[1]{\textbf{Let: } #1}
     							   %  if you're setting it to be true
\newcommand{\supp}[1]{\textbf{Suppose: } #1}
     							   %  Suppose (if it'll end up false)
\newcommand{\wts}[1]{\textbf{Want to show: } #1}
     							   %  Want to show
\newcommand{\as}[1]{\textbf{Assume: } #1}
     							   %  if you think it follows from truth
\newcommand{\bpth}[1]{\textbf{(#1)}}

\newcommand{\step}[2]{\begin{equation}\tag{#2}#1\end{equation}}
\newcommand{\epf}{\end{proof}}

\newcommand{\dbs}[3]{\mt{#1_{#2_#3}}}

\newcommand{\sidenote}[1]{-----------------------------------------------------------------Side Note----------------------------------------------------------------
#1 \

---------------------------------------------------------------------------------------------------------------------------------------------}

% Analysis / Logical commands:

\newcommand{\br}{\mt{\mathbb{R}} }       % |R
\newcommand{\bq}{\mt{\mathbb{Q}} }       % |Q
\newcommand{\bn}{\mt{\mathbb{N}} }       % |N
\newcommand{\bc}{\mt{\mathbb{C}} }       % |C
\newcommand{\bz}{\mt{\mathbb{Z}} }       % |Z

\newcommand{\ep}{\mt{\epsilon} }         % epsilon
\newcommand{\fa}{\mt{\forall} }          % for all
\newcommand{\afa}{\mt{\alpha} }
\newcommand{\bta}{\mt{\beta} }
\newcommand{\mem}{\mt{\in} }
\newcommand{\exs}{\mt{\exists} }

\newcommand{\es}{\mt{\emptyset} }        % empty set
\newcommand{\sbs}{\mt{\subset} }         % subset of
\newcommand{\fs}[2]{\{\uw{#1}{1}, \uw{#1}{2}, ... \uw{#1}{#2}\}}

\newcommand{\lra}{ \mt{\longrightarrow} } % implies ----->
\newcommand{\rar}{ \mt{\Rightarrow} }     % implies -->

\newcommand{\lla}{ \mt{\longleftarrow} }  % implies <-----
\newcommand{\lar}{ \mt{\Leftarrow} }      % implies <--

\newcommand{\av}[1]{\mt{|}#1\mt{|}}  % absolute value

\newcommand{\prn}[1]{(#1)}
\newcommand{\bk}[1]{\{#1\}}

\newcommand{\ps}{\mt{+} }
\newcommand{\ms}{\mt{-} }

\newcommand{\ls}{\mt{<} }
\newcommand{\gr}{\mt{>} }

\newcommand{\lse}{\mt{\leq} }
\newcommand{\gre}{\mt{\geq} }

\newcommand{\eql}{\mt{=} }

\newcommand{\pr}{\mt{^\prime} } 		   % prime (i.e. R')
\newcommand{\uw}[2]{#1\mt{_{#2}}}
\newcommand{\uf}[2]{#1\mt{^{#2}}}
\newcommand{\frc}[2]{\mt{\frac{#1}{#2}}}
\newcommand{\lmti}[1]{\mt{\displaystyle{\lim_{#1 \to \infty}}}}
\newcommand{\limt}[2]{\mt{\displaystyle{\lim_{#1 \to #2}}}}

\newcommand{\bnm}[2]{\mt{#1\setminus{#2}}}
\newcommand{\bnt}[2]{\mt{\textrm{#1}\setminus{\textrm{#2}}}}
\newcommand{\bi}{\bnm{\mathbb{R}}{\mathbb{Q}}}

\newcommand{\urng}[2]{\mt{\bigcup_{#1}^{#2}}}
\newcommand{\nrng}[2]{\mt{\bigcap_{#1}^{#2}}}
\newcommand{\nck}[2]{\mt{{#1 \choose #2}}}

\newcommand{\nbho}[3]{\textrm{N(}#1, #2\textrm{) }\cap \textrm{ #3} \neq \emptyset}
     							   %  N(x, eps) intersect S \= emptyset
\newcommand{\nbhe}[3]{\textrm{N(}#1, #2\textrm{) }\cap \textrm{ #3} = \emptyset}
     							   %  N(x, eps) intersect S  = emptyset
\newcommand{\dnbho}[3]{\textrm{N*(}#1, #2\textrm{) }\cap \textrm{ #3} \neq \emptyset}
     							   %  N*(x, eps) intersect S \= emptyset
\newcommand{\dnbhe}[3]{\textrm{N*(}#1, #2\textrm{) }\cap \textrm{ #3} = \emptyset}
     							   %  N*(x, eps) intersect S = emptyset
     							 
% ----------

\begin{document}

Homework Due 10/5/17 (7 problems): 

Section 4.1 pages 169 - 170; 1, 6(b), 7(f), 9(a), 11, 12, 15

\bsc{\#1}{

Mark each statement True or False. Justify each answer.

\balist
\item If (\uw{s}{n}) is a sequence and \uw{s}{i} \eql \uw{s}{j} then i \eql j.
	
	False.
	
	\lt{\prn{\uw{s}{n}}} \eql \bk{\uf{1}{n}}
\item If \uw{s}{n} \lra s, then, for every \ep $>$ 0, \exs N \mem \bn st n $\geq$ N implies $|$\uw{s}{n} $-$ s$|$ $<$ \ep.
	
	True.
	
	A sequence \{\uw{s}{n}\} is said to \textbf{converge} to s \mem \br provided that \fa \ep $>$ 0

	\exs N \mem \bn $\leq$ n st 

	$|\uw{s}{n} - s|$ $<$ \ep \textrm{   }
	
	This is the definition of convergence, so this implies that \uw{s}{n} \lra s
\item If \uw{s}{n} \lra k and \uw{t}{n} \lra k, then \uw{s}{n} \eql \uw{t}{n} \fa n \mem \bn.
	
	False.
	
	\lt{\uw{s}{n} \eql $\sum_{i = 0}^\infty$\frc{1}{2^i}, \uw{t}{n} \eql 2 \ms $\sum_{i = 0}^\infty$\frc{1}{2^i} }
\item Every convergent sequence is bounded.
	
	By Theorem 4.1.13, this is true.
\elist
}
\newpage

\bsc{6(b)}{

\textbf{Definition 4.1.2}

A sequence \{\uw{s}{n}\} is said to \textbf{converge} to s \mem \br provided that \fa \ep $>$ 0

\exs N \mem \bn st

n \gre N \lra $|\uw{s}{n} - s|$ $<$ \ep \textrm{   } \

\

Using only definition 4.1.2, prove the following:

For k $>$ 0, k \mem \br, \lmti{n}(\frc{1}{\uf{n}{k}}) \eql 0

\bgpf

\lt{ \bk{\uw{s}{n}} \eql \frc{1}{\mt{n^k}}, s \eql 0}

\av{\uw{s}{n} \ms s} \eql \av{\frc{1}{\mt{n^k}} \ms 0} \eql \av{\frc{1}{\uf{n}{k}}}

\wts{\fa \ep \gr 0, \exs N \mem \bn st n \gre N \lra \av{\frc{1}{\uf{n}{k}}} \ls \ep}

\lt{\ep \gr 0, N \mem \bn, k \mem \br \gr 0}

\wts{\exs N \mem \bn st \av{\frc{1}{\uf{N}{k}}} \ls \ep}

\lt{\av{\frc{1}{\uf{N}{k}}} \ls \ep}

\frc{1}{\av{\uf{N}{k}}} \ls \ep

\frc{1}{\ep} \ls \av{\uf{N}{k}} \

\av{\uf{N}{k}} \eql \uf{N}{k} since N \mem \bn and k \gr 0 \bpth{1}

\frc{1}{\ep} \ls \uf{N}{k}

\prn{\frc{1}{\ep}}$^\frac{1}{k}$ \ls N

If N is the ceiling of \prn{\frc{1}{\ep}}$^\frac{1}{k}$ \ps 1, then N exists. \

\

\wts{\av{\frc{1}{\uf{(N + 1)}{k}}} \ls \ep}

If we know that \av{\frc{1}{\uf{N}{k}}} \ls \ep, 

then showing 

\av{\frc{1}{\uf{(N + 1)}{k}}} \ls \av{\frc{1}{\uf{N}{k}}} \

\

shows \

\

\av{\frc{1}{\uf{(N + 1)}{k}}} \ls \ep \

\

\av{\frc{1}{\uf{(N + 1)}{k}}} \ls \av{\frc{1}{\uf{N}{k}}}

\frc{1}{\av{\uf{(N + 1)}{k}}} \ls \frc{1}{\av{\uf{N}{k}}}

\av{\uf{N}{k}} \ls \av{\uf{(N + 1)}{k}}

From \bpth{1}, 

\av{\uf{N}{k}} \eql \uf{N}{k} \ls \av{\uf{(N + 1)}{k}} \eql \uf{(N + 1)}{k}

\uf{N}{k} \ls \uf{(N + 1)}{k}

This is true since N \mem \bn and k \gr 0 \

So, \av{\frc{1}{\uf{N}{k}}} decreases as N grows.

Since \fa \ep \gr 0, \exs N \mem \bn st n \gre N implies \av{\frc{1}{\uf{n}{k}}} \ls \ep,

\lmti{n}\frc{1}{\uf{n}{k}} \eql 0

\epf

}

\newpage

\bsc{7(f)}{

Using any of the results in this section (4.1), prove the following:

If \av{x} \ls 1, then \lmti{n}\uf{x}{n} \eql 0

\bgpf

\av{x} \ls 1 implies  0 \lse \av{x} \ls 1 \bpth{1}

\lt{\uw{s}{n} \eql \uf{x}{n}, s \eql 0}

\wts{\fa \ep \gr 0, \exs N \mem \bn st n \gre N implies \av{\uw{s}{n} \ms s} \ls \ep}

\lt{\ep \gr 0}

\av{\uw{s}{n} \ms s} \ls \ep \eql \av{\uf{x}{n}} \ls \ep

\wts{\exs N \mem \bn st \av{\uf{x}{N}} \ls \ep}

\av{\uf{x}{N}} \ls \ep \

\av{\av{\uf{x}{N}}} \ls \av{ \ep}

We know that because of \bpth{1} and because N \mem \bn, 

\av{\uf{x}{N + 1}} \ls \av{\uf{x}{N}}

We also know that \ep \gr 0

So, 0 \ls \av{\uf{x}{N +\textrm{ }k}} \ls ... \ls \av{\uf{x}{N + 1}} \ls \av{\uf{x}{N}} where k \mem \bn



\epf

}

\bsc{9(a)}{
For each of the following, prove or give a counter example:

If (\uw{s}{n}) converges to s, then \prn{\av{\uw{s}{n}}} converges to \av{s}.

\bgpf

If \uw{s}{n} converges to s, then by definition,

\fa \ep \gr 0, \exs N \mem \bn st N \lse n implies \av{\uw{s}{n} \ms s} \ls \ep

\wts{\av{\av{\uw{s}{n}} \ms \av{s}} \ls \ep}

Case 1: \uw{s}{n} and s are the same sign.

\av{\av{\uw{s}{n}} \ms \av{s}} \eql \av{\uw{s}{n} \ms s}

Therefore, N \lse n implies \av{\av{\uw{s}{n}} \ms \av{s}} \ls \ep

If we let \uw{s}{n} \eql \av{\uw{s}{n}} and \av{s} \eql s, then \av{\uw{s}{n}} converges to \av{s}.

Case 2: \uw{s}{n} and s are different signs.

\av{\av{\uw{s}{n}} \ms \av{s}} \lse \av{\uw{s}{n} \ms s} \ls \ep

\av{\av{\uw{s}{n}} \ms \av{s}} \ls \ep

Therefore, N \lse n implies \av{\av{\uw{s}{n}} \ms \av{s}} \ls \ep

If we let \uw{s}{n} \eql \av{\uw{s}{n}} and \av{s} \eql s, then \av{\uw{s}{n}} converges to \av{s}.

Hence, result.

\epf

}

\newpage

\bsc{11}{
Given the sequence \prn{\uw{s}{n}}, k \mem \bn, let \prn{\uw{t}{n}} be the sequence defined by \uw{t}{n} \eql \uw{s}{n + k}. That is, the terms in (\uw{t}{n}) are the same as that of the terms in (\uw{s}{n}) after the first k terms have been skipped. Prove that \prn{\uw{t}{n}} converges iff \prn{\uw{s}{n}} converges, and if they converge, show that lim \uw{t}{n} \eql lim \uw{s}{n}. Thus, the convergence of a sequence is not affected by omitting (or changing) a finite number of terms.

\bgpf

\lra

\prn{\uw{t}{n}} converges \lra \prn{\uw{s}{n}} converges

If \uw{t}{n} converges, then by definition,

\fa \ep \gr 0, \exs N \mem \bn st n \gre N implies \av{\uw{t}{n} \ms L} \ls \ep

(or)

\fa \ep \gr 0, \exs N \mem \bn st \av{\uw{t}{n} \ms L} \ls \ep \fa n \gre N

Since \uw{t}{n} \eql \uw{s}{n + k},

we know that \uw{s}{n + k} converges.

\lt{\uw{n}{1} \eql n \ps k}

So,

\fa \ep \gr 0, \exs N \mem \bn st \uw{n}{1} \gre N implies \av{\uw{s}{\uw{n}{1}} \ms L} \ls \ep

\fa \ep \gr 0, \exs N \mem \bn st n \ps k \gre N implies \av{\uw{s}{\uw{n}{1}} \ms L} \ls \ep

\fa \ep \gr 0, \exs N \mem \bn st n \gre N \ms k implies \av{\uw{s}{\uw{n}{1}} \ms L} \ls \ep

Notice that N \ms k \mem \bn. Let's call it \uw{N}{1}

\fa \ep \gr 0, \exs \uw{N}{1} \mem \bn st n \gre \uw{N}{1} implies \av{\uw{s}{\uw{n}{1}} \ms L} \ls \ep

Since there is still a natural number \uw{N}{1} st n \gre \uw{N}{1} implies \av{\uw{s}{\uw{n}{1}} \ms L} \ls \ep,

If \uw{t}{n} converges, then \uw{s}{n} converges.

\lla

\prn{\uw{s}{n}} converges \lra \prn{\uw{t}{n}} converges

If \uw{s}{n} converges, then by definition,

\fa \ep \gr 0, \exs N \mem \bn st n \gre N implies \av{\uw{s}{n} \ms L} \ls \ep

Since \uw{t}{n} \eql \uw{s}{n + k}, \uw{t}{n - k} \eql \uw{s}{n}

So since \uw{s}{n} converges, \uw{t}{n - k} converges.

If we let \uw{n}{1} \eql n \ms k,

\fa \ep \gr 0, \exs N \mem \bn st \uw{n}{1} \gre N implies \av{\uw{t}{\uw{n}{1}} \ms L} \ls \ep

\fa \ep \gr 0, \exs N \mem \bn st n \ms k \gre N implies \av{\uw{t}{\uw{n}{1}} \ms L} \ls \ep

\fa \ep \gr 0, \exs N \mem \bn st n \gre (N \ps k) implies \av{\uw{t}{\uw{n}{1}} \ms L} \ls \ep

Notice that N \ps k \mem \bn. Let's call it \uw{N}{1}

\fa \ep \gr 0, \exs \uw{N}{1} \mem \bn st n \gre \uw{N}{1} implies \av{\uw{t}{\uw{n}{1}} \ms L} \ls \ep

Since there is still a natural number \uw{N}{1} st n \gre \uw{N}{1} implies \av{\uw{t}{\uw{n}{1}} \ms L} \ls \ep,

If \uw{s}{n} converges, then \uw{t}{n} converges.

\epf

}

\bsc{12}{

\balist
\item Assume that lim \uw{s}{n} \eql 0. If \prn{\uw{t}{n}} is a bounded sequence, prove that lim(\uw{s}{n}\uw{t}{n}) \eql 0.
	
	If lim \uw{s}{n} \eql 0, then by definition,
	
	\fa \ep \gr 0, \exs N \mem \bn st if n \gre N, then \av{\uw{s}{n} - 0} \ls \ep
	
	If \uw{t}{n} is a bounded sequence, then \fa n \mem \bn, a \lse \uw{t}{n} \lse b, where a, b \mem \br
	
	We know that \uw{t}{n} will always be between two constants a and b, so lets let \uw{t}{n} \eql c, where a \lse c \lse b.
	
	Since \uw{s}{n} converges,
	
	\fa \ep \gr 0, \exs N \mem \bn st if n \gre N, then \av{\uw{s}{n} - 0} \ls \ep
	
	can be simplified to
	
	\fa \ep \gr 0, \exs N \mem \bn st if n \gre N, then \av{\uw{s}{n}} \ls \ep
	
	\wts{lim(\uw{s}{n}\uw{t}{n}) \eql 0}
	
	lim(\uw{s}{n}\uw{t}{n}) \eql 0 if
	
	\fa \ep \gr 0, \exs N \mem \bn st if n \gre N, then \av{\uw{s}{n}\uw{t}{n}} \ls \ep
	
	Since we let \uw{t}{n} \eql c, some bounded real number, this is equivalent to
	
	\fa \ep \gr 0, \exs N \mem \bn st if n \gre N, then \av{c\uw{s}{n}} \ls \ep
	
	\fa \ep \gr 0, \exs N \mem \bn st if n \gre N, then \av{\uw{s}{n}} \ls \av{c}\ep
	
	\fa \ep \gr 0, \exs N \mem \bn st if n \gre N, then \av{\uw{s}{n}} \ls \uw{\ep}{1}
	
	which is equivalent to
	
	\fa \ep \gr 0, \exs N \mem \bn st if n \gre N, then \av{\uw{s}{n}} \ls \ep
	
	Hence, result.
	
\item Show by example that the boundedness of \prn{\uw{t}{n}} is a necessary condition in part (a).
	
	If lim \uw{s}{n} \eql 0, then by definition,
	
	\fa \ep \gr 0, \exs N \mem \bn st if n \gre N, then \av{\uw{s}{n} - 0} \ls \ep
	
	\fa \ep \gr 0, \exs N \mem \bn st if n \gre N, then \av{\uw{s}{n}} \ls \ep
	
	However, if we let \uw{t}{n} be unbounded (i.e. let \uw{t}{n} \eql \uf{e}{n}), this doesn't work. See below:
	
	\uw{s}{n}\uw{t}{n} is bounded if 
	
	\fa \ep \gr 0, \exs N \mem \bn st if n \gre N, then \av{\uw{s}{n}\uw{t}{n}} \ls \ep
	
	\supp{\uw{s}{n} \eql \frc{1}{n}}
	
	Then \uw{s}{n}\uw{t}{n} \eql \frc{\uf{e}{n}}{n}
	
	Since \uf{e}{n} grows faster than \frc{1}{n}, \uw{s}{n}\uw{t}{n} grows overall as n approaches infinity.
	
	Hence, the boundedness of \uw{t}{n} is necessary.
	
\elist

}

\newpage

\bsc{15}{

\balist
\item Prove that x is an accumulation point of a set S iff \exs a sequence \prn{\uw{s}{n}} of points in \bnt{S}{\bk{x}} st \prn{\uw{s}{n}} converges to x.

	\lra
	
	\lt{x \mem S\pr}
	
	This means that $\dnbho{x}{\ep}{S}$, \fa \ep \gr 0. \bpth{1}
	
	N*(x, \ep) means \bk{y \mem \br : 0 \ls \av{y \ms x} \ls \ep}
	
	If \fa \ep \gr 0, \exs N \mem \bn st n \gre N implies \av{\uw{s}{n} - x} \ls \ep,
	
	Then \prn{\uw{s}{n}} converges to x.
	
	\lt{\uw{s}{n} \mem $\dnbho{x}{\frc{1}{n}}{S}$}
	
	Then
	
	\av{\uw{s}{n} \ms x} \ls \frc{1}{n} and \uw{s}{n} \mem \bnt{S}{\bk{x}}
	
	\lt{\ep \gr 0}
	
	\exs N(\ep) \mem \bn st \frc{1}{N} \ls \ep (By AP)
	
	Thus, from \bpth{1}, 
	
	\av{\uw{s}{n} \ms x} \ls \ep, \fa n \gre N.
	
	Hence, lim \uw{s}{n} \eql x and \uw{s}{n} \mem \bnt{S}{\bk{x}} \fa n \mem \bn
	
	\lla
	
	Conversely,
	
	\as{\bk{\uw{s}{n}} is a sequence in \bnt{S}{\bk{x}} st \lmti{n}\uw{s}{n} \eql x}
	
	\wts{x \mem S\pr}
	
	\fa \ep \gr 0, \exs N(\ep) (as in N is chosen based on \ep) \mem \bn st
	
	\av{\uw{s}{n} \ms x} \ls \ep \fa n \gre N and \uw{s}{n} \mem \bnt{S}{\bk{X}}
	
	\uw{s}{n} \mem (x \ms \ep, x \ps \ep), \uw{s}{n} $\neq$ x
	
	(theorem 4.2.1 is a possibility on test)
	
	4.2.4, 4.2.7 not on exam
	
	Thus, $\dnbho{x}{\ep}{S}$
	
	So, x \mem S\pr
	
	Hence result.
	
\item Prove that a set S is closed iff, whenever \prn{\uw{s}{n}} is a convergent sequence of points in S, it follows that lim \uw{s}{n} is in S.
	
	\lra 
	
	\lt{S be a closed set.}
	
	\wts{\prn{\uw{s}{n}} is a sequence in S st \lmti{n}\uw{s}{n} \eql s implies s \mem S}
	
	If S is closed, then S \eql cl S \eql S $\cup$ S\pr, bd S \sbs S
	
	\fa \ep \gr 0, \exs N(\ep) \mem \bn st \av{\uw{s}{n} \ms s} \ls \ep \fa n \gre N
	
	\wts{s \mem S}
	
	Case
	
	\bilist
	\item s \mem S.
		
		In this case, we are done.
	\item s $\not\in$ S
		
		Hence, \prn{\uw{s}{n}} is a sequence in \bnt{S}{\bk{s}}  st \lmti{n} \uw{s}{n} \eql s. By \bpth{a}, s \mem S\pr. 
		
		Since S is closed, s \mem S.
	\elist
	
	\textbf{Note: The above is what we did in class. Below (until \lla) is my original answer. Can you tell me if the next 11 or so lines are valid?}
	
	\supp{lim \uw{s}{n} $\not\in$ S}
	
	This implies that s $\not\in$ S where
	
	\fa \ep \gr 0, \exs N \mem \bn st n \gre N implies \av{\uw{s}{n} \ms s} \ls \ep
	
	Since S is closed,
	
	\lt{u be the closest boundary point to s}
	
	Now, let \ep \eql \av{\frc{u \ms s}{2}}
	
	We know that \av{\uw{s}{n} \ms s} \ls \ep for this epsilon.
	
	\av{\uw{s}{n} \ms s} \ls \av{\frc{u \ms L}{2}}
	
	Which implies that the distance between \uw{s}{n} and s is less than the distance between \uw{s}{n} and the nearest boundary point of S.
	
	This means there is an \uw{s}{n} st \uw{s}{n} $\not\in$ S, a contradiction.
	
	So, \uw{s}{n} is not a convergent sequence of points in S if lim \uw{s}{n} is not in S. 
	
	\lla
	
	Conversely, 
	
	\as{whenever \prn{\uw{s}{n}} is a sequence in S st \lmti{n}\uw{s}{n} \eql s, then s \mem S}
	
	\wts{S is closed}
	
	We will use Theorem 3.4.17 (a): S is closed iff S\pr \sbs S
	
	\lt{s \mem S\pr}
	
	s \mem S\pr means \fa \ep \gr 0, $\dnbho{s}{\ep}{S}$
	
	\lt{\uw{s}{n} \mem $\dnbho{s}{\frc{1}{n}}{S}$, n \mem \bn}
	
	So,
	
	\av{\uw{s}{n} \ms s} \ls \frc{1}{n}, \uw{s}{n} \mem S \fa n \mem \bn 
	
	Hence \lmti{n}\uw{s}{n} \eql s
	
	We know \exs N \mem \bn st \frc{1}{N} \ls \ep by AP.
	
	From \bpth{1}, \av{\uw{s}{n} \ms s} \ls \frc{1}{n} \ls \ep \fa n \gre N
	
	Hence, \lmti{n} \uw{s}{n} \eql s, \uw{s}{n} \mem S, \fa n \mem \bn
	
	By assuming s \mem S, S is closed.
	
\elist

}

\end{document}