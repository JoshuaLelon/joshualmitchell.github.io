% Thank you Josh Davis for this template!
% https://github.com/jdavis/latex-homework-template/blob/master/homework.tex

\documentclass{article}

\newcommand{\hmwkTitle}{HW\ \#8}

% % ----------

% Packages

\usepackage{fancyhdr}
\usepackage{extramarks}
\usepackage{amsmath}
\usepackage{amssymb}
\usepackage{amsthm}
\usepackage{amsfonts}
\usepackage{tikz}
\usepackage[plain]{algorithm}
\usepackage{algpseudocode}
\usepackage{enumitem}
\usepackage{chngcntr}

% Libraries

\usetikzlibrary{automata, positioning, arrows}

%
% Basic Document Settings
%

\topmargin=-0.45in
\evensidemargin=0in
\oddsidemargin=0in
\textwidth=6.5in
\textheight=9.0in
\headsep=0.25in

\linespread{1.1}

\pagestyle{fancy}
\lhead{\hmwkAuthorName}
\chead{}
\rhead{\hmwkClass\ (\hmwkClassInstructor): \hmwkTitle}
\lfoot{\lastxmark}
\cfoot{\thepage}

\renewcommand\headrulewidth{0.4pt}
\renewcommand\footrulewidth{0.4pt}

\setlength\parindent{0pt}
\setcounter{secnumdepth}{0}

\newcommand{\hmwkClass}{MATH 3380 / Analysis 1}        % Class
\newcommand{\hmwkClassInstructor}{Dr. Welsh}           % Instructor
\newcommand{\hmwkAuthorName}{\textbf{Joshua Mitchell}} % Author

%
% Title Page
%

\title{
    \vspace{2in}
    \textmd{\textbf{\hmwkClass:\ \hmwkTitle}}\\
    \normalsize\vspace{0.1in}\small\vspace{0.1in}\large{\textit{\hmwkClassInstructor}}
    \vspace{3in}
}

\author{\hmwkAuthorName}
\date{}

\renewcommand{\part}[1]{\textbf{\large Part \Alph{partCounter}}\stepcounter{partCounter}\\}

% Integral dx
\newcommand{\dx}{\mathrm{d}x}

%
% Various Helper Commands
%

% For derivatives
\newcommand{\deriv}[1]{\frac{\mathrm{d}}{\mathrm{d}x} (#1)}

% For partial derivatives
\newcommand{\pderiv}[2]{\frac{\partial}{\partial #1} (#2)}


% Alias for the Solution section header
\newcommand{\solution}{\textbf{\large Solution}}

% Probability commands: Expectation, Variance, Covariance, Bias
\newcommand{\E}{\mathrm{E}}
\newcommand{\Var}{\mathrm{Var}}
\newcommand{\Cov}{\mathrm{Cov}}
\newcommand{\Bias}{\mathrm{Bias}}

% Formatting commands:

\newcommand{\mt}[1]{\ensuremath{#1}}
\newcommand{\nm}[1]{\textrm{#1}}

\newcommand\bsc[2][\DefaultOpt]{%
  \def\DefaultOpt{#2}%
  \section[#1]{#2}%
}
\newcommand\ssc[2][\DefaultOpt]{%
  \def\DefaultOpt{#2}%
  \subsection[#1]{#2}%
}
\newcommand{\bgpf}{\begin{proof} $ $\newline}

\newcommand{\bgeq}{\begin{equation*}}
\newcommand{\eeq}{\end{equation*}}	

\newcommand{\balist}{\begin{enumerate}[label=\alph*.]}
\newcommand{\elist}{\end{enumerate}}

\newcommand{\bilist}{\begin{enumerate}[label=\roman*)]}	

\newcommand{\bgsp}{\begin{split}}
% \newcommand{\esp}{\end{split}} % doesn't work for some reason.

\newcommand\prs[1]{~~~\textbf{(#1)}}

\newcommand{\lt}[1]{\textbf{Let: } #1}
     							   %  if you're setting it to be true
\newcommand{\supp}[1]{\textbf{Suppose: } #1}
     							   %  Suppose (if it'll end up false)
\newcommand{\wts}[1]{\textbf{Want to show: } #1}
     							   %  Want to show
\newcommand{\as}[1]{\textbf{Assume: } #1}
     							   %  if you think it follows from truth
\newcommand{\bpth}[1]{\textbf{(#1)}}

\newcommand{\step}[2]{\begin{equation}\tag{#2}#1\end{equation}}
\newcommand{\epf}{\end{proof}}

\newcommand{\dbs}[3]{\mt{#1_{#2_#3}}}

\newcommand{\sidenote}[1]{-----------------------------------------------------------------Side Note----------------------------------------------------------------
#1 \

---------------------------------------------------------------------------------------------------------------------------------------------}

% Analysis / Logical commands:

\newcommand{\br}{\mt{\mathbb{R}} }       % |R
\newcommand{\bq}{\mt{\mathbb{Q}} }       % |Q
\newcommand{\bn}{\mt{\mathbb{N}} }       % |N
\newcommand{\bc}{\mt{\mathbb{C}} }       % |C
\newcommand{\bz}{\mt{\mathbb{Z}} }       % |Z

\newcommand{\ep}{\mt{\epsilon} }         % epsilon
\newcommand{\fa}{\mt{\forall} }          % for all
\newcommand{\afa}{\mt{\alpha} }
\newcommand{\bta}{\mt{\beta} }
\newcommand{\mem}{\mt{\in} }
\newcommand{\exs}{\mt{\exists} }

\newcommand{\es}{\mt{\emptyset}}        % empty set
\newcommand{\sbs}{\mt{\subset} }         % subset of
\newcommand{\fs}[2]{\{\uw{#1}{1}, \uw{#1}{2}, ... \uw{#1}{#2}\}}

\newcommand{\lra}{ \mt{\longrightarrow} } % implies ----->
\newcommand{\rar}{ \mt{\Rightarrow} }     % implies -->

\newcommand{\lla}{ \mt{\longleftarrow} }  % implies <-----
\newcommand{\lar}{ \mt{\Leftarrow} }      % implies <--

\newcommand{\eql}{\mt{=} }
\newcommand{\pr}{\mt{^\prime} } 		   % prime (i.e. R')
\newcommand{\uw}[2]{#1\mt{_{#2}}}
\newcommand{\frc}[2]{\mt{\frac{#1}{#2}}}

\newcommand{\bnm}[2]{\mt{#1\setminus{#2}}}
\newcommand{\bnt}[2]{\mt{\textrm{#1}\setminus{\textrm{#2}}}}
\newcommand{\bi}{\bnm{\mathbb{R}}{\mathbb{Q}}}

\newcommand{\urng}[2]{\mt{\bigcup_{#1}^{#2}}}
\newcommand{\nrng}[2]{\mt{\bigcap_{#1}^{#2}}}

\newcommand{\nbho}[3]{\textrm{N(}#1, #2\textrm{) }\cap \textrm{ #3} \neq \emptyset}
     							   %  N(x, eps) intersect S \= emptyset
\newcommand{\nbhe}[3]{\textrm{N(}#1, #2\textrm{) }\cap \textrm{ #3} = \emptyset}
     							   %  N(x, eps) intersect S  = emptyset
\newcommand{\dnbho}[3]{\textrm{N*(}#1, #2\textrm{) }\cap \textrm{ #3} \neq \emptyset}
     							   %  N*(x, eps) intersect S \= emptyset
\newcommand{\dnbhe}[3]{\textrm{N*(}#1, #2\textrm{) }\cap \textrm{ #3} = \emptyset}
     							   %  N*(x, eps) intersect S = emptyset
     							 


% ----------

% ----------

% Packages

\usepackage{fancyhdr}
\usepackage{extramarks}
\usepackage{amsmath}
\usepackage{amssymb}
\usepackage{amsthm}
\usepackage{amsfonts}
\usepackage{tikz}
\usepackage[plain]{algorithm}
\usepackage{algpseudocode}
\usepackage{enumitem}
\usepackage{chngcntr}

% Libraries

\graphicspath{{/Users/jm/iclouddrive/3380pics/}}

\usetikzlibrary{automata, positioning, arrows}

%
% Basic Document Settings
%

\topmargin=-0.45in
\evensidemargin=0in
\oddsidemargin=0in
\textwidth=6.5in
\textheight=9.0in
\headsep=0.25in

\linespread{1.1}

\pagestyle{fancy}
\lhead{\hmwkAuthorName}
\chead{}
\rhead{\hmwkClass\ (\hmwkClassInstructor): \hmwkTitle}
\lfoot{\lastxmark}
\cfoot{\thepage}

\renewcommand\headrulewidth{0.4pt}
\renewcommand\footrulewidth{0.4pt}

\setlength\parindent{0pt}
\setcounter{secnumdepth}{0}

\newcommand{\hmwkClass}{MATH 3380 / Analysis 1}        % Class
\newcommand{\hmwkClassInstructor}{Dr. Welsh}           % Instructor
\newcommand{\hmwkAuthorName}{\textbf{Joshua Mitchell}} % Author

%
% Title Page
%

\title{
    \vspace{2in}
    \textmd{\textbf{\hmwkClass:\ \hmwkTitle}}\\
    \normalsize\vspace{0.1in}\small\vspace{0.1in}\large{\textit{\hmwkClassInstructor}}
    \vspace{3in}
}

\author{\hmwkAuthorName}
\date{}

\renewcommand{\part}[1]{\textbf{\large Part \Alph{partCounter}}\stepcounter{partCounter}\\}

% Integral dx
\newcommand{\dx}{\mathrm{d}x}

%
% Various Helper Commands
%

% For derivatives
\newcommand{\deriv}[1]{\frac{\mathrm{d}}{\mathrm{d}x} (#1)}

% For partial derivatives
\newcommand{\pderiv}[2]{\frac{\partial}{\partial #1} (#2)}


% Alias for the Solution section header
\newcommand{\solution}{\textbf{\large Solution}}

% Probability commands: Expectation, Variance, Covariance, Bias
\newcommand{\E}{\mathrm{E}}
\newcommand{\Var}{\mathrm{Var}}
\newcommand{\Cov}{\mathrm{Cov}}
\newcommand{\Bias}{\mathrm{Bias}}

% Formatting commands:

\newcommand{\mt}[1]{\ensuremath{#1}}
\newcommand{\nm}[1]{\textrm{#1}}

\newcommand\bsc[2][\DefaultOpt]{%
  \def\DefaultOpt{#2}%
  \section[#1]{#2}%
}
\newcommand\ssc[2][\DefaultOpt]{%
  \def\DefaultOpt{#2}%
  \subsection[#1]{#2}%
}
\newcommand{\bgpf}{\begin{proof} $ $\newline}

\newcommand{\bgeq}{\begin{equation*}}
\newcommand{\eeq}{\end{equation*}}	

\newcommand{\balist}{\begin{enumerate}[label=\alph*.]}
\newcommand{\elist}{\end{enumerate}}

\newcommand{\bilist}{\begin{enumerate}[label=\roman*)]}	

\newcommand{\bgsp}{\begin{split}}
% \newcommand{\esp}{\end{split}} % doesn't work for some reason.

\newcommand\prs[1]{~~~\textbf{(#1)}}

\newcommand{\lt}[1]{\textbf{Let: } #1}
     							   %  if you're setting it to be true
\newcommand{\supp}[1]{\textbf{Suppose: } #1}
     							   %  Suppose (if it'll end up false)
\newcommand{\wts}[1]{\textbf{Want to show: } #1}
     							   %  Want to show
\newcommand{\as}[1]{\textbf{Assume: } #1}
     							   %  if you think it follows from truth
\newcommand{\bpth}[1]{\textbf{(#1)}}

\newcommand{\step}[2]{\begin{equation}\tag{#2}#1\end{equation}}
\newcommand{\epf}{\end{proof}}

\newcommand{\dbs}[3]{\mt{#1_{#2_#3}}}

\newcommand{\sidenote}[1]{-----------------------------------------------------------------Side Note----------------------------------------------------------------
#1 \

---------------------------------------------------------------------------------------------------------------------------------------------}

% Analysis / Logical commands:

\newcommand{\br}{\mt{\mathbb{R}} }       % |R
\newcommand{\bq}{\mt{\mathbb{Q}} }       % |Q
\newcommand{\bn}{\mt{\mathbb{N}} }       % |N
\newcommand{\bc}{\mt{\mathbb{C}} }       % |C
\newcommand{\bz}{\mt{\mathbb{Z}} }       % |Z

\newcommand{\ep}{\mt{\epsilon} }         % epsilon
\newcommand{\fa}{\mt{\forall} }          % for all
\newcommand{\afa}{\mt{\alpha} }
\newcommand{\bta}{\mt{\beta} }
\newcommand{\mem}{\mt{\in} }
\newcommand{\exs}{\mt{\exists} }

\newcommand{\es}{\mt{\emptyset} }        % empty set
\newcommand{\sbs}{\mt{\subset} }         % subset of
\newcommand{\fs}[2]{\{\uw{#1}{1}, \uw{#1}{2}, ... \uw{#1}{#2}\}}

\newcommand{\lra}{ \mt{\longrightarrow} } % implies ----->
\newcommand{\rar}{ \mt{\Rightarrow} }     % implies -->

\newcommand{\lla}{ \mt{\longleftarrow} }  % implies <-----
\newcommand{\lar}{ \mt{\Leftarrow} }      % implies <--

\newcommand{\av}[1]{\mt{|}#1\mt{|}}  % absolute value

\newcommand{\prn}[1]{(#1)}
\newcommand{\bk}[1]{\{#1\}}

\newcommand{\ps}{\mt{+} }
\newcommand{\ms}{\mt{-} }

\newcommand{\ls}{\mt{<} }
\newcommand{\gr}{\mt{>} }

\newcommand{\lse}{\mt{\leq} }
\newcommand{\gre}{\mt{\geq} }

\newcommand{\eql}{\mt{=} }

\newcommand{\pr}{\mt{^\prime} } 		   % prime (i.e. R')
\newcommand{\uw}[2]{#1\mt{_{#2}}}
\newcommand{\uf}[2]{#1\mt{^{#2}}}
\newcommand{\frc}[2]{\mt{\frac{#1}{#2}}}
\newcommand{\lmti}[1]{\mt{\displaystyle{\lim_{#1 \to \infty}}}}
\newcommand{\limt}[2]{\mt{\displaystyle{\lim_{#1 \to #2}}}}

\newcommand{\bnm}[2]{\mt{#1\setminus{#2}}}
\newcommand{\bnt}[2]{\mt{\textrm{#1}\setminus{\textrm{#2}}}}
\newcommand{\bi}{\bnm{\mathbb{R}}{\mathbb{Q}}}

\newcommand{\urng}[2]{\mt{\bigcup_{#1}^{#2}}}
\newcommand{\nrng}[2]{\mt{\bigcap_{#1}^{#2}}}
\newcommand{\nck}[2]{\mt{{#1 \choose #2}}}

\newcommand{\nbho}[3]{\textrm{N(}#1, #2\textrm{) }\cap \textrm{ #3} \neq \emptyset}
     							   %  N(x, eps) intersect S \= emptyset
\newcommand{\nbhe}[3]{\textrm{N(}#1, #2\textrm{) }\cap \textrm{ #3} = \emptyset}
     							   %  N(x, eps) intersect S  = emptyset
\newcommand{\dnbho}[3]{\textrm{N*(}#1, #2\textrm{) }\cap \textrm{ #3} \neq \emptyset}
     							   %  N*(x, eps) intersect S \= emptyset
\newcommand{\dnbhe}[3]{\textrm{N*(}#1, #2\textrm{) }\cap \textrm{ #3} = \emptyset}
     							   %  N*(x, eps) intersect S = emptyset
     							   
\newcommand{\eqn}[1]{\[#1\]}
\newcommand{\splt}[1]{\begin{split}#1\end{split}}

\newcommand{\infy}{\mt{\infty} }
     							 
% ----------

\begin{document}
HW 8: pages 193, \#1, 2, 3, 5, 9, 10, 17

For 2(c), see Theorem 1 and Example 9 from Lecture 15

Make sure when you do these problems, justify the answer by either writing down the theorem name or providing a counter example.

\bsc{Exercise 1}{

Mark each statement True or False. Justify each answer.

\balist
\item A sequence \prn{\uw{s}{n}} converges to s iff every subsequence of (\uw{s}{n}) converges to s.
	
	\textbf{True.} By Theorem 4.4.4.
	
	
\item Every bounded sequence is convergent.
	
	\textbf{False.}
	
	Counter example: \prn{\uw{s}{n}} \eql \uf{(-1)}{n}
\item Let (\uw{s}{n}) be a bounded sequence. If \prn{\uw{s}{n}} oscillates, then the set S of subsequential limits of \prn{\uw{s}{n}} contains at least two points.

	\textbf{True.} If S oscillates, then lim inf S \ls lim sup S. This implies that these are two different points.
\item Let \prn{\uw{s}{n}} be a bounded sequence and let m \eql lim sup \uw{s}{n}.
	
	Then, \fa \ep \gr 0, \exs N \mem \bn st N \gre n implies \uw{s}{n} \gr m \ms \ep
	
	\textbf{True.}
	\bgpf
	\lt{\ep \gr 0}
	
	Since \uw{s}{n} is bounded, let S be the set containing the range of \uw{s}{n}.
	
	By definition, \exs some \dbs{s}{n}{k} st lim \dbs{s}{n}{k} \eql m where k \mem \bn 
	
	Since lim \dbs{s}{n}{k} \eql m,
	
	\exs N \mem \bn st N \gre \uw{n}{k} implies \av{\dbs{s}{n}{k} \ms m} \ls \ep
	
	\av{\dbs{s}{n}{k} \ms m} \ls \ep
	
	$-$\ep \ls \dbs{s}{n}{k} \ms m \ls \ep
	
	m \ms \ep \ls \dbs{s}{n}{k} \ls m \ps \ep \bpth{1}
	
	So, by \bpth{1},
	
	\exs some N \mem \bn st n \gre N implies \uw{s}{n} \gr m \ms \ep
	
	\epf
\item If \prn{\uw{s}{n}} is unbounded above, then \prn{\uw{s}{n}} contains a subsequence that has \infy as a limit.
	
	\textbf{True.} By Theorem 4.4.8.
\elist

}

\bsc{Exercise 2}{

Mark each statement True or False. Justify each answer.

\balist
\item Every sequence has a convergent subsequence.
	
	\textbf{False.} Let \uw{s}{n} \eql n
\item The set of subsequential limits of a bounded sequence is always nonempty.
	
	\textbf{True.} By Theorem 4.4.8
\item \prn{\uw{s}{n}} converges to s iff lim inf \uw{s}{n} \eql lim sup \uw{s}{n} \eql s
	
	\textbf{True.} By Definition 4.4.9 and exercise 9.
\item Let (\uw{s}{n}) be a bounded sequence and let m \eql lim sup \uw{s}{n}. Then, \fa \ep \gr 0, there are infinitely many terms in the sequence greater than m \ms \ep.
	
	\textbf{True.} By Theorem 4.4.7, \uw{s}{n} has a convergent subsequence.
	
	Let \uw{t}{n} be a subsequence of \uw{s}{n} st \lmti{n} \uw{t}{n} \eql m
	
	By definition,
	
	\fa \ep \gr 0, \exs N \mem \bn st n \gre N implies \av{\uw{t}{n} \ms m} \ls \ep
	
	so,
	
	$-$\ep \ls \uw{t}{n} \ms m \ls \ep
	
	m $-$ \ep \ls \uw{t}{n}
	
	Pick \uw{\ep}{2} to be \frc{\ep}{2}
	
	Then, 
	
	\exs N(\uw{\ep}{2}) st m $-$ \ep \ls \uw{t}{N(\uw{\ep}{2})}
	
	Inductively, we can let \uw{\ep}{3} \eql \frc{\epsilon_2}{2}, and so on.
	
	Hence, since there are infinitely many terms in \uw{t}{n} greater than m \ms \ep, the same is true for \uw{s}{n}.
	
	
\item If \prn{\uw{s}{n}} is unbounded above, then lim inf \uw{s}{n} \eql lim sup \uw{s}{n} \eql \infy
	
	\textbf{True.}
	
	\supp{\uw{s}{n} has a subsequence \uw{t}{n} such that \lmti{n} \uw{t}{n} \eql t where t $\neq$ \infy (but could be negative infinity)}
	
	So,
	
	\fa \ep \gr 0, \exs N \mem \bn st n \gre N implies \av{\uw{t}{n} \ms t} \ls \ep
	
	Notice also, that since \uw{s}{n} is unbounded above,
	
	\fa m \mem \br, \exs \uw{N}{m} \mem \bn st \dbs{s}{N}{m} \gr m
	
	That means that \exs some N for \uw{t}{n} st \uw{t}{N} \gr m
	
	If we let m \eql t, then
	
	\exs some \uw{N}{1} for \uw{t}{n} st \uw{t}{N_1} \gr t \eql m
	
	If we let m \eql t \ps 1, then
	
	\exs some \uw{N}{2} for \uw{t}{n} st \uw{t}{N_2} \gr m \eql t \ps 1
	
	Inductively, \uw{t}{n} has an infinite amount of values above t, and is increasing: a contradiction.
	
	Thus, \uw{t}{n} is unbounded above.
\elist 

}

\bsc{Exercise 3}{

For each sequence, find the set S of subsequential limits, the limit inferior, and the limit superior.

\balist
\item \uw{s}{n} \eql 1 + $(-1)^n$
	
	S \eql \bk{0, 2}, \uw{s}{*} \eql 0, \uf{s}{*} \eql 2 
\item \uw{t}{n} \eql (0, \frc{1}{2}, \frc{2}{3}, \frc{1}{4}, \frc{4}{5}, \frc{1}{6}, \frc{6}{7})
	
	S \eql \bk{0, \frc{1}{2}, \frc{2}{3}, \frc{1}{4}, \frc{4}{5}, \frc{1}{6}, \frc{6}{7}}, \uw{s}{*} \eql 0, \uf{s}{*} \eql \frc{6}{7}
\item \uw{u}{n} \eql \uf{n}{2}($-1$ \ps ($-1$)$^n$)
	
	S \eql \bk{0}, \uw{s}{*} \eql $-\infty$, \uf{s}{*} \eql 0
\item \uw{v}{n} \eql n sin \frc{n\pi}{2}
	
	S \eql \bk{0}, \uw{s}{*} \eql $-\infty$, \uf{s}{*} \eql $\infty$
\elist

}

\bsc{Exercise 5}{

Use exercise 4.3.14 to find the limit of each sequence:

\textbf{Known:} \uw{t}{n} \eql \prn{1 \ps \frc{1}{n}}$^n$  and \lmti{n} \uw{t}{n} \eql e

\balist
\item \uw{s}{n} \eql (1 + \frc{1}{2n})$^{2n}$
	
	We can just think of \uw{s}{n} as a subsequence of \uw{t}{n} (the original e sequence),
	
	so therefore it has the same limit: e.
\item \uw{s}{n} \eql (1 + \frc{1}{n})$^{2n}$
	
	\eql ((1 + \frc{1}{n})$^n$)$^2$
	
	so, \lmti{n} \uw{s}{n} \eql e$^2$
\item \uw{s}{n} \eql (1 + \frc{1}{n})$^{n - 1}$

	\eql (1 + \frc{1}{n})$^{n}$(1 + \frc{1}{n})$^{-1}$
	
	so, \lmti{n} \uw{s}{n} \eql e * 1 \eql e
\item \uw{s}{n} \eql (\frc{n	}{n + 1})$^{n}$
	
	\eql \frc{1}{(\frc{n + 1}{n})^n}
	
	\eql \frc{1}{(1 + \frc{1}{n})^n}
	
	so, \lmti{n} \uw{s}{n} \eql \frc{1}{e}
\item \uw{s}{n} \eql (1 + \frc{1	}{2n})$^{n}$
	
	\eql ((1 + \frc{1}{2n})$^{2n}$)$^\frac{1}{2}$
	
	so, \lmti{n} \uw{s}{n} \eql $\sqrt{e}$
	
\item \uw{s}{n} \eql (\frc{n + 2	}{n + 1})$^{n + 3}$
	
	\eql (\frc{n + 2}{n + 1})$^{n}$(\frc{n + 2	}{n + 1})$^{3}$
	
	\eql (\frc{n}{n + 1} \ps \frc{2}{n + 1})$^{n}$(\frc{n + 2	}{n + 1})$^{3}$
	
	Now, \lmti{n} (\frc{n}{n + 1} \ps \frc{2}{n + 1})$^{n}$(\frc{n + 2	}{n + 1})$^{3}$ \eql (e + 0) $\times$ 1 by \bpth{d}
	
	so, \lmti{n} \uw{s}{n} \eql e
\elist

}

\bsc{Exercise 9}{

Let \prn{\uw{s}{n}} be a bounded sequence.

\as{lim inf \uw{s}{n} \eql lim sup \uw{s}{n} \eql s}

Prove that \prn{\uw{s}{n}} is convergent and that lim \uw{s}{n} \eql s

\

Let S \sbs \br be the range of limits for any subsequence of \uw{s}{n}.

Since lim inf \uw{s}{n} \eql s, inf S \eql s.

Since lim sup \uw{s}{n} \eql s, sup S \eql s.

By Corollary 4.4.12, S contains s.

Since inf S \eql sup S \eql s, the range of S is just \bk{s}. \bpth{1}

Since \uw{s}{n} is bounded, it can't diverge to \infy or $-\infty$.

However, suppose \uw{s}{n} diverges in general.

Then, \exs \ep(\uw{s}{n}) st \av{\uw{s}{n} \ms s} \gr \ep(\uw{s}{n}) for all n \gre some N \mem \bn

Since there are infinitely many n \gre N, \exs \dbs{s}{n}{k} (a subsequence of \uw{s}{n}) st 

\av{\dbs{s}{n}{k} \ms s} \gre \ep(\uw{s}{n}) where \uw{n}{k} \eql N \ps k, k \mem \bn

Since \dbs{s}{n}{k} is bounded (because \uw{s}{n} is bounded), it itself has a convergent subsequence (for notation reasons lets call it \dbs{t}{n}{k})

Notice that \dbs{t}{n}{k} is a convergent subsequence of \uw{s}{n}, but it's limit is not s (since \exs an \ep st \av{\uw{s}{n} \ms s} \gr \ep), a contradiction.

Hence, \uw{s}{n} must converge to s.

}

\bsc{Exercise 10}{

\as{x \gr 1}

Prove that lim \uf{x}{\frc{1}{n}} \eql 1

\lmti{n}\uf{x}{\frc{1}{n}} \eql 1 if

\fa \ep \gr 0, \exs N \mem \bn st N \gre n implies \av{\uf{x}{\frc{1}{n}} \ms 1} \ls \ep

\lt{\ep \gr 0}

\av{\uf{x}{\frc{1}{n}} \ms 1} \ls \ep

Since x \gr 1 and n \mem \bn,

\uf{x}{\frc{1}{n}} \ms 1 \ls \ep

\uf{x}{\frc{1}{n}} \ls \ep \ps 1

(\uf{x}{\frc{1}{n}})$^n$ \ls (\ep \ps 1)$^n$

x \ls (\ep \ps 1)$^n$

ln x \ls n ln (\ep \ps 1)

\frc{\textrm{ln x}}{\textrm{ln (\ep + 1)}} \ls n

So, if \frc{\textrm{ln x}}{\textrm{ln (\ep + 1)}} \ls N,

then \exs N st \av{\uf{x}{\frc{1}{n}} \ms 1} \ls \ep

Hence, result.
}

\bsc{Exercise 17}{

Prove that if lim sup \uw{s}{n} \eql \infy and k \gr 0, then lim sup (k\uw{s}{n}) \eql \infy

\sidenote{
\textbf{Question}: Is it a valid proof to say that since
\eqn{t_n = \sum_{i = 1}^n \frac{1}{n}}
is the slowest possible diverging sequence (without constants of course),

since
\eqn{\lmti{n} kt_n = k \infty = \infty}
then \lmti{n} of k $\times$ any sequence diverging to \infy is also $\infty$?

So, therefore lim sup (k * any sequence diverging to \infy) is also \infy?
}

\lt{\uw{t}{n} be a subsequence of \uw{s}{n} st \lmti{n} \uw{t}{n} \eql \infy}

Algebraically, k \lmti{n} \uw{t}{n} \eql \lmti{n} k\uw{t}{n} \eql k\infy \eql \infy

Hence, lim sup (k\uw{s}{n}) \eql \infy


}

\end{document}