% Thank you Josh Davis for this template!
% https://github.com/jdavis/latex-homework-template/blob/master/homework.tex

\documentclass{article}

\newcommand{\hmwkTitle}{Lec\ \#19}

% % ----------

% Packages

\usepackage{fancyhdr}
\usepackage{extramarks}
\usepackage{amsmath}
\usepackage{amssymb}
\usepackage{amsthm}
\usepackage{amsfonts}
\usepackage{tikz}
\usepackage[plain]{algorithm}
\usepackage{algpseudocode}
\usepackage{enumitem}
\usepackage{chngcntr}

% Libraries

\usetikzlibrary{automata, positioning, arrows}

%
% Basic Document Settings
%

\topmargin=-0.45in
\evensidemargin=0in
\oddsidemargin=0in
\textwidth=6.5in
\textheight=9.0in
\headsep=0.25in

\linespread{1.1}

\pagestyle{fancy}
\lhead{\hmwkAuthorName}
\chead{}
\rhead{\hmwkClass\ (\hmwkClassInstructor): \hmwkTitle}
\lfoot{\lastxmark}
\cfoot{\thepage}

\renewcommand\headrulewidth{0.4pt}
\renewcommand\footrulewidth{0.4pt}

\setlength\parindent{0pt}
\setcounter{secnumdepth}{0}

\newcommand{\hmwkClass}{MATH 3380 / Analysis 1}        % Class
\newcommand{\hmwkClassInstructor}{Dr. Welsh}           % Instructor
\newcommand{\hmwkAuthorName}{\textbf{Joshua Mitchell}} % Author

%
% Title Page
%

\title{
    \vspace{2in}
    \textmd{\textbf{\hmwkClass:\ \hmwkTitle}}\\
    \normalsize\vspace{0.1in}\small\vspace{0.1in}\large{\textit{\hmwkClassInstructor}}
    \vspace{3in}
}

\author{\hmwkAuthorName}
\date{}

\renewcommand{\part}[1]{\textbf{\large Part \Alph{partCounter}}\stepcounter{partCounter}\\}

% Integral dx
\newcommand{\dx}{\mathrm{d}x}

%
% Various Helper Commands
%

% For derivatives
\newcommand{\deriv}[1]{\frac{\mathrm{d}}{\mathrm{d}x} (#1)}

% For partial derivatives
\newcommand{\pderiv}[2]{\frac{\partial}{\partial #1} (#2)}


% Alias for the Solution section header
\newcommand{\solution}{\textbf{\large Solution}}

% Probability commands: Expectation, Variance, Covariance, Bias
\newcommand{\E}{\mathrm{E}}
\newcommand{\Var}{\mathrm{Var}}
\newcommand{\Cov}{\mathrm{Cov}}
\newcommand{\Bias}{\mathrm{Bias}}

% Formatting commands:

\newcommand{\mt}[1]{\ensuremath{#1}}
\newcommand{\nm}[1]{\textrm{#1}}

\newcommand\bsc[2][\DefaultOpt]{%
  \def\DefaultOpt{#2}%
  \section[#1]{#2}%
}
\newcommand\ssc[2][\DefaultOpt]{%
  \def\DefaultOpt{#2}%
  \subsection[#1]{#2}%
}
\newcommand{\bgpf}{\begin{proof} $ $\newline}

\newcommand{\bgeq}{\begin{equation*}}
\newcommand{\eeq}{\end{equation*}}	

\newcommand{\balist}{\begin{enumerate}[label=\alph*.]}
\newcommand{\elist}{\end{enumerate}}

\newcommand{\bilist}{\begin{enumerate}[label=\roman*)]}	

\newcommand{\bgsp}{\begin{split}}
% \newcommand{\esp}{\end{split}} % doesn't work for some reason.

\newcommand\prs[1]{~~~\textbf{(#1)}}

\newcommand{\lt}[1]{\textbf{Let: } #1}
     							   %  if you're setting it to be true
\newcommand{\supp}[1]{\textbf{Suppose: } #1}
     							   %  Suppose (if it'll end up false)
\newcommand{\wts}[1]{\textbf{Want to show: } #1}
     							   %  Want to show
\newcommand{\as}[1]{\textbf{Assume: } #1}
     							   %  if you think it follows from truth
\newcommand{\bpth}[1]{\textbf{(#1)}}

\newcommand{\step}[2]{\begin{equation}\tag{#2}#1\end{equation}}
\newcommand{\epf}{\end{proof}}

\newcommand{\dbs}[3]{\mt{#1_{#2_#3}}}

\newcommand{\sidenote}[1]{-----------------------------------------------------------------Side Note----------------------------------------------------------------
#1 \

---------------------------------------------------------------------------------------------------------------------------------------------}

% Analysis / Logical commands:

\newcommand{\br}{\mt{\mathbb{R}} }       % |R
\newcommand{\bq}{\mt{\mathbb{Q}} }       % |Q
\newcommand{\bn}{\mt{\mathbb{N}} }       % |N
\newcommand{\bc}{\mt{\mathbb{C}} }       % |C
\newcommand{\bz}{\mt{\mathbb{Z}} }       % |Z

\newcommand{\ep}{\mt{\epsilon} }         % epsilon
\newcommand{\fa}{\mt{\forall} }          % for all
\newcommand{\afa}{\mt{\alpha} }
\newcommand{\bta}{\mt{\beta} }
\newcommand{\mem}{\mt{\in} }
\newcommand{\exs}{\mt{\exists} }

\newcommand{\es}{\mt{\emptyset}}        % empty set
\newcommand{\sbs}{\mt{\subset} }         % subset of
\newcommand{\fs}[2]{\{\uw{#1}{1}, \uw{#1}{2}, ... \uw{#1}{#2}\}}

\newcommand{\lra}{ \mt{\longrightarrow} } % implies ----->
\newcommand{\rar}{ \mt{\Rightarrow} }     % implies -->

\newcommand{\lla}{ \mt{\longleftarrow} }  % implies <-----
\newcommand{\lar}{ \mt{\Leftarrow} }      % implies <--

\newcommand{\eql}{\mt{=} }
\newcommand{\pr}{\mt{^\prime} } 		   % prime (i.e. R')
\newcommand{\uw}[2]{#1\mt{_{#2}}}
\newcommand{\frc}[2]{\mt{\frac{#1}{#2}}}

\newcommand{\bnm}[2]{\mt{#1\setminus{#2}}}
\newcommand{\bnt}[2]{\mt{\textrm{#1}\setminus{\textrm{#2}}}}
\newcommand{\bi}{\bnm{\mathbb{R}}{\mathbb{Q}}}

\newcommand{\urng}[2]{\mt{\bigcup_{#1}^{#2}}}
\newcommand{\nrng}[2]{\mt{\bigcap_{#1}^{#2}}}

\newcommand{\nbho}[3]{\textrm{N(}#1, #2\textrm{) }\cap \textrm{ #3} \neq \emptyset}
     							   %  N(x, eps) intersect S \= emptyset
\newcommand{\nbhe}[3]{\textrm{N(}#1, #2\textrm{) }\cap \textrm{ #3} = \emptyset}
     							   %  N(x, eps) intersect S  = emptyset
\newcommand{\dnbho}[3]{\textrm{N*(}#1, #2\textrm{) }\cap \textrm{ #3} \neq \emptyset}
     							   %  N*(x, eps) intersect S \= emptyset
\newcommand{\dnbhe}[3]{\textrm{N*(}#1, #2\textrm{) }\cap \textrm{ #3} = \emptyset}
     							   %  N*(x, eps) intersect S = emptyset
     							 


% ----------

% ----------

% Packages

\usepackage{fancyhdr}
\usepackage{extramarks}
\usepackage{amsmath}
\usepackage{amssymb}
\usepackage{amsthm}
\usepackage{amsfonts}
\usepackage{tikz}
\usepackage[plain]{algorithm}
\usepackage{algpseudocode}
\usepackage{enumitem}
\usepackage{chngcntr}

% Libraries

\graphicspath{{/Users/jm/iclouddrive/3380pics/}}

\usetikzlibrary{automata, positioning, arrows}

%
% Basic Document Settings
%

\topmargin=-0.45in
\evensidemargin=0in
\oddsidemargin=0in
\textwidth=6.5in
\textheight=9.0in
\headsep=0.25in

\linespread{1.1}

\pagestyle{fancy}
\lhead{\hmwkAuthorName}
\chead{}
\rhead{\hmwkClass\ (\hmwkClassInstructor): \hmwkTitle}
\lfoot{\lastxmark}
\cfoot{\thepage}

\renewcommand\headrulewidth{0.4pt}
\renewcommand\footrulewidth{0.4pt}

\setlength\parindent{0pt}
\setcounter{secnumdepth}{0}

\newcommand{\hmwkClass}{MATH 3380 / Analysis 1}        % Class
\newcommand{\hmwkClassInstructor}{Dr. Welsh}           % Instructor
\newcommand{\hmwkAuthorName}{\textbf{Joshua Mitchell}} % Author

%
% Title Page
%

\title{
    \vspace{2in}
    \textmd{\textbf{\hmwkClass:\ \hmwkTitle}}\\
    \normalsize\vspace{0.1in}\small\vspace{0.1in}\large{\textit{\hmwkClassInstructor}}
    \vspace{3in}
}

\author{\hmwkAuthorName}
\date{}

\renewcommand{\part}[1]{\textbf{\large Part \Alph{partCounter}}\stepcounter{partCounter}\\}

% Integral dx
\newcommand{\dx}{\mathrm{d}x}

%
% Various Helper Commands
%

% For derivatives
\newcommand{\deriv}[1]{\frac{\mathrm{d}}{\mathrm{d}x} (#1)}

% For partial derivatives
\newcommand{\pderiv}[2]{\frac{\partial}{\partial #1} (#2)}


% Alias for the Solution section header
\newcommand{\solution}{\textbf{\large Solution}}

% Probability commands: Expectation, Variance, Covariance, Bias
\newcommand{\E}{\mathrm{E}}
\newcommand{\Var}{\mathrm{Var}}
\newcommand{\Cov}{\mathrm{Cov}}
\newcommand{\Bias}{\mathrm{Bias}}

% Formatting commands:

\newcommand{\mt}[1]{\ensuremath{#1}}
\newcommand{\nm}[1]{\textrm{#1}}

\newcommand\bsc[2][\DefaultOpt]{%
  \def\DefaultOpt{#2}%
  \section[#1]{#2}%
}
\newcommand\ssc[2][\DefaultOpt]{%
  \def\DefaultOpt{#2}%
  \subsection[#1]{#2}%
}
\newcommand{\bgpf}{\begin{proof} $ $\newline}

\newcommand{\bgeq}{\begin{equation*}}
\newcommand{\eeq}{\end{equation*}}	

\newcommand{\balist}{\begin{enumerate}[label=\alph*.]}
\newcommand{\elist}{\end{enumerate}}

\newcommand{\bilist}{\begin{enumerate}[label=\roman*)]}	

\newcommand{\bgsp}{\begin{split}}
% \newcommand{\esp}{\end{split}} % doesn't work for some reason.

\newcommand\prs[1]{~~~\textbf{(#1)}}

\newcommand{\lt}[1]{\textbf{Let: } #1}
     							   %  if you're setting it to be true
\newcommand{\supp}[1]{\textbf{Suppose: } #1}
     							   %  Suppose (if it'll end up false)
\newcommand{\wts}[1]{\textbf{Want to show: } #1}
     							   %  Want to show
\newcommand{\as}[1]{\textbf{Assume: } #1}
     							   %  if you think it follows from truth
\newcommand{\bpth}[1]{\textbf{(#1)}}

\newcommand{\step}[2]{\begin{equation}\tag{#2}#1\end{equation}}
\newcommand{\epf}{\end{proof}}

\newcommand{\dbs}[3]{\mt{#1_{#2_#3}}}

\newcommand{\sidenote}[1]{-----------------------------------------------------------------Side Note----------------------------------------------------------------
#1 \

---------------------------------------------------------------------------------------------------------------------------------------------}

% Analysis / Logical commands:

\newcommand{\br}{\mt{\mathbb{R}} }       % |R
\newcommand{\bq}{\mt{\mathbb{Q}} }       % |Q
\newcommand{\bn}{\mt{\mathbb{N}} }       % |N
\newcommand{\bc}{\mt{\mathbb{C}} }       % |C
\newcommand{\bz}{\mt{\mathbb{Z}} }       % |Z

\newcommand{\ep}{\mt{\epsilon} }         % epsilon
\newcommand{\fa}{\mt{\forall} }          % for all
\newcommand{\afa}{\mt{\alpha} }
\newcommand{\bta}{\mt{\beta} }
\newcommand{\mem}{\mt{\in} }
\newcommand{\exs}{\mt{\exists} }

\newcommand{\es}{\mt{\emptyset} }        % empty set
\newcommand{\sbs}{\mt{\subset} }         % subset of
\newcommand{\fs}[2]{\{\uw{#1}{1}, \uw{#1}{2}, ... \uw{#1}{#2}\}}

\newcommand{\lra}{ \mt{\longrightarrow} } % implies ----->
\newcommand{\rar}{ \mt{\Rightarrow} }     % implies -->

\newcommand{\lla}{ \mt{\longleftarrow} }  % implies <-----
\newcommand{\lar}{ \mt{\Leftarrow} }      % implies <--

\newcommand{\av}[1]{\mt{|}#1\mt{|}}  % absolute value

\newcommand{\prn}[1]{(#1)}
\newcommand{\bk}[1]{\{#1\}}

\newcommand{\ps}{\mt{+} }
\newcommand{\ms}{\mt{-} }

\newcommand{\ls}{\mt{<} }
\newcommand{\gr}{\mt{>} }

\newcommand{\lse}{\mt{\leq} }
\newcommand{\gre}{\mt{\geq} }

\newcommand{\eql}{\mt{=} }

\newcommand{\pr}{\mt{^\prime} } 		   % prime (i.e. R')
\newcommand{\uw}[2]{#1\mt{_{#2}}}
\newcommand{\uf}[2]{#1\mt{^{#2}}}
\newcommand{\frc}[2]{\mt{\frac{#1}{#2}}}
\newcommand{\lmti}[1]{\mt{\displaystyle{\lim_{#1 \to \infty}}}}
\newcommand{\limt}[2]{\mt{\displaystyle{\lim_{#1 \to #2}}}}

\newcommand{\bnm}[2]{\mt{#1\setminus{#2}}}
\newcommand{\bnt}[2]{\mt{\textrm{#1}\setminus{\textrm{#2}}}}
\newcommand{\bi}{\bnm{\mathbb{R}}{\mathbb{Q}}}

\newcommand{\urng}[2]{\mt{\bigcup_{#1}^{#2}}}
\newcommand{\nrng}[2]{\mt{\bigcap_{#1}^{#2}}}
\newcommand{\nck}[2]{\mt{{#1 \choose #2}}}

\newcommand{\nbho}[3]{\textrm{N(}#1, #2\textrm{) }\cap \textrm{ #3} \neq \emptyset}
     							   %  N(x, eps) intersect S \= emptyset
\newcommand{\nbhe}[3]{\textrm{N(}#1, #2\textrm{) }\cap \textrm{ #3} = \emptyset}
     							   %  N(x, eps) intersect S  = emptyset
\newcommand{\dnbho}[3]{\textrm{N*(}#1, #2\textrm{) }\cap \textrm{ #3} \neq \emptyset}
     							   %  N*(x, eps) intersect S \= emptyset
\newcommand{\dnbhe}[3]{\textrm{N*(}#1, #2\textrm{) }\cap \textrm{ #3} = \emptyset}
     							   %  N*(x, eps) intersect S = emptyset
     							   
\newcommand{\eqn}[1]{\[#1\]}
\newcommand{\splt}[1]{\begin{split}#1\end{split}}

\newcommand{\infy}{\mt{\infty} }
     							 
% ----------

\begin{document}
HW 9: page 203 - 205, \#1, 2, 3(a)(c)(e)(g), 7(c), 13, 16, 18, 19

\bsc{Chapter 5 Continued:}{
\ssc{Theorem 5.1.8}{

Let f : D \lra \br and let c \mem D\pr

Then,

\limt{x}{c} f(x) \eql L \mem \br iff for \textbf{every} sequence \bk{\uw{s}{n}} in D st \uw{s}{n} $\neq$ c \fa n \mem \bn and \lmti{n} \uw{s}{n} \eql c it follows that \lmti{n} \bk{f(\uw{s}{n})} \eql L

So,

\limt{x}{c} f(x) \eql L

for \ep \gr 0, \exs $\delta$ \gr 0 st

\av{f(x) \ms L} \ls \ep (i.e. L \ms \ep \ls f(x) \ls L \ps \ep) whenever 0 \ls \av{x	 \ms c} \ls $\delta$
}

\ssc{Corollary 5.1.9}{

If f : D \lra \br and if c \mem D\pr,

then

if \limt{x}{c} f(x) \eql L, then L is unique.

\bgpf
Assume that

\step{\limt{x}{c} f(x) = L_1}{1}
and
\step{\limt{x}{c} f(x) = L_2}{2}

Let \bk{\uw{s}{n}} be a sequence in D st

\uw{s}{n} $\neq$ c \fa n \mem \bn and \lmti{n} \uw{s}{n} \eql c

By \bpth{1} and Theorem 5.1.8, \lmti{n} f(\uw{s}{n}) \eql \uw{L}{1}.

And by \bpth{2} and Theorem 5.1.8, \lmti{n} f(\uw{s}{n}) \eql \uw{L}{2}

However, by Theorem 4.1.14, if a sequence converges, then its limit is unique.

So, \uw{L}{1} \eql \uw{L}{2}, hence, uniqueness.
\epf

}

\ssc{Theorem 5.1.10}{

Let f : D \lra \br and let c \mem D\pr

Then the following are equivalent:

\balist
\item f does not have a limit at c
\item \exs a sequence \bk{\uw{s}{n}} in D st \uw{s}{n} $\neq$ c \fa n \mem \bn and \lmti{n} \uw{s}{n} \eql c but \bk{f(\uw{s}{n})} is not convergent in \br

	(looks like the second part of Thm 5.1.8 except the opposite)
\elist

\bgpf
\lra 

We first prove that a \rar b by using the contrapositive. (i.e. not b implies not a)

Assume \bpth{b} is false.

Thus, for every sequence \bk{\uw{s}{n}} in D st \uw{s}{n} $\neq$ c \fa n \mem \bn and \lmti{n} \uw{s}{n} \eql c

it follows that \bk{f(\uw{s}{n})} converges in \br

\wts{\limt{x}{c} f(x) exists}

Let \bk{\uw{s}{n}} and \bk{\uw{t}{n}} be sequences in D st \uw{s}{n} $\neq$ c and \uw{t}{n} $\neq$ c \fa n \mem \bn in \lmti{n} \uw{s}{n} \eql c, \lmti{n} \uw{t}{n} \eql c.

Thus,

\exs \uw{L}{1}, \uw{L}{2} \mem \br st \lmti{n} f(\uw{s}{n}) \eql \uw{L}{1} and \lmti{n} f(\uw{t}{n}) \eql \uw{L}{2}

\wts{\uw{L}{1} \eql \uw{L}{2}}

Define the sequence \bk{\uw{u}{n}} in D by

\bk{\uw{u}{n}} \eql \uw{s}{1}, \uw{t}{1}, \uw{s}{2}, \uw{t}{2}, ... 

Then \uw{u}{n} $\neq$ c \fa n \mem \bn (should be obvious) and \lmti{n} \uw{u}{n} \eql c

So \exs L \mem \br st \lmti{n} f(\uw{u}{n}) \eql L

Since \uw{s}{n} and \uw{t}{n} are subsequences of \uw{u}{n}, \uw{s}{n} and \uw{t}{n} must also converge to L.

Thus, 

\uw{L}{1} \eql \uw{L}{2}

\sidenote{
To see that \lmti{n} \uw{u}{n} \eql c, 

\lt{\ep \gr 0}

Then \exs \uw{N}{1}, \uw{N}{2} \mem \bn st \av{\uw{s}{n} \ms c} \ls \ep for n \gre \uw{N}{1}, and

\av{\uw{t}{n} \ms c} \ls \ep for n \gre \uw{N}{2}

Let N \eql max \bk{\uw{N}{1}, \uw{N}{2}}

\

Also, consider \av{\uw{u}{n} \ms c}

Case:
\bilist
\item n is even
	
	Then n \eql 2k for some k \mem \bn and
	\eqn{|u_n - c| = |u_{2k} - c| = |t_k - c| < \epsilon \textrm{ for } k \gre N}
	So,
	\step{|u_n - c| < \epsilon \textrm{ for } n \gre 2N}{1}
\item n is odd
	
	Then n \eql 2k \ms 1 for some k \mem \bn and
	\eqn{|u_n - c| = |u_{2k - 1} - c| = |s_k - c| < \epsilon \textrm{ for } k \gre N}
	So,
	\step{|u_n - c| < \epsilon \textrm{ for } n = 2k - 1 \gre 2N - 1}{2}
\elist

From \bpth{1} and \bpth{2}, \lmti{n} \uw{u}{n} \eql c
}

Since \bk{f(\uw{u}{n})} converges to L and \bk{f(\uw{s}{n})}, \bk{f(\uw{t}{n})} are subsequences of \bk{f(\uw{f}{n})}, 

it follows by Theorem 4.4.4 that \uw{L}{1} \eql \uw{L}{2} \eql L

Hence, by Theorem 5.1.8, \limt{x}{c} f(x) \eql L
\epf
\lla

Direct proof of \bpth{b} implies \bpth{a}.

Assume \bpth{a} is false.

Then,

\exs L \mem \br st \limt{x}{c} f(x) \eql L. The result follows directly from Theorem 5.1.8

}

Recall: a iff b \lra not a iff not b

\ssc{Example 5.1.11}{

\lt{f(x) \eql sin(\frc{1}{x}) for x \gr 0}

Prove that \limt{x}{0} f(x) does not exist.

\bgpf

\lt{\uw{x}{n} \eql \frc{2}{n\pi} for n \mem \bn}

Then, 

\bk{\uw{x}{n}} is a sequence in D (x \gr 0) st

\uw{x}{n} $\neq$ 0 \fa n \mem \bn and \lmti{n} \uw{x}{n} \eql 0, but,  \fa n \mem \bn,

\eqn{f(x_n) = \sin(\frac{1}{x_n}) = \sin(\frac{n\pi}{2})}
Now, \bk{f(\uw{x}{n})} \eql 1, 0, $-$1, 0, 1, 0, $-1$, 0 ...

Notice that \bk{f(\uw{x}{n})} does not converge since it possesses subsequences that converge to different limits.

(i.e. \lmti{k} f(\uw{x}{2k}) \eql 0, \lmti{k} f(\uw{x}{4k - 3}) \eql 1, etc.)

By Theorem 5.1.10, f(x) does not have a limit at x \eql 0.
\epf

}

\ssc{Definition 5.1.12}{
Let f : D \lra \br and g : D \lra \br

Define:

\balist
\item The \textbf{sum} f \ps g : D \lra \br by (f \ps g)(x) \eql f(x) \ps g(x) \fa x \mem D
\item The \textbf{product} fg : D \lra \br by (fg)(x) \eql f(x)g(x) \fa x \mem D
\item The \textbf{multiple} kf : D \lra \br (kf)(x) \eql kf(x) \fa x \mem D, k \mem \br
\item The \textbf{quotient} \frc{f}{g} : D \lra \br (\frc{f}{g})(x) \eql \frc{f(x)}{g(x)} \fa x \mem D provided that g(x) $\neq$ 0 \fa x \mem D
\elist
}

\newpage

\ssc{Theorem 5.1.13}{

Let f : D \lra \br, g : D \lra \br and let c \mem D\pr

If \limt{x}{c} f(x) \eql L and \limt{x}{c} g(x) \eql M, then

\balist
\item \limt{x}{c} (f \ps g) \eql L \ps M
\item Let k \mem \br, \limt{x}{c} kf \eql kL
\item \limt{x}{c} (fg) \eql LM
\item \limt{x}{c} (\frc{f}{g}) \eql \frc{L}{M}, provided that M $\neq$ 0
\elist

\bgpf
\bpth{a} through \bpth{c} are similar to \bpth{d}.

\bpth{d}: Let \bk{\uw{s}{n}} be a sequence in D st \uw{s}{n} $\neq$ c \fa n \mem \bn and \lmti{n} \uw{s}{n} \eql c.

Then, by Theorem 5.1.8, \lmti{n} f(\uw{s}{n}) \eql L.

Now, \lmti{n} g(x) \eql M $\neq$ 0.

So \exs N \mem \bn st
\eqn{g(s_n) \neq 0 \textrm{ for } n \gre N}
(ask why? next time)

Then, \lmti{n} (\frc{f}{g})(\uw{s}{n}) \eql \lmti{n} \frc{f(s_n)}{g(s_n)} \eql \frc{\lmti{n} f(s_n)}{\lmti{n} g(s_n)} (by Theorem 4.2.11d) \eql \frc{L}{M}

Recall:

\av{x} \ms \av{y} \lse \av{\av{x} \ms \av{y}} \lse \av{x \ms y}

\av{y} \gre \av{x} \ms \av{x \ms y}

So,

\av{g(\uw{s}{n})} \gre \av{M} \ms \av{M \ms g(\uw{s}{n})}

and since,

\lmti{n} g(\uw{s}{n}) \eql M $\neq$ 0

\av{g(\uw{s}{n}) \ms M} \ls \frc{\av{M}}{2}

$-$\av{g(\uw{s}{n}) \ms M} \gr \frc{-\av{M}}{2}

for n \gre N

So,

\av{g(\uw{s}{n})} \gr \av{M} \ms \frc{\av{M}}{2} \eql \frc{\av{M}}{2} for n \gre N 
\epf 
}

\

Also, for the homework:

\limt{x}{c} P(x) \eql P(c) where P is a polynomial.
}

\end{document}