% Thank you Josh Davis for this template!
% https://github.com/jdavis/latex-homework-template/blob/master/homework.tex

\documentclass{article}

\newcommand{\hmwkTitle}{Lec\ \#8}

% ----------

% Packages

\usepackage{fancyhdr}
\usepackage{extramarks}
\usepackage{amsmath}
\usepackage{amssymb}
\usepackage{amsthm}
\usepackage{amsfonts}
\usepackage{tikz}
\usepackage[plain]{algorithm}
\usepackage{algpseudocode}
\usepackage{enumitem}
\usepackage{chngcntr}

% Libraries

\usetikzlibrary{automata, positioning, arrows}

%
% Basic Document Settings
%

\topmargin=-0.45in
\evensidemargin=0in
\oddsidemargin=0in
\textwidth=6.5in
\textheight=9.0in
\headsep=0.25in

\linespread{1.1}

\pagestyle{fancy}
\lhead{\hmwkAuthorName}
\chead{}
\rhead{\hmwkClass\ (\hmwkClassInstructor): \hmwkTitle}
\lfoot{\lastxmark}
\cfoot{\thepage}

\renewcommand\headrulewidth{0.4pt}
\renewcommand\footrulewidth{0.4pt}

\setlength\parindent{0pt}
\setcounter{secnumdepth}{0}

\newcommand{\hmwkClass}{MATH 3380 / Analysis 1}        % Class
\newcommand{\hmwkClassInstructor}{Dr. Welsh}           % Instructor
\newcommand{\hmwkAuthorName}{\textbf{Joshua Mitchell}} % Author

%
% Title Page
%

\title{
    \vspace{2in}
    \textmd{\textbf{\hmwkClass:\ \hmwkTitle}}\\
    \normalsize\vspace{0.1in}\small\vspace{0.1in}\large{\textit{\hmwkClassInstructor}}
    \vspace{3in}
}

\author{\hmwkAuthorName}
\date{}

\renewcommand{\part}[1]{\textbf{\large Part \Alph{partCounter}}\stepcounter{partCounter}\\}

% Integral dx
\newcommand{\dx}{\mathrm{d}x}

%
% Various Helper Commands
%

% For derivatives
\newcommand{\deriv}[1]{\frac{\mathrm{d}}{\mathrm{d}x} (#1)}

% For partial derivatives
\newcommand{\pderiv}[2]{\frac{\partial}{\partial #1} (#2)}


% Alias for the Solution section header
\newcommand{\solution}{\textbf{\large Solution}}

% Probability commands: Expectation, Variance, Covariance, Bias
\newcommand{\E}{\mathrm{E}}
\newcommand{\Var}{\mathrm{Var}}
\newcommand{\Cov}{\mathrm{Cov}}
\newcommand{\Bias}{\mathrm{Bias}}

% Formatting commands:

\newcommand{\mt}[1]{\ensuremath{#1}}
\newcommand{\nm}[1]{\textrm{#1}}

\newcommand\bsc[2][\DefaultOpt]{%
  \def\DefaultOpt{#2}%
  \section[#1]{#2}%
}
\newcommand\ssc[2][\DefaultOpt]{%
  \def\DefaultOpt{#2}%
  \subsection[#1]{#2}%
}
\newcommand{\bgpf}{\begin{proof} $ $\newline}

\newcommand{\bgeq}{\begin{equation*}}
\newcommand{\eeq}{\end{equation*}}	

\newcommand{\balist}{\begin{enumerate}[label=\alph*.]}
\newcommand{\elist}{\end{enumerate}}

\newcommand{\bilist}{\begin{enumerate}[label=\roman*)]}	

\newcommand{\bgsp}{\begin{split}}
% \newcommand{\esp}{\end{split}} % doesn't work for some reason.

\newcommand\prs[1]{~~~\textbf{(#1)}}

\newcommand{\lt}[1]{\textbf{Let: } #1}
     							   %  if you're setting it to be true
\newcommand{\supp}[1]{\textbf{Suppose: } #1}
     							   %  Suppose (if it'll end up false)
\newcommand{\wts}[1]{\textbf{Want to show: } #1}
     							   %  Want to show
\newcommand{\as}[1]{\textbf{Assume: } #1}
     							   %  if you think it follows from truth
\newcommand{\bpth}[1]{\textbf{(#1)}}

\newcommand{\step}[2]{\begin{equation}\tag{#2}#1\end{equation}}
\newcommand{\epf}{\end{proof}}

\newcommand{\dbs}[3]{\mt{#1_{#2_#3}}}

\newcommand{\sidenote}[1]{-----------------------------------------------------------------Side Note----------------------------------------------------------------
#1 \

---------------------------------------------------------------------------------------------------------------------------------------------}

% Analysis / Logical commands:

\newcommand{\br}{\mt{\mathbb{R}} }       % |R
\newcommand{\bq}{\mt{\mathbb{Q}} }       % |Q
\newcommand{\bn}{\mt{\mathbb{N}} }       % |N
\newcommand{\bc}{\mt{\mathbb{C}} }       % |C
\newcommand{\bz}{\mt{\mathbb{Z}} }       % |Z

\newcommand{\ep}{\mt{\epsilon} }         % epsilon
\newcommand{\fa}{\mt{\forall} }          % for all
\newcommand{\afa}{\mt{\alpha} }
\newcommand{\bta}{\mt{\beta} }
\newcommand{\mem}{\mt{\in} }
\newcommand{\exs}{\mt{\exists} }

\newcommand{\es}{\mt{\emptyset}}        % empty set
\newcommand{\sbs}{\mt{\subset} }         % subset of
\newcommand{\fs}[2]{\{\uw{#1}{1}, \uw{#1}{2}, ... \uw{#1}{#2}\}}

\newcommand{\lra}{ \mt{\longrightarrow} } % implies ----->
\newcommand{\rar}{ \mt{\Rightarrow} }     % implies -->

\newcommand{\lla}{ \mt{\longleftarrow} }  % implies <-----
\newcommand{\lar}{ \mt{\Leftarrow} }      % implies <--

\newcommand{\eql}{\mt{=} }
\newcommand{\pr}{\mt{^\prime} } 		   % prime (i.e. R')
\newcommand{\uw}[2]{#1\mt{_{#2}}}
\newcommand{\frc}[2]{\mt{\frac{#1}{#2}}}

\newcommand{\bnm}[2]{\mt{#1\setminus{#2}}}
\newcommand{\bnt}[2]{\mt{\textrm{#1}\setminus{\textrm{#2}}}}
\newcommand{\bi}{\bnm{\mathbb{R}}{\mathbb{Q}}}

\newcommand{\urng}[2]{\mt{\bigcup_{#1}^{#2}}}
\newcommand{\nrng}[2]{\mt{\bigcap_{#1}^{#2}}}

\newcommand{\nbho}[3]{\textrm{N(}#1, #2\textrm{) }\cap \textrm{ #3} \neq \emptyset}
     							   %  N(x, eps) intersect S \= emptyset
\newcommand{\nbhe}[3]{\textrm{N(}#1, #2\textrm{) }\cap \textrm{ #3} = \emptyset}
     							   %  N(x, eps) intersect S  = emptyset
\newcommand{\dnbho}[3]{\textrm{N*(}#1, #2\textrm{) }\cap \textrm{ #3} \neq \emptyset}
     							   %  N*(x, eps) intersect S \= emptyset
\newcommand{\dnbhe}[3]{\textrm{N*(}#1, #2\textrm{) }\cap \textrm{ #3} = \emptyset}
     							   %  N*(x, eps) intersect S = emptyset
     							 


% ----------

\begin{document}

Homework: page 148-149, \#1-4, 6, 8

\bsc{Heine-Borel Theorem}{

\es $\neq$ S \sbs \br is compact iff S is closed and bounded.

\bgpf
\lra

Done. \

\

\lla

\supp{S is closed and bounded.} \

\lt{S \sbs \urng{\afa \mem I}{} \uw{G}{\afa} where \uw{G}{\afa} is open \fa \afa \mem I}

Since is is bounded, sup S, inf S \mem \br both exist.

Define, for x \mem \br, \

\uw{S}{x} \eql S $\cap$ ($-\infty$, x].

S \sbs \urng{x \mem S}{} N(x, \ep)

\bta \eql \{ x \mem \br : \uw{S}{x} has a finite subcover from the \uw{G}{\afa}'s\}

\bta $\neq$ \es, inf S \mem \bta

\uw{S}{inf S} \eql S $\cap$ ($-\infty$, inf S]

We need to prove that S has a finite subcover of the \uw{G}{\afa}'s.

If \bta is unbounded above, then \exs z \mem \bta st z $>$ sup S.

Then \uw{S}{z} \eql S $\cap$ ($-\infty$, z] \eql S

Since \uw{S}{z} \eql S has a finite subcover of the \uw{G}{\afa}'s, we see that, in this case, S is compact.

We prove that \bta is unbounded above using contradiction. \

\

\supp{\bta is bounded above.} \

Thus, sup \bta \mem \br exists. \

\

Case i): sup \bta \mem S. \

\

In this case, \exs \ep \mem I st sup \bta \mem \dbs{G}{\afa}{0}

Since \dbs{G}{\afa}{0} is open, \exs \uw{\ep}{0} $>$ 0 st

N(sup \bta, \uw{\ep}{0}) \eql (sup \bta $-$ \uw{\ep}{0}, sup \bta $+$ \uw{\ep}{0}) \sbs \dbs{G}{\afa}{0}

By the definition of the supremum,

\exs \uw{x}{0} \mem \bta st \

sup \bta $-$ \uw{\ep}{0} $<$ \uw{y}{0} $\leq$ sup B $<$ sup B $+$ \frc{\uw{\ep}{0}}{2} $<$ sup \bta + \uw{\ep}{0}

Since \uw{x}{0} \mem \uw{\bta}{1}, \exs k \mem \bn and \{\uw{\afa}{1}, \uw{\afa}{2}, ... \uw{\afa}{n}\} \sbs I

st \dbs{S}{x}{0} \sbs \urng{i=1}{k} \dbs{G}{\afa}{i}

\sidenote{
\dbs{S}{x}{0} \eql S $\cap$ ($-\infty$, \uw{x}{0}]

\uw{S}{sup \bta} + \frc{\uw{\ep}{0}}{2}

\eql S $\cap$ ($-\infty$, sup \bta $+$ \frc{\uw{\ep}{0}}{2}]
}
This produces the contradiction that (sup \bta $+$ \frc{\uw{\ep}{0}}{2}) \mem \bta \

\newpage

Case ii): \

\

sup \bta \mem \bnt{\br}{S}, which is open since S is closed.

Thus, \exs \uw{\ep}{1} $>$ 0 st N(sup \bta, \uw{\ep}{1}) \sbs \bnt{\br}{S}

\sidenote{
----(-----|---|----)----

sup - ep1, sup B, sup B $+$ ep1/2, supB $+$ ep1
}

As in case i), \exs \uw{x}{1} \mem \bta st

sup \bta $-$ \uw{\ep}{1} $<$ \uw{x}{1} $\leq$ sup \bta $<$ sup \bta $+$ \frc{\uw{\ep}{1}}{2} $<$ sup \bta $+$ \uw{\ep}{1}

From \bpth{1}, N(sup \bta, \uw{\ep}{1}) \eql (sup \bta $-$ \uw{\ep}{1}, sup \bta $+$ \uw{\ep}{1} $\cap$ S \eql \es

----(------]---|----]----)---
supB-ep0, x0inB, supB, supBplusEpOver2, supBplusEp0

Notice that:

\dbs{S}{x}{1} \eql S $\cap$ ($-\infty$, \uw{x}{1}] \eql S $\cap$ ($-\infty$, sup \bta $+$ \frc{\uw{\ep}{1}}{2}] 

% \eql \uw{S}{sup $+$ \frc{\uw{\ep}{1}}{2}}

Again we obtain the contradiction that (sup \bta $+$ \frc{\uw{\ep}{1}}{2}) \mem \bta

Hence, result by contradiction.

\epf

}

\bsc{Theorem 3.5.6: Bolzano - Weierstrass Theorem}{

If a bounded set S \sbs \br contains an infinite number of points, then there exists at least one point in \br that is an accumulation point of S.

\bgpf

\supp{\exs S \sbs \br where S has an infinite number of points and S is bounded but S\pr \eql \es}

Since cl S \eql S $\cup$ S\pr \eql S $\cup$ \es \eql S, we can see by Theorem 3.4.17 a) that S is closed. \ 

Since S is also bounded, it follows by the Heire-Borel theorem that S is compact.

\lt{x \mem S}

Then x $\not\in$ S\pr, so \exs \uw{\ep}{x} $>$ 0 st 

N(x, \uw{\ep}{x}) $\cap$ S \eql \{x\}

\sidenote{
-----(----|--|---)-----

x-ep(x?), x, yMemS, xplusep(x?)

If x \mem S\pr, then:

$\neg$[\fa \ep $>$ 0, $\dnbho{x}{\ep}{S}$]

\exs \ep $>$ 0 st N(x, \ep) $\cap$ S \eql \{x\}
}

Then: 

S \sbs \urng{x \mem S}{} N(x, \uw{\ep}{x})

Since S is compact,

\exs k \mem \bn and \{\uw{x}{1}, \uw{x}{2}, ... \uw{x}{k}\} \sbs S

S \sbs \urng{i=1}{k} N(\dbs{x}{i}{1}, \dbs{\ep}{i}{1})

However, S $\cap$ ( \urng{i=1}{k} N(\dbs{x}{i}{1}, \dbs{\ep}{i}{1}) ) \eql \{\uw{x}{1}, \uw{x}{2}, ... \uw{x}{k}\}

This produces the contradiction that S contains a \textbf{finite} number of points.

Hence, S\pr $\neq$ \es
\epf
}

\bsc{Theorem 3.5.7 (F.I.P.)}{

\lt{\uw{\{\uw{K}{\afa}\}}{\afa \mem I} be a family of compact sets, where I is an index.}

Suppose that the intersection of any finite subfamily of the \uw{K}{\afa}'s has a nonempty intersection.

Then \nrng{\afa \mem I}{} \uw{K}{\afa} $\neq$ \es

\bgpf

Assume that \nrng{\afa \mem I}{} \uw{K}{\afa} \eql \es

Then \bnt{\br}{(\nrng{\afa \mem I}{} \uw{K}{\afa})} \eql \urng{\afa \mem I}{} (\bnt{\br}{\uw{K}{\afa}}) \eql \br

Notice, by the Heine-Borel Theorem that \bnt{\br}{\uw{K}{\afa}} is open \fa \afa \mem I.

\lt{\uw{\afa}{0} \mem I}

Since \dbs{K}{\afa}{0} is compact,

\exs k \mem \bn and \{\uw{\afa}{1}, \uw{\afa}{2}, ... \uw{\afa}{n}\} \sbs I st.

\dbs{K}{\afa}{0} \sbs \urng{\afa \mem I}{} (\bnt{\br}{\uw{K}{\afa}}) \sbs \urng{i = 1}{k} (\bnt{\br}{\dbs{K}{\afa}{0}})

\sidenote{
If A \sbs B, then \bnt{\br}{B} \sbs \bnt{\br}{A}

Let x \mem \bnt{\br}{B}.

Then x $\not\in$ B.

So, x $\not\in$ A.

Thus, x \mem \bnt{\br}{A}
}
\bnt{\br}{
	(\urng{i=1}{k}(
		\bnt{\br}{\uw{K}{\afa}})
	)
} \sbs \bnt{\br}{\dbs{K}{\afa}{0}}

So,

\nrng{i = 1}{k} \dbs{K}{\afa}{i} \sbs \bnt{\br}{\dbs{K}{\afa}{0}}

We obtain the contradiction that:

\nrng{i = 0}{k} \dbs{K}{\afa}{i} \eql \es

Hence, result.
\epf
}

\bsc{Corollary 3.5.8 Nested Intervals Theorem}{

\lt{\{\uw{A}{n}\}$^\infty_{n = 1}$ be a family of nonempty closed bounded intervals in \br st A$_{n + 1}$ \sbs A$_n$ \fa n \mem \bn}

Then:

\nrng{n = 1}{\infty} \uw{A}{n} $\neq$ \es

\bgpf

We use Theorem 3.5.7.

Will this be contradiction?

\supp{\fa k \mem \bn, that \{\uw{n}{1}, \uw{n}{2}, ... \uw{n}{k}\} \sbs \bn}

Then,

\nrng{i = 1}{k} \uw{A}{ni} \eql \uw{A}{m} $\neq$ \es

where

m \eql max \{\uw{n}{1}, \uw{n}{2}, ... \uw{n}{k}\}

\sidenote{
----[-----[---[---]---]---]----
not imp, not imp, not imp, A3, A2, A1
}

\epf

}

\end{document}