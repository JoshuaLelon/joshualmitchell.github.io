% Thank you Josh Davis for this template!
% https://github.com/jdavis/latex-homework-template/blob/master/homework.tex

\documentclass{article}

\newcommand{\hmwkTitle}{Lec\ \#13}

% % ----------

% Packages

\usepackage{fancyhdr}
\usepackage{extramarks}
\usepackage{amsmath}
\usepackage{amssymb}
\usepackage{amsthm}
\usepackage{amsfonts}
\usepackage{tikz}
\usepackage[plain]{algorithm}
\usepackage{algpseudocode}
\usepackage{enumitem}
\usepackage{chngcntr}

% Libraries

\usetikzlibrary{automata, positioning, arrows}

%
% Basic Document Settings
%

\topmargin=-0.45in
\evensidemargin=0in
\oddsidemargin=0in
\textwidth=6.5in
\textheight=9.0in
\headsep=0.25in

\linespread{1.1}

\pagestyle{fancy}
\lhead{\hmwkAuthorName}
\chead{}
\rhead{\hmwkClass\ (\hmwkClassInstructor): \hmwkTitle}
\lfoot{\lastxmark}
\cfoot{\thepage}

\renewcommand\headrulewidth{0.4pt}
\renewcommand\footrulewidth{0.4pt}

\setlength\parindent{0pt}
\setcounter{secnumdepth}{0}

\newcommand{\hmwkClass}{MATH 3380 / Analysis 1}        % Class
\newcommand{\hmwkClassInstructor}{Dr. Welsh}           % Instructor
\newcommand{\hmwkAuthorName}{\textbf{Joshua Mitchell}} % Author

%
% Title Page
%

\title{
    \vspace{2in}
    \textmd{\textbf{\hmwkClass:\ \hmwkTitle}}\\
    \normalsize\vspace{0.1in}\small\vspace{0.1in}\large{\textit{\hmwkClassInstructor}}
    \vspace{3in}
}

\author{\hmwkAuthorName}
\date{}

\renewcommand{\part}[1]{\textbf{\large Part \Alph{partCounter}}\stepcounter{partCounter}\\}

% Integral dx
\newcommand{\dx}{\mathrm{d}x}

%
% Various Helper Commands
%

% For derivatives
\newcommand{\deriv}[1]{\frac{\mathrm{d}}{\mathrm{d}x} (#1)}

% For partial derivatives
\newcommand{\pderiv}[2]{\frac{\partial}{\partial #1} (#2)}


% Alias for the Solution section header
\newcommand{\solution}{\textbf{\large Solution}}

% Probability commands: Expectation, Variance, Covariance, Bias
\newcommand{\E}{\mathrm{E}}
\newcommand{\Var}{\mathrm{Var}}
\newcommand{\Cov}{\mathrm{Cov}}
\newcommand{\Bias}{\mathrm{Bias}}

% Formatting commands:

\newcommand{\mt}[1]{\ensuremath{#1}}
\newcommand{\nm}[1]{\textrm{#1}}

\newcommand\bsc[2][\DefaultOpt]{%
  \def\DefaultOpt{#2}%
  \section[#1]{#2}%
}
\newcommand\ssc[2][\DefaultOpt]{%
  \def\DefaultOpt{#2}%
  \subsection[#1]{#2}%
}
\newcommand{\bgpf}{\begin{proof} $ $\newline}

\newcommand{\bgeq}{\begin{equation*}}
\newcommand{\eeq}{\end{equation*}}	

\newcommand{\balist}{\begin{enumerate}[label=\alph*.]}
\newcommand{\elist}{\end{enumerate}}

\newcommand{\bilist}{\begin{enumerate}[label=\roman*)]}	

\newcommand{\bgsp}{\begin{split}}
% \newcommand{\esp}{\end{split}} % doesn't work for some reason.

\newcommand\prs[1]{~~~\textbf{(#1)}}

\newcommand{\lt}[1]{\textbf{Let: } #1}
     							   %  if you're setting it to be true
\newcommand{\supp}[1]{\textbf{Suppose: } #1}
     							   %  Suppose (if it'll end up false)
\newcommand{\wts}[1]{\textbf{Want to show: } #1}
     							   %  Want to show
\newcommand{\as}[1]{\textbf{Assume: } #1}
     							   %  if you think it follows from truth
\newcommand{\bpth}[1]{\textbf{(#1)}}

\newcommand{\step}[2]{\begin{equation}\tag{#2}#1\end{equation}}
\newcommand{\epf}{\end{proof}}

\newcommand{\dbs}[3]{\mt{#1_{#2_#3}}}

\newcommand{\sidenote}[1]{-----------------------------------------------------------------Side Note----------------------------------------------------------------
#1 \

---------------------------------------------------------------------------------------------------------------------------------------------}

% Analysis / Logical commands:

\newcommand{\br}{\mt{\mathbb{R}} }       % |R
\newcommand{\bq}{\mt{\mathbb{Q}} }       % |Q
\newcommand{\bn}{\mt{\mathbb{N}} }       % |N
\newcommand{\bc}{\mt{\mathbb{C}} }       % |C
\newcommand{\bz}{\mt{\mathbb{Z}} }       % |Z

\newcommand{\ep}{\mt{\epsilon} }         % epsilon
\newcommand{\fa}{\mt{\forall} }          % for all
\newcommand{\afa}{\mt{\alpha} }
\newcommand{\bta}{\mt{\beta} }
\newcommand{\mem}{\mt{\in} }
\newcommand{\exs}{\mt{\exists} }

\newcommand{\es}{\mt{\emptyset}}        % empty set
\newcommand{\sbs}{\mt{\subset} }         % subset of
\newcommand{\fs}[2]{\{\uw{#1}{1}, \uw{#1}{2}, ... \uw{#1}{#2}\}}

\newcommand{\lra}{ \mt{\longrightarrow} } % implies ----->
\newcommand{\rar}{ \mt{\Rightarrow} }     % implies -->

\newcommand{\lla}{ \mt{\longleftarrow} }  % implies <-----
\newcommand{\lar}{ \mt{\Leftarrow} }      % implies <--

\newcommand{\eql}{\mt{=} }
\newcommand{\pr}{\mt{^\prime} } 		   % prime (i.e. R')
\newcommand{\uw}[2]{#1\mt{_{#2}}}
\newcommand{\frc}[2]{\mt{\frac{#1}{#2}}}

\newcommand{\bnm}[2]{\mt{#1\setminus{#2}}}
\newcommand{\bnt}[2]{\mt{\textrm{#1}\setminus{\textrm{#2}}}}
\newcommand{\bi}{\bnm{\mathbb{R}}{\mathbb{Q}}}

\newcommand{\urng}[2]{\mt{\bigcup_{#1}^{#2}}}
\newcommand{\nrng}[2]{\mt{\bigcap_{#1}^{#2}}}

\newcommand{\nbho}[3]{\textrm{N(}#1, #2\textrm{) }\cap \textrm{ #3} \neq \emptyset}
     							   %  N(x, eps) intersect S \= emptyset
\newcommand{\nbhe}[3]{\textrm{N(}#1, #2\textrm{) }\cap \textrm{ #3} = \emptyset}
     							   %  N(x, eps) intersect S  = emptyset
\newcommand{\dnbho}[3]{\textrm{N*(}#1, #2\textrm{) }\cap \textrm{ #3} \neq \emptyset}
     							   %  N*(x, eps) intersect S \= emptyset
\newcommand{\dnbhe}[3]{\textrm{N*(}#1, #2\textrm{) }\cap \textrm{ #3} = \emptyset}
     							   %  N*(x, eps) intersect S = emptyset
     							 


% ----------

% ----------

% Packages

\usepackage{fancyhdr}
\usepackage{extramarks}
\usepackage{amsmath}
\usepackage{amssymb}
\usepackage{amsthm}
\usepackage{amsfonts}
\usepackage{tikz}
\usepackage[plain]{algorithm}
\usepackage{algpseudocode}
\usepackage{enumitem}
\usepackage{chngcntr}

% Libraries

\usetikzlibrary{automata, positioning, arrows}

%
% Basic Document Settings
%

\topmargin=-0.45in
\evensidemargin=0in
\oddsidemargin=0in
\textwidth=6.5in
\textheight=9.0in
\headsep=0.25in

\linespread{1.1}

\pagestyle{fancy}
\lhead{\hmwkAuthorName}
\chead{}
\rhead{\hmwkClass\ (\hmwkClassInstructor): \hmwkTitle}
\lfoot{\lastxmark}
\cfoot{\thepage}

\renewcommand\headrulewidth{0.4pt}
\renewcommand\footrulewidth{0.4pt}

\setlength\parindent{0pt}
\setcounter{secnumdepth}{0}

\newcommand{\hmwkClass}{MATH 3380 / Analysis 1}        % Class
\newcommand{\hmwkClassInstructor}{Dr. Welsh}           % Instructor
\newcommand{\hmwkAuthorName}{\textbf{Joshua Mitchell}} % Author

%
% Title Page
%

\title{
    \vspace{2in}
    \textmd{\textbf{\hmwkClass:\ \hmwkTitle}}\\
    \normalsize\vspace{0.1in}\small\vspace{0.1in}\large{\textit{\hmwkClassInstructor}}
    \vspace{3in}
}

\author{\hmwkAuthorName}
\date{}

\renewcommand{\part}[1]{\textbf{\large Part \Alph{partCounter}}\stepcounter{partCounter}\\}

% Integral dx
\newcommand{\dx}{\mathrm{d}x}

%
% Various Helper Commands
%

% For derivatives
\newcommand{\deriv}[1]{\frac{\mathrm{d}}{\mathrm{d}x} (#1)}

% For partial derivatives
\newcommand{\pderiv}[2]{\frac{\partial}{\partial #1} (#2)}


% Alias for the Solution section header
\newcommand{\solution}{\textbf{\large Solution}}

% Probability commands: Expectation, Variance, Covariance, Bias
\newcommand{\E}{\mathrm{E}}
\newcommand{\Var}{\mathrm{Var}}
\newcommand{\Cov}{\mathrm{Cov}}
\newcommand{\Bias}{\mathrm{Bias}}

% Formatting commands:

\newcommand{\mt}[1]{\ensuremath{#1}}
\newcommand{\nm}[1]{\textrm{#1}}

\newcommand\bsc[2][\DefaultOpt]{%
  \def\DefaultOpt{#2}%
  \section[#1]{#2}%
}
\newcommand\ssc[2][\DefaultOpt]{%
  \def\DefaultOpt{#2}%
  \subsection[#1]{#2}%
}
\newcommand{\bgpf}{\begin{proof} $ $\newline}

\newcommand{\bgeq}{\begin{equation*}}
\newcommand{\eeq}{\end{equation*}}	

\newcommand{\balist}{\begin{enumerate}[label=\alph*.]}
\newcommand{\elist}{\end{enumerate}}

\newcommand{\bilist}{\begin{enumerate}[label=\roman*)]}	

\newcommand{\bgsp}{\begin{split}}
% \newcommand{\esp}{\end{split}} % doesn't work for some reason.

\newcommand\prs[1]{~~~\textbf{(#1)}}

\newcommand{\lt}[1]{\textbf{Let: } #1}
     							   %  if you're setting it to be true
\newcommand{\supp}[1]{\textbf{Suppose: } #1}
     							   %  Suppose (if it'll end up false)
\newcommand{\wts}[1]{\textbf{Want to show: } #1}
     							   %  Want to show
\newcommand{\as}[1]{\textbf{Assume: } #1}
     							   %  if you think it follows from truth
\newcommand{\bpth}[1]{\textbf{(#1)}}

\newcommand{\step}[2]{\begin{equation}\tag{#2}#1\end{equation}}
\newcommand{\epf}{\end{proof}}

\newcommand{\dbs}[3]{\mt{#1_{#2_#3}}}

\newcommand{\sidenote}[1]{-----------------------------------------------------------------Side Note----------------------------------------------------------------
#1 \

---------------------------------------------------------------------------------------------------------------------------------------------}

% Analysis / Logical commands:

\newcommand{\br}{\mt{\mathbb{R}} }       % |R
\newcommand{\bq}{\mt{\mathbb{Q}} }       % |Q
\newcommand{\bn}{\mt{\mathbb{N}} }       % |N
\newcommand{\bc}{\mt{\mathbb{C}} }       % |C
\newcommand{\bz}{\mt{\mathbb{Z}} }       % |Z

\newcommand{\ep}{\mt{\epsilon} }         % epsilon
\newcommand{\fa}{\mt{\forall} }          % for all
\newcommand{\afa}{\mt{\alpha} }
\newcommand{\bta}{\mt{\beta} }
\newcommand{\mem}{\mt{\in} }
\newcommand{\exs}{\mt{\exists} }

\newcommand{\es}{\mt{\emptyset} }        % empty set
\newcommand{\sbs}{\mt{\subset} }         % subset of
\newcommand{\fs}[2]{\{\uw{#1}{1}, \uw{#1}{2}, ... \uw{#1}{#2}\}}

\newcommand{\lra}{ \mt{\longrightarrow} } % implies ----->
\newcommand{\rar}{ \mt{\Rightarrow} }     % implies -->

\newcommand{\lla}{ \mt{\longleftarrow} }  % implies <-----
\newcommand{\lar}{ \mt{\Leftarrow} }      % implies <--

\newcommand{\av}[1]{\mt{|}#1\mt{|}}  % absolute value

\newcommand{\prn}[1]{(#1)}
\newcommand{\bk}[1]{\{#1\}}

\newcommand{\ps}{\mt{+} }
\newcommand{\ms}{\mt{-} }

\newcommand{\ls}{\mt{<} }
\newcommand{\gr}{\mt{>} }

\newcommand{\lse}{\mt{\leq} }
\newcommand{\gre}{\mt{\geq} }

\newcommand{\eql}{\mt{=} }

\newcommand{\pr}{\mt{^\prime} } 		   % prime (i.e. R')
\newcommand{\uw}[2]{#1\mt{_{#2}}}
\newcommand{\uf}[2]{#1\mt{^{#2}}}
\newcommand{\frc}[2]{\mt{\frac{#1}{#2}}}
\newcommand{\lmti}[1]{\mt{\displaystyle{\lim_{#1 \to \infty}}}}
\newcommand{\limt}[2]{\mt{\displaystyle{\lim_{#1 \to #2}}}}

\newcommand{\bnm}[2]{\mt{#1\setminus{#2}}}
\newcommand{\bnt}[2]{\mt{\textrm{#1}\setminus{\textrm{#2}}}}
\newcommand{\bi}{\bnm{\mathbb{R}}{\mathbb{Q}}}

\newcommand{\urng}[2]{\mt{\bigcup_{#1}^{#2}}}
\newcommand{\nrng}[2]{\mt{\bigcap_{#1}^{#2}}}
\newcommand{\nck}[2]{\mt{{#1 \choose #2}}}

\newcommand{\nbho}[3]{\textrm{N(}#1, #2\textrm{) }\cap \textrm{ #3} \neq \emptyset}
     							   %  N(x, eps) intersect S \= emptyset
\newcommand{\nbhe}[3]{\textrm{N(}#1, #2\textrm{) }\cap \textrm{ #3} = \emptyset}
     							   %  N(x, eps) intersect S  = emptyset
\newcommand{\dnbho}[3]{\textrm{N*(}#1, #2\textrm{) }\cap \textrm{ #3} \neq \emptyset}
     							   %  N*(x, eps) intersect S \= emptyset
\newcommand{\dnbhe}[3]{\textrm{N*(}#1, #2\textrm{) }\cap \textrm{ #3} = \emptyset}
     							   %  N*(x, eps) intersect S = emptyset
     							   
\newcommand{\eqn}[1]{\[#1\]}
\newcommand{\splt}[1]{\begin{split}#1\end{split}}
     							 
% ----------

\begin{document}

Homework 7: pages 184 - 185 numbers 1, 21(a)(b), 3(e), 4, 10, 13, 14 \lla 14 is difficult, but not impossible! (want to show that lim (1 \ps \frc{1}{n})$^n$ exists)

Hint:

\uf{\prn{1 \ps b}}{n} \eql 1 \ps nb \ps \frc{n(n - 1}{2!}\uf{b}{n} \ps ... \ps \frc{n(n - 1)... (n - (r - 1))}{r!}\uf{b}{r} \ps ... \ps \uf{b}{n}

In our problem, b \eql \frc{1}{n}

Look at it as 1 \ps $\sum_{r = 1}^n$ \frc{n(n - 1)...(n - (r - 1))}{r!}\frc{1}{n^r}

\prn{1 \ps \frc{1}{n}}$^n$ goes in there somewhere somehow.

About the last homework (HW 6), problem 9:

If \uw{s}{n} \lse \uw{t}{n} \fa n \mem \bn and \lmti{n} \uw{s}{n} \eql $\infty$,

then \lmti{n} \uw{t}{n} \eql $\infty$

So, \fa M \mem \br, \exs N \mem \bn st

\uw{s}{n} \gr M, \fa n \gre N

Notice that:

\uw{t}{n} \gre \uw{s}{n} \gr M, \fa n \gre N

So by definition, \lmti{n} \uw{t}{n} \eql $\infty$

\bsc{Section 4.3: Monotone Sequences and Cauchy Sequences}{

\ssc{Definition 4.3.1}{

A sequence \prn{\uw{s}{n}} is \textbf{increasing} (or \textbf{decreasing}) if \uw{s}{n} \lse \uw{s}{n + 1} (or \uw{s}{n + 1} \lse \uw{s}{n}) \fa n \mem \bn. A sequence is \textbf{monotonic} if it is increasing or decreasing.
}

\ssc{Example 4.3.2}{
\balist
\item \uw{a}{n} \eql n, \fa n \mem \bn
	
	increasing
\item \uw{b}{n} \eql \uf{2}{n}, \fa n \mem \bn
	
	increasing
\item \uw{c}{n} \eql 2 \ms \frc{1}{n}, \fa n \mem \bn
	
	increasing
\item \prn{\uw{d}{n}} \eql 1, 1, 2, 2, 3, 3...
	
	increasing
\item \uw{e}{n} \eql \frc{2}{n}, \fa ...
	
	decreasing
\item \uw{f}{n} \eql \ms3n
	
	decreasing
\item \prn{\uw{g}{n}} \eql 1, 1, 1, ... (\uw{g}{n} \eql 1, \fa n \mem \bn)
	
	increasing and decreasing
\item \uw{h}{n} \eql \uf{-1}{n}, \fa n \mem \bn
	
	not monotonic
\item \uw{i}{n} \eql cos(\frc{n\pi}{3}) \fa n \mem \bn
	
	not monotonic
\elist
}

\ssc{Theorem 4.3.3 (Monotone Convergence Theorem)}{

A \textbf{monotonic sequence} is convergent iff it is bounded.

\bgpf
\lt{\bk{\uw{s}{n}} be a monotonically increasing sequence}

\lra 

Assume \bk{\uw{s}{n}} is convergent.

By Theorem 4.1.13, \bk{\uw{s}{n}} is bounded.

\lla

Conversely, assume \bk{\uw{s}{n}} is bounded.

\wts{\bk{\uw{s}{n}} converges}

Let the range of \bk{\uw{s}{n}} be denoted by S \eql \bk{\uw{s}{n} : n \mem \bn}

Since \bk{\uw{s}{n}} is bounded, S is bounded above.

Thus, sup S exists.

\wts{\bk{\uw{s}{n}} converges to sup S}

Recall: The supremum is the least upper bound.

Thus,

\uw{s}{n} \lse sup S, \fa n \mem \bn \bpth{1}

and for \ep \gr 0, \exs N( \ep) \mem \bn st \fa n \gre N,

sup S \ms \ep \ls \uw{s}{n}

sup S \ms \ep \ls \uw{s}{N} \lse \uw{s}{n} \lse sup S \ls sup S \ps \ep \bpth{2}

Since \bk{\uw{s}{n}} is increasing and, using \bpth{1}, 

From \bpth{2}, we see that

\ms\ep \ls \uw{s}{n} \ms sup S \ls \ep, \fa n \gre N

Hence, 

\av{\uw{s}{n} \ms sup S} \ls \ep, \fa n \gre N,

which is equivalent to \lmti{n} \uw{s}{n} \eql sup S

(since \av{x} \ls a iff \ms a \ls x \ls a)

\epf

}


\textbf{The difficult homework problem is going to come from here.}

Additional help:

\uw{s}{n} \eql (1 \ps \frc{1}{n})$^n$

First thing, show that it's increasing:

a \ls \uw{s}{n} \lse \uw{s}{n + 1}

\prn{1 \ps \frc{1}{n}}$^n$ \lse \prn{1 \ps \frc{1}{n + 1}}$^{n + 1}$

Second thing, show this:

(mini hint: \av{\uw{s}{n} \ms s} \ls \ep \fa n \gre W (written on board, maybe he means M?)

\uw{s}{n} \ls 3 \fa n \mem \bn 

Turns up naturally:

\lmti{n} \prn{1 \ps \frc{1}{n}}$^n$ \eql e

\newpage

\ssc{Example 4.3.4}{

\lt{\uw{s}{1} \eql 1, \uw{s}{n + 1} \eql $\sqrt{1 + s_n}$ \fa n \mem \bn with n \gre 2}

Prove that \bk{\uw{s}{n}} converges and find its limit.

\uw{s}{1} \eql 1, \uw{s}{2} \eql $\sqrt{2}$, \uw{s}{3} \eql $\sqrt{1 \ps \sqrt{2}}$, \uw{s}{4} \eql $\sqrt{1 + \sqrt{1 \ps \sqrt{2}}}$ ...

}

\ssc{Conjecture}{

\bk{\uw{s}{n}} is increasing and 1 \lse \uw{s}{n} \lse 2, \fa n \mem \bn

Proposition as a function of n [P(n)]:

\uw{s}{n} \lse \uw{s}{n + 1}, \fa n \mem \bn 

\uw{s}{1} \eql 1 \ls $\sqrt{2}$ \eql \uw{s}{2}

Suppose that, \fa k \mem \bn,

\eqn{\sqrt{1 + s_k} \lse \sqrt{1 + s_{k + 1}}}

Now,

\eqn{s_{k + 1} = \sqrt{1	 + s_k} \lse \sqrt{1 + s_{k + 1}} = s_{k + 2}}

So,

\eqn{s_k \lse s_{k + 1}}

Hence, by induction, P(n): \uw{s}{n} \lse \uw{s}{n + 1} is true \fa n \mem \bn 

Q(n): \uw{s}{n} \lse 2 \fa n \mem \bn 

\uw{s}{1} \eql 1 \ls 2

Assume for k \mem \bn that \uw{s}{k} \ls 2

Consider:

\eqn{s_{k + 1} = \sqrt{1 + s_k} < \sqrt{1 + 3} = \sqrt{2 + 2} = 2}

Hence, by induction, Q(n): \uw{s}{n} \ls 2 is true \fa n \mem \bn 

}

By the Montone Convergence Theorem,

\exs s \mem \br st

\lmti{n} \uw{s}{n} \eql s

By HW problem 11, page 170.

Thus,

\lmti{n} \uw{s}{n + 1} \eql \lmti{n} \uw{s}{n} \eql s

\newpage

\sidenote{
\bk{\uw{s}{n}} \lra s

\bk{\uw{t}{n}}: \uw{t}{n} \eql \uw{s}{n + k}, k \mem \bn 
}

So, we claim that \lmti{n} \uw{s}{n + 1} \eql s \eql \lmti{n} $\sqrt{1 + s_n}$ \eql $\sqrt{1 + s}$

From Example 4.2.6,

\lmti{n} $\sqrt{\uw{t}{n}}$ \eql $\sqrt{t}$ if \lmti{n} \uw{t}{n} \eql t

Also, by Theorem 4.2.1 (b), \lmti{n} $\sqrt{1 + s_n}$ \eql $\sqrt{1 + s}$

(which is like saying \lmti{n} \uw{t}{n} \eql t)

Hence,

\eqn{
	\splt{
		s & = \sqrt{1 + s} \\
		s^2 & = 1 + s \\
		s^2 - s - 1 & = 0 \\
		s & = \frac{1 (+/-) \sqrt{1 - (-4)}}{2} \\
		& = \frac{1 (+/-) \sqrt{5}}{2} 
	}
}

But one of those limits can't be true since limits are unique.

Since \uw{s}{n} \gre 0, \fa n \mem \bn,

then \lmti{n} \uw{s}{n} \eql s \gre 0, \fa n \mem \bn 

(By Corollary 4.2.5)

Hence, 

s \eql $\frac{1 + \sqrt{5}}{2}$

\bk{\uw{s}{n}} is Cauchy if for \ep \gr 0, \exs N \mem \bn

st

\av{\uw{s}{n} - \uw{s}{m}} \ls \ep \fa m, n \gre N

}

\end{document}