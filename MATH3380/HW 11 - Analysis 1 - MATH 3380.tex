% Thank you Josh Davis for this template!
% https://github.com/jdavis/latex-homework-template/blob/master/homework.tex

\documentclass{article}

\newcommand{\hmwkTitle}{HW\ \#11}

% % ----------

% Packages

\usepackage{fancyhdr}
\usepackage{extramarks}
\usepackage{amsmath}
\usepackage{amssymb}
\usepackage{amsthm}
\usepackage{amsfonts}
\usepackage{tikz}
\usepackage[plain]{algorithm}
\usepackage{algpseudocode}
\usepackage{enumitem}
\usepackage{chngcntr}

% Libraries

\usetikzlibrary{automata, positioning, arrows}

%
% Basic Document Settings
%

\topmargin=-0.45in
\evensidemargin=0in
\oddsidemargin=0in
\textwidth=6.5in
\textheight=9.0in
\headsep=0.25in

\linespread{1.1}

\pagestyle{fancy}
\lhead{\hmwkAuthorName}
\chead{}
\rhead{\hmwkClass\ (\hmwkClassInstructor): \hmwkTitle}
\lfoot{\lastxmark}
\cfoot{\thepage}

\renewcommand\headrulewidth{0.4pt}
\renewcommand\footrulewidth{0.4pt}

\setlength\parindent{0pt}
\setcounter{secnumdepth}{0}

\newcommand{\hmwkClass}{MATH 3380 / Analysis 1}        % Class
\newcommand{\hmwkClassInstructor}{Dr. Welsh}           % Instructor
\newcommand{\hmwkAuthorName}{\textbf{Joshua Mitchell}} % Author

%
% Title Page
%

\title{
    \vspace{2in}
    \textmd{\textbf{\hmwkClass:\ \hmwkTitle}}\\
    \normalsize\vspace{0.1in}\small\vspace{0.1in}\large{\textit{\hmwkClassInstructor}}
    \vspace{3in}
}

\author{\hmwkAuthorName}
\date{}

\renewcommand{\part}[1]{\textbf{\large Part \Alph{partCounter}}\stepcounter{partCounter}\\}

% Integral dx
\newcommand{\dx}{\mathrm{d}x}

%
% Various Helper Commands
%

% For derivatives
\newcommand{\deriv}[1]{\frac{\mathrm{d}}{\mathrm{d}x} (#1)}

% For partial derivatives
\newcommand{\pderiv}[2]{\frac{\partial}{\partial #1} (#2)}


% Alias for the Solution section header
\newcommand{\solution}{\textbf{\large Solution}}

% Probability commands: Expectation, Variance, Covariance, Bias
\newcommand{\E}{\mathrm{E}}
\newcommand{\Var}{\mathrm{Var}}
\newcommand{\Cov}{\mathrm{Cov}}
\newcommand{\Bias}{\mathrm{Bias}}

% Formatting commands:

\newcommand{\mt}[1]{\ensuremath{#1}}
\newcommand{\nm}[1]{\textrm{#1}}

\newcommand\bsc[2][\DefaultOpt]{%
  \def\DefaultOpt{#2}%
  \section[#1]{#2}%
}
\newcommand\ssc[2][\DefaultOpt]{%
  \def\DefaultOpt{#2}%
  \subsection[#1]{#2}%
}
\newcommand{\bgpf}{\begin{proof} $ $\newline}

\newcommand{\bgeq}{\begin{equation*}}
\newcommand{\eeq}{\end{equation*}}	

\newcommand{\balist}{\begin{enumerate}[label=\alph*.]}
\newcommand{\elist}{\end{enumerate}}

\newcommand{\bilist}{\begin{enumerate}[label=\roman*)]}	

\newcommand{\bgsp}{\begin{split}}
% \newcommand{\esp}{\end{split}} % doesn't work for some reason.

\newcommand\prs[1]{~~~\textbf{(#1)}}

\newcommand{\lt}[1]{\textbf{Let: } #1}
     							   %  if you're setting it to be true
\newcommand{\supp}[1]{\textbf{Suppose: } #1}
     							   %  Suppose (if it'll end up false)
\newcommand{\wts}[1]{\textbf{Want to show: } #1}
     							   %  Want to show
\newcommand{\as}[1]{\textbf{Assume: } #1}
     							   %  if you think it follows from truth
\newcommand{\bpth}[1]{\textbf{(#1)}}

\newcommand{\step}[2]{\begin{equation}\tag{#2}#1\end{equation}}
\newcommand{\epf}{\end{proof}}

\newcommand{\dbs}[3]{\mt{#1_{#2_#3}}}

\newcommand{\sidenote}[1]{-----------------------------------------------------------------Side Note----------------------------------------------------------------
#1 \

---------------------------------------------------------------------------------------------------------------------------------------------}

% Analysis / Logical commands:

\newcommand{\br}{\mt{\mathbb{R}} }       % |R
\newcommand{\bq}{\mt{\mathbb{Q}} }       % |Q
\newcommand{\bn}{\mt{\mathbb{N}} }       % |N
\newcommand{\bc}{\mt{\mathbb{C}} }       % |C
\newcommand{\bz}{\mt{\mathbb{Z}} }       % |Z

\newcommand{\ep}{\mt{\epsilon} }         % epsilon
\newcommand{\fa}{\mt{\forall} }          % for all
\newcommand{\afa}{\mt{\alpha} }
\newcommand{\bta}{\mt{\beta} }
\newcommand{\mem}{\mt{\in} }
\newcommand{\exs}{\mt{\exists} }

\newcommand{\es}{\mt{\emptyset}}        % empty set
\newcommand{\sbs}{\mt{\subset} }         % subset of
\newcommand{\fs}[2]{\{\uw{#1}{1}, \uw{#1}{2}, ... \uw{#1}{#2}\}}

\newcommand{\lra}{ \mt{\longrightarrow} } % implies ----->
\newcommand{\rar}{ \mt{\Rightarrow} }     % implies -->

\newcommand{\lla}{ \mt{\longleftarrow} }  % implies <-----
\newcommand{\lar}{ \mt{\Leftarrow} }      % implies <--

\newcommand{\eql}{\mt{=} }
\newcommand{\pr}{\mt{^\prime} } 		   % prime (i.e. R')
\newcommand{\uw}[2]{#1\mt{_{#2}}}
\newcommand{\frc}[2]{\mt{\frac{#1}{#2}}}

\newcommand{\bnm}[2]{\mt{#1\setminus{#2}}}
\newcommand{\bnt}[2]{\mt{\textrm{#1}\setminus{\textrm{#2}}}}
\newcommand{\bi}{\bnm{\mathbb{R}}{\mathbb{Q}}}

\newcommand{\urng}[2]{\mt{\bigcup_{#1}^{#2}}}
\newcommand{\nrng}[2]{\mt{\bigcap_{#1}^{#2}}}

\newcommand{\nbho}[3]{\textrm{N(}#1, #2\textrm{) }\cap \textrm{ #3} \neq \emptyset}
     							   %  N(x, eps) intersect S \= emptyset
\newcommand{\nbhe}[3]{\textrm{N(}#1, #2\textrm{) }\cap \textrm{ #3} = \emptyset}
     							   %  N(x, eps) intersect S  = emptyset
\newcommand{\dnbho}[3]{\textrm{N*(}#1, #2\textrm{) }\cap \textrm{ #3} \neq \emptyset}
     							   %  N*(x, eps) intersect S \= emptyset
\newcommand{\dnbhe}[3]{\textrm{N*(}#1, #2\textrm{) }\cap \textrm{ #3} = \emptyset}
     							   %  N*(x, eps) intersect S = emptyset
     							 


% ----------

% ----------

% Packages

\usepackage{fancyhdr}
\usepackage{extramarks}
\usepackage{amsmath}
\usepackage{amssymb}
\usepackage{amsthm}
\usepackage{amsfonts}
\usepackage{tikz}
\usepackage[plain]{algorithm}
\usepackage{algpseudocode}
\usepackage{enumitem}
\usepackage{chngcntr}

% Libraries

\graphicspath{{/Users/jm/iclouddrive/3380pics/}}

\usetikzlibrary{automata, positioning, arrows}

%
% Basic Document Settings
%

\topmargin=-0.45in
\evensidemargin=0in
\oddsidemargin=0in
\textwidth=6.5in
\textheight=9.0in
\headsep=0.25in

\linespread{1.1}

\pagestyle{fancy}
\lhead{\hmwkAuthorName}
\chead{}
\rhead{\hmwkClass\ (\hmwkClassInstructor): \hmwkTitle}
\lfoot{\lastxmark}
\cfoot{\thepage}

\renewcommand\headrulewidth{0.4pt}
\renewcommand\footrulewidth{0.4pt}

\setlength\parindent{0pt}
\setcounter{secnumdepth}{0}

\newcommand{\hmwkClass}{MATH 3380 / Analysis 1}        % Class
\newcommand{\hmwkClassInstructor}{Dr. Welsh}           % Instructor
\newcommand{\hmwkAuthorName}{\textbf{Joshua Mitchell}} % Author

%
% Title Page
%

\title{
    \vspace{2in}
    \textmd{\textbf{\hmwkClass:\ \hmwkTitle}}\\
    \normalsize\vspace{0.1in}\small\vspace{0.1in}\large{\textit{\hmwkClassInstructor}}
    \vspace{3in}
}

\author{\hmwkAuthorName}
\date{}

\renewcommand{\part}[1]{\textbf{\large Part \Alph{partCounter}}\stepcounter{partCounter}\\}

% Integral dx
\newcommand{\dx}{\mathrm{d}x}

%
% Various Helper Commands
%

% For derivatives
\newcommand{\deriv}[1]{\frac{\mathrm{d}}{\mathrm{d}x} (#1)}

% For partial derivatives
\newcommand{\pderiv}[2]{\frac{\partial}{\partial #1} (#2)}


% Alias for the Solution section header
\newcommand{\solution}{\textbf{\large Solution}}

% Probability commands: Expectation, Variance, Covariance, Bias
\newcommand{\E}{\mathrm{E}}
\newcommand{\Var}{\mathrm{Var}}
\newcommand{\Cov}{\mathrm{Cov}}
\newcommand{\Bias}{\mathrm{Bias}}

% Formatting commands:

\newcommand{\mt}[1]{\ensuremath{#1}}
\newcommand{\nm}[1]{\textrm{#1}}

\newcommand\bsc[2][\DefaultOpt]{%
  \def\DefaultOpt{#2}%
  \section[#1]{#2}%
}
\newcommand\ssc[2][\DefaultOpt]{%
  \def\DefaultOpt{#2}%
  \subsection[#1]{#2}%
}
\newcommand{\bgpf}{\begin{proof} $ $\newline}

\newcommand{\bgeq}{\begin{equation*}}
\newcommand{\eeq}{\end{equation*}}	

\newcommand{\balist}{\begin{enumerate}[label=\alph*.]}
\newcommand{\elist}{\end{enumerate}}

\newcommand{\bilist}{\begin{enumerate}[label=\roman*)]}	

\newcommand{\bgsp}{\begin{split}}
% \newcommand{\esp}{\end{split}} % doesn't work for some reason.

\newcommand\prs[1]{~~~\textbf{(#1)}}

\newcommand{\lt}[1]{\textbf{Let: } #1}
     							   %  if you're setting it to be true
\newcommand{\supp}[1]{\textbf{Suppose: } #1}
     							   %  Suppose (if it'll end up false)
\newcommand{\wts}[1]{\textbf{Want to show: } #1}
     							   %  Want to show
\newcommand{\as}[1]{\textbf{Assume: } #1}
     							   %  if you think it follows from truth
\newcommand{\bpth}[1]{\textbf{(#1)}}

\newcommand{\step}[2]{\begin{equation}\tag{#2}#1\end{equation}}
\newcommand{\epf}{\end{proof}}

\newcommand{\dbs}[3]{\mt{#1_{#2_#3}}}

\newcommand{\sidenote}[1]{-----------------------------------------------------------------Side Note----------------------------------------------------------------
#1 \

---------------------------------------------------------------------------------------------------------------------------------------------}

% Analysis / Logical commands:

\newcommand{\br}{\mt{\mathbb{R}} }       % |R
\newcommand{\bq}{\mt{\mathbb{Q}} }       % |Q
\newcommand{\bn}{\mt{\mathbb{N}} }       % |N
\newcommand{\bc}{\mt{\mathbb{C}} }       % |C
\newcommand{\bz}{\mt{\mathbb{Z}} }       % |Z

\newcommand{\ep}{\mt{\epsilon} }         % epsilon
\newcommand{\fa}{\mt{\forall} }          % for all
\newcommand{\afa}{\mt{\alpha} }
\newcommand{\bta}{\mt{\beta} }
\newcommand{\dta}{\mt{\delta} }
\newcommand{\mem}{\mt{\in} }
\newcommand{\exs}{\mt{\exists} }

\newcommand{\es}{\mt{\emptyset} }        % empty set
\newcommand{\sbs}{\mt{\subset} }         % subset of
\newcommand{\fs}[2]{\{\uw{#1}{1}, \uw{#1}{2}, ... \uw{#1}{#2}\}}

\newcommand{\lra}{ \mt{\longrightarrow} } % implies ----->
\newcommand{\rar}{ \mt{\Rightarrow} }     % implies -->

\newcommand{\lla}{ \mt{\longleftarrow} }  % implies <-----
\newcommand{\lar}{ \mt{\Leftarrow} }      % implies <--

\newcommand{\av}[1]{\mt{|}#1\mt{|}}  % absolute value

\newcommand{\prn}[1]{(#1)}
\newcommand{\bk}[1]{\{#1\}}

\newcommand{\ps}{\mt{+} }
\newcommand{\ms}{\mt{-} }

\newcommand{\ls}{\mt{<} }
\newcommand{\gr}{\mt{>} }

\newcommand{\lse}{\mt{\leq} }
\newcommand{\gre}{\mt{\geq} }

\newcommand{\eql}{\mt{=} }

\newcommand{\pr}{\mt{^\prime} } 		   % prime (i.e. R')
\newcommand{\uw}[2]{#1\mt{_{#2}}}
\newcommand{\uf}[2]{#1\mt{^{#2}}}
\newcommand{\frc}[2]{\mt{\frac{#1}{#2}}}
\newcommand{\lmti}[1]{\mt{\displaystyle{\lim_{#1 \to \infty}}}}
\newcommand{\limt}[2]{\mt{\displaystyle{\lim_{#1 \to #2}}}}

\newcommand{\bnm}[2]{\mt{#1\setminus{#2}}}
\newcommand{\bnt}[2]{\mt{\textrm{#1}\setminus{\textrm{#2}}}}
\newcommand{\bi}{\bnm{\mathbb{R}}{\mathbb{Q}}}

\newcommand{\urng}[2]{\mt{\bigcup_{#1}^{#2}}}
\newcommand{\nrng}[2]{\mt{\bigcap_{#1}^{#2}}}
\newcommand{\nck}[2]{\mt{{#1 \choose #2}}}

\newcommand{\nbho}[3]{\textrm{N(}#1, #2\textrm{) }\cap \textrm{ #3} \neq \emptyset}
     							   %  N(x, eps) intersect S \= emptyset
\newcommand{\nbhe}[3]{\textrm{N(}#1, #2\textrm{) }\cap \textrm{ #3} = \emptyset}
     							   %  N(x, eps) intersect S  = emptyset
\newcommand{\dnbho}[3]{\textrm{N*(}#1, #2\textrm{) }\cap \textrm{ #3} \neq \emptyset}
     							   %  N*(x, eps) intersect S \= emptyset
\newcommand{\dnbhe}[3]{\textrm{N*(}#1, #2\textrm{) }\cap \textrm{ #3} = \emptyset}
     							   %  N*(x, eps) intersect S = emptyset
     							   
\newcommand{\eqn}[1]{\[#1\]}
\newcommand{\splt}[1]{\begin{split}#1\end{split}}

\newcommand{\infy}{\mt{\infty} }
\newcommand{\unn}{\mt{\cup} }
\newcommand{\inn}{\mt{\cap} }
\newcommand\tab[1][1cm]{\hspace*{#1}}

\newcommand{\wit}[1]{\mt{\widetilde{#1}}}
     							 
% ----------

\begin{document}

HW 11: page 220 - 221, \#1, 2, 5 and page 226-227, \# 1 - 3, 4(a)(b), 5, 11

\bsc{Exercise 1 (pages 220 - 221)}{

Mark each statement True or False. Justify each answer.
\balist
\item Let D be a compact subset of \br and suppose that f : D \lra \br is continuous. Then f(D) is compact.
	
	\textbf{True, by Theorem 5.3.2.}
\item Suppose that f : D \lra R is continuous. Then, there exists a point \uw{x}{1} in D st f(\uw{x}{1}) \gre f(x) \fa x \mem D
	
	\textbf{False.}
	
	\lt{f(x) \eql x and D \eql \br}
	
	\supp{\exs \uw{x}{1} \mem D st f(\uw{x}{1}) \gre f(x) \fa x \mem D}
	
	Notice that (f(\uw{x}{1}) \ps 1) \mem \br, and if \uw{x}{2} \eql (f(\uw{x}{1}) \ps 1), then f(\uw{x}{2}) \eql (f(\uw{x}{1}) \ps 1) \gr f(\uw{x}{1}). A contradiction.
\item Let D be a bounded subset of \br and assume that f : D \lra \br is continuous. Then f(D) is bounded.

	\textbf{False.}
	
	\lt{f : (0, \infy) \lra \br be defined by f(x) \eql \frc{1}{x}}
	
	\supp{\exs f(\uw{x}{1}) st f(\uw{x}{1}) \gre f(x) \fa x \mem (0, \infy)}
	
	Notice that (f(\uw{x}{1}) \ps 1) \mem \br, and if \uw{x}{2} \eql \frc{1}{f(\uw{x}{1}) \ps 1}, then f(\uw{x}{2}) \eql (f(\uw{x}{1}) \ps 1) \gr f(\uw{x}{1}). A contradiction.
\elist
}

\bsc{Exercise 2 (pages 220 - 221)}{

Mark each statement True or False. Justify each answer.

\balist
\item Let f : [a,b] \lra \br be continuous and assume f(a) \ls 0 \ls f(b). Then there exists a point c \mem (a, b) st f(c) \eql 0.

	\textbf{True, by Theorem 5.3.6 (IVT).}

\item Let f : [a,b] \lra \br be continuous and assume f(a) \lse k \lse f(b). Then there exists a point c \mem [a,b] st f(c) \eql k.

	\textbf{True, by Theorem 5.3.6 (IVT). Also because this statement is just (a) above with k \eql 0, except weaker.}
	
\item If f : D \lra \br is continuous and bounded on D, then f assumes maximum and minimum values on D.
	
	\textbf{False.}
	
	\lt{f : (0, 1) \lra \br be defined by f(x) \eql x}
	
	\supp{f has x \mem D, a maximum value on D}
	
	Notice that 0 \ls x \ls 1, and that x \ls x \ps \frc{1 - x}{2}.
	
	However, notice also that x \ps \frc{1 - x}{2} \ls 1
	
	But x is a maximum value on D. A contradiction.
	
	WLOG, a minimum value on D is similar.
	
\elist
}

\newpage

\bsc{Exercise 5 (pages 220 - 221)}{
Show that the equation \uf{5}{x} = \uf{x}{4} has at least one real solution.

\lt{f : [$-1$, 0] \lra \br be defined by f(x) \eql \uf{5}{x} \ms \uf{x}{4}}

Notice that f($-1$) \eql $-0.8$ and f(0) \eql 1

Since \uf{5}{x} \ms \uf{x}{4} \eql 0 means \uf{5}{x} \eql \uf{x}{4}, and $-0.8$ \ls 0 \ls 1,

by Theorem 5.3.6, since f(x) is continuous on \br,

\exs c \mem [$-1$, 0] st f(c) \eql 0.

}

\bsc{Exercise 1 (pages 226 - 227)}{

Let f : D \lra \br. Mark each statement True or False. Justify each answer.

\balist
\item f is uniformly continuous on D iff for every \ep \gr 0 there exists a \dta \gr 0 st \av{f(x) \ms f(y)} \ls \dta whenever \av{x \ms y} \ls \ep and x, y \mem D.

\textbf{This isn't the definition, but I can't find a counter example for it...}

\item If D \eql \bk{x}, then f is uniformly continuous at x.
	
\textbf{True. Since x is the only element in the domain, and since f is a function, f(x) is the only element in the range of f which makes \av{f(x) \ms f(y)} always less than any \ep \gr 0 since there is only one object in the range, making them the same object in any possible case.}
	
\item If f is continuous and D is compact, then f is uniformly continuous on D.
	
	\textbf{True, by Theorem 5.4.6.}

\elist
}

\bsc{Exercise 2 (pages 226 - 227)}{
Let f : D \lra \br. Mark each statement True or False. Justify each answer.
\balist
\item In the definition of uniform continuity, the positive \dta depends only on the function f and the given \ep \gr 0.

\textbf{False. The positive \dta depends on the given x, y \mem D as well.}

\item If f is continuous and (\uw{x}{n}) is a Cauchy sequence in D, then (f(\uw{x}{n})) is a Cauchy sequence.

\textbf{False.}

\lt{\uw{x}{n} \eql \frc{1}{n}, n \mem \bn and f : (0, 1] \lra \br be defined by f(x) \eql \frc{1}{x}}

Notice that f(\uw{x}{n}) \eql 1, 2, 3...

This is not a Cauchy sequence.

\item If f : (a, b) \lra \br can be extended to a function that is continuous on [a, b], then f is uniformly continuous on (a,b).

\textbf{True, by Theorem 5.4.9.}

\elist
}

\newpage

\bsc{Exercise 3 (pages 226 - 227)}{

Determine which of the following continuous functions are uniformly continuous on the given set. Justify your answers.

\balist
\item f(x) \eql x on [2, 5] \textbf{since f is continuous and D is compact, f is uniformly continuous (by Theorem 5.4.6)}
\item f(x) \eql x on (0, 2) \textbf{since \wit{f} : [0, 2] \lra \br is continuous, f is uniformly continuous (by Theorem 5.4.9)}
\item f(x) \eql \uf{x}{2} \ps 2x \ms 7 on [0, 5] \textbf{since f is continuous and D is compact, f is uniformly continuous (by Theorem 5.4.6)}
\item f(x) \eql \uf{x}{2} \ps 2x \ms 7 on (1, 4) \textbf{since \wit{f} : [1, 4] \lra \br is continuous, f is uniformly continuous (by Theorem 5.4.9)}
\item f(x) \eql \frc{1}{x^2} on (0, 1) \textbf{Since \limt{x}{0} f(x) does not exist, f(x) cannot be extended to a continuous function. Therefore, f is not uniformly continuous.}
\item f(x) \eql \frc{1}{x^2} on (0, \infy) \textbf{Since \limt{x}{0} f(x) does not exist, f(x) cannot be extended to a continuous function. Therefore, f is not uniformly continuous.}
\item f(x) \eql \frc{x^2 - 4}{x - 2} on (2, 4) \textbf{Since \limt{x}{2} f(x) and \limt{x}{4} f(x) exist, f(x) can be extended to a continuous function. Therefore, f is uniformly continuous.}
\item f(x) \eql x sin(\frc{1}{x}) on (0, 1) \textbf{Since \limt{x}{0} f(x) \eql 0 and \limt{x}{1} f(x) \eql sin(1), f(x) can be extended to a continuous function. Therefore, f is uniformly continuous.}
\elist

}

\newpage 

\bsc{Exercise 4(a)(b) (pages 226 - 227)}{
Prove that each function is uniformly continuous on the given set by directly verifying the \ep - \dta property in Definition 4.1.

\textbf{Definition 5.4.1:}

f : D \lra \br is uniformly continuous on D if

\fa \ep \gr 0, \exs \dta \gr 0 st 0 \ls \av{x \ms y} \ls \dta and x, y \mem D implies \av{f(x) \ms f(y)} \ls \ep

\balist
\item f(x) \eql \uf{x}{3} on [0, 2]

\fa \ep \gr 0, \exs \dta \gr 0 st 0 \ls \av{x \ms y} \ls \dta and x, y \mem D implies \av{\uf{x}{3} \ms \uf{y}{3}} \ls \ep

\eqn{\av{\uf{x}{3} \ms \uf{y}{3}}}
\eqn{\av{(x - y)(x^2 + xy + y^2)}}
\eqn{\av{(x - y)(x^2 + xy + y^2)} \lse |(x - y)|(|x^2| + |xy| + |y^2|) \lse 12|(x - y)| \ls \ep}
so, whenever \av{x \ms y} \ls \dta \eql \frc{\ep}{12}, \av{\uf{x}{3} \ms \uf{y}{3}} \ls \ep

\item f(x) \eql \frc{1}{x} on [2, \infy)

\fa \ep \gr 0, \exs \dta \gr 0 st 0 \ls \av{x \ms y} \ls \dta and x, y \mem D implies \av{\frc{1}{x} \ms \frc{1}{y}} \ls \ep

\eqn{|\frac{1}{x} - \frac{1}{y}| = |\frac{y - x}{xy}|}
Since all elements in the domain are positive,
\eqn{|\frac{y - x}{xy}| = |y - x| \frac{1}{xy} = |x - y| \frac{1}{xy} \ls \ep}
So, since \frc{1}{x} is maximum at x \eql 2 and \frc{1}{y} is maximum at y \eql 2,
\eqn{|x - y| \ls xy\ep}
\eqn{|x - y| \ls (2)(2)\ep}
\eqn{|x - y| \ls \dta \eql 4\ep}
so, whenever \av{x \ms y} \ls \dta \eql 4\ep, \av{\frc{1}{x} \ms \frc{1}{y}} \ls \ep


\elist

}
\bsc{Exercise 5 (pages 226 - 227)}{

Prove that f(x) = $\sqrt{x}$ is uniformly continuous on [0, \infy).

In other words, show that

\fa \ep \gr 0, \exs \dta \gr 0 st \av{x \ms y} \ls \dta and x, y \mem [0, \infy) implies \av{$\sqrt{x}$ \ms $\sqrt{y}$} \ls \ep
\eqn{|\sqrt{x} - \sqrt{y}|^2 \lse |\sqrt{x} - \sqrt{y}||\sqrt{x} + \sqrt{y}| = |x - y| \ls \ep^2}
so, if we let
\eqn{\dta \eql \epsilon^2}
then \av{$\sqrt{x}$ \ms $\sqrt{y}$} \ls \ep

}

\newpage

\bsc{Exercise 11 (pages 226 - 227)}{

Let f : D \lra \br be uniformly continuous on the bounded set D. Prove that f is bounded on D. 

Use Theorem 4.4.6, 5.4.8 (but there is no theorem 4.4.6, figure out which one it is).

The hint is that it's bounded.

\ssc{Theorem 5.4.8}{

\lt{f : D \lra \br be uniformly continuous on D}

\as{\bk{\uw{x}{n}} is a Cauchy sequence in D}

Then,

\bk{f(\uw{x}{n})} is a Cauchy sequence.
}

\ssc{Lemma 4.3.11}{
Every Cauchy sequence is bounded.
}

\bgpf

Any Cauchy sequence \uw{x}{n} in D means that \bk{f(\uw{x}{n})} is a Cauchy sequence, and if \bk{f(\uw{x}{n})} is a Cauchy sequence then it's bounded.

So, our strategy will be to somehow make a Cauchy sequence \uw{x}{n} that has a limit at c such that f(c) \eql max(f(D)) and, WLOG, d such that f(d) \eql min(f(D)).

\epf


}

\end{document}