% Thank you Josh Davis for this template!
% https://github.com/jdavis/latex-homework-template/blob/master/homework.tex

\documentclass{article}

\newcommand{\hmwkTitle}{HW\ \#2}

% % ----------

% Packages

\usepackage{fancyhdr}
\usepackage{extramarks}
\usepackage{amsmath}
\usepackage{amssymb}
\usepackage{amsthm}
\usepackage{amsfonts}
\usepackage{tikz}
\usepackage[plain]{algorithm}
\usepackage{algpseudocode}
\usepackage{enumitem}
\usepackage{chngcntr}

% Libraries

\usetikzlibrary{automata, positioning, arrows}

%
% Basic Document Settings
%

\topmargin=-0.45in
\evensidemargin=0in
\oddsidemargin=0in
\textwidth=6.5in
\textheight=9.0in
\headsep=0.25in

\linespread{1.1}

\pagestyle{fancy}
\lhead{\hmwkAuthorName}
\chead{}
\rhead{\hmwkClass\ (\hmwkClassInstructor): \hmwkTitle}
\lfoot{\lastxmark}
\cfoot{\thepage}

\renewcommand\headrulewidth{0.4pt}
\renewcommand\footrulewidth{0.4pt}

\setlength\parindent{0pt}
\setcounter{secnumdepth}{0}

\newcommand{\hmwkClass}{MATH 3380 / Analysis 1}        % Class
\newcommand{\hmwkClassInstructor}{Dr. Welsh}           % Instructor
\newcommand{\hmwkAuthorName}{\textbf{Joshua Mitchell}} % Author

%
% Title Page
%

\title{
    \vspace{2in}
    \textmd{\textbf{\hmwkClass:\ \hmwkTitle}}\\
    \normalsize\vspace{0.1in}\small\vspace{0.1in}\large{\textit{\hmwkClassInstructor}}
    \vspace{3in}
}

\author{\hmwkAuthorName}
\date{}

\renewcommand{\part}[1]{\textbf{\large Part \Alph{partCounter}}\stepcounter{partCounter}\\}

% Integral dx
\newcommand{\dx}{\mathrm{d}x}

%
% Various Helper Commands
%

% For derivatives
\newcommand{\deriv}[1]{\frac{\mathrm{d}}{\mathrm{d}x} (#1)}

% For partial derivatives
\newcommand{\pderiv}[2]{\frac{\partial}{\partial #1} (#2)}


% Alias for the Solution section header
\newcommand{\solution}{\textbf{\large Solution}}

% Probability commands: Expectation, Variance, Covariance, Bias
\newcommand{\E}{\mathrm{E}}
\newcommand{\Var}{\mathrm{Var}}
\newcommand{\Cov}{\mathrm{Cov}}
\newcommand{\Bias}{\mathrm{Bias}}

% Formatting commands:

\newcommand{\mt}[1]{\ensuremath{#1}}
\newcommand{\nm}[1]{\textrm{#1}}

\newcommand\bsc[2][\DefaultOpt]{%
  \def\DefaultOpt{#2}%
  \section[#1]{#2}%
}
\newcommand\ssc[2][\DefaultOpt]{%
  \def\DefaultOpt{#2}%
  \subsection[#1]{#2}%
}
\newcommand{\bgpf}{\begin{proof} $ $\newline}

\newcommand{\bgeq}{\begin{equation*}}
\newcommand{\eeq}{\end{equation*}}	

\newcommand{\balist}{\begin{enumerate}[label=\alph*.]}
\newcommand{\elist}{\end{enumerate}}

\newcommand{\bilist}{\begin{enumerate}[label=\roman*)]}	

\newcommand{\bgsp}{\begin{split}}
% \newcommand{\esp}{\end{split}} % doesn't work for some reason.

\newcommand\prs[1]{~~~\textbf{(#1)}}

\newcommand{\lt}[1]{\textbf{Let: } #1}
     							   %  if you're setting it to be true
\newcommand{\supp}[1]{\textbf{Suppose: } #1}
     							   %  Suppose (if it'll end up false)
\newcommand{\wts}[1]{\textbf{Want to show: } #1}
     							   %  Want to show
\newcommand{\as}[1]{\textbf{Assume: } #1}
     							   %  if you think it follows from truth
\newcommand{\bpth}[1]{\textbf{(#1)}}

\newcommand{\step}[2]{\begin{equation}\tag{#2}#1\end{equation}}
\newcommand{\epf}{\end{proof}}

\newcommand{\dbs}[3]{\mt{#1_{#2_#3}}}

\newcommand{\sidenote}[1]{-----------------------------------------------------------------Side Note----------------------------------------------------------------
#1 \

---------------------------------------------------------------------------------------------------------------------------------------------}

% Analysis / Logical commands:

\newcommand{\br}{\mt{\mathbb{R}} }       % |R
\newcommand{\bq}{\mt{\mathbb{Q}} }       % |Q
\newcommand{\bn}{\mt{\mathbb{N}} }       % |N
\newcommand{\bc}{\mt{\mathbb{C}} }       % |C
\newcommand{\bz}{\mt{\mathbb{Z}} }       % |Z

\newcommand{\ep}{\mt{\epsilon} }         % epsilon
\newcommand{\fa}{\mt{\forall} }          % for all
\newcommand{\afa}{\mt{\alpha} }
\newcommand{\bta}{\mt{\beta} }
\newcommand{\mem}{\mt{\in} }
\newcommand{\exs}{\mt{\exists} }

\newcommand{\es}{\mt{\emptyset}}        % empty set
\newcommand{\sbs}{\mt{\subset} }         % subset of
\newcommand{\fs}[2]{\{\uw{#1}{1}, \uw{#1}{2}, ... \uw{#1}{#2}\}}

\newcommand{\lra}{ \mt{\longrightarrow} } % implies ----->
\newcommand{\rar}{ \mt{\Rightarrow} }     % implies -->

\newcommand{\lla}{ \mt{\longleftarrow} }  % implies <-----
\newcommand{\lar}{ \mt{\Leftarrow} }      % implies <--

\newcommand{\eql}{\mt{=} }
\newcommand{\pr}{\mt{^\prime} } 		   % prime (i.e. R')
\newcommand{\uw}[2]{#1\mt{_{#2}}}
\newcommand{\frc}[2]{\mt{\frac{#1}{#2}}}

\newcommand{\bnm}[2]{\mt{#1\setminus{#2}}}
\newcommand{\bnt}[2]{\mt{\textrm{#1}\setminus{\textrm{#2}}}}
\newcommand{\bi}{\bnm{\mathbb{R}}{\mathbb{Q}}}

\newcommand{\urng}[2]{\mt{\bigcup_{#1}^{#2}}}
\newcommand{\nrng}[2]{\mt{\bigcap_{#1}^{#2}}}

\newcommand{\nbho}[3]{\textrm{N(}#1, #2\textrm{) }\cap \textrm{ #3} \neq \emptyset}
     							   %  N(x, eps) intersect S \= emptyset
\newcommand{\nbhe}[3]{\textrm{N(}#1, #2\textrm{) }\cap \textrm{ #3} = \emptyset}
     							   %  N(x, eps) intersect S  = emptyset
\newcommand{\dnbho}[3]{\textrm{N*(}#1, #2\textrm{) }\cap \textrm{ #3} \neq \emptyset}
     							   %  N*(x, eps) intersect S \= emptyset
\newcommand{\dnbhe}[3]{\textrm{N*(}#1, #2\textrm{) }\cap \textrm{ #3} = \emptyset}
     							   %  N*(x, eps) intersect S = emptyset
     							 


% ----------

% ----------

% Packages

\usepackage{fancyhdr}
\usepackage{extramarks}
\usepackage{amsmath}
\usepackage{amssymb}
\usepackage{amsthm}
\usepackage{amsfonts}
\usepackage{tikz}
\usepackage[plain]{algorithm}
\usepackage{algpseudocode}
\usepackage{enumitem}
\usepackage{chngcntr}

% Libraries

\graphicspath{{/Users/jm/iclouddrive/3380pics/}}

\usetikzlibrary{automata, positioning, arrows}

%
% Basic Document Settings
%

\topmargin=-0.45in
\evensidemargin=0in
\oddsidemargin=0in
\textwidth=6.5in
\textheight=9.0in
\headsep=0.25in

\linespread{1.1}

\pagestyle{fancy}
\lhead{\hmwkAuthorName}
\chead{}
\rhead{\hmwkClass\ (\hmwkClassInstructor): \hmwkTitle}
\lfoot{\lastxmark}
\cfoot{\thepage}

\renewcommand\headrulewidth{0.4pt}
\renewcommand\footrulewidth{0.4pt}

\setlength\parindent{0pt}
\setcounter{secnumdepth}{0}

\newcommand{\hmwkClass}{MATH 5358 / Applied Discrete Math}        % Class
\newcommand{\hmwkClassInstructor}{Dr. Rusnak}           % Instructor
\newcommand{\hmwkAuthorName}{\textbf{Joshua Mitchell}} % Author

%
% Title Page
%

\title{
    \vspace{2in}
    \textmd{\textbf{\hmwkClass:\ \hmwkTitle}}\\
    \normalsize\vspace{0.1in}\small\vspace{0.1in}\large{\textit{\hmwkClassInstructor}}
    \vspace{3in}
}

\author{\hmwkAuthorName}
\date{}

\renewcommand{\part}[1]{\textbf{\large Part \Alph{partCounter}}\stepcounter{partCounter}\\}

% Integral dx
\newcommand{\dx}{\mathrm{d}x}

%
% Various Helper Commands
%

% For derivatives
\newcommand{\deriv}[1]{\frac{\mathrm{d}}{\mathrm{d}x} (#1)}

% For partial derivatives
\newcommand{\pderiv}[2]{\frac{\partial}{\partial #1} (#2)}


% Alias for the Solution section header
\newcommand{\solution}{\textbf{\large Solution}}

% Probability commands: Expectation, Variance, Covariance, Bias
\newcommand{\E}{\mathrm{E}}
\newcommand{\Var}{\mathrm{Var}}
\newcommand{\Cov}{\mathrm{Cov}}
\newcommand{\Bias}{\mathrm{Bias}}

% Formatting commands:

\newcommand{\mt}[1]{\ensuremath{#1}}
\newcommand{\nm}[1]{\textrm{#1}}

\newcommand\bsc[2][\DefaultOpt]{%
  \def\DefaultOpt{#2}%
  \section[#1]{#2}%
}
\newcommand\ssc[2][\DefaultOpt]{%
  \def\DefaultOpt{#2}%
  \subsection[#1]{#2}%
}
\newcommand{\bgpf}{\begin{proof} $ $\newline}

\newcommand{\bgeq}{\begin{equation*}}
\newcommand{\eeq}{\end{equation*}}	

\newcommand{\balist}{\begin{enumerate}[label=\alph*.]}
\newcommand{\elist}{\end{enumerate}}

\newcommand{\bilist}{\begin{enumerate}[label=\roman*)]}	

\newcommand{\bgsp}{\begin{split}}
% \newcommand{\esp}{\end{split}} % doesn't work for some reason.

\newcommand\prs[1]{~~~\textbf{(#1)}}

\newcommand{\lt}[1]{\textbf{Let: } #1}
     							   %  if you're setting it to be true
\newcommand{\supp}[1]{\textbf{Suppose: } #1}
     							   %  Suppose (if it'll end up false)
\newcommand{\wts}[1]{\textbf{Want to show: } #1}
     							   %  Want to show
\newcommand{\as}[1]{\textbf{Assume: } #1}
     							   %  if you think it follows from truth
\newcommand{\bpth}[1]{\textbf{(#1)}}

\newcommand{\step}[2]{\begin{equation}\tag{#2}#1\end{equation}}
\newcommand{\epf}{\end{proof}}

\newcommand{\dbs}[3]{\mt{#1_{#2_#3}}}

\newcommand{\sidenote}[1]{-----------------------------------------------------------------Side Note----------------------------------------------------------------
#1 \

---------------------------------------------------------------------------------------------------------------------------------------------}

% Analysis / Logical commands:

\newcommand{\br}{\mt{\mathbb{R}} }       % |R
\newcommand{\bq}{\mt{\mathbb{Q}} }       % |Q
\newcommand{\bn}{\mt{\mathbb{N}} }       % |N
\newcommand{\bc}{\mt{\mathbb{C}} }       % |C
\newcommand{\bz}{\mt{\mathbb{Z}} }       % |Z

\newcommand{\ep}{\mt{\epsilon} }         % epsilon
\newcommand{\fa}{\mt{\forall} }          % for all
\newcommand{\afa}{\mt{\alpha} }
\newcommand{\bta}{\mt{\beta} }
\newcommand{\dta}{\mt{\delta} }
\newcommand{\mem}{\mt{\in} }
\newcommand{\exs}{\mt{\exists} }

\newcommand{\es}{\mt{\emptyset} }        % empty set
\newcommand{\sbs}{\mt{\subset} }         % subset of
\newcommand{\fs}[2]{\{\uw{#1}{1}, \uw{#1}{2}, ... \uw{#1}{#2}\}}

\newcommand{\lra}{ \mt{\longrightarrow} } % implies ----->
\newcommand{\rar}{ \mt{\Rightarrow} }     % implies -->

\newcommand{\lla}{ \mt{\longleftarrow} }  % implies <-----
\newcommand{\lar}{ \mt{\Leftarrow} }      % implies <--

\newcommand{\av}[1]{\mt{|}#1\mt{|}}  % absolute value

\newcommand{\prn}[1]{(#1)}
\newcommand{\bk}[1]{\{#1\}}

\newcommand{\ps}{\mt{+} }
\newcommand{\ms}{\mt{-} }

\newcommand{\ls}{\mt{<} }
\newcommand{\gr}{\mt{>} }

\newcommand{\lse}{\mt{\leq} }
\newcommand{\gre}{\mt{\geq} }

\newcommand{\eql}{\mt{=} }

\newcommand{\pr}{\mt{^\prime} } 		   % prime (i.e. R')
\newcommand{\uw}[2]{#1\mt{_{#2}}}
\newcommand{\uf}[2]{#1\mt{^{#2}}}
\newcommand{\frc}[2]{\mt{\frac{#1}{#2}}}
\newcommand{\lmti}[1]{\mt{\displaystyle{\lim_{#1 \to \infty}}}}
\newcommand{\limt}[2]{\mt{\displaystyle{\lim_{#1 \to #2}}}}

\newcommand{\bnm}[2]{\mt{#1\setminus{#2}}}
\newcommand{\bnt}[2]{\mt{\textrm{#1}\setminus{\textrm{#2}}}}
\newcommand{\bi}{\bnm{\mathbb{R}}{\mathbb{Q}}}

\newcommand{\urng}[2]{\mt{\bigcup_{#1}^{#2}}}
\newcommand{\nrng}[2]{\mt{\bigcap_{#1}^{#2}}}
\newcommand{\nck}[2]{\mt{{#1 \choose #2}}}

\newcommand{\nbho}[3]{\textrm{N(}#1, #2\textrm{) }\cap \textrm{ #3} \neq \emptyset}
     							   %  N(x, eps) intersect S \= emptyset
\newcommand{\nbhe}[3]{\textrm{N(}#1, #2\textrm{) }\cap \textrm{ #3} = \emptyset}
     							   %  N(x, eps) intersect S  = emptyset
\newcommand{\dnbho}[3]{\textrm{N*(}#1, #2\textrm{) }\cap \textrm{ #3} \neq \emptyset}
     							   %  N*(x, eps) intersect S \= emptyset
\newcommand{\dnbhe}[3]{\textrm{N*(}#1, #2\textrm{) }\cap \textrm{ #3} = \emptyset}
     							   %  N*(x, eps) intersect S = emptyset
     							   
\newcommand{\eqn}[1]{\[#1\]}
\newcommand{\splt}[1]{\begin{split}#1\end{split}}

\newcommand{\infy}{\mt{\infty} }
\newcommand{\unn}{\mt{\cup} }
\newcommand{\inn}{\mt{\cap} }
\newcommand\tab[1][1cm]{\hspace*{#1}}
\newcommand{\rln}{ \mt{\sim} }
\newcommand{\dvd}{ \mt{\vert} }
\newcommand{\ndvd}{ \mt{\not\vert} }
\newcommand{\eqw}{ \mt{ \equiv } }
\newcommand{\lcg}{ \mt{\gamma} }
\newcommand{\smm}[2]{ \mt{\sum_{#1}^{#2}}}
\newcommand{\ff}[2]{[\mt{#1}]\mt{_{(\underline{#2})}}}

\newcommand{\wit}[1]{\mt{\widetilde{#1}}}
     							 
% ----------

\begin{document}

\begin{enumerate}
  \item Prove Pascal's Formula \nck{\afa}{k} \eql \nck{\afa - 1}{k - 1} \ps \nck{\afa - 1}{k} for any \afa \mem \br and k \mem \uw{\bz}{\gre 0}. (Note: You will need to use the falling factorial definition.)
  \eqn{
  	\splt{
  		\nck{\afa}{k} & = \nck{\afa - 1}{k - 1} + \nck{\afa - 1}{k} \\
  		& = \frac{(\alpha - 1)!}{((\alpha - 1) - (k - 1))!(k - 1)!} + \frac{(\alpha - 1)!}{((\alpha - 1) - k)!k!} \\
  		& = \frac{(\alpha - 1)!}{(\alpha - k)!(k - 1)!} + \frac{(\alpha - 1)!}{(\alpha - 1 - k)!k!} \\
  		& = \frac{(\alpha - 1)!}{(\alpha - k)!(k - 1)!} + \frac{(\alpha - 1)!(\alpha - k)\frac{1}{k}}{(\alpha - k)!(k - 1)!} \\
  		& = \frac{(\alpha - 1)! + (\alpha - 1)!(\alpha - k)\frac{1}{k}}{(\alpha - k)!(k - 1)!} \\
  		& = \frac{k(\alpha - 1)! + (\alpha - 1)!(\alpha - k)}{(\alpha - k)!k!} \\
  		& = \frac{\alpha(\alpha - 1)!}{(\alpha - k)!k!} \\
  		& = \frac{\alpha!}{(\alpha - k)!k!} \\
  	}
  }
  
  
  \item Determine the generating function for each of the following sequences:
   \balist
   \item 1, r, \uf{r}{2}, \uf{r}{3}, ...
   
   1 \ps rx \ps \uf{r}{2}\uf{x}{2} ... \lra \frc{1}{1 - rx}
   
   \item 1, \ms 1, 1, \ms 1, ... 
   
   1 \ms x \ps \uf{x}{2} \ms \uf{x}{3} \lra \frc{1}{1 + x}
   
   \item \nck{\afa}{0}, \ms\nck{\afa}{1}, \nck{\afa}{2}, \ms\nck{\afa}{3}, ...
   
   \nck{\afa}{0} \ms \nck{\afa}{1}x \ps \nck{\afa}{2}\uf{x}{2} \ms \nck{\afa}{3}\uf{x}{3} ...
   
   1 \ms \afa x \ps \frc{\afa(\afa - 1)}{2 * 1}\uf{x}{2} \ms \frc{\afa(\afa - 1)(\afa - 2)}{3 * 2 * 1}\uf{x}{3} ...
   
   1 \ms \afa x \ps \frc{\ff{\afa}{2}}{\ff{2}{2}}\uf{x}{2} \ms \frc{\ff{\afa}{3}}{\ff{3}{3}}\uf{x}{3} ...
   
   \mt{\sum_{k = 0}^\infy (-1)^k\nck{\afa}{k}x^k}
   
   \mt{(1 - x)^\afa}
   
   \item 1, \frc{1}{1!}, \frc{1}{2!}, \frc{1}{3!}, ...
   
   1 \ps \frc{1}{1!}x \ps \frc{1}{2!}\uf{x}{2} \ps \frc{1}{3!}\uf{x}{3}...
   
   \uf{e}{x}
   
   \item 1, \frc{-1}{1!}, \frc{1}{2!}, \frc{-1}{3!}, \frc{1}{4!}, ...
   
   1 \ms \frc{1}{1!}x \ps \frc{1}{2!}\uf{x}{2} \ms \frc{1}{3!}\uf{x}{3}...
   
   1 \ps \frc{1}{1!}x \ps \frc{1}{2!}\uf{x}{2} \ps \frc{1}{3!}\uf{x}{3}... \ms 2(\frc{1}{1!}x \ps \frc{1}{3!}\uf{x}{3}...)
   
   \uf{e}{x} \ms sinh x
   
   \item \nck{0}{2}, \nck{1}{2}, \nck{2}{2}, \nck{3}{2}, ...
   
   \nck{0}{2} \ps \nck{1}{2}x \ps \nck{2}{2}\uf{x}{2} \ps \nck{3}{2}\uf{x}{3} ...
   
   -\frc{1}{2} \ps 0x \ps \frc{\ff{2}{2}}{\ff{2}{2}}\uf{x}{2} \ps \frc{\ff{3}{2}}{\ff{2}{2}}\uf{x}{3} ...
   
   -\frc{1}{2} \ps 0x \ps \frc{\ff{2}{2}}{2}\uf{x}{2} \ps \frc{\ff{3}{2}}{2}\uf{x}{3} ... 

   \elist
   
   \textbf{Is this the right process?} \textbf{How do you know when to use EGF vs GF?}
   \item Given the Fibonacci sequence \uw{f}{n} \eql \uw{f}{n - 1} \ps \uw{f}{n - 2} with initial conditions \uw{f}{0} \eql 0 and \uw{f}{1} \eql 1,
   \balist
   \item Solve the recursion by writing it as a linear homogenous recursion and finding the characteristic polynomial. Write your answer in the form \uw{c}{1}\uf{\uw{q}{1}}{n} \ps \uw{c}{2}\uf{\uw{q}{2}}{n}. (Note: we have already solved this up to finding the constants in class. Finish the problem.)
   
   \uw{f}{n} \eql \uw{f}{n - 1} \ps \uw{f}{n - 2}
   
   0 \eql \uw{f}{n} - \uw{f}{n - 1} \ms \uw{f}{n - 2}
   
   \uf{q}{n} \ms \uf{q}{n - 1} \ms \uf{q}{n - 2} \eql 0
   
   \uf{q}{n - 2}(\uf{q}{2} \ms \uf{q}{1} \ms 1) \eql 0
   
   Thus, the solution has the form \uw{f}{n} \eql \uw{c}{1}\uf{(?)}{n} \uw{c}{2}\uf{(?)}{n}.
   
   q \eql \frc{1 \pm \sqrt{5}}{2}
   
   \uw{f}{n} \eql \uw{c}{1}\frc{1 + \sqrt{5}}{2} \ps \uw{c}{2}\frc{1 - \sqrt{5}}{2}
   
   \uw{f}{0} \eql \uw{c}{1} \ps \uw{c}{2}
   
   \uw{f}{1} \eql \uw{c}{1}\uf{(\frc{1 + \sqrt{5}}{2})}{1} \ps \uw{c}{2}\uf{(\frc{1 - \sqrt{5}}{2})}{1}
   
   Let \uw{f}{0} \eql 0, \uw{f}{1} \eql 1. Solving for \uw{c}{1} and \uw{c}{2} gives us \uw{c}{1} \eql \frc{1}{\sqrt{5}}, \uw{c}{2} \eql \frc{-1}{\sqrt{5}}
   
   Thus, \uw{f}{n} \eql \frc{1}{\sqrt{5}}\uf{(\frc{1 + \sqrt{5}}{2})}{n} \ps \frc{-1}{\sqrt{5}}\uf{(\frc{1 - \sqrt{5}}{2})}{n}
   
   \item Solve the recursion by using generating functions. (Note: Use a partial fraction decomposition to finish the problem.)
   
   \uw{f}{n} \eql \uw{f}{n - 1} \ps \uw{f}{n - 2}
   
   \uw{h}{n} \eql \uw{h}{n - 1} \ps \uw{h}{n - 2}
   
   0 \eql \uw{h}{n} \ms \uw{h}{n - 1} \ms \uw{h}{n - 2}
   
   Let g(x) \eql \uw{h}{0} \ps \uw{h}{1}\uf{x}{1} \ps \uw{h}{2}\uf{x}{2} ... 
   
   Then,
   \eqn{
   		\splt{
   			g(x) = &  \uw{h}{0} \ps \uw{h}{1}\uf{x}{1} \ps \uw{h}{2}\uf{x}{2} ... \\
   			-xg(x) = & -h_0x^1 - h_1x^2 - h_2x^3 ... \\
   			-x^2g(x) = & -h_0x^2 - h_1x^3 - h_2x^4 ...
   		}
   }
   Thus,
   \eqn{(1 - x - x^2)g(x) = h_0 + (h_1 - h_0)x^1 + (h_2 - h_1 - h_0)x^2 + (h_3 - h_2 - h_1)x^3 + ...}
   
   But since 0 \eql \uw{h}{n} \ms \uw{h}{n - 1} \ms \uw{h}{n - 2},
   
   \eqn{(1 - x - x^2)g(x) = h_0 + (h_1 - h_0)x^1}
   \eqn{g(x) = \frac{h_0 + (h_1 - h_0)x}{(1 - x - x^2)}}
   Plugging in \mt{h_0 = 0} and \mt{h_1 = 1},
   \eqn{g(x) = \frac{x}{(1 - x - x^2)}}
   \eqn{g(x) = \frac{x}{(1 - x - x^2)}}
   \eqn{g(x) = \frac{x}{(x + (-\frac{1}{2} + \frac{\sqrt{5}}{2})(x - (-\frac{1}{2} + \frac{\sqrt{5}}{2})}}
   \eqn{g(x) = \frac{A}{(x + (-\frac{1}{2} + \frac{\sqrt{5}}{2})} + \frac{B}{(x - (-\frac{1}{2} + \frac{\sqrt{5}}{2})}}
   \eqn{g(x) = \frac{1/2}{(x + (-\frac{1}{2} + \frac{\sqrt{5}}{2})} + \frac{1/2}{(x - (-\frac{1}{2} + \frac{\sqrt{5}}{2})}}
   \eqn{g(x) = \frac{1}{2(x + (-\frac{1}{2} + \frac{\sqrt{5}}{2})} + \frac{1}{2(x - (-\frac{1}{2} + \frac{\sqrt{5}}{2})}}
   \eqn{g(x) = \frac{1}{2}\frac{1}{((-\frac{1}{2} + \frac{\sqrt{5}}{2}) + x)}  -\frac{1}{2}\frac{1}{((-\frac{1}{2} + \frac{\sqrt{5}}{2}) - x)}}
   
   \textbf{At this point, I'm not sure how to convert to Power Series}
   
   \uw{f}{n} \eql \frc{1}{\sqrt{5}}\uf{(\frc{1 + \sqrt{5}}{2})}{n} \ps \frc{-1}{\sqrt{5}}\uf{(\frc{1 - \sqrt{5}}{2})}{n}
   
   
   \elist
   \item Prove that the Fibonacci number \uw{f}{n} is even if, and only if, divisible by 3.
   
   \textbf{Wait.. 2 is a fibonacci number that is even and not divisible by 3.. So is 8.}
   
   \lra
   
   \as{\uw{f}{n} is even (i.e. \exs t \mem \bz such that \uw{f}{n} \eql 2t)}
   
   \lla
   
   \as{3 divides \uw{f}{n} (i.e. \exs t \mem \bz such that \uw{f}{n} \eql 3t)}
   
   \item Consider a 1-by-n chessboard. Suppose we color each square of the chessboard with one of the colors red, white, or blue. Let \uw{h}{n} be the number of colorings in which there is an even number of red squares (the example from class).
   \balist
   \item Reproduce the exponential generating function solution from class.
   
   Colors: R, W, B. R is even.
   
   EGF:
   \eqn{(1 \ps \frac{x^2}{2!} + \frac{x^4}{4!}...)(1 \ps \frac{x^1}{1!} + \frac{x^2}{2!}...)^2}
   \eqn{\frac{1}{2}(e^x + e^{-x})e^xe^x}
   \eqn{\frac{1}{2}(e^{3x} + e^x)}
   \eqn{\frac{1}{2}(\sum\frac{3^nx^n}{n!} + \sum\frac{x^n}{n!})}
   \eqn{\frac{1}{2}(\sum(3^n + 1)\frac{x^n}{n!})}
   \eqn{\rar \sum(\frac{3^n + 1}{2})\frac{x^n}{n!}}
   \item Solve this by using a standard generating function and partial fractions.
   
   GF:
   \eqn{(1 + x^2 + x^4 ...)(1 + x + x^2 ...)^2}
   \eqn{\frac{1}{1 - x^2}\frac{1}{1 - x}\frac{1}{1 - x}}
   \eqn{\frac{1}{(1 + x)(1 - x)^3}}
   \eqn{\frac{A}{1 + x} + \frac{B}{1 - x} + \frac{C}{(1 - x)^2} + \frac{D}{(1 - x)^3}}
   After partial fractions:
   
   \mt{
   	\begin{matrix}
  	1 & 1 & 1 & 1 & | & 1\\
  	-3 & -1 & 0 & 1 & | & 0 \\
  	3 & -1 & -1 & 0 & | & 0 \\
  	-1 & 1 & 0 & 0 & | & 0
	\end{matrix}
	}
	
	A \eql \frc{1}{8}, B \eql \frc{1}{8}, C \eql \frc{1}{4}, D \eql \frc{1}{2}
	
	\eqn{\rar \frac{1}{8}\sum(-1)^nx^n + \frac{1}{8}\sum x^n + \frac{1}{4}\sum nx^{n - 1} + \frac{1}{2}\sum n(n - 1)x^{n - 2}}
	\textbf{How do I get from above to below?}
	\eqn{\sum (\frac{3^n + 1}{2})x^n}
   
   \item Reproduce the associated recursion for \uw{h}{n}.
   \item Using your answer from part c, solve the recursion using the generating function method for non-homogeneous recursions.
   \elist
   \item Consider a 1-by-n chessboard. Suppose we color each square of the chessboard with one of the colors red or blue. Let \uw{h}{n} be the number of colorings in which no two squares that are colored red are adjacent. Find a recurrence relation that \uw{h}{n} satisfies, then derive a formula for \uw{h}{n}.
   
   Colors: R, B. 
   
   \uw{h}{n} is the number of colorings such that no two adjacent squares are red.
   
   In other words, B \rar R or B and R \rar B.
   
   \uw{h}{1} \eql 2, \uw{h}{2} \eql 3, \uw{h}{3} \eql 5, \uw{h}{4} \eql 8, \uw{h}{5} \eql 13...
   
   It looks like the number of ways that color the nth square blue is just \uw{h}{n - 1}, and the number of ways to color the nth square red is the number of ways that color the (n-1)th square blue, which is just \uw{h}{n - 2}.
   
   So it looks like a recurrence relation would be \uw{h}{n} \eql \uw{h}{n - 1} \ps \uw{h}{n - 2}
   
   Since this is Fibonacci (and we already did this):
   
   \uw{h}{n} \eql \frc{1}{\sqrt{5}}\uf{(\frc{1 + \sqrt{5}}{2})}{n} \ps \frc{-1}{\sqrt{5}}\uf{(\frc{1 - \sqrt{5}}{2})}{n}
   
   
   \item Determine the generating function for the number \uw{h}{n} of bags of fruit of apples, oranges, bananas, and pears in which apples \% 2 \eql 0, oranges \lse 2, bananas \% 3 \eql 0, and pears \lse 1. Then find a formula for \uw{h}{n} from the generating function.
   
   GF:
   
   \eqn{(1 \ps x^2 + x^4 + ...)(1 + x + x^2)(1 + x^3 + x^6 ...)(1 + x)}
   \eqn{\frac{1}{(1 - x)^2}\frac	{1 - x^3	}{1 - x}\frac{1}{(1 - x)^3}\frac{1 - x^2}{1 - x}}
   \eqn{\frac{1}{(1 + x)(1 - x)}\frac{(1 - x)(1 + x + x^2)}{1 - x}\frac{1}{(1 - x)^3}\frac{(1 - x)(1 + x)}{1 - x}}
   
   \eqn{\frac{(1 + x + x^2)}{(1 - x)^4}}
   
   \eqn{\frac{1}{(1 - x)^2} + \frac{-3}{(1 - x)^3} + \frac{3}{(1 - x)^4}}
   
   \eqn{\sum x^{2n} - 3\sum x^{3n} + 3\sum x^{4n} \lra \sum x^{2n} - 3x^{3n} + 3x^{4n}}
   
   \item Determine the exponential generating function for the following sequence:
   \balist
   \item 0!, 1!, 2!, ...
   
   \eqn{\uf{g}{(e)}(x) \eql \frac{0!}{0!} + \frac{1!}{1!}x + \frac{2!}{2!}x^2 ... }
   \eqn{\uf{g}{(e)}(x) \eql 1 + x + x^2 ...}
   
   \item \ff{\alpha}{0}, \ff{\alpha}{1}, \ff{\alpha}{2}, \ff{\alpha}{3}, ... (Note: \ff{\alpha}{n} is the falling factorial.)
   
   \eqn{\uf{g}{(e)}(x) \eql \frac{\afa}{0!} + \frac{\afa(\afa - 1)}{1!}x + \frac{\alpha(\alpha - 1)(\alpha - 2)}{2!}x^2 ... }
   \eqn{\uf{g}{(e)}(x) \eql \sum_{n = 0}^{\infty} \frac{\alpha!}{(\alpha - n - 1)!n!}}
   
   
   \elist
   \item Let \uw{h}{n} denote the number of ways to color the square of a 1-by-n board with the colors red, white, blue, and green in such a way that the numbers of squares colored red is even and the number of squares colored white is odd. Determine the exponential generating function for the sequence, then find a simple formula for \uw{h}{n}.
   
   Colors: RWBG. R is even, W is odd.
   
   EGF:
   \eqn{(1 + \frac{x^2}{2!} + \frac{x^4}{4!}...)(x + \frac{x^3}{3!} + \frac{x^5}{5!}...)(1 \ps x \ps \frac{x^2}{2!} + \frac{x^3}{3!}...)^2}
   \eqn{\frac{1}{2}(e^x + e^{-x}) * \frac{1}{2}(e^x - e^{-x}) * e^x * e^x}
   \eqn{\frac{1}{4}(e^{2x} - e^{-2x})* e^{2x}}
   \eqn{\frac{1}{4}(e^{4x} - 1)}
   
   f(x) \eql \mt{\frac{1}{4}(e^{4x} - 1)} \lra f'(x) \eql \frc{1}{4}(4\uf{e}{4x}) \lra f''(x) \eql \frc{1}{4}16\uf{e}{4x} \lra f'''(x) \eql \frc{1}{4}64\uf{e}{4x} \lra \uf{f}{n}(x) \eql \frc{1}{4}(\uf{4}{n}\uf{e}{4x})
   
   So,
   
   \uw{h}{n} \eql \frc{4^n}{4n!} \textbf{Which doesn't seem right since n! grows faster than \uf{4}{n}. Also, where goes - 1?}
   
   
   \item Determine the number of ways to color the squares of a 1-by-n board using the colors red, blue, green, and orange if an even number of squares is to be colored red and an even number is to be colored green.
   
   Colors: RGBO. R, G are even.
   
   GF:
   \eqn{(1 + x^2 + x^4...)^2(1 + x + x^2 + x^3...)^2}
   \eqn{\frac{1}{1 - x^2}\frac{1}{1 - x^2}\frac{1}{1 - x}\frac{1}{1 - x}}
   \eqn{\frac{1}{(1 - x)^4(1 + x)^2}}
   \eqn{\frac{A}{(1 - x)} + \frac{B}{(1 - x)^2} + \frac{C}{(1 - x)^3} + \frac{D}{(1 - x)^4} + \frac{E}{(1 + x)} + \frac{F}{(1 + x)^2}}
   \eqn{A\sum x^n + E\sum(-1)^nx^n}
   
   \item Determine the number of n-digit numbers with all digits odd, such that 1 and 3 each occur a nonzero, even number of times.
   \item Solve the recurrence relation:
   \balist
   \item \uw{h}{n} \eql 4\uw{h}{n - 2}, \uw{h}{0} \eql 0, \uw{h}{1} \eql 1, and n \gre 2.
   
   0, 1, 0, 4, 0, 16, 0, 64...
   
   \uw{h}{n} \ms 4\uw{h}{n - 2} \eql 0
   
   \uf{q}{n - 2}(\uf{q}{2} \ms 4) \eql 0
   
   \uw{h}{n} \eql a\uf{(2)}{n} \ps b\uf{(-2)}{n}
   
   0 \eql a \ps b and 1 \eql 2a \ms 2b
   
   b \eql -\frc{1}{4}, a \eql \frc{1}{4}
   
   \uw{h}{n} \eql \frc{1}{4}\uf{2}{n} \ms \frc{1}{4}\uf{(-2)}{n}
   
   \item \uw{h}{n} \eql \uw{h}{n - 1} \ps 9\uw{h}{n - 2} \ms 9\uw{h}{n - 3}, \uw{h}{0} \eql 0, \uw{h}{1} \eql 1, and \uw{h}{2} \eql 2. n \gre 3.
   
   \uf{q}{n - 3}(\uf{q}{3} \ms \uf{q}{2} \ms 9\uf{q}{1} \ms 9) \eql 0
   
   (\uf{q}{2} \ms 9)(\uf{q}{1} \ps 1) \eql 0
   
   (q \ms 3)(q \ps 3)(q \ps 1) \eql 0
   
   \uw{h}{n} \eql a\uf{(3)}{n} \ps b\uf{(-3)}{n} \ps c\uf{(-1)}{n}
   
   So, 0 \eql a \ps b \ps c, 1 \eql 3a \ms 3b \ms c, 2 \eql 9a \ps 9b \ps c
   
   Hence, a \eql \frc{1}{4}, b \eql 0, c \eql -\frc{1}{4}
   
   \uw{h}{n} \eql \frc{1}{4}\uf{(3)}{n} \ps -\frc{1}{4}\uf{(-1)}{n}
   
   \item \uw{h}{n} \eql 4\uw{h}{n - 1} \ps \uf{4}{n}, \uw{h}{0} \eql 3 and n \gre 1.
   
   3, 16, 80, 384...
   
   \elist
   \item Let \uw{h}{n} \eql the number of ternary strings of length n made up of 0's, 1's, and 2's, such that the substrings 00, 01, 10, and 11 never occur. Prove that
   \eqn{h_n = h_{n - 1} + 2h_{n - 2}}
   with \uw{h}{0} \eql 1, \uw{h}{1} \eql 3, and then find a formula for \uw{h}{n}.
   \item Compute the Stirling numbers of the first and second kind up to n \eql 6 using their recursive formulas.
   
   \textbf{But stirling numbers take 2 parameters: s(p, k); where does n fit?}
   
   \item Prove the Stirling numbers of the second kind satisfy:

   Recall: S(p, k) \eql k S(p \ms 1, k) \ps S(p \ms 1, k \ms 1)
   \balist
   \item S(n, 1) \eql 1
   \item S(n, 2) \eql \uf{2}{n - 1} \ms 1
   \item S(n, n \ms 1) \eql \nck{n}{2}
   \elist
   \item Prove the Stirling numbers of the first kind satisfy:
   \balist
   \item s(n, 1) \eql (n \ms 1)!
   \item s(n, n \ms 1) \eql \nck{n}{2}
   \elist
   \item Write \ff{n}{k} as a polynomial in n for k \eql 5, 6, 7. (Do not use distribution!)
   
   \ff{n}{k} \eql n(n \ms 1)(n \ms 2)...(n \ms k)
   
   \ff{n}{k} \eql \mt{\sum_{p = 0}^k (-1)^{k - p}s(k, p)n^p}
   
   \ff{n}{5} \eql \mt{\sum_{p = 0}^5 (-1)^{5 - p}s(5, p)n^p}
   
   \ff{n}{5} \eql \mt{-s(5, 0)} \ps \mt{s(5, 1)n} \ms \mt{s(5, 2)n^2} \ps \mt{s(5, 3)n^3} \ms \mt{s(5, 4)n^4} \ps \mt{s(5, 5)n^5}
   
   \ff{n}{5} \eql \mt{4!n} \ms \mt{s(5, 2)n^2} \ps \mt{s(5, 3)n^3} \ms \mt{\nck{5}{2}n^4} \ps \mt{n^5}
   
   s(5, 2) \eql 4s(4, 2) \ps 3! and s(5, 3) \eql 4\nck{4}{2} \ps s(4, 2)
   
   
   
   \item Find a closed formula for the sequence: 1, 6, 15, 28, 45, 66, 91, ... (Use a difference table.)
   
   \begin{tabular}{l|ccccccc}
  & 1 & 6 & 15 & 28 & 45 & 66 & 91 \\
  \hline
  & & 5 & 9 & 13 & 17 & 21 & 25 \\
  & & & 4 & 4 & 4 & 4 & 4 \\
  & & & & 0 & 0 & 0 & 0
\end{tabular}

\uw{h}{n} \eql 1\nck{n}{0} \ps 5\nck{n}{1} \ps 4\nck{n}{2}

   
\end{enumerate}

\end{document}