% Thank you Josh Davis for this template!
% https://github.com/jdavis/latex-homework-template/blob/master/homework.tex

\documentclass{article}

\usepackage{fancyhdr}
\usepackage{extramarks}
\usepackage{amsmath}
\usepackage{amssymb}
\usepackage{amsthm}
\usepackage{amsfonts}
\usepackage{tikz}
\usepackage[plain]{algorithm}
\usepackage{algpseudocode}
\usepackage{enumitem}
\usepackage{chngcntr}

\usetikzlibrary{automata,positioning}

%
% Basic Document Settings
%

\topmargin=-0.45in
\evensidemargin=0in
\oddsidemargin=0in
\textwidth=6.5in
\textheight=9.0in
\headsep=0.25in

\linespread{1.1}

\pagestyle{fancy}
\lhead{\hmwkAuthorName}
\chead{}
\rhead{\hmwkClass\ (\hmwkClassInstructor): \hmwkTitle}
\lfoot{\lastxmark}
\cfoot{\thepage}

\renewcommand\headrulewidth{0.4pt}
\renewcommand\footrulewidth{0.4pt}

\setlength\parindent{0pt}

%
% Create Problem Sections
%

\newcommand{\enterProblemHeader}[1]{
    \nobreak\extramarks{}{Problem \arabic{#1} continued on next page\ldots}\nobreak{}
    \nobreak\extramarks{Problem \arabic{#1} (continued)}{Problem \arabic{#1} continued on next page\ldots}\nobreak{}
}

\newcommand{\exitProblemHeader}[1]{
    \nobreak\extramarks{Problem \arabic{#1} (continued)}{Problem \arabic{#1} continued on next page\ldots}\nobreak{}
    \stepcounter{#1}
    \nobreak\extramarks{Problem \arabic{#1}}{}\nobreak{}
}

\setcounter{secnumdepth}{0}
\newcounter{partCounter}
\newcounter{homeworkProblemCounter}
\setcounter{homeworkProblemCounter}{1}
\nobreak\extramarks{Problem \arabic{homeworkProblemCounter}}{}\nobreak{}

\counterwithin*{equation}{section}
\counterwithin*{equation}{subsection}

%
% Homework Problem Environment
%
% This environment takes an optional argument. When given, it will adjust the
% problem counter. This is useful for when the problems given for your
% assignment aren't sequential. See the last 3 problems of this template for an
% example.
%
\newenvironment{homeworkProblem}[1][-1]{
    \ifnum#1>0
        \setcounter{homeworkProblemCounter}{#1}
    \fi
    \section{Problem \arabic{homeworkProblemCounter}}
    \setcounter{partCounter}{1}
    \enterProblemHeader{homeworkProblemCounter}
}{
    \exitProblemHeader{homeworkProblemCounter}
}

%
% Homework Details
%   - Title
%   - Class
%   - Instructor
%   - Author
%

\newcommand{\hmwkTitle}{Lecture\ \#6}
\newcommand{\hmwkClass}{MATH 3380 / Analysis 1}
\newcommand{\hmwkClassInstructor}{Dr. Welsh}
\newcommand{\hmwkAuthorName}{\textbf{Joshua Mitchell}}

%
% Title Page
%

\title{
    \vspace{2in}
    \textmd{\textbf{\hmwkClass:\ \hmwkTitle}}\\
    \normalsize\vspace{0.1in}\small\vspace{0.1in}\large{\textit{\hmwkClassInstructor}}
    \vspace{3in}
}

\author{\hmwkAuthorName}
\date{}

\renewcommand{\part}[1]{\textbf{\large Part \Alph{partCounter}}\stepcounter{partCounter}\\}

%
% Various Helper Commands
%

% For derivatives
\newcommand{\deriv}[1]{\frac{\mathrm{d}}{\mathrm{d}x} (#1)}

% For partial derivatives
\newcommand{\pderiv}[2]{\frac{\partial}{\partial #1} (#2)}

% Integral dx
\newcommand{\dx}{\mathrm{d}x}

% Alias for the Solution section header
\newcommand{\solution}{\textbf{\large Solution}}

% Probability commands: Expectation, Variance, Covariance, Bias
\newcommand{\E}{\mathrm{E}}
\newcommand{\Var}{\mathrm{Var}}
\newcommand{\Cov}{\mathrm{Cov}}
\newcommand{\Bias}{\mathrm{Bias}}

% Formatting commands:
\newcommand\bsc[2][\DefaultOpt]{%
  \def\DefaultOpt{#2}%
  \section[#1]{#2}%
}
\newcommand\ssc[2][\DefaultOpt]{%
  \def\DefaultOpt{#2}%
  \subsection[#1]{#2}%
}
\newcommand{\bgpf}{\begin{proof} $ $\newline}

\newcommand{\bgeq}{\begin{equation*}}
\newcommand{\eeq}{\end{equation*}}	

\newcommand{\balist}{\begin{enumerate}[label=\alph*.]}
\newcommand{\elist}{\end{enumerate}}

\newcommand{\bilist}{\begin{enumerate}[label=\roman*)]}	

\newcommand{\bgsp}{\begin{split}}
% \newcommand{\esp}{\end{split}} % doesn't work for some reason.

\newcommand\prs[1]{~~~\textbf{(#1)}}

\newcommand{\lt}[1]{\textbf{Let: } #1}
     							   %  if you're setting it to be true
\newcommand{\supp}[1]{\textbf{Suppose: } #1}
     							   %  Suppose (if it'll end up false)
\newcommand{\wts}[1]{\textbf{Want to show: } #1}
     							   %  Want to show
\newcommand{\as}[1]{\textbf{Assume: } #1}
     							   %  if you think it follows from truth
\newcommand{\bpth}[1]{\textbf{(#1)}}

\newcommand{\step}[2]{\begin{equation}\tag{#2}#1\end{equation}}
\newcommand{\epf}{\end{proof}}



% Analysis / Logical commands:

\newcommand{\br}{\mathbb{R}}       % |R
\newcommand{\bq}{\mathbb{Q}}       % |Q
\newcommand{\bn}{\mathbb{N}}       % |N
\newcommand{\bc}{\mathbb{C}}       % |C

\newcommand{\ep}{\epsilon}         % epsilon
\newcommand{\fa}{\forall}          % for all

\newcommand{\es}{\emptyset}        % empty set
\newcommand{\sbs}{\subset}         % subset of

\newcommand{\lra}{\longrightarrow} % implies ----->
\newcommand{\rar}{\Rightarrow}     % implies -->

\newcommand{\lla}{\longleftarrow}  % implies <-----
\newcommand{\lar}{\Leftarrow}      % implies <--

\newcommand{\pr}{^\prime} 		   % prime (i.e. R')
\newcommand{\butnot}[2]{#1\setminus{\textrm{#2}}}

\newcommand{\nbho}[3]{\textrm{N(}#1, #2\textrm{) }\cap \textrm{ #3} \neq \emptyset}
     							   %  N(x, eps) intersect S \= emptyset
\newcommand{\nbhe}[3]{\textrm{N(}#1, #2\textrm{) }\cap \textrm{ #3} = \emptyset}
     							   %  N(x, eps) intersect S  = emptyset
\newcommand{\dnbho}[3]{\textrm{N*(}#1, #2\textrm{) }\cap \textrm{ #3} \neq \emptyset}
     							   % N*(x, eps) intersect S \= emptyset
\newcommand{\dnbhe}[3]{\textrm{N*(}#1, #2\textrm{) }\cap \textrm{ #3} = \emptyset}
     							   % N*(x, eps) intersect S = emptyset
%%%% ACTUAL BEGINNING OF DOCUMENT %%%%

\begin{document}

\bsc{Misc. Notes:}

Ex 3.4.8. e) \ 

$\br$ is both open and closed.

int $\br = \br\pr$

$\es =$ bd $\br \sbs \br$

-----

S $\sbs \br$ \par
$s \in S\pr$ if, $\fa \ep > 0$, $\dnbho{x}{\epsilon}{S}$

-----

HW: pages 141 - 142, numbers 6, 7, 15, 17, 19, 21

\bsc{Theorem 3.4.17 - pg 118}

\lt{$S \sbs \br$} 

Then

\balist
\item S is closed iff S$\pr \sbs$ S
\item cl S is a closed set
\item S is closed iff S = cl S
\item cl S = S U S$\pr$ = S $\cup$ bd S
\elist

\bgpf
\ssc{a)}
S is closed iff S$\pr \sbs$ S \

$\lra$

\supp{S is closed.}

\wts{S$\pr \sbs$ S} \

\lt{x $\in$ S'} \
 
Thus, $\fa \ep > 0$ 

\step{\nbhe{x}{\epsilon}{S}}{1}

\wts{x $\in S$} \

\as{x $\not\in$ S} \

Then, from \bpth{1},
\step{\nbho{x}{\epsilon}{S}}{2}
and
\step{\nbho{x}{\epsilon}{$\neg$S}}{3}

From \bpth{2} and \bpth{3}, \

x $\in$ bd S $\subset$ S by definition of a closed set. This is a contradiction. \

Hence, x $\in$ S.

This proves:
 
\[S\pr \subset S\]

$\lla$ \

Conversely, \

\supp{S$\pr \sbs $S}

\wts{$\butnot{\br}{S}$ is open $\rar$ S is closed.} \

\lt{x $\in \butnot{\br}{S}$}


\wts{$\exists \ep > 0$ st N(x, $\ep$) $\sbs \butnot{\br}{S}$}

Since x $\not\in$ S, we see that x not $\not\in$ S'.

Thus, $\exists \ep > 0$ st $\nbhe{x}{\epsilon}{S}$ \

Since x $\not\in$ S, we have:

\step{\nbhe{x}{\epsilon}{S}}{1}

Hence, N(x, $\ep$) $\sbs \butnot{\br}{S}$ , which proves that $\butnot{\br}{S}$ is open, or, equivalently, that S is closed. \

This completes the proof of a).

\ssc{b)}
cl S is a closed set \

Recall that cl S = S $\cup$ S$\pr$.

\wts{$\butnot{\br}{cl S}$ is open $\rar$ cl S is closed}

\lt{x $\in$ cl ($\butnot{\br}{S}$) (aka (S $\cup$ S$\pr$) Compliment)} \

We must find an $\ep > 0$ st N(x, $\ep$) $\subset$ cl ($\butnot{\br}{S}$) \

Now x $\not\in$ S and x $\not\in$ S$\pr$. \

$\exists \ep > 0$ st $\dnbhe{x}{\epsilon}{S}$ \

However, x $\not\in$ S, so \

\step{\nbhe{x}{\epsilon}{S}}{1}

We claim that N(x, $\ep$) $\cap$ S' $= \emptyset$ \

Since:

\[\neg [x \in S \cup S\pr]\]
\[\neg [x \in S \textrm{ or }x \in S\pr]\]
\[x \not\in S \textrm{ and }x \not\in S\pr\]

which is equivalent to N(x, $\ep$) $\sbs$ $\butnot{\br}{S$\pr$}$ \

\lt{y $\in$ N(x, $\ep$)}

By Theorem 2(a), the set N(x, $\ep$) is open. \

So $\exists \hat{\epsilon} > 0$ st N(y, $\hat{\epsilon}$) $\subset$ N(x,$\ep$). \

In particular, y $\not\in$ N(x, $\ep$). \

From \bpth{1} \

N*(y, $\hat{\epsilon}$) $\cap$ S $= \emptyset$. \

So, y $\not\in$ S$\pr$ or, equivalently, y $\in \butnot{\br}{S$\pr$}$.

This proves that N(x, $\ep$) $\subset$ $\butnot{\br}{S$\pr$}$ or, equivalently,

\step{\nbhe{x}{\epsilon}{S$\pr$}}{2}

From \bpth{1} and \bpth{2}, N(x, $\ep$) $\cap$ (S $\cup$ S$\pr$) $= \es$. \

Hence, 
\step{N(x, \ep) \sbs (S \cup S\pr)^C = \textrm{ cl }S^C}{3}

Thus, \bpth{3} and * prove that cl S$^C$ is open. \

Hence, by Theorem 3.4.7, cl S is closed.

\ssc{c)} S is closed iff S $=$ cl S ($=$ S $\cup$ S') \

$\lra$ \

\supp{S is closed.}

\wts{S = S $\cup$ S$\pr$.} \

By definition, S $\subset$ S $\cup$ S$\pr$. \

\wts{S $\cup$ S$\pr \subset$ S}

Let x $\in$ S $\cup$ S'. \

If x $\in$ S, then we are finished. \

If x $\in$ $\butnot{S\pr}{S}$ Venn Diagram: (S     (    )xxS') \

Then by a), S' $\subset$ S, since S is closed.

Hence, x $\in$ S, and we are finished.\

$\lla$ \

Conversely, \

\supp{S $= S \cup S\pr$} \

\wts{S is closed.}\

By (b), cl S is closed. \

Since, S $= S \cup S\pr$ $=$ cl S, S is also closed.



\ssc{d)} cl S $=$ S $\cup$ S' $=$ S $\cup$ bd S \

\lt{x $\in$ S $\cup$ S'}

If x $\in$ S, then x $\in$ S $\cup$ bd S.\

So, S $\cup$ S\ $\sbs$ S $\cup$ bd S in this case. \

If x $\in$ $\butnot{S\pr}{S}$, then $\fa \ep > 0$, $\nbho{x}{\epsilon}{S}$, which implies x $\in$ $\butnot{\br}{S}$ and $\nbho{x}{\epsilon}{$\butnot{\br}{S}$}$ \

Thus, x $\in$ bd S $\subset$ S $\cup$ bd S. \

Hence, S $\cup$ S' $\subset$ S $\cup$ bd S. \

For the reverse conclusion, let x $\in$ S $\cup$ bd S. \

If x $\in$ S, then x $\in$ S $\cup$ S$\pr$. \
So, in this case, S $\cup$ bd S $\subset$ S $\cup$ S$\pr$ $=$ cl S. \

if x $\in$ $\butnot{\textrm{bd } S}{S}$, then, in particular, \

$\fa \ep > 0$,

\[N*(x, \ep) \cap S \neq \emptyset\]

which implies that x $\in$ S' $\subset$ S $\cup$ S'. \

Hence, S $\cup$ bd S $\subset$ S $\cup$ S$\pr$. \

Hence, result.



\epf
\end{document}