% Thank you Josh Davis for this template!
% https://github.com/jdavis/latex-homework-template/blob/master/homework.tex

\documentclass{article}

\usepackage{fancyhdr}
\usepackage{extramarks}
\usepackage{amsmath}
\usepackage{amssymb}
\usepackage{amsthm}
\usepackage{amsfonts}
\usepackage{tikz}
\usepackage[plain]{algorithm}
\usepackage{algpseudocode}
\usepackage{enumitem}
\usepackage{chngcntr}

\usetikzlibrary{automata,positioning}

%
% Basic Document Settings
%

\topmargin=-0.45in
\evensidemargin=0in
\oddsidemargin=0in
\textwidth=6.5in
\textheight=9.0in
\headsep=0.25in

\linespread{1.1}

\pagestyle{fancy}
\lhead{\hmwkAuthorName}
\chead{}
\rhead{\hmwkClass\ (\hmwkClassInstructor): \hmwkTitle}
\lfoot{\lastxmark}
\cfoot{\thepage}

\renewcommand\headrulewidth{0.4pt}
\renewcommand\footrulewidth{0.4pt}

\setlength\parindent{0pt}

%
% Create Problem Sections
%

\newcommand{\enterProblemHeader}[1]{
    \nobreak\extramarks{}{Problem \arabic{#1} continued on next page\ldots}\nobreak{}
    \nobreak\extramarks{Problem \arabic{#1} (continued)}{Problem \arabic{#1} continued on next page\ldots}\nobreak{}
}

\newcommand{\exitProblemHeader}[1]{
    \nobreak\extramarks{Problem \arabic{#1} (continued)}{Problem \arabic{#1} continued on next page\ldots}\nobreak{}
    \stepcounter{#1}
    \nobreak\extramarks{Problem \arabic{#1}}{}\nobreak{}
}

\setcounter{secnumdepth}{0}
\newcounter{partCounter}
\newcounter{homeworkProblemCounter}
\setcounter{homeworkProblemCounter}{1}
\nobreak\extramarks{Problem \arabic{homeworkProblemCounter}}{}\nobreak{}

\counterwithin*{equation}{section}
\counterwithin*{equation}{subsection}

%
% Homework Problem Environment
%
% This environment takes an optional argument. When given, it will adjust the
% problem counter. This is useful for when the problems given for your
% assignment aren't sequential. See the last 3 problems of this template for an
% example.
%
\newenvironment{homeworkProblem}[1][-1]{
    \ifnum#1>0
        \setcounter{homeworkProblemCounter}{#1}
    \fi
    \section{Problem \arabic{homeworkProblemCounter}}
    \setcounter{partCounter}{1}
    \enterProblemHeader{homeworkProblemCounter}
}{
    \exitProblemHeader{homeworkProblemCounter}
}

%
% Homework Details
%   - Title
%   - Class
%   - Instructor
%   - Author
%

\newcommand{\hmwkTitle}{Lecture\ \#1}
\newcommand{\hmwkClass}{MATH 3380 / Analysis 1}
\newcommand{\hmwkClassInstructor}{Dr. Welsh}
\newcommand{\hmwkAuthorName}{\textbf{Joshua Mitchell}}

%
% Title Page
%

\title{
    \vspace{2in}
    \textmd{\textbf{\hmwkClass:\ \hmwkTitle}}\\
    \normalsize\vspace{0.1in}\small\vspace{0.1in}\large{\textit{\hmwkClassInstructor}}
    \vspace{3in}
}

\author{\hmwkAuthorName}
\date{}

\renewcommand{\part}[1]{\textbf{\large Part \Alph{partCounter}}\stepcounter{partCounter}\\}

%
% Various Helper Commands
%

% For derivatives
\newcommand{\deriv}[1]{\frac{\mathrm{d}}{\mathrm{d}x} (#1)}

% For partial derivatives
\newcommand{\pderiv}[2]{\frac{\partial}{\partial #1} (#2)}

% Integral dx
\newcommand{\dx}{\mathrm{d}x}

% Alias for the Solution section header
\newcommand{\solution}{\textbf{\large Solution}}

% Probability commands: Expectation, Variance, Covariance, Bias
\newcommand{\E}{\mathrm{E}}
\newcommand{\Var}{\mathrm{Var}}
\newcommand{\Cov}{\mathrm{Cov}}
\newcommand{\Bias}{\mathrm{Bias}}

% Formatting commands:
\newcommand\bsc[2][\DefaultOpt]{%
  \def\DefaultOpt{#2}%
  \section[#1]{#2}%
}
\newcommand\ssc[2][\DefaultOpt]{%
  \def\DefaultOpt{#2}%
  \subsection[#1]{#2}%
}
\newcommand{\bgpf}{\begin{proof} $ $\newline}

\newcommand{\bgeq}{\begin{equation*}}
\newcommand{\eeq}{\end{equation*}}	

\newcommand{\balist}{\begin{enumerate}[label=\alph*.]}
\newcommand{\elist}{\end{enumerate}}

\newcommand{\bilist}{\begin{enumerate}[label=\roman*)]}	

\newcommand{\bgsp}{\begin{split}}
% \newcommand{\esp}{\end{split}} % doesn't work for some reason.

\newcommand\prs[1]{~~~\textbf{(#1)}}

\newcommand{\lt}[1]{\textbf{Let: } #1}
     							   %  if you're setting it to be true
\newcommand{\supp}[1]{\textbf{Suppose: } #1}
     							   %  Suppose (if it'll end up false)
\newcommand{\wts}[1]{\textbf{Want to show: } #1}
     							   %  Want to show
\newcommand{\as}[1]{\textbf{Assume: } #1}
     							   %  if you think it follows from truth
\newcommand{\bpth}[1]{\textbf{(#1)}}

\newcommand{\step}[2]{\begin{equation}\tag{#2}#1\end{equation}}
\newcommand{\epf}{\end{proof}}



% Analysis / Logical commands:

\newcommand{\br}{\mathbb{R}}       % |R
\newcommand{\bq}{\mathbb{Q}}       % |Q
\newcommand{\bn}{\mathbb{N}}       % |N
\newcommand{\bc}{\mathbb{C}}       % |C

\newcommand{\ep}{\epsilon}         % epsilon
\newcommand{\fa}{\forall}          % for all

\newcommand{\es}{\emptyset}        % empty set
\newcommand{\sbs}{\subset}         % subset of

\newcommand{\lra}{\longrightarrow} % implies ----->
\newcommand{\rar}{\Rightarrow}     % implies -->

\newcommand{\lla}{\longleftarrow}  % implies <-----
\newcommand{\lar}{\Leftarrow}      % implies <--

\newcommand{\pr}{^\prime} 		   % prime (i.e. R')
\newcommand{\butnot}[2]{#1\setminus{\textrm{#2}}}

\newcommand{\nbho}[3]{\textrm{N(}#1, #2\textrm{) }\cap \textrm{ #3} \neq \emptyset}
     							   %  N(x, eps) intersect S \= emptyset
\newcommand{\nbhe}[3]{\textrm{N(}#1, #2\textrm{) }\cap \textrm{ #3} = \emptyset}
     							   %  N(x, eps) intersect S  = emptyset
\newcommand{\dnbho}[3]{\textrm{N*(}#1, #2\textrm{) }\cap \textrm{ #3} \neq \emptyset}
     							   % N*(x, eps) intersect S \= emptyset
\newcommand{\dnbhe}[3]{\textrm{N*(}#1, #2\textrm{) }\cap \textrm{ #3} = \emptyset}
     							   % N*(x, eps) intersect S = emptyset
%%%% ACTUAL BEGINNING OF DOCUMENT %%%%

\begin{document}
\bsc{Theorem 3.2.8 - pg 118}

Let x, y $\in$ $\mathbb{R}$

\balist
\item If x $\leq$ y + $\ep$ $\fa$ $\ep$ $>$ $0$, then x $\leq$ y
\item If $|x - y|$ $\leq$ $\ep$ $\fa$ $\ep$ $>$ $0$, then $|x - y|$ $=$ $0$ or, evidently, $x = y$
\elist

\ssc{a)}
If x $\leq$ y + $\ep$ $\fa$ $\ep$ $>$ $0$, then x $\leq$ y
\bgpf

Suppose that:
$$ x > y $$
$$ x - y > 0 $$

\bigskip

Let $$ \ep = \frac{x + y}{2} > 0$$

See that
\[
            \bgsp
                y + \ep  & 
                \\
                &= y + \frac{x + y}{2}
                \\
                &= y + \frac{x}{2} - \frac{y}{2}
                \\
                &= \frac{x}{2} + \frac{y}{2}
                \\
                &< \frac{x}{2} + \frac{x}{2}
                \\
                &< x
                \\
                y + \ep  &< x \prs{1}
            \end{split}
\]

Thus, by contrapositive, the result is true.

\epf
\pagebreak
\ssc{b)}

If $|x - y|$ $\leq$ $\ep$ $\fa$ $\ep$ $>$ $0$, then $|x - y|$ $=$ $0$ or, evidently, $x = y$

\bgpf

Suppose that:

$$|x - y| > 0$$

Let

$$ \ep = \frac{|x - y|}{2} $$

See that

\[
            \bgsp
                &1 > \frac{1}{2}
                \\
                &|x - y| > \frac{1}{2}|x - y|
                \\
                &|x - y| > \epsilon
            \end{split}
\]

Thus, by contrapositive, the result is true.

\epf

\pagebreak

\bsc{Definition 3.2.9}

If x $\in \br$,

\bgeq
  |x|=\begin{cases}
    x, & \text{if $x \geq 0$}.\\
    -x, & \text{if $x < 0$}.
  \end{cases}
\eeq

\bsc{Theorem 3.2.10}
Let x, y $\in \br$ and a $\geq 0$ \\
\\
Then
\balist
\item $|x| \geq 0$
\item $|x| \leq a iff -a \leq x \leq a$
\item $|xy| = |x||y|$
\item $|x + y| \leq |x| + |y|$
\elist

\ssc{a)}
$|x| \geq 0$
\bgpf

Case:
\bilist
\item $x \geq 0$: \\
	then $|x| = x \geq 0$
\item $x < 0 \rar -x > 0$ \\
	then $|x| = -x \geq 0$
\elist

Hence, result
\epf

\ssc{b)}

$|x| \leq a$ iff $-a \leq x \leq a$ \\

Since it's a biconditional, first we prove p $\rar$ q, then q $\rar$ p.
\bgpf

Notice that:
$$ -a \leq -|x|$$
Case:
\bilist
\item $x \geq 0$ \\
	then $0 \leq x = |x|$ \\
	and  $\therefore |x| \leq a$
	
	Also, since $x = |x| \geq 0$,
	$-a \leq x$ or $-a \leq 0$
	$-a \leq x \leq a$
\item $x < 0$
	$|x| = -x \leq a
	x \geq -a
	\therefore -a \leq x
	-a \leq x \leq a$
\elist

Hence, result.

$\lla$

Conversely, we shall prove that q $\rar$ p

\supp{$-a \leq x \leq a$}

Then:

\bilist
\item x $\geq 0$, then $|x| = x \leq a$
\item x $< 0$, then $|x| = -x \leq a$
\elist

Hence, result.

\ssc{c)} $|xy| = |x||y|$

Notice that if x $=$ 0 (p) or y $=$ 0 (q), then $|xy| = 0 = |x||y|$.

WLOG, assume that not [p or q] $=$ not p $\cap$ not q.

\bilist
\item $x > 0$ and $y > 0$
	then $|x| = x$, $|y| = y$
	Also, $xy > 0$
	So, $|xy| = xy = |x||y|$
\item $x < 0$, $y < 0$
	then $|x| = -x, |y| = -y, xy > 0$
	So, $|xy| = xy = -|x|(-|y|) = |x||y|$
\item $x > 0$, $y < 0$ OR $y > 0$, $x < 0$
	WLOG, let $x > 0$, $y < 0$
	$|xy| = |x||y|$
	$|yx| = |y||x|$
	$|x| = x$, $|y| = -y$, $xy < 0$
	So, $|xy| = -(xy)$ \
		$-[|x|(-|y|)]$ \ 
		$-[-|x||y|]$ \
		$|x||y|$ \ 
\elist
	
\ssc{d)} $|x + y| \leq |x| + |y|$ \

\lt{Z $= x + y$, and a $= |x| + |y|$}

If $a \geq 0$, then $|Z| \leq a$ iff $-a \leq Z \leq a$

$-(|x| + |y|) \leq x + y \leq |x| + |y|$ \

From b), since $|x| + |y| \geq 0$,

\wts{$-(|x| + |y|) \leq x + y \leq |x| + |y|$}

Notice: $-|x| \leq x \leq |x|$

$|x| = x$ or $|x| = -x$ or $-|x| = x$ \

Then \

$-|x| - |y| \leq x + y \leq |x| + |y|$ \

$-(|x| + |y|) \leq x + y \leq |x| + |y|$

By b), this is equivalent to \

$|x + y| \leq |x| + |y|$

\epf
\end{document}